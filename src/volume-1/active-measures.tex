\section{Russian ``Active Measures'' Social Media Campaign}

The first form of Russian election influence came principally from the Internet Research Agency, LLC (IRA), a Russian organization funded by Yevgeniy Viktorovich Prigozhin and companies he controlled, including Concord Management and Consulting LLC and Concord Catering (collectively "Concord").% 2
\footnote{The Office is aware of reports that other Russian entities engaged in similar active measures operations targeting the United States.
Some evidence collected by the Office corroborates those reports, and the Office has shared that evidence with other offices in the Department of Justice and FBI.}
The IRA conducted social media operations targeted at large U.S. audiences with the goal of sowing discord in the U.S. political system.% 3
\footnote{\xblackout{Harm to Ongoing Matter: Lorem ipsum dolor sit amet, consectetur adipiscing elit}
see also SM-2230634, serial 44 (analysis).
The FBI case number cited here, and other FBI case numbers identified in the report, should be treated as law enforcement sensitive given the context.
The report contains additional law enforcement sensitive information.}
These operations constituted "active measures" (активные мероприятия), a term that typically refers to operations conducted by Russian security services aimed at influencing the course of international affairs.% 4
\footnote{As discussed in Part V below, the active measures investigation has resulted in criminal charges against 13 individual Russian nationals and three Russian entities, principally for conspiracy to defraud the United States, in violation of 18 U.S.C. § 371.
See Volume I, Section V.A, infra; Indictment, United States v. Internet Research Agency, et al., 1 :18-cr-32 (D.D.C. Feb. 16, 2018), Doc. 1 ("Internet Research Agency Indictment").}

The IRA and its employees began operations targeting the United States as early as 2014.
Using fictitious U.S. personas, IRA employees operated social media accounts and group pages designed to attract U.S. audiences.
These groups and accounts, which addressed divisive U.S. political and social issues, falsely claimed to be controlled by U.S. activists.
Over time, these social media accounts became a means to reach large U.S. audiences.
IRA employees travelled to the United States in mid-2014 on an intelligence-gathering mission to obtain information and photographs for use in their social media posts.

IRA employees posted derogatory information about a  number of candidates in the 2016 U.S. presidential election.
By early to mid-2016, IRA operations included supporting the Trump Campaign and disparaging candidate Hillary Clinton.
The IRA made various expenditures to carry out those activities, including buying political advertisements on social media in the names of U.S. persons and entities.
Some IRA employees, posing as U.S. persons and without revealing their Russian association, communicated electronically with individuals associated with the Trump Campaign and with other political activists to seek to coordinate political activities, including the staging of political rallies.% 5
\footnote{Internet Research Agency Indictment \S\S 52, 54, 55(a), 56, 74; \xblackout{Harm to Ongoing Matter: Lorem ipsum dolor sit amet, consectetur adipiscing elit, sed do eiusmod tempor}}
The investigation did not identify evidence that any U.S. persons knowingly or intentionally coordinated with the IRA's interference operation.

By the end of the 2016 U.S. election, the IRA had the ability to reach millions of U.S. persons through their social media accounts.
Multiple IRA-controlled Facebook groups and Instagram accounts had hundreds of thousands of U.S. participants.
IRA-controlled Twitter accounts separately had tens of thousands of followers, including multiple U.S. political figures who retweeted IRA-created content.
In November 2017, a Facebook representative testified that Facebook had identified 470 IRA-controlled Facebook accounts that collectively made 80,000 posts between January 2015 and August 2017.
Facebook estimated the IRA reached as many as 126 million persons through its Face book accounts.% 6
\footnote{Social Media Influence in the 2016 US. Election, Hearing Before the Senate Select Committee on Intelligence, 115th Cong. 13 (11/1/17) (testimony of Colin Stretch, General Counsel of Facebook)
("We estimate that roughly 29 million people were served content in their News Feeds directly from the IRA's 80,000 posts over the two years.
Posts from these Pages were also shared, liked, and followed by people on Facebook, and, as a result, three times more people may have been exposed to a story that originated from the Russian operation.
Our best estimate is that approximately 126 million people may have been served content from a Page associated with the IRA at some point during the two-year period.").
The Facebook representative also testified that Facebook had identified 170 Instagram accounts that posted approximately 120,000 pieces of content during that time.
Facebook did not offer an estimate of the audience reached via Instagram.}
In January 2018, Twitter announced that it had identified 3,814 IRA-controlled Twitter accounts and notified approximately 1 .4 million people Twitter believed may have been in contact with an iRA-controlled account.% 7
\footnote{Twitter, Update on Twitter's Review of the 2016 US Election (Jan. 31, 2018).}

\subsection{Structure of the Internet Research Agency}

\xblackout{Harm to Ongoing Matter: Lorem ipsum dolor sit amet, consectetur adipiscing elit, sed do eiusmod tempor}% 8
\footnote{See SM-2230634, serial 92.}
\xblackout{Harm to Ongoing Matter: Lorem ipsum dolor sit amet, consectetur adipiscing elit, sed do eiusmod tempor}% 9
\footnote{\xblackout{Harm to Ongoing Matter: Lorem ipsum dolor sit amet, consectetur adipiscing elit, sed do eiusmod tempor}}
\xblackout{Harm to Ongoing Matter: Lorem ipsum dolor sit amet, consectetur adipiscing elit, sed do eiusmod tempor}% 10
\footnote{\xblackout{Harm to Ongoing Matter: Lorem ipsum dolor sit amet, consectetur adipiscing elit, sed do eiusmod tempor}}

The organization quickly grew.
\xblackout{Harm to Ongoing Matter: Lorem ipsum dolor sit amet, consectetur adipiscing elit, sed do eiusmod tempor}% 11
\footnote{See SM-2230634, serial 86 \xblackout{Harm to Ongoing Matter: Lorem ipsum dolor sit amet, consectetur adipiscing elit, sed do eiusmod tempor}}
\xblackout{Harm to Ongoing Matter: Lorem ipsum dolor sit amet, consectetur adipiscing elit, sed do eiusmod tempor}% 12
\footnote{\xblackout{Harm to Ongoing Matter: Lorem ipsum dolor sit amet, consectetur adipiscing elit, sed do eiusmod tempor}}

The growth of the organization also led to a more detailed organizational structure.
\xblackout{Lorem ipsum dolor sit amet, consectetur adipiscing elit, sed do eiusmod tempor incididunt ut labore et dolore magna aliqua. Ut enim ad minim veniam, quis nostrud exercitation ullamco laboris nisi ut aliquip ex ea commodo consequat. Duis aute irure dolor in reprehenderit in voluptate velit esse cillum dolore eu fugiat nulla pariatur. Excepteur sint occaecat cupidatat non proident, sunt in culpa qui officia deserunt mollit anim id est laborum.}% 13
\footnote{\xblackout{Harm to Ongoing Matter: Lorem ipsum dolor sit amet, consectetur adipiscing elit, sed do eiusmod tempor}}

Two individuals headed the IRA's management: its general director, Mikhail Bystrov, and its executive director, Mikhail Burchik.
\xblackout{Harm to Ongoing Matter: Lorem ipsum dolor sit amet, consectetur adipiscing elit, sed do eiusmod tempor}% 14
\footnote{See, e.g., SM-2230634, serials 9, 113 \& 180 \xblackout{Harm to Ongoing Matter: Lorem ipsum dolor sit amet, consectetur adipiscing elit, sed do eiusmod tempor}}
\xblackout{Harm to Ongoing Matter: Lorem ipsum dolor sit amet, consectetur adipiscing elit, sed do eiusmod tempor}% 15
\footnote{\xblackout{Harm to Ongoing Matter: Lorem ipsum dolor sit amet, consectetur adipiscing elit, sed do eiusmod tempor}}

As early as spring of 2014, the IRA began to hids its funding and activities.
\xblackout{Lorem ipsum dolor sit amet, consectetur adipiscing elit, sed do eiusmod tempor incididunt ut labore et dolore magna aliqua. Ut enim ad minim veniam, quis nostrud exercitation ullamco laboris nisi ut aliquip ex ea commodo consequat. Duis aute irure dolor in reprehenderit in voluptate velit esse cillum dolore eu fugiat nulla pariatur. Excepteur sint occaecat cupidatat non proident, sunt in culpa qui officia deserunt mollit anim id est laborum.}% 16
\footnote{\xblackout{Harm to Ongoing Matter: Lorem ipsum dolor sit amet, consectetur adipiscing elit, sed do eiusmod tempor.} See SM-2230634, serials 131 \& 204.}

The IRA's U.S. operations are part of a larger set of interlocking operations known as ``Project Lakhta,'' \xblackout{Harm to Ongoing Matter: Lorem ipsum dolor sit amet, consectetur adipiscing elit, sed do eiusmod tempor}% 17
\footnote{\xblackout{Harm to Ongoing Matter: Lorem ipsum dolor sit amet, consectetur adipiscing elit, sed do eiusmod tempor}}
\xblackout{Harm to Ongoing Matter: Lorem ipsum dolor sit amet, consectetur adipiscing elit, sed do eiusmod tempor}% 18
\footnote{\xblackout{Harm to Ongoing Matter: Lorem ipsum dolor sit amet, consectetur adipiscing elit, sed do eiusmod tempor}}

\subsection{Funding and Oversight from Concord and Prigozhin}

Until at least February 2018, Yevgeniy Viktorovich Prigozhin and two Concord companies funded the IRA.
Prigozhin is a wealthy Russian businessman who served as the head of Concord.
\xblackout{Harm to Ongoing Matter: Lorem ipsum dolor sit amet, consectetur adipiscing elit, sed do eiusmod tempor}
Prigozhin was sanctioned by the U.S. Treasury Deparment in December 2016,% 19
\footnote{U.S. Treasury Deprutment, "Treasury Sanctions Individuals and Entities in Connection with Russia's Occupation of Crimea and the Conflict in Ukraine" (Dec. 20, 2016).}
\xblackout{Harm to Ongoing Matter: Lorem ipsum dolor sit amet, consectetur adipiscing elit, sed do eiusmod tempor}% 20
\footnote{\xblackout{Harm to Ongoing Matter: Lorem ipsum dolor sit amet, consectetur adipiscing elit, sed do eiusmod tempor}}
\xblackout{Harm to Ongoing Matter: Lorem ipsum dolor sit amet, consectetur adipiscing elit, sed do eiusmod tempor}% 21
\footnote{\xblackout{Harm to Ongoing Matter: Lorem ipsum dolor sit amet, consectetur adipiscing elit, sed do eiusmod tempor}}
Numerous media sources have reported on Prigozhin's ties to Putin, and the two have appeared together in public photographs.% 22
\footnote{See, e.g., Neil MacFarquhar, Yevgeny Prigozhin, Russian Oligarch Indicted by US., Is Known as "Putin's Cook", New York Times (Feb. 16, 2018).}

\xblackout{Harm to Ongoing Matter: Lorem ipsum dolor sit amet, consectetur adipiscing elit, sed do eiusmod tempor}% 23
\footnote{\xblackout{Harm to Ongoing Matter: Lorem ipsum dolor sit amet, consectetur adipiscing elit, sed do eiusmod tempor}}
\xblackout{Harm to Ongoing Matter: Lorem ipsum dolor sit amet, consectetur adipiscing elit, sed do eiusmod tempor}


\xblackout{Harm to Ongoing Matter: Lorem ipsum dolor sit amet, consectetur adipiscing elit, sed do eiusmod tempor}% 24
\footnote{\xblackout{Harm to Ongoing Matter: Lorem ipsum dolor sit amet, consectetur adipiscing elit, sed do eiusmod tempor}}
\xblackout{Harm to Ongoing Matter: Lorem ipsum dolor sit amet, consectetur adipiscing elit, sed do eiusmod tempor}% 25
\footnote{\xblackout{Harm to Ongoing Matter: Lorem ipsum dolor sit amet, consectetur adipiscing elit, sed do eiusmod tempor} see also SM-2230634, serial 113 \xblackout{HOM}}
\xblackout{Harm to Ongoing Matter: Lorem ipsum dolor sit amet, consectetur adipiscing elit, sed do eiusmod tempor}

\xblackout{Harm to Ongoing Matter: Lorem ipsum dolor sit amet, consectetur adipiscing elit, sed do eiusmod tempor}
\xblackout{Harm to Ongoing Matter: Lorem ipsum dolor sit amet, consectetur adipiscing elit, sed do eiusmod tempor}


\xblackout{Harm to Ongoing Matter: Lorem ipsum dolor sit amet, consectetur adipiscing elit, sed do eiusmod tempor}% 26
\footnote{\xblackout{Harm to Ongoing Matter: Lorem ipsum dolor sit amet, consectetur adipiscing elit, sed do eiusmod tempor}}
\xblackout{Harm to Ongoing Matter: Lorem ipsum dolor sit amet, consectetur adipiscing elit, sed do eiusmod tempor}% 27
\footnote{\xblackout{Harm to Ongoing Matter: Lorem ipsum dolor sit amet, consectetur adipiscing elit, sed do eiusmod tempor}}

\xblackout{Harm to Ongoing Matter: Lorem ipsum dolor sit amet, consectetur adipiscing elit, sed do eiusmod tempor incididunt ut labore et dolore magna aliqua. Ut enim ad minim veniam, quis nostrud exercitation ullamco laboris nisi ut aliquip ex ea commodo consequat. Duis aute irure dolor in reprehenderit in voluptate velit esse cillum dolore eu fugiat nulla pariatur. Excepteur sint occaecat cupidatat non proident, sunt in culpa qui officia deserunt mollit anim id est laborum.}

\xblackout{Harm to Ongoing Matter: Lorem ipsum dolor sit amet, consectetur adipiscing elit, sed do eiusmod tempor}% 28
\footnote{The term "troll" refers to internet users - in this context, paid operatives - who post inflammatory or otherwise disruptive content on social media or other websites. }
\xblackout{Harm to Ongoing Matter: Lorem ipsum dolor sit amet, consectetur adipiscing elit, sed do eiusmod tempor}

IRA employees were aware that Prigozhin was involved in the IRA's U.S. operations, \xblackout{Harm to Ongoing Matter: Lorem ipsum dolor sit amet, consectetur adipiscing elit, sed do eiusmod tempor}% 29
\footnote{\xblackout{Investigative Technique: Lorem ipsum.} See SM-2230634, serials 131 \& 204.}
\xblackout{Harm to Ongoing Matter: Lorem ipsum dolor sit amet, consectetur adipiscing elit, sed do eiusmod tempor}% 30
\footnote{See SM-2230634, serial 156.}
In May 2016, IRA employees, claiming to be U.S. social activists and administrators of Facebook groups, recruited U.S. persons to hold signs (including one in front of the White House) that read "Happy 55th Birthday Dear Boss," as an homage to Prigozhin (whose 55th birthday was on June 1, 2016).% 31
\footnote{Internet Research Agency Indictment \P 12(b); \textit{see also} 5/26/16 Facebook Messages, ID 1479936895656747 (United Muslims of America) \& \xblackout{Personal Privacy: Lorem ipsum}}
\xblackout{Harm to Ongoing Matter: Lorem ipsum dolor sit amet, consectetur adipiscing elit, sed do eiusmod tempor}% 32
\footnote{\xblackout{Harm to Ongoing Matter: Lorem ipsum dolor sit amet, consectetur adipiscing elit, sed do eiusmod tempor} see also SM-2230634, serial 189. \xblackout{Harm to Ongoing Matter: Lorem ipsum dolor sit amet, consectetur adipiscing elit, sed do eiusmod tempor}.}

\xblackout{Harm to Ongoing Matter: Lorem ipsum dolor sit amet, consectetur adipiscing elit, sed do eiusmod tempor incididunt ut labore et dolore magna aliqua. Ut enim ad minim veniam, quis nostrud exercitation ullamco laboris nisi ut aliquip ex ea commodo consequat. Duis aute irure dolor in reprehenderit in voluptate velit esse cillum dolore eu fugiat nulla pariatur. Excepteur sint occaecat cupidatat non proident, sunt in culpa qui officia deserunt mollit anim id est laborum.}

\subsection{The IRA Targets U.S. Elections}

\subsubsection{The IRA Ramps Up U.S. Operations as Early as 2014}

The IRA's U.S. operatiosn sought to influence public opinion through online media and forums.
By the spring of 2014, the IRA began to consolidate U.S. operations within a single general department, known internally as the ``Translator'' (переводчик) department.
\xblackout{Harm to Ongoing Matter: Lorem ipsum dolor sit amet, consectetur adipiscing elit, sed do eiusmod tempor}
IRA subdivided the Translator Department into different responsibilities, ranging from operations on different social media platforms to analytics to graphics and IT.

\xblackout{Harm to Ongoing Matter: Lorem ipsum dolor sit amet, consectetur adipiscing elit, sed do eiusmod tempor}% 33
\footnote{33}
\xblackout{Harm to Ongoing Matter: Lorem ipsum dolor sit amet, consectetur adipiscing elit, sed do eiusmod tempor}% 34
\footnote{34}

\xblackout{Harm to Ongoing Matter: Lorem ipsum dolor sit amet, consectetur adipiscing elit, sed do eiusmod tempor incididunt ut labore et dolore magna aliqua. Ut enim ad minim veniam, quis nostrud exercitation ullamco laboris nisi ut aliquip ex ea commodo consequat. Duis aute irure dolor in reprehenderit in voluptate velit esse cillum dolore eu fugiat nulla pariatur. Excepteur sint occaecat cupidatat non proident, sunt in culpa qui officia deserunt mollit anim id est laborum.}

\xblackout{Harm to Ongoing Matter: Lorem ipsum dolor sit amet, consectetur adipiscing elit, sed do eiusmod tempor}% 35
\footnote{35}
\xblackout{Harm to Ongoing Matter: Lorem ipsum dolor sit amet, consectetur adipiscing elit, sed do eiusmod tempor}% 36
\footnote{36}

\xblackout{Harm to Ongoing Matter: Lorem ipsum dolor sit amet, consectetur adipiscing elit, sed do eiusmod tempor}

\begin{quote}
\xblackout{Harm to Ongoing Matter: Lorem ipsum dolor sit amet, consectetur adipiscing elit, sed do eiusmod tempor}

\xblackout{Harm to Ongoing Matter: Lorem ipsum dolor sit amet, consectetur adipiscing elit, sed do eiusmod tempor}
\end{quote}

\xblackout{Harm to Ongoing Matter}

\xblackout{Harm to Ongoing Matter: Lorem ipsum dolor sit amet, consectetur adipiscing elit, sed do eiusmod tempor}

\xblackout{Harm to Ongoing Matter}

\xblackout{Harm to Ongoing Matter: Lorem ipsum dolor sit amet, consectetur adipiscing elit, sed do eiusmod tempor}% 37
\footnote{37}

IRA employees also traveled to the United States on intelligence-gathering missions.
In June 2014, four IRA employees applied to the U.S. Department of State to enter the United States, while lying about the purpose of their trip and claiming to be four friends who had met at a party.% 38
\footnote{38}
Ultimately, two IRA employees-Anna Bogacheva and Aleksandra Krylova-received visas and entered the United States on June 4, 2014.

Prior to traveling, Krylova and Bogacheva compiled itineraries and instructions for the trip.
\xblackout{Harm to Ongoing Matter: Lorem ipsum dolor sit amet, consectetur adipiscing elit, sed do eiusmod tempor}% 39
\footnote{39}
\xblackout{Harm to Ongoing Matter: Lorem ipsum dolor sit amet, consectetur adipiscing elit, sed do eiusmod tempor}
\xblackout{Harm to Ongoing Matter: Lorem ipsum dolor sit amet, consectetur adipiscing elit, sed do eiusmod tempor}% 40
\footnote{40}
\xblackout{Harm to Ongoing Matter: Lorem ipsum dolor sit amet, consectetur adipiscing elit, sed do eiusmod tempor}% 41
\footnote{41}

\subsubsection{U.S. Operations Through IRA-Controlled Social Media Accounts}

Dozens of IRA employees were responsible for operating accounts and personas on different U.S. social media platforms.
The IRA referred to employees assigned to operate the social media accounts as "specialists."% 42
\footnote{42}
Starting as early as 2014, the IRA's U.S. operations included social media specialists focusing on Facebook, YouTube, and Twitter.% 43
\footnote{43}
The IRA later added specialists who operated on Tumblr and Instagram accounts.% 44
\footnote{44}

Initially, the IRA created social media accounts that pretended to be the personal accounts of U.S. persons.% 45
\footnote{45}
By early 2015, the IRA began to create larger social media groups or public social media pages that claimed (falsely) to be affiliated with U.S. political and grassroots organizations.
In certain cases, the IRA created accounts that mimicked real U.S. organizations.
For example, one IRA-controlled Twitter account, @TEN\_GOP, purported to be connected to the Tennessee Republican Party.% 46
\footnote{46}
More commonly, the IRA created accounts in the names of fictitious U.S. organizations and grassroots groups and used these accounts to pose as anti-immigration groups, Tea Party activists, Black Lives Matter protestors, and other U.S. social and political activists.

The IRA closely monitored the activity of its social media accounts.
\xblackout{Harm to Ongoing Matter: Lorem ipsum dolor sit amet, consectetur adipiscing elit, sed do eiusmod tempor incididunt ut labore et dolore magna aliqua. Ut enim ad minim veniam, quis nostrud exercitation ullamco laboris nisi ut aliquip ex ea commodo consequat. Duis aute irure dolor in reprehenderit in voluptate velit esse cillum dolore eu fugiat nulla pariatur. Excepteur sint occaecat cupidatat non proident, sunt in culpa qui officia deserunt mollit anim id est laborum.}% 47
\footnote{47}
\xblackout{Harm to Ongoing Matter: Lorem ipsum dolor sit amet, consectetur adipiscing elit, sed do eiusmod tempor}% 48
\footnote{48}

\xblackout{Harm to Ongoing Matter: Lorem ipsum dolor sit amet, consectetur adipiscing elit, sed do eiusmod tempor incididunt ut labore et dolore magna aliqua. Ut enim ad minim veniam, quis nostrud exercitation ullamco laboris nisi ut aliquip ex ea commodo consequat. Duis aute irure dolor in reprehenderit in voluptate velit esse cillum dolore eu fugiat nulla pariatur. Excepteur sint occaecat cupidatat non proident, sunt in culpa qui officia deserunt mollit anim id est laborum.}

By February 2016, internal IRA documents referred to support for the Trump Campaign and opposition to candidate Clinton.% 49
\footnote{49}
For example, \xblackout{Harm to Ongoing Matter} directions to IRA operators \xblackout{Harm to Ongoing Matter: Lorem ipsum dolor sit amet, consectetur adipiscing elit, sed do eiusmod tempor.}
``Main idea: Use any opportunity to criticize Hillary [Clinton] and the rest (except Sanders and Trump - we support them).''% 50
\footnote{50}
\xblackout{Harm to Ongoing Matter: Lorem ipsum dolor sit amet, consectetur adipiscing elit, sed do eiusmod tempor}

The focus on the U.S. presidential campaign continued through 2016.  In \xblackout{HOM} 2016 internal \xblackout{HOM} reviewing the IRA-controlled Facebook group ``Secured Borders,'' the author criticized the "lower number of posts dedicated to criticizing Hillary Clinton" and reminded the Facebook specialist "it is imperative to intensify criticizing Hillary Clinton."% 51
\footnote{51}

IRA employees also acknowledged that their work focused on influencing the U.S. presidential election.
\xblackout{Harm to Ongoing Matter: Lorem ipsum dolor sit amet, consectetur adipiscing elit, sed do eiusmod tempor}

\xblackout{Harm to Ongoing Matter: Lorem ipsum dolor sit amet, consectetur adipiscing elit, sed do eiusmod tempor incididunt ut labore et dolore magna aliqua. Ut enim ad minim veniam, quis nostrud exercitation ullamco laboris nisi ut aliquip ex ea commodo consequat. Duis aute irure dolor in reprehenderit in voluptate velit esse cillum dolore eu fugiat nulla pariatur. Excepteur sint occaecat cupidatat non proident, sunt in culpa qui officia deserunt mollit anim id est laborum.}% 52
\footnote{52}

\subsubsection{U.S. Operations Through Facebook}

Many IRA operations used Facebook accounts created and operated by its specialists.
\xblackout{Harm to Ongoing Matter: Lorem ipsum dolor sit amet, consectetur adipiscing elit, sed do eiusmod tempor}

\xblackout{Harm to Ongoing Matter: Lorem ipsum dolor sit amet, consectetur adipiscing elit, sed do eiusmod tempor}

\begin{enumerate}[i]
    \item \xblackout{Harm to Ongoing Matter: Lorem ipsum dolor sit amet, consectetur adipiscing elit, sed do eiusmod tempor}
    \item \xblackout{Harm to Ongoing Matter: Lorem ipsum dolor sit amet, consectetur adipiscing elit, sed do eiusmod tempor}
    \item \xblackout{Harm to Ongoing Matter: Lorem ipsum dolor sit amet, consectetur adipiscing elit, sed do eiusmod tempor}
\end{enumerate}

\xblackout{Harm to Ongoing Matter: Lorem ipsum dolor sit amet, consectetur adipiscing elit, sed do eiusmod tempor}% 53
\footnote{53}

\xblackout{Harm to Ongoing Matter: Lorem ipsum dolor sit amet, consectetur adipiscing elit, sed do eiusmod tempor incididunt ut labore et dolore magna aliqua. Ut enim ad minim veniam, quis nostrud exercitation ullamco laboris nisi ut aliquip ex ea commodo consequat.}% 54
\footnote{54}
The IRA Facebook groups active during the 2016 campaign covered a range of political issues and included purported conservative groups (with names such as "Being Patriotic,"  "Stop All Immigrants," "Secured Borders," and "Tea Party News"), purported Black social justice groups ("Black Matters,"  "Blacktivist," and "Don't Shoot Us"), LGBTQ groups ("LGBT United"), and religious groups ("United Muslims of America").

Throughout 2016, IRA accounts published an increasing number of materials supporting the Trump Campaign and opposing the Clinton Campaign.
For example, on May 31, 2016, the operational account "Matt Skiber" began to privately message dozens of pro-Trump Facebook groups asking them to help plan a "pro-Trump rally near Trump Tower."% 55
\footnote{55}

To reach larger U.S. audiences, the IRA purchased advertisements from Facebook that promoted the IRA groups on the newsfeeds of U.S. audience members.
According to Facebook, the IRA purchased over 3,500 advertisements, and the expenditures totaled approximately \$100,000.% 56
\footnote{56}

During the U.S. presidential campaign, many IRA-purchased advertisements explicitly supported or opposed a  presidential candidate or promoted U.S. rallies organized by the IRA (discussed below).
As early as March 2016, the IRA purchased advertisements that overtly opposed the Clinton Campaign.
For example, on March 18, 2016, the IRA purchased an advertisement depicting candidate Clinton and a caption that read in part, "If one day God lets this liar enter the White House as a president - that day would be a  real national tragedy."% 57
\footnote{57}
Similarly, on April 6, 2016, the IRA purchased advertisements for its account "Black Matters" calling for a  "flashmob" of U.S. persons to "take a photo with \#HillaryClintonForPrison2016 or \#nohillary2016."% 58
\footnote{58}
IRA-purchased advertisements featuring Clinton were, with very few exceptions, negative.% 59
\footnote{59}

IRA-purchased advertisements referencing candidate Trump largely supported his campaign.
The first known IRA advertisement explicitly endorsing the Trump Campaign was purchased on April 19, 2016.
The IRA bought an advertisement for its Instagram account "Tea Party News" asking U.S. persons to help them "make a patriotic team of young Trump supporters" by uploading photos with the hashtag "\#KIDS4TRUMP."% 60
\footnote{60}
In subsequent months, the IRA purchased dozens of advertisements supporting the Trump Campaign, predominantly through the Facebook groups "Being Patriotic," "Stop All Invaders," and "Secured Borders."

Collectively, the IRA's social media accounts reached tens of millions of U.S. persons.
Individual IRA social media accounts attracted hundreds of thousands of followers.
For example, at the time they were deactivated by Facebook in mid-2017, the IRA's "United Muslims of America" Facebook group had over 300,000 followers, the "Don't Shoot Us" Facebook group had over 250,000 followers, the "Being Patriotic" Facebook group had over 200,000 followers, and the "Secured Borders" Facebook group had over 130,000 followers.% 61
\footnote{61}
According to Facebook, in total the IRA-controlled accounts made over 80,000 posts before their deactivation in August 2017, and these posts reached at least 29 million U.S persons and" may have reached an estimated 126 million people."% 62
\footnote{62}

\subsubsection{U.S. Operations Through Twitter}

A number of IRA employees assigned to the Translator Department served as Twitter specialists.
\xblackout{Harm to Ongoing Matter: Lorem ipsum dolor sit amet, consectetur adipiscing elit, sed do eiusmod tempor}% 63
\footnote{63}

The IRA's Twitter operations involved two strategies.
First, IRA specialists operated certain Twitter accounts to create individual U.S. personas, \xblackout{Harm to Ongoing Matter: Lorem ipsum dolor sit amet, consectetur adipiscing elit, sed do eiusmod tempor}% 64
\footnote{64}
Separately, the IRA operated a network of automated Twitter accounts (commonly referred to as a bot network) that enabled the IRA to amplify existing content on Twitter.

\paragraph{Individualized Accounts}

\xblackout{Harm to Ongoing Matter: Lorem ipsum dolor sit amet, consectetur adipiscing elit, sed do eiusmod tempor}% 65
\footnote{65}
\xblackout{Harm to Ongoing Matter: Lorem ipsum dolor sit amet, consectetur adipiscing elit, sed do eiusmod tempor}% 66
\footnote{66}
The IRA operated individualized Twitter accounts similar to the operation of its Facebook accounts, by continuously posting original content to the accounts while also communicating with U.S. Twitter users directly (through public tweeting or Twitter's private messaging).

The IRA used many of these accounts to attempt to influence U.S. audiences on the election.
Individualized accounts used to influence the U.S. presidential election included \@TEN\_GOP (described above); \@jenn\_abrams (claiming to be a Virginian Trump supporter with 70,000 followers); \@Pamela\_Moore13 (claiming to be a Texan Trump supporter with 70,000 followers); and \@America\_1st\_ (an anti-immigration persona with 24,000 followers).% 67
\footnote{67}
In May 2016, the IRA created the Twitter account \@march\_for\_trump, which promoted IRA-organized rallies in support of the Trump Campaign (described below).% 68
\footnote{68}

\xblackout{Harm to Ongoing Matter: Lorem ipsum dolor sit amet, consectetur adipiscing elit, sed do eiusmod tempor incididunt ut labore et dolore magna aliqua. Ut enim ad minim veniam, quis nostrud exercitation ullamco laboris nisi ut aliquip ex ea commodo consequat. Duis aute irure dolor in reprehenderit in voluptate velit esse cillum dolore eu fugiat nulla pariatur. Excepteur sint occaecat cupidatat non proident, sunt in culpa qui officia deserunt mollit anim id est laborum.}

\begin{quote}
\xblackout{Harm to Ongoing Matter: Lorem ipsum dolor sit amet, consectetur adipiscing elit, sed do eiusmod tempor incididunt ut labore et dolore magna aliqua. Ut enim ad minim veniam, quis nostrud exercitation ullamco laboris nisi ut aliquip ex ea commodo consequat. Duis aute irure dolor in reprehenderit in voluptate velit esse cillum dolore eu fugiat nulla pariatur. Excepteur sint occaecat cupidatat non proident, sunt in culpa qui officia deserunt mollit anim id est laborum.}% 69
\footnote{69}
\end{quote}

Using these accounts and others, the IRA provoked reactions from users and the media.
Multiple IRA-posted tweets gained popularity.% 70
\footnote{70}
U.S. media outlets also quoted tweets from IRA-controlled accounts and attributed them to the reactions of real U.S. persons.% 71
\footnote{71}
Similarly, numerous high-profile U.S. persons, including former Ambassador Michael McFaul,% 72
\footnote{72} Roger Stone,% 73
\footnote{73} Sean Hannity,% 74
\footnote{74} and Michael Flynn Jr.,% 75
\footnote{75} retweeted or responded to tweets posted to these IRA-controlled accounts.
Multiple individuals affiliated with the Trump Campaign also promoted IRA tweets (discussed below).

\paragraph{IRA Botnet Activities}

\xblackout{Harm to Ongoing Matter: Lorem ipsum dolor sit amet, consectetur adipiscing elit, sed do eiusmod tempor}% 76
\footnote{76}

\begin{quote}

\xblackout{Harm to Ongoing Matter: Lorem ipsum dolor sit amet, consectetur adipiscing elit, sed do eiusmod tempor}
\xblackout{Harm to Ongoing Matter: Lorem ipsum dolor sit amet, consectetur adipiscing elit, sed do eiusmod tempor incididunt ut labore et dolore magna aliqua. Ut enim ad minim veniam, quis nostrud exercitation ullamco laboris nisi ut aliquip ex ea commodo consequat. Duis aute irure dolor in reprehenderit in voluptate velit esse cillum dolore eu fugiat nulla pariatur. Excepteur sint occaecat cupidatat non proident, sunt in culpa qui officia deserunt mollit anim id est laborum.}% 77
\footnote{77}

\xblackout{Harm to Ongoing Matter: Lorem ipsum dolor sit amet, consectetur adipiscing elit, sed do eiusmod tempor incididunt ut labore et dolore magna aliqua. Ut enim ad minim veniam, quis nostrud exercitation ullamco laboris nisi ut aliquip ex ea commodo consequat.}% 78
\footnote{78}

\end{quote}

In January 2018, Twitter publicly identified 3,814 Twitter accounts associated with the IRA.% 79
\footnote{79}
According to Twitter, in the ten weeks before the 2016 U.S. presidential election, these accounts posted approximately 175,993 tweets, "approximately 8.4\% of which were election-related."% 80
\footnote{80}
Twitter also announced that it had notified approximately 1.4 million people who Twitter believed may have been in contact with an IRA-controlled account.% 81
\footnote{81}

\subsubsection{U.S. Operations Involving Political Rallies}

The IRA organized and promoted political rallies inside the United States while posing as U.S. grassroots activists.
First, the IRA used one of its preexisting social media personas (Facebook groups and Twitter accounts, for example) to announce and promote the event.
The IRA then sent a large number of direct messages to followers of its social media account asking them to attend the event.
From those who responded with interest in attending, the IRA then sought a U.S. person to serve as the event's coordinator.
In most cases, the IRA account operator would tell the U.S. person that they personally could not attend the event due to some preexisting conflict or because they were somewhere else in the United States.% 82
\footnote{82}
The IRA then further promoted the event by contacting U.S. media about the event and directing them to speak with the coordinator.% 83
\footnote{83}
After the event, the IRA posted videos and photographs of the event to the IRA's  social media accounts.% 84
\footnote{84}

The Office identified dozens of U.S. rallies organized by the IRA. The earliest evidence of a rally was a "confederate rally" in November 2015.% 85
\footnote{85}
The IRA continued to organize rallies even after the 2016 U.S. presidential election.
The attendance at rallies varied.
Some rallies appear to have drawn few (if any) participants while others drew hundreds.
The reach and success of these rallies was closely monitored \xblackout{Harm to Ongoing Matter: Lorem ipsum dolor sit amet, consectetur adipiscing elit, sed do eiusmod tempor}

\xblackout{Harm to Ongoing Matter: Lorem ipsum dolor sit amet, consectetur adipiscing elit, sed do eiusmod tempor incididunt ut labore et dolore magna aliqua. Ut enim ad minim veniam, quis nostrud exercitation ullamco laboris nisi ut aliquip ex ea commodo consequat. Duis aute irure dolor in reprehenderit in voluptate velit esse cillum dolore eu fugiat nulla pariatur. Excepteur sint occaecat cupidatat non proident, sunt in culpa qui officia deserunt mollit anim id est laborum. Lorem ipsum dolor sit amet, consectetur adipiscing elit, sed do eiusmod tempor incididunt ut labore et dolore magna aliqua. Ut enim ad minim veniam, quis nostrud exercitation ullamco laboris nisi ut aliquip ex ea commodo consequat. Duis aute irure dolor in reprehenderit in voluptate velit esse cillum dolore eu fugiat nulla pariatur. Excepteur sint occaecat cupidatat non proident, sunt in culpa qui officia deserunt mollit anim id est laborum. Lorem ipsum dolor sit amet, consectetur adipiscing elit, sed do eiusmod tempor incididunt ut labore et dolore magna aliqua. Ut enim ad minim veniam, quis nostrud exercitation ullamco laboris nisi ut aliquip ex ea commodo consequat. Duis aute irure dolor in reprehenderit in voluptate velit esse cillum dolore eu fugiat nulla pariatur. Excepteur sint occaecat cupidatat non proident, sunt in culpa qui officia deserunt mollit anim id est laborum.}

% placeholder: IRA Poster for Pennsylvania Rallies organized by the IRA

From June 2016 until the end of the presidential campaign, almost all of the U.S. rallies organized by the IRA focused on the U.S. election, often promoting the Trump Campaign and opposing the Clinton Campaign.
Pro-Trump rallies included three in New York; a series of pro-Trump rallies in Florida in August 2016; and a series of pro-Trump rallies in October 2016 in Pennsylvania.
The Florida rallies drew the attention of the Trump Campaign, which posted about the Miami rally on candidate Trump's Facebook account (as discussed below).% 86
\footnote{86}

Many of the same IRA employees who oversaw the IRA's social media accounts also conducted the day-to-day recruiting for political rallies inside the United States.
\xblackout{Harm to Ongoing Matter: Lorem ipsum dolor sit amet, consectetur adipiscing elit, sed do eiusmod tempor incididunt ut labore et dolore magna aliqua. Ut enim ad minim veniam, quis nostrud exercitation ullamco laboris nisi ut aliquip ex ea commodo consequat.}% 87
\footnote{87}

\subsubsection{Targeting and Recruitment of U.S. Persons}
