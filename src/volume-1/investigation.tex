\section{The Special Counsel's Investigation}

On May~17, 2017, Deputy Attorney General Rod J. Rosenstein---then serving as Acting Attorney General for the Russia investigation following the recusal of former Attorney General Jeff Sessions on March~2, 2016---appointed the Special Counsel "to investigate Russian interference with the 2016 presidential election and related matters."
Office of the Deputy Att'y Gen., Order No.~3915-2017, Appointment of Special Counsel to Investigate Russian Interference with the 2016 Presidential Election and Related Matters, May~17, 2017) ("Appointment Order").
Relying on "the authority vested" in the Acting Attorney General," including 28 U.S.C. \S\S 509, 510, and 515," the Acting Attorney General ordered the appointment of a Special Counsel " in order to discharge [the Acting Attorney General' s] responsibility to provide supervision and management of the Department of Justice, and to ensure a full and thorough investigation of the Russian government's efforts to interfere in the 2016 presidential election." Appointment Order (introduction).
"The Special Counsel," the Order stated, "is authorized to conduct the investigation confirmed by then-FBI Director James B. Comey in testimony before the House Permanent Select Committee on Intelligence on March~20, 2017," including:

\begin{enumerate}[i]
  \item any links and/or coordination between the Russian government and individuals associated with the campaign of President Donald Trump; and
  \item any matters that arose or may arise directly from the investigation; and
  \item any other matters within the scope of 28 C.F.R. \S 600.4(a).
\end{enumerate}

Appointment Order \P (b). Section 600.4 affords the Special Counsel "the authority to investigate and prosecute federal crimes committed in the course of, and with intent to interfere with, the Special Counsel's investigation, such as perjury, obstruction of justice, destruction of evidence, and intimidation of witnesses." 28 C.F.R. \S 600.4(a).
The authority to investigate "any matters that arose ... directly from the investigation," Appointment Order \P (b)(ii), covers similar crimes that may have occurred during the course of the FBI's confirmed investigation before the Special Counsel's appointment.
"If the Special Counsel believes it is necessary and appropriate," the Order further provided, "the Special Counsel is authorized to prosecute federal crimes arising from the investigation of these matters." Id.~\P ( c ).
Finally, the Acting Attorney General made applicable " Sections 600.4 through 600.10 of Title 28 of the Code of Federal Regulations." Id.~\P (d).

The Acting Attorney General further clarified the scope of the Special Counsel's investigatory authority in two subsequent memoranda.
A memorandum dated August~2, 2017, explained that the Appointment Order had been "worded categorically in order to permit its public release without confirming specific investigations involving specific individuals."
It then confirmed that the Special Counsel had been authorized since his appointment to investigate allegations that three Trump campaign officials---Carter Page, Paul Manafort, and George Papadopoulos---"committed a crime or crimes by colluding with Russian government officials with respect to the Russian government's efforts to interfere with the 2016 presidential election."
The memorandum also confirmed the Special Counsel's authority to investigate certain other matters, including two additional sets of allegations involving Manafort (crimes arising from payments he received from the Ukrainian government and crimes arising from his receipt of loans from a bank whose CEO was then seeking a position in the Trump Administration); allegations that Papadopoulos committed a crime or crimes by acting as an unregistered agent of the Israeli government; and four sets of allegations involving Michael Flynn, the former National Security Advisor to President Trump.

On October~20, 2017, the Acting Attorney General confirmed in a memorandum the Special Counsel's investigative authority as to several individuals and entities.
First," as part of a full and thorough investigation of the Russian government's efforts to interfere in the 2016 presidential election," the Special Counsel was authorized to investigate "the pertinent activities of Michael Cohen, Richard Gates, \xblackout{Personal Privacy}, Roger Stone, and \xblackout{Personal Privacy}" "Confirmation of the authorization to investigate such individuals," the memorandum stressed, "does not suggest that the Special Counsel has made a determination that any of them has committed a crime."
Second, with respect to Michael Cohen, the memorandum recognized the Special Counsel's authority to investigate "leads relate[d] to Cohen's establishment and use of Essential Consultants LLC to, inter alia, receive funds from Russian-backed entities."
Third, the memorandum memorialized the Special Counsel's authority to investigate individuals and entities who were possibly engaged in "jointly undertaken activity" with existing subjects of the investigation, including Paul Manafort.
Finally, the memorandum described an FBI investigation opened before the Special Counsel's appointment into "allegations that [then-Attorney General Jeff Sessions] made false statements to the United States Senate[,]" and confirmed the Special Counsel's authority to investigate that matter.

The Special Counsel structured the investigation in view of his power and authority "to exercise all investigative and prosecutorial functions of any United States Attorney." 28 C.F.R: § 600.6. Like a U.S. Attorney's Office, the Special Counsel's Office considered a range of classified and unclassified information available to the FBI in the course of the Office's Russia investigation, and the Office structured that work around evidence for possible use in prosecutions of federal crimes (assuming that one or more crimes were identified that warranted prosecution).
There was substantial evidence immediately available to the Special Counsel at the inception of the investigation in May 2017 because the FBI had, by that time, already investigated Russian election interference for nearly 10 months.
The Special Counsel's Office exercised its judgment regarding what to investigate and did not, for instance, investigate every public report of a contact between the Trump Campaign and Russian-affiliated individuals and entities.

The Office has concluded its investigation into links and coordination between the Russian government and individuals associated with the Trump Campaign.
Certain proceedings associated with the Office's work remain ongoing.
After consultation with the Office of the Deputy Attorney General, the Office has transferred responsibility for those remaining issues to other components of the Department of Justice and FBI\null.
Appendix D lists those transfers.

Two district courts confirmed the breadth of the Special Counsel's authority to investigate Russia election interference and links and/or coordination with the Trump Campaign.
See United States v.\ Manafort, 312 F. Supp.~3d 60, 79--83 (D.D.C. 2018); United States v.\ Manafort, 321 F. Supp.~3d 640, 650--655 (E.D. Va.~2018).
In the course of conducting that investigation, the Office periodically identified evidence of potential criminal activity that was outside the scope of the Special Counsel's authority established by the Acting Attorney General.
After consultation with the Office of the Deputy Attorney General, the Office referred that evidence to appropriate law enforcement authorities, principally other components of the Department of Justice and to the FBI\null.
Appendix D summarizes those referrals.

\hr

To carry out the investigation and prosecution of the matters assigned to him, the Special Counsel assembled a team that at its high point included 19 attorneys---five of whom joined the Office from private practice and 14 on detail or assigned from other Department of Justice components.
These attorneys were assisted by a filter team of Department lawyers and FBI personnel who screened materials obtained via court process for privileged information before turning those materials over to investigators; a support staff of three paralegals on detail from the Department's Antitrust Division; and an administrative staff of nine responsible for budget, finance, purchasing, human resources, records, facilities, security, information technology, and administrative support.
The Special Counsel attorneys and support staff were co-located with and worked alongside approximately 40 FBI agents, intelligence analysts, forensic accountants, a paralegal, and professional staff assigned by the FBI to assist the Special Counsel's investigation.
Those "assigned" FBI employees remained under FBI supervision at all times; the matters on which they assisted were supervised by the Special Counsel.% 1
\footnote{FBI personnel assigned to the Special Counsel's Office were required to adhere to all applicable federal law and all Department and FBI regulations, guidelines, and policies.
An FBI attorney worked on FBI-related matters for the Office, such as FBI compliance with all FBI policies and procedures, including the FBI's Domestic Investigations and Operations Guide (DIOG).
That FBI attorney worked under FBI legal supervision, not the Special Counsel's supervision.}

During its investigation the Office issued more than 2,800 subpoenas under the auspices of a grand jury sitting in the District of Columbia; executed nearly 500 search-and-seizure warrants; obtained more than 230 orders for communications records under 18 U.S.C. § 2703(d); obtained almost 50 orders authorizing use of pen registers; made 13 requests to foreign governments pursuant to Mutual Legal Assistance Treaties; and interviewed approximately 500 witnesses, including almost 80 before a grand jury.

\hr

From its inception, the Office recognized that its investigation could identify foreign intelligence and counterintelligence information relevant to the FBI's broader national security mission.
FBI personnel who assisted the Office established procedures to identify and convey such information to the FBI\null.
The FBI's Counterintelligence Division met with the Office regularly for that purpose for most of the Office's tenure.
For more than the past year, the FBI also embedded personnel at the Office who did not work on the Special Counsel's investigation, but whose purpose was to review the results of the investigation and to send---in writing---summaries of foreign intelligence and counterintelligence information to FBIHQ and FBI Field Offices.
Those communications and other correspondence between the Office and the FBI contain information derived from the investigation, not all of which is contained in this Volume.
This Volume is a summary.
It contains, in the Office's judgment, that information necessary to account for the Special Counsel's prosecution and declination decisions and to describe the investigation's main factual results.
