\section{Prosecution and Declination Decisions}

The Appointment Order authorized the Special Counsel's Office ``to prosecute federal crimes arising from [its] investigation'' of the matters assigned to it.
In deciding whether to exercise this prosecutorial authority, the Office has been guided by the Principles of Federal Prosecution set forth in the Justice (formerly U.S. Attorney's) Manual.
In particular, the Office has evaluated whether the conduct of the individuals considered for prosecution constituted a federal offense and whether admissible evidence would probably be sufficient to obtain and sustain a conviction for such an offense.
Justice Manual \S 9-27.220 (2018).
Where the answer to those questions was yes, the Office further considered whether the prosecution would serve a substantial federal interest, the individuals were subject to effective prosecution in another jurisdiction, and there existed an adequate non-criminal alternative to prosecution.
\textit{Id}.

As explained below, those considerations led the Office to seek charges against two sets of Russian nationals for their roles in perpetrating the active-measures social media campaign and computer-intrusion operations.
\blackout{Harm to Ongoing Matter}
The Office similarly determined that the contacts between Campaign officials and Russia-linked individuals either did not involve the commission of a federal crime or, in the case of campaign-finance offenses, that our evidence was not sufficient to obtain and sustain a criminal conviction.
At the same time, the Office concluded that the Principles of Federal Prosecution supported charging certain individuals connected to the Campaign with making false statements or otherwise obstructing this investigation or parallel congressional investigations.

\subsection{Russian ``Active Measures'' Social Media Campaign}

On February~16, 2018, a federal grand jury in the District of Columbia returned an indictment charging 13 Russian nationals and three Russian entities---including the Internet Research Agency (IRA) and Concord Management and Consulting LLC (Concord)---with violating U.S. criminal laws in order to interfere with U.S. elections and political processes.%1276
\footnote{A more detailed explanation of the charging decision in this case is set forth in a separate memorandum provided to the Acting Attorney General before the indictment.}
The indictment charges all of the defendants with conspiracy to defraud the United States (Count One), three defendants with conspiracy to commit wire fraud and bank fraud (Count Two), and five defendants with aggravated identity theft (Counts Three through Eight).
\textit{Internet Research Agency} Indictment.
Concord, which is one of the entities charged in the Count One conspiracy, entered an appearance through U.S. counsel and moved to dismiss the charge on multiple grounds.
In orders and memorandum opinions issued on August~13 and November~15, 2018, the district court denied Concord's motions to dismiss.
\textit{United States v.\ Concord Management \& Consulting LLC}, 347 F. Supp.~3d 38 (D.D.C. 2018).
\textit{United States v.\ Concord Management \& Consulting LLC}, 317 F. Supp.~3d 598 (D.D.C. 2018).
As of this writing, the prosecution of Concord remains ongoing before the U.S. District Court for the District of Columbia.
The other defendants remain at large.

Although members of the IRA had contact with individuals affiliated with the Trump Campaign, the indictment does not charge any Trump Campaign official or any other U.S. person with participating in the conspiracy.
That is because the investigation did not identify evidence that any U.S. person who coordinated or communicated with the IRA knew that he or she was speaking with Russian nationals engaged in the criminal conspiracy.
The Office therefore determined that such persons did not have the knowledge or criminal purpose required to charge them in the conspiracy to defraud the United States (Count One) or in the separate count alleging a wire- and bank-fraud conspiracy involving the IRA and two individual Russian nationals (Count Two).
The Office did, however, charge one U.S. national for his role in supplying false or stolen bank account numbers that allowed the IRA conspirators to access U.S. online payment systems by circumventing those systems' security features.
On February~12, 2018, Richard Pinedo pleaded guilty, pursuant to a single-count information, to identity fraud, in violation of 18 U.S.C. \S 1028(a)(7) and (b)(1)(D).
Plea Agreement, \textit{United States v.\ Richard Pinedo}, No.~1:18-cr-24 (D.D.C. Feb.~12, 2018), Doc.~10.
The investigation did not establish that Pinedo was aware of the identity of the IRA members who purchased bank account numbers from him.
Pinedo's sales of account numbers enabled the IRA members to anonymously access a financial network through which they transacted with U.S. persons and companies.
\textit{See} Gov't Sent.~Mem.~at~3, \textit{United States v.\ Richard Pinedo}, No.~1:18-cr-24 (D.D.C. Sept.~26, 2018), Doc.~24.
On October~10, 2018, Pinedo was sentenced to six months of imprisonment, to be followed by six months of home confinement, and was ordered to complete 100 hours of community service.

\subsection{Russian Hacking and Dumping Operations}

\subsubsection{Section 1030 Computer-Intrusion Conspiracy}

\paragraph{Background}

On July~13, 2018, a federal grand jury in the District of Columbia returned an indictment charging Russian military intelligence officers from the GRU with conspiring to hack into various U.S. computers used by the Clinton Campaign, DNC, DCCC, and other U.S. persons, in violation of 18 U.S.C. \S\S 1030 and 371 (Count One); committing identity theft and conspiring to commit money laundering in furtherance of that hacking conspiracy, in violation of 18 U.S.C. \S\S 1028A and 1956(h) (Counts Two through Ten); and a separate conspiracy to hack into the computers of U.S. persons and entities responsible for the administration of the 2016 U.S. election, in violation of 18 U.S.C. \S\S 1030 and 371 (Count Eleven).
\textit{Netyksho} Indictment.%1277
\footnote{The Office provided a more detailed explanation of the charging decision in this case in meetings with the Office of the Acting Attorney General before the indictment.}
As of this writing, all 12 defendants remain at large.

The \textit{Netyksho} indictment alleges that the defendants conspired with one another and with others to hack into the computers of U.S. persons and entities involved in the 2016 U.S. presidential election, steal documents from those computers, and stage releases of the stolen documents to interfere in the election.
\textit{Netyksho Indictment}~\P 2.
The indictment also describes how, in staging the releases, the defendants used the Guccifer~2.0 persona to disseminate documents through WikiLeaks.
On July~22, 2016, WikiLeaks released over 20,000 emails and other documents that the hacking conspirators had stolen from the DNC\null.
\textit{Netyksho Indictment} \P 48.
In addition, on October~7, 2016, WikiLeaks began releasing emails that some conspirators had stolen from Clinton Campaign chairman John Podesta after a successful spearphishing operation.
\textit{Netyksho Indictment} \S 49.

\blackout{Harm to Ongoing Matter}

\blackout{Grand Jury}

\blackout{Harm to Ongoing Matter}

\paragraph{Charging Decision As to [$\blacksquare\blacksquare\blacksquare\blacksquare\blacksquare\blacksquare\blacksquare\blacksquare$: Harm to Ongoing Matter]}
\blackout{Harm to Ongoing Matter}%1278
\footnote{The Office also considered, but ruled out, charges on the theory that the post-hacking sharing and dissemination of emails could constitute trafficking in or receipt of stolen property under the National Stolen Property Act (NSPA), 18 U.S.C. \S\S 2314 and 2315.
The statutes comprising the NSPA cover “goods, wares, or merchandise,” and lower courts have largely understood that phrase to be limited to tangible items since the Supreme Court's decision in \textit{Dowling v.\ United States}, 473 U.S. 207 (1985).
\textit{See United States v.\ Yijia Zhang}, 995F, Supp.~2d 340, 344--48 (E.D. Pa.~2014) (collecting cases).
One of those post-\textit{Dowling} decisions---\textit{United States v.\ Brown}, 925 F.2d 1301 (10th Cir.~1991)---specifically held that the NSPA does not reach “a computer program in source code form,” even though that code was stored in tangible items (\textit{i.e.}, a hard disk and in a three-ring notebook).
\textit{Id}. at 1302--03.
Congress, in turn, cited the Brown opinion in explaining the need for amendments to 18 U.S.C. \S 1030(a)(2) that “would ensure that the theft of intangible information by the unauthorized use of a computer is prohibited in the same way theft of physical items [is] protected.”
S. Rep.~104-357, at 7 (1996).
That sequence of events would make it difficult to argue that hacked emails in electronic form, which are the relevant stolen items here, constitute “goods, wares, or merchandise” within the meaning of the NSPA.}

\blackout{Harm to Ongoing Matter}

\blackout{Harm to Ongoing Matter}

\blackout{Harm to Ongoing Matter}

\blackout{Harm to Ongoing Matter}%1279
\footnote{\blackout{Harm to Ongoing Investigation}}

\blackout{Harm to Ongoing Matter}

\blackout{Harm to Ongoing Matter}

\blackout{Harm to Ongoing Matter}

\blackout{Harm to Ongoing Matter}


\subsubsection{Potential Section 1030 Violation By [$\blacksquare\blacksquare\blacksquare\blacksquare\blacksquare\blacksquare\blacksquare\blacksquare$: Personal Privacy]}
\blackout{Personal Privacy}

\blackout{Personal Privacy}
See \textit{United States v.\ Willis}, 476 F.3d 1121, 1125 n.1 (10th Cir.~2007) (explaining that the 1986 amendments to Section 1030 reflect Congress's desire to reach ``intentional acts of unauthorized access---rather than mistaken, inadvertent, or careless ones'') (quoting S. Rep.~99-432, at 5 (1986)).
In addition, the computer \blackout{Personal Privacy} likely qualifies as a ``protected'' one under the statute, which reaches ``effectively all computers with Internet access.''
\textit{United States v.\ Nosal}, 676 F.3d 854, 859 (9th Cir.~2012) (en banc).
\blackout{Personal Privacy}

Applying the Principles of Federal Prosecution, however, the Office determined that prosecution of this potential violation was not warranted.
Those Principles instruct prosecutors to consider, among other things, the nature and seriousness of the offense, the person's culpability in connection with the offense, and the probable sentence to be imposed if the prosecution is successful.
Justice Manual \S 9-27.230. \blackout{Personal Privacy}

\subsection{Russian Government Outreach and Contacts}
As explained in Section IV above, the Office's investigation uncovered evidence of numerous links (i.e., contacts) between Trump Campaign officials and individuals having or claiming to have ties to the Russian government.
The Office evaluated the contacts under several sets of federal laws, including conspiracy laws and statutes governing foreign agents who operate in the United States.
After considering the available evidence, the Office did not pursue charges under these statutes against any of the individuals discussed in Section IV above---with the exception of FARA charges against Paul Manafort and Richard Gates based on their activities on behalf of Ukraine.
One of the interactions between the Trump Campaign and Russian-affiliated individuals---the June~9, 2016 meeting between high-ranking campaign officials and Russians promising derogatory information on Hillary Clinton---implicates an additional body of law: campaign finance statutes.
Schemes involving the solicitation or receipt of assistance from foreign sources raise difficult statutory and constitutional questions.
As explained below, the Office evaluated those questions in connection with the June~9 meeting \blackout{Harm to Ongoing Matter}.
The Office ultimately concluded that, even if the principal legal questions were resolved favorably to the government, a prosecution would encounter difficulties proving that Campaign officials or individuals connected to the Campaign willfully violated the law.
Finally, although the evidence of contacts between Campaign officials and Russian-affiliated individuals may not have been sufficient to establish or sustain criminal charges, several U.S. persons connected to the Campaign made false statements about those contacts and took other steps to obstruct the Office's investigation and those of Congress.
This Office has therefore charged some of those individuals with making false statements and obstructing justice.

\subsubsection{Potential Coordination: Conspiracy and Collusion}
As an initial matter, this Office evaluated potentially criminal conduct that involved the collective action of multiple individuals not under the rubric of ``collusion,'' but through the lens of conspiracy law.
In so doing, the Office recognized that the word ``collud[e]'' appears in the Acting Attorney General's August~2, 2017 memorandum; it has frequently been invoked in public reporting; and it is sometimes referenced in antitrust law, see, \textit{e.g., Brooke Group v.\ Brown \& Williamson Tobacco Corp.}, 509 U.S. 209, 227 (1993).
But collusion is not a specific offense or theory of liability found in the U.S. Code; nor is it a term of art in federal criminal law.
To the contrary, even as defined in legal dictionaries, collusion is largely synonymous with conspiracy as that crime is set forth in the general federal conspiracy statute, 18 U.S.C. \S 371.
See \textit{Black's Law Dictionary} 321 (10th ed.~2014) (collusion is ``[a]n agreement to defraud another or to do or obtain something forbidden by law''); 1 Alexander Burrill, \textit{A Law Dictionary and Glossary} 311 (1871) (``An agreement between two or more persons to defraud another by the forms of law, or to employ such forms as means of accomplishing some unlawful object.''); 1 \textit{Bouvier's Law Dictionary} 352 (1897) (``An agreement between two or more persons to defraud a person of his rights by the forms of law, or to obtain an object forbidden by law.'').

For that reason, this Office's focus in resolving the question of joint criminal liability was on conspiracy as defined in federal law, not the commonly discussed term ``collusion.''
The Office considered in particular whether contacts between Trump Campaign officials and Russia-linked individuals could trigger liability for the crime of conspiracy---either under statutes that have their own conspiracy language (e.g., 18 U.S.C. \S\S 1349, 1951(a)), or under the general conspiracy statute (18 U.S.C. \S 371).
The investigation did not establish that the contacts described in Volume I, Section IV, \textit{supra}, amounted to an agreement to commit any substantive violation of federal criminal law---including foreign-influence and campaign-finance laws, both of which are discussed further below.
The Office therefore did not charge any individual associated with the Trump Campaign with conspiracy to commit a federal offense arising from Russia contacts, either under a specific statute or under Section 371's offenses clause.

The Office also did not charge any campaign official or associate with a conspiracy under Section 371's defraud clause.
That clause criminalizes participating in an agreement to obstruct a lawful function of the U.S. government or its agencies through deceitful or dishonest means.
See \textit{Dennis v.\ United States}, 384 U.S. 855, 861 (1966); \textit{Hammerschmidt v.\ United States}, 2605 U.S. 182, 188 (1924); see also \textit{United States v.\ Concord Mgmt. \& Consulting LLC}, 347 F. Supp.~3d 38, 46 (D.D.C. 2018).
The investigation did not establish any agreement among Campaign officials---or between such officials and Russia-linked individuals---to interfere with or obstruct a lawful function of a government agency during the campaign or transition period.
And, as discussed in Volume I, Section V.A, \textit{supra}, the investigation did not identify evidence that any Campaign official or associate knowingly and intentionally participated in the conspiracy to defraud that the Office charged, namely, the active-measures conspiracy described in Volume I, Section II, \textit{supra}.
Accordingly, the Office did not charge any Campaign associate or other U.S. person with conspiracy to defraud the United States based on the Russia-related contacts described in Section IV above.

\subsubsection{Potential Coordination: Foreign Agent Statutes (FARA and 18 U.S.C. § 951)}
The Office next assessed the potential liability of Campaign-affiliated individuals under federal statutes regulating actions on behalf of, or work done for, a foreign government.

\paragraph{Governing Law}
Under 18 U.S.C. \S 951, it is generally illegal to act in the United States as an agent of a foreign government without providing notice to the Attorney General.
Although the defendant must act on behalf of a foreign government (as opposed to other kinds of foreign entities), the acts need not involve espionage; rather, acts of any type suffice for liability.
See \textit{United States v.\ Duran}, 596 F.3d 1283, 1293--94 (11th Cir.~2010); \textit{United States v.\ Latchin}, 554 F.3d 709, 715 (7th Cir.~2009); \textit{United States v.\ Dumeisi}, 424 F.3d 566, 581 (7th Cir.~2005).
An ``agent of a foreign government'' is an ``individual'' who ``agrees to operate'' in the United States ``subject to the direction or control of a foreign government or official.'' 18 U.S.C. \S 951(d).

The crime defined by Section 951 is complete upon knowingly acting in the United States as an unregistered foreign-government agent.
18 U.S.C. \S 951(a).
The statute does not require willfulness, and knowledge of the notification requirement is not an element of the offense.
\textit{United States v.\ Campa}, 529 F.3d 980, 998--99 (11th Cir.~2008); \textit{Duran}, 596 F.3d at 1291--94; \textit{Dumeisi}, 424 F.3d at 581.

The Foreign Agents Registration Act (FARA) generally makes it illegal to act as an agent of a foreign principal by engaging in certain (largely political) activities in the United States without registering with the Attorney General. 22 U.S.C. \S\S 611--621. The triggering agency relationship must be with a foreign principal or ``a person any of whose activities are directly or indirectly supervised, directed, controlled, financed, or subsidized in whole or in major part by a foreign principal.'' 22 U.S.C. \S 611(c)(1).
That includes a foreign government or political party and various foreign individuals and entities.
22 U.S.C. \S 611(b).
A covered relationship exists if a person ``acts as an agent, representative, employee, or servant'' or ``in any other capacity at the order, request, or under the [foreign principal's] direction or control.''
22 U.S.C. \S 611(c)(1).
It is sufficient if the person ``agrees, consents, assumes or purports to act as, or who is or holds himself out to be, whether or not pursuant to contractual relationship, an agent of a foreign principal.''
22 U.S.C. \S 611(c)(2).

The triggering activity is that the agent ``directly or through any other person'' in the United States (1) engages in ``political activities for or in the interests of [the] foreign principal,'' which includes attempts to influence federal officials or the public; (2) acts as ``public relations counsel, publicity agent, information-service employee or political consultant for or in the interests of such foreign principal''; (3) ``solicits, collects, disburses, or dispenses contributions, loans, money, or other things of value for or in the interest of such foreign principal''; or (4) ``represents the interests of such foreign principal'' before any federal agency or official.
22 U.S.C. \S 611(c)(1).

It is a crime to engage in a ``[w]illful violation of any provision of the Act or any regulation thereunder.''
22 U.S.C. \S 618(a)(1).
It is also a crime willfully to make false statements or omissions of material facts in FARA registration statements or supplements.
22 U.S.C. \S 618(a)(2).
Most violations have a maximum penalty of five years of imprisonment and a \$10,000 fine.
22 U.S.C. \S 618.

\paragraph{Application}
The investigation uncovered extensive evidence that Paul Manafort's and Richard Gates's pre-campaign work for the government of Ukraine violated FARA\null.
Manafort and Gates were charged for that conduct and admitted to it when they pleaded guilty to superseding criminal informations in the District of Columbia prosecution.%1280
\footnote{\textit{Gates} Superseding Criminal Information;
Waiver of Indictment, \textit{United States v.\ Richard W. Gates III}, 1:17-cr-201 (D.D.C. Feb.~23, 2018), Doc.~203;
Waiver of Trial by Jury, \textit{United States v.\ Richard W. Gates III}, 1:17-cr-201 (D.D.C. Feb.~23, 2018), Doc.~204;
Gates Plea Agreement;
Statement of Offense, \textit{United States v.\ Richard W. Gates III}, 1:17-cr-201 (D.D.C. Feb.~23, 2018), Doc.~206;
Plea Agreement, \textit{United States v.\ Paul J. Manafort, Jr.}, 1:17-cr-201 (D.D.C. Sept.~14, 2018), Doc.~422;
Statement of Offense, \textit{United States v.\ Paul J. Manafort, Jr.}, 1:17-cr-201 (D.D.C. Sept.~14, 2018), Doc.~423.}
The evidence underlying those charges is not addressed in this report because it was discussed in public court documents and in a separate prosecution memorandum submitted to the Acting Attorney General before the original indictment in that case.
In addition, the investigation produced evidence of FARA violations involving Michael Flynn.
Those potential violations, however, concerned a country other than Russia (\textit{i.e.}, Turkey) and were resolved when Flynn admitted to the underlying facts in the Statement of Offense that accompanied his guilty plea to a false-statements charge.
Statement of Offense, \textit{United States v.\ Michael T. Flynn}, No.~1:17-cr-232 (D.D.C. Dec.~1, 2017), Doc.~4 (``\textit{Flynn} Statement of Offense'').%1281
\footnote{\blackout{Harm to Ongoing Investigation}}
The investigation did not, however, yield evidence sufficient to sustain any charge that any individual affiliated with the Trump Campaign acted as an agent of a foreign principal within the meaning of FARA or, in terms of Section 951, subject to the direction or control of the government of Russia, or any official thereof.
In particular, the Office did not find evidence likely to prove beyond a reasonable doubt that Campaign officials such as Paul Manafort, George Papadopoulos, and Carter Page acted as agents of the Russian government---or at its direction, control, or request---during the relevant time period.%1282
\footnote{On four occasions, the Foreign Intelligence Surveillance Court (FISC) issued warrants based on a finding of probable cause to believe that Page was an agent of a foreign power.
50 U.S.C. \S\S 1801(b), 1805(a)(2)(A).
The FISC's probable-cause finding was based on a different (and lower) standard than the one governing the Office's decision whether to bring charges against Page, which is whether admissible evidence would likely be sufficient to prove beyond a reasonable doubt that Page acted as an agent of the Russian Federation during the period at issue.
\textit{Cf.~United States v.\ Cardoza}, 713 F.3d 656, 660 (D.C. Cir.~2013) (explaining that probable cause requires only “a fair probability,” and not “certainty, or proof beyond a reasonable doubt, or proof by a preponderance of the evidence”).}
\blackout{Personal Privacy}
As a result, the Office did not charge \blackout{PP} any other Trump Campaign official with violating FARA or Section 951, or attempting or conspiring to do so, based on contacts with the Russian government or a Russian principal.

Finally, the Office investigated whether one of the above campaign advisors---George Papadopoulos---acted as an agent of, or at the direction and control of, the government of Israel.
While the investigation revealed significant ties between Papadopoulos and Israel (and search warrants were obtained in part on that basis), the Office ultimately determined that the evidence was not sufficient to obtain and sustain a conviction under FARA or Section 951.

\subsubsection{Campaign Finance}
Several areas of the Office's investigation involved efforts or offers by foreign nationals to provide negative information about candidate Clinton to the Trump Campaign or to distribute that information to the public, to the anticipated benefit of the Campaign.
As explained below, the Office considered whether two of those efforts in particular---the June~9, 2016 meeting at Trump Tower \blackout{Harm to Ongoing Matter}---constituted prosecutable violations of the campaign-finance laws.
The Office determined that the evidence was not sufficient to charge either incident as a criminal violation.

\paragraph{Overview Of Governing Law}

``[T]he United States has a compelling interest\dots in limiting the participation of foreign citizens in activities of democratic self-government, and in thereby preventing foreign influence over the U.S. political process.''
\textit{Bluman v.\ FEC}, 800 F. Supp.~2d 281, 288 (D.D.C. 2011) (Kavanaugh, J., for three-judge court), \textit{aff'd}, 565 U.S. 1104 (2012).
To that end, federal campaign finance law broadly prohibits foreign nationals from making contributions, donations, expenditures, or other disbursements in connection with federal, state, or local candidate elections, and prohibits anyone from soliciting, accepting, or receiving such contributions or donations.
As relevant here, foreign nationals may not make---and no one may ``solicit, accept, or receive'' from them---``a contribution or donation of money or other thing of value'' or ``an express or implied promise to make a contribution or donation, in connection with a Federal, State, or local election.''
52 U.S.C. \S 30121 (a)(1)(A), (a)(2).%1283
\footnote{Campaign-finance law also places financial limits on contributions, 52 U.S.C. \S 30116(a), and prohibits contributions from corporations, banks, and labor unions, 52 U.S.C. § 30118(a);
see Citizens United v.\ FEC, 558 U.S. 310, 320 (2010).
Because the conduct that the Office investigated involved possible electoral activity by foreign nationals, the foreign-contributions ban is the most readily applicable provision.}
The term ``contribution,'' which is used throughout the campaign-finance law, ``includes'' ``any gift, subscription, loan, advance, or deposit of money or anything of value made by any person for the purpose of influencing any election for Federal office.''
52 U.S.C. \S 30101(8)(A)(i).
It excludes, among other things, ``the value of [volunteer] services.'' 52 U.S.C. \S 30101(8)(B)G).

Foreign nationals are also barred from making ``an expenditure, independent expenditure, or disbursement for an electioneering communication.''
52 U.S.C. \S 30121(a)(1)(C).
The term ``expenditure'' ``includes'' ``any purchase, payment, distribution, loan, advance, deposit, or gift of money or anything of value, made by any person for the purpose of influencing any election for Federal office.''
52 U.S.C. \S 30101(9)(A)(i).
It excludes, among other things, news stories and non-partisan get-out-the-vote activities.
52 U.S.C. \S 30101(9)(B)(i)--(ii).
An ``independent expenditure'' is an expenditure ``expressly advocating the election or defeat of a clearly identified candidate'' and made independently of the campaign.
52 U.S.C. \S 30101(17).
An ``electioneering communication'' is a broadcast communication that ``refers to a clearly identified candidate for Federal office'' and is made within specified time periods and targeted at the relevant electorate.

The statute defines ``foreign national'' by reference to FARA and the Immigration and Nationality Act, with minor modification.
52 U.S.C. \S 30121(b) (cross-referencing 22 U.S.C. \S 611(b)(1)--(3) and 8 U.S.C. \S 1101(a)(20), (22)).
That definition yields five, sometimes overlapping categories of foreign nationals, which include all of the individuals and entities relevant for present purposes---namely, foreign governments and political parties, individuals outside of the U.S. who are not legal permanent residents, and certain non-U.S., entities located outside of the U.S.

A ``knowing[] and willful[]'' violation involving an aggregate of \$25,000 or more in a calendar year is a felony.
52 U.S.C. \S 30109(d)(1)(A)(i); \textit{see Bluman}, 800 F. Supp.~2d at 292 (noting that a willful violation will require some ``proof of the defendant's knowledge of the law''); \textit{United States v.\ Danielczyk}, 917 F. Supp.~2d 573, 577 (E.D. Va.~2013) (applying willfulness standard drawn from \textit{Bryan v.\ United States}, 524 U.S. 184, 191--92 (1998)); \textit{see also Wagner v.\ FEC}, 793 F.3d 1, 19 n.23 (D.C. Cir.~2015) (en banc) (same).

A ``knowing[] and willful[]'' violation involving an aggregate of \$2,000 or more in a calendar year, but less than \$25,000, is a misdemeanor.
52 U.S.C. \S 30109(d)(1)(A)(ii).

\paragraph{Application to June~9 Trump Tower Meeting}
The Office considered whether to charge Trump Campaign officials with crimes in connection with the June~9 meeting described in Volume I, Section IV.A.5, \textit{supra}.
The Office concluded that, in light of the government's substantial burden of proof on issues of intent (``knowing'' and ``willful''), and the difficulty of establishing the value of the offered information, criminal charges would not meet the Justice Manual standard that ``the admissible evidence will probably be sufficient to obtain and sustain a conviction.''
Justice Manual \S 9-27.220.

In brief, the key facts are that, on June~3, 2016, Robert Goldstone emailed Donald Trump~Jr., to pass along from Emin and Aras Agalarov an ``offer'' from Russia's ``Crown prosecutor'' to ``the Trump campaign'' of ``official documents and information that would incriminate Hillary and her dealings with Russia and would be very useful to [Trump~Jr.'s] father.''
The email described this as ``very high level and sensitive information'' that is ``part of Russia and its government's support to Mr.~Trump---helped along by Aras and Emin.''
Trump~Jr.\ responded: ``if it's what you say I love it especially later in the summer.''
Trump~Jr.\ and Emin Agalarov had follow-up conversations and, within days, scheduled a meeting with Russian representatives that was attended by Trump~Jr., Manafort, and Kushner.
The communications setting up the meeting and the attendance by high-level Campaign representatives support an inference that the Campaign anticipated receiving derogatory documents and information from official Russian sources that could assist candidate Trump's electoral prospects.

This series of events could implicate the federal election-law ban on contributions and donations by foreign nationals, 52 U.S.C. \S 30121(a)(1)(A).
Specifically, Goldstone passed along an offer purportedly from a Russian government official to provide ``official documents and information'' to the Trump Campaign for the purposes of influencing the presidential election.
Trump~Jr.\ appears to have accepted that offer and to have arranged a meeting to receive those materials.
Documentary evidence in the form of email chains supports the inference that Kushner and Manafort were aware of that purpose and attended the June~9 meeting anticipating the receipt of helpful information to the Campaign from Russian sources.

The Office considered whether this evidence would establish a conspiracy to violate the foreign contributions ban, in violation of 18 U.S.C. \S 371; the solicitation of an illegal foreign source contribution; or the acceptance or receipt of ``an express or implied promise to make a [foreign-source] contribution,'' both in violation of 52 U.S.C. \S 30121(a)(1)(A), (a)(2).
There are reasonable arguments that the offered information would constitute a ``thing of value'' within the meaning of these provisions, but the Office determined that the government would not be likely to obtain and sustain a conviction for two other reasons: first, the Office did not obtain admissible evidence likely to meet the government's burden to prove beyond a reasonable doubt that these individuals acted ``willfully,'' i.e., with general knowledge of the illegality of their conduct; and, second, the government would likely encounter difficulty in proving beyond a reasonable doubt that the value of the promised information exceeded the threshold for a criminal violation, see 52 USS.C. \S 30109(d)(1)(A)G).

\subparagraph{Thing-of-Value Element}
A threshold legal question is whether providing to a campaign ``documents and information'' of the type involved here would constitute a prohibited campaign contribution.
The foreign contribution ban is not limited to contributions of money.
It expressly prohibits ``a contribution or donation of money or \textit{other thing of value}.''
52 U.S.C. \S 30121(a)(1)(A), (a)(2) (emphasis added).
And the term ``contribution'' is defined throughout the campaign-finance laws to ``include[]'' ``any gift, subscription, loan, advance, or deposit of money or \textit{anything of value}.''
52 U.S.C. \S 30101(8)(A)(i) (emphasis added).
The phrases ``thing of value'' and ``anything of value'' are broad and inclusive enough to encompass at least some forms of valuable information.
Throughout the United States Code, these phrases serve as ``term[s] of art'' that are construed ``broad[ly].''
\textit{United States v.\ Nilsen}, 967 F.2d 539, 542 (11th Cir.~1992) (per curiam) (``thing of value'' includes ``both tangibles and intangibles''); \textit{see also, e.g.}, 18 U.S.C. \S\S 201(b)(1), 666(a)(2) (bribery statutes); \textit{id.} \S 641 (theft of government property).
For example, the term ``thing of value'' encompasses law enforcement reports that would reveal the identity of informants, \textit{United States v.\ Girard}, 601 F.2d 69, 71 (2d Cir.~1979); classified materials, \textit{United States v.\ Fowler}, 932 F.2d 306, 310 (4th Cir.~1991); confidential information about a competitive bid, \textit{United States v.\ Matzkin}, 14 F.3d 1014, 1020 (4th Cir.~1994); secret grand jury information, \textit{United States v.\ Jeter}, 775 F.2d 670, 680 (6th Cir.~1985); and information about a witness's whereabouts, \textit{United States v.\ Sheker}, 618 F.2d 607, 609 (9th Cir.~1980) (per curiam).
And in the public corruption context, ```thing of value' is defined broadly to include the value which the defendant subjectively attaches to the items received.''
\textit{United States v.\ Renzi}, 769 F.3d 731, 744 (9th Cir.~2014) (internal quotation marks omitted).

Federal Election Commission (FEC) regulations recognize the value to a campaign of at least some forms of information, stating that the term ``anything of value'' includes ``the provision of any goods or services without charge,'' such as ``membership lists'' and ``mailing lists.''
11 C.F.R. \S 100.52(d)(1).
The FEC has concluded that the phrase includes a state-by-state list of activists.
\textit{See Citizens for Responsibility and Ethics in Washington v.\ FEC}, 475 F.3d 337, 338 (D.C. Cir.~2007) (describing the FEC's findings).
Likewise, polling data provided to a campaign constitutes a ``contribution.''
FEC Advisory Opinion 1990-12 (Strub), 1990 WL 153454 (citing 11 C.F.R. \S 106.4(b)).
And in the specific context of the foreign-contributions ban, the FEC has concluded that ``election materials used in previous Canadian campaigns,'' including ``flyers, advertisements, door hangers, tri-folds, signs, and other printed material,'' constitute ``anything of value,'' even though ``the value of these materials may be nominal or difficult to ascertain.''
FEC Advisory Opinion 2007-22 (Hurysz), 2007 WL 5172375, at $\ast$5.
These authorities would support the view that candidate-related opposition research given to a campaign for the purpose of influencing an election could constitute a contribution to which the foreign-source ban could apply.
A campaign can be assisted not only by the provision of funds, but also by the provision of derogatory information about an opponent.
Political campaigns frequently conduct and pay for opposition research.
A foreign entity that engaged in such research and provided resulting information to a campaign could exert a greater effect on an election, and a greater tendency to ingratiate the donor to the candidate, than a gift of money or tangible things of value.
At the same time, no judicial decision has treated the voluntary provision of uncompensated opposition research or similar information as a thing of value that could amount to a contribution under campaign-finance law.
Such an interpretation could have implications beyond the foreign-source ban, see 52 U.S.C. \S 30116(a) (imposing monetary limits on campaign contributions), and raise First Amendment questions.
Those questions could be especially difficult where the information consisted simply of the recounting of historically accurate facts.
It is uncertain how courts would resolve those issues.

\subparagraph{Willfulness}
Even assuming that the promised ``documents and information that would incriminate Hillary'' constitute a ``thing of value'' under campaign-finance law, the government would encounter other challenges in seeking to obtain and sustain a conviction.
Most significantly, the government has not obtained admissible evidence that is likely to establish the scienter requirement beyond a reasonable doubt.
To prove that a defendant acted ``knowingly and willfully,'' the government would have to show that the defendant had general knowledge that his conduct was unlawful.
U.S. Department of Justice, \textit{Federal Prosecution of Election Offenses}~123 (8th ed.\ Dec.~2017) (``\textit{Election Offenses}''); \textit{see Bluman}, 800 F. Supp. 2d at 292 (noting that a willful violation requires ``proof of the defendant's knowledge of the law''); \textit{Danielczyk}, 917 F. Supp.~2d at 577 (``knowledge of general unlawfulness'').
``This standard creates an elevated scienter element requiring, at the very least, that application of the law to the facts in question be fairly clear.
When there is substantial doubt concerning whether the law applies to the facts of a particular matter, the offender is more likely to have an intent defense.''
\textit{Election Offenses}~123.
On the facts here, the government would unlikely be able to prove beyond a reasonable doubt that the June~9 meeting participants had general knowledge that their conduct was unlawful.
The investigation has not developed evidence that the participants in the meeting were familiar with the foreign-contribution ban or the application of federal law to the relevant factual context.
The government does not have strong evidence of surreptitious behavior or efforts at concealment at the time of the June~9 meeting.
While the government has evidence of later efforts to prevent disclosure of the nature of the June~9 meeting that could circumstantially provide support for a showing of scienter, \textit{see} Volume II, Section II.G, \textit{infra}, that concealment occurred more than a year later, involved individuals who did not attend the June~9 meeting, and may reflect an intention to avoid political consequences rather than any prior knowledge of illegality.
Additionally, in light of the unresolved legal questions about whether giving ``documents and information'' of the sort offered here constitutes a campaign contribution, Trump~Jr.\ could mount a factual defense that he did not believe his response to the offer and the June~9 meeting itself violated the law.
Given his less direct involvement in arranging the June~9 meeting, Kushner could likely mount a similar defense.
And, while Manafort is experienced with political campaigns, the Office has not developed evidence showing that he had relevant knowledge of these legal issues.

\subparagraph{Difficulties in Valuing Promised Information}
The Office would also encounter difficulty proving beyond a reasonable doubt that the value of the promised documents and information exceeds the \$2,000 threshold for a criminal violation, as well as the \$25,000 threshold for felony punishment.
See 52 U.S.C. \S 30109(d)(1).
The type of evidence commonly used to establish the value of non-monetary contributions---such as pricing the contribution on a commercial market or determining the upstream acquisition cost or the cost of distribution---would likely be unavailable or ineffective in this factual setting.
Although damaging opposition research is surely valuable to a campaign, it appears that the information ultimately delivered in the meeting was not valuable.
And while value in a conspiracy may well be measured by what the participants expected to receive at the time of the agreement, \textit{see, e.g., United States v.\ Tombrello}, 666 F.2d 485, 489 (11th Cir.~1982), Goldstone's description of the offered material here was quite general.
His suggestion of the information's value---i.e., that it would ``incriminate Hillary'' and ``would be very useful to [Trump~Jr.'s] father''---was nonspecific and may have been understood as being of uncertain worth or reliability, given Goldstone's lack of direct access to the original source.
The uncertainty over what would be delivered could be reflected in Trump~Jr.'s response (``\textit{if it's what you say} I love it'') (emphasis added).

Accordingly, taking into account the high burden to establish a culpable mental state in a campaign-finance prosecution and the difficulty in establishing the required valuation, the Office decided not to pursue criminal campaign-finance charges against Trump~Jr.\ or other campaign officials for the events culminating in the June~9 meeting.

\paragraph{Application to [$\blacksquare\blacksquare\blacksquare\blacksquare\blacksquare\blacksquare\blacksquare\blacksquare$: HOM]}
\blackout{Harm to Ongoing Matter}

\subparagraph{Questions Over [$\blacksquare\blacksquare\blacksquare\blacksquare\blacksquare\blacksquare\blacksquare\blacksquare$: Harm to Ongoing Matter]}
\blackout{Harm to Ongoing Matter}

\blackout{Harm to Ongoing Matter}

\subparagraph{Willfulness}
As discussed, to establish a criminal campaign-finance violation, the government must prove that the defendant acted ``knowingly and willfully.''
52 U.S.C. \S 30109(d)(1)(A)(i).
That standard requires proof that the defendant knew generally that his conduct was unlawful.
\textit{Election Offenses}~123.
Given the uncertainties noted above, the ``willfulness'' requirement would pose a substantial barrier to prosecution.

\subparagraph{Constitutional Considerations}
Finally, the First Amendment could pose constraints on a prosecution.
\blackout{Harm to Ongoing Matter}

\subparagraph{Analysis [$\blacksquare\blacksquare\blacksquare\blacksquare\blacksquare\blacksquare\blacksquare\blacksquare$: HOM]}
\blackout{Harm to Ongoing Matter}

\blackout{Harm to Ongoing Matter}

\blackout{Harm to Ongoing Matter}

\subsubsection{False Statements and Obstruction of the Investigation}
The Office determined that certain individuals associated with the Campaign lied to investigators about Campaign contacts with Russia and have taken other actions to interfere with the investigation.
As explained below, the Office therefore charged some U.S. persons connected to the Campaign with false statements and obstruction offenses.

\paragraph{Overview Of Governing Law}
\textit{False Statements}.
The principal federal statute criminalizing false statements to government investigators is 18 U.S.C. \S 1001.
As relevant here, under Section 1001(a)(2), it is a crime to knowingly and willfully ``make[] any materially false, fictitious, or fraudulent statement or representation'' ``in any matter within the jurisdiction of the executive \dots branch of the Government.''
An FBI investigation is a matter within the Executive Branch's jurisdiction.
\textit{United States v.\ Rodgers}, 466 U.S. 475, 479 (1984).
The statute also applies to a subset of legislative branch actions---\textit{viz.}, administrative matters and ``investigation[s] or review[s]'' conducted by a congressional committee or subcommittee.
18 U.S.C. \S 1001(c)(1) and~(2); \textit{see United States v.\ Pickett}, 353 F.3d 62, 66 (D.C. Cir.~2004).

Whether the statement was made to law enforcement or congressional investigators, the government must prove beyond a reasonable doubt the same basic non-jurisdictional elements: the statement was false, fictitious, or fraudulent; the defendant knew both that it was false and that it was unlawful to make a false statement; and the false statement was material.
\textit{See, e.g., United States v.\ Smith}, 831 F.3d 1207, 1222 n.27 (9th Cir.~2017) (listing elements); \textit{see also} Ninth Circuit Pattern Instruction 8.73 \& cmt.\ (explaining that the Section 1001 jury instruction was modified in light of the Department of Justice's position that the phrase ``knowingly and willfully'' in the statute requires the defendant's knowledge that his or her conduct was unlawful).
In the D.C. Circuit, the government must prove that the statement was actually false; a statement that is misleading but ``literally true'' does not satisfy Section 1001(a)(2).
\textit{See United States v.\ Milton}, 8 F.3d 39, 45 (D.C. Cir.~1993); \textit{United States v.\ Dale}, 991 F.2d 819, 832--33 \& n.22 (D.C. Cir.~1993).
For that false statement to qualify as ``material,'' it must have a natural tendency to influence, or be capable of influencing, a discrete decision or any other function of the agency to which it is addressed.
\textit{See United States v.\ Gaudin}, 515 U.S. 506, 509 (1995); \textit{United States v.\ Moore}, 612 F.3d 698, 701 (D.C. Cir.~2010).

\textit{Perjury}.
Under the federal perjury statutes, it is a crime for a witness testifying under oath before a grand jury to knowingly make any false material declaration.
\textit{See} 18 U.S.C. \S 1623.
The government must prove four elements beyond a reasonable doubt to obtain a conviction under Section 1623(a): the defendant testified under oath before a federal grand jury; the defendant's testimony was false in one or more respects; the false testimony concerned matters that were material to the grand jury investigation; and the false testimony was knowingly given.
\textit{United States v.\ Bridges}, 717 F.2d 1444, 1449 n.30 (D.C. Cir.~1983).
The general perjury statute, 18 U.S.C. \S 1621, also applies to grand jury testimony and has similar elements, except that it requires that the witness have acted willfully and that the government satisfy ``strict common-law requirements for establishing falsity.''
\textit{See Dunn v.\ United States}, 442 U.S. 100, 106 \& n.6 (1979) (explaining ``the two-witness rule'' and the corroboration that it demands).

\textit{Obstruction of Justice}.
Three basic elements are common to the obstruction statutes pertinent to this Office's charging decisions: an obstructive act; some form of nexus between the obstructive act and an official proceeding; and criminal (\textit{i.e.}, corrupt) intent.
A detailed discussion of those elements, and the law governing obstruction of justice more generally, is included in Volume II of the report.

\paragraph{Application to Certain Individuals}

\subparagraph{George Papadopoulos}
Investigators approached Papadopoulos for an interview based on his role as a foreign policy advisor to the Trump Campaign and his suggestion to a foreign government representative that Russia had indicated that it could assist the Campaign through the anonymous release of information damaging to candidate Clinton.
On January~27, 2017, Papadopoulos agreed to be interviewed by FBI agents, who informed him that the interview was part of the investigation into potential Russian government interference in the 2016 presidential election.
During the interview, Papadopoulos lied about the timing, extent, and nature of his communications with Joseph Mifsud, Olga Polonskaya, and Ivan Timofeev.
With respect to timing, Papadopoulos acknowledged that he had met Mifsud and that Mifsud told him the Russians had ``dirt'' on Clinton in the form of ``thousands of emails.''
But Papadopoulos stated multiple times that those communications occurred before he joined the Trump Campaign and that it was a ``very strange coincidence'' to be told of the ``dirt'' before he started working for the Campaign.
This account was false.
Papadopoulos met Mifsud for the first time on approximately March~14, 2016, after Papadopoulos had already learned he would be a foreign policy advisor for the Campaign.
Mifsud showed interest in Papadopoulos only after learning of his role on the Campaign.
And Mifsud told Papadopoulos about the Russians possessing ``dirt'' on candidate Clinton in late April 2016, more than a month after Papadopoulos had joined the Campaign and been publicly announced by candidate Trump.
Statement of Offense \P\P 25--26, \textit{United States v.\ George Papadopoulos}, No.~1:17-cr-182 (D.D.C. Oct.~5, 2017), Doc.~19 (``\textit{Papadopoulos} Statement of Offense'').

Papadopoulos also made false statements in an effort to minimize the extent and importance of his communications with Mifsud.
For example, Papadopoulos stated that ``[Mifsud]'s a nothing,'' that he thought Mifsud was ``just a guy talk[ing] up connections or something,'' and that he believed Mifsud was ``BS'ing to be completely honest with you.''
In fact, however, Papadopoulos understood Mifsud to have substantial connections to high-level Russian government officials and that Mifsud spoke with some of those officials in Moscow before telling Papadopoulos about the ``dirt.''
Papadopoulos also engaged in extensive communications over a period of months with Mifsud about foreign policy issues for the Campaign, including efforts to arrange a ``history making'' meeting between the Campaign and Russian government officials.
In addition, Papadopoulos failed to inform investigators that Mifsud had introduced him to Timofeev, the Russian national who Papadopoulos understood to be connected to the Russian Ministry of Foreign Affairs, despite being asked if he had met with Russian nationals or ``[a]nyone with a Russian accent'' during the campaign.
\textit{Papadopoulos} Statement of Offense \P\P 27--29.

Papadopoulos also falsely claimed that he met Polonskaya before he joined the Campaign, and falsely told the FBI that he had ``no'' relationship at all with her.
He stated that the extent of their communications was her sending emails---``Just, `Hi, how are you?' That's it.''
In truth, however, Papadopoulos met Polonskaya on March~24, 2016, after he had joined the Campaign; he believed that she had connections to high-level Russian government officials and could help him arrange a potential foreign policy trip to Russia.
During the campaign he emailed and spoke with her over Skype on numerous occasions about the potential foreign policy trip to Russia.
\textit{Papadopoulos} Statement of Offense \P\P 30--31.

Papadopoulos's false statements in January 2017 impeded the FBI's investigation into Russian interference in the 2016 presidential election.
Most immediately, those statements hindered investigators' ability to effectively question Mifsud when he was interviewed in the lobby of a Washington, D.C. hotel on February~10, 2017.
\textit{See} Gov't Sent.~Mem.~at 6, \textit{United States v.\ George Papadopoulos}, No.~1:17-cr-182 (D.D.C. Aug.~18, 2017), Doc.~44.
During that interview, Mifsud admitted to knowing Papadopoulos and to having introduced him to Polonskaya and Timofeev.
But Mifsud denied that he had advance knowledge that Russia was in possession of emails damaging to candidate Clinton, stating that he and Papadopoulos had discussed cybersecurity and hacking as a larger issue and that Papadopoulos must have misunderstood their conversation.
Mifsud also falsely stated that he had not seen Papadopoulos since the meeting at which Mifsud introduced him to Polonskaya, even though emails, text messages, and other information show that Mifsud met with Papadopoulos on at least two other occasions---April~12 and April~26, 2016.
In addition, Mifsud omitted that he had drafted (or edited) the follow-up message that Polonskaya sent to Papadopoulos following the initial meeting and that, as reflected in the language of that email chain (``Baby, thank you!''), Mifsud may have been involved in a personal relationship with Polonskaya at the time.
The false information and omissions in Papadopoulos's January 2017 interview undermined investigators' ability to challenge Mifsud when he made these inaccurate statements.

Given the seriousness of the lies and omissions and their effect on the FBI's investigation, the Office charged Papadopoulos with making false statements to the FBI, in violation of 18 U.S.C. \S 1001.
Information, \textit{United States v.\ George Papadopoulos}, No.~1:17-cr-182 (D.D.C. Oct.~3, 2017), Doc.~8.
On October~7, 2017, Papadopoulos pleaded guilty to that charge pursuant to a plea agreement.
On September~7, 2018, he was sentenced to 14 days of imprisonment, a \$9,500 fine, and 200 hours of community service.

\subparagraph{[$\blacksquare\blacksquare\blacksquare\blacksquare\blacksquare\blacksquare\blacksquare\blacksquare$: Personal Privacy]}
\blackout{Grand Jury}

\blackout{Grand Jury}

\subparagraph{Michael Flynn}
Michael Flynn agreed to be interviewed by the FBI on January~24, 2017, four days after he had officially assumed his duties as National Security Advisor to the President.
During the interview, Flynn made several false statements pertaining to his communications with the Russian ambassador.
First, Flynn made two false statements about his conversations with Russian Ambassador Kislyak in late December 2016, at a time when the United States had imposed sanctions on Russia for interfering with the 2016 presidential election and Russia was considering its response.
\textit{See Flynn} Statement of Offense.
Flynn told the agents that he did not ask Kislyak to refrain from escalating the situation in response to the United States's imposition of sanctions.
That statement was false.
On December~29, 2016, Flynn called Kislyak to request Russian restraint.
Flynn made the call immediately after speaking to a senior Transition Team official (K.T. McFarland) about what to communicate to Kislyak.
Flynn then spoke with McFarland again after the Kislyak call to report on the substance of that conversation.
Flynn also falsely told the FBI that he did not remember a follow-up conversation in which Kislyak stated that Russia had chosen to moderate its response to the U.S. sanctions as a result of Flynn's request.
On December~31, 2016, Flynn in fact had such a conversation with Kislyak, and he again spoke with McFarland within hours of the call to relay the substance of his conversation with Kislyak.
\textit{See Flynn} Statement of Offense \P 3.

Second, Flynn made false statements about calls he had previously made to representatives of Russia and other countries regarding a resolution submitted by Egypt to the United Nations Security Council on December~21, 2016.
Specifically, Flynn stated that he only asked the countries' positions on how they would vote on the resolution and that he did not request that any of the countries take any particular action on the resolution.
That statement was false.
On December~22, 2016, Flynn called Kislyak, informed him of the incoming Trump Administration's opposition to the resolution, and requested that Russia vote against or delay the resolution.
Flynn also falsely stated that Kislyak never described Russia's response to his December~22 request regarding the resolution.
Kislyak in fact told Flynn in a conversation on December~23, 2016, that Russia would not vote against the resolution if it came to a vote.
\textit{See Flynn} Statement of Offense \P 4.

Flynn made these false statements to the FBI at a time when he was serving as National Security Advisor and when the FBI had an open investigation into Russian interference in the 2016 presidential election, including the nature of any links between the Trump Campaign and Russia.
Flynn's false statements and omissions impeded and otherwise had a material impact on that ongoing investigation.
\textit{Flynn} Statement of Offense \P\P 1--2.
They also came shortly before Flynn made separate submissions to the Department of Justice, pursuant to FARA, that also contained materially false statements and omissions.
\textit{Id}.~\P 5.
Based on the totality of that conduct, the Office decided to charge Flynn with making false statements to the FBI, in violation of 18 U.S.C. \S 1001(a).
On December~1, 2017, and pursuant to a plea agreement, Flynn pleaded guilty to that charge and also admitted his false statements to the Department in his FARA filing.
\textit{See id}.; Plea Agreement, \textit{United States v.\ Michael T. Flynn}, No.~1:17-cr-232 (D.D.C. Dec.~1, 2017), Doc.~3.
Flynn is awaiting sentencing.

\subparagraph{Michael Cohen}
Michael Cohen was the executive vice president and special counsel to the Trump Organization when Trump was president of the Trump Organization.
Information \P 1, \textit{United States v.\ Cohen}, No.~1:18-cr-850 (S.D.N.Y. Nov.~29, 2018), Doc.~2 (``\textit{Cohen} Information'').
From the fall of 2015 through approximately June 2016, Cohen was involved in a project to build a Trump-branded tower and adjoining development in Moscow.
The project was known as Trump Tower Moscow.
In 2017, Cohen was called to testify before the House Permanent Select Committee on Intelligence (HPSCI) and the Senate Select Committee on Intelligence (SSCI), both of which were investigating Russian interference in the 2016 presidential election and possible links between Russia and the presidential campaigns.
In late August 2017, in advance of his testimony, Cohen caused a two-page statement to be sent to SSCI and HPSCI addressing Trump Tower Moscow.
\textit{Cohen} Information \P\P 2--3.
The letter contained three representations relevant here.
First, Cohen stated that the Trump Moscow project had ended in January 2016 and that he had briefed candidate Trump on the project only three times before making the unilateral decision to terminate it.
Second, Cohen represented that he never agreed to travel to Russia in connection with the project and never considered asking Trump to travel for the project.
Third, Cohen stated that he did not recall any Russian government contact about the project, including any response to an email that he had sent to a Russian government email account.
\textit{Cohen} Information \P 4.
Cohen later asked that his two-page statement be incorporated into his testimony's transcript before SSCI, and he ultimately gave testimony to SSCI that was consistent with that statement.
\textit{Cohen} Information \P 5.

Each of the foregoing representations in Cohen's two-page statement was false and misleading.
Consideration of the project had extended through approximately June 2016 and included more than three progress reports from Cohen to Trump.
Cohen had discussed with Felix Sater his own travel to Russia as part of the project, and he had inquired about the possibility of Trump traveling there---both with the candidate himself and with senior campaign official Corey Lewandowski.
Cohen did recall that he had received a response to the email that he sent to Russian government spokesman Dmitry Peskov---in particular, that he received an email reply and had a follow-up phone conversation with an English-speaking assistant to Peskov in mid-January 2016.
\textit{Cohen} Information \P 7.
Cohen knew the statements in the letter to be false at the time, and admitted that he made them in an effort (1) to minimize the links between the project and Trump (who by this time was President), and (2) to give the false impression that the project had ended before the first vote in the Republican Party primary process, in the hopes of limiting the ongoing Russia investigations.
\textit{Id}.

Given the nature of the false statements and the fact that he repeated them during his initial interview with the Office, we charged Cohen with violating Section~1001.
On November~29, 2018, Cohen pleaded guilty pursuant to a plea agreement to a single-count information charging him with making false statements in a matter within the jurisdiction of the legislative branch, in violation of 18 U.S.C. \S 1001(a)(2) and (c).
\textit{Cohen} Information.
The case was transferred to the district judge presiding over the separate prosecution of Cohen pursued by the Southern District of New York (after a referral from our Office).
On December~7, 2018, this Office submitted a letter to that judge recommending that Cohen's cooperation with our investigation be taken into account in sentencing Cohen on both the false-statements charge and the offenses in the Southern District prosecution.
On December~12, 2018, the judge sentenced Cohen to two months of imprisonment on the false-statements count, to run concurrently with a 36-month sentence imposed on the other counts.

\subparagraph{[$\blacksquare\blacksquare\blacksquare\blacksquare\blacksquare\blacksquare\blacksquare\blacksquare$: HOM]}
\blackout{Harm to Ongoing Matter}

\blackout{Harm to Ongoing Matter}

\blackout{Harm to Ongoing Matter}

\blackout{Harm to Ongoing Matter}

\subparagraph{Jeff Sessions}
As set forth in Volume I, Section IV.A.6, \textit{supra}, the investigation established that, while a U.S. Senator and a Trump Campaign advisor, former Attorney General Jeff Sessions interacted with Russian Ambassador Kislyak during the week of the Republican National Convention in July 2016 and again at a meeting in Sessions's Senate office in September 2016.
The investigation also established that Sessions and Kislyak both attended a reception held before candidate Trump's foreign policy speech at the Mayflower Hotel in Washington, D.C., in April 2016, and that it is possible that they met briefly at that reception.
The Office considered whether, in light of these interactions, Sessions committed perjury before, or made false statements to, Congress in connection with his confirmation as Attorney General.
In January 2017 testimony during his confirmation hearing, Sessions stated in response to a question about Trump Campaign communications with the Russian government that he had ``been called a surrogate at a time or two in that campaign and I didn't have -- did not have communications with the Russians.''
In written responses submitted on January~17, 2017, Sessions answered ``[n]o'' to a question asking whether he had ``been in contact with anyone connected to any part of the Russian government about the 2016 election, either before or after election day.''
And, in a March 2017 supplement to his testimony, Sessions identified two of the campaign-period contacts with Ambassador Kislyak noted above, which had been reported in the media following the January 2017 confirmation hearing.
Sessions stated in the supplemental response that he did ``not recall any discussions with the Russian Ambassador, or any other representatives of the Russian government, regarding the political campaign on these occasions or any other occasion.''

Although the investigation established that Sessions interacted with Kislyak on the occasions described above and that Kislyak mentioned the presidential campaign on at least one occasion, the evidence is not sufficient to prove that Sessions gave knowingly false answers to Russia-related questions in light of the wording and context of those questions.
With respect to Sessions's statements that he did ``not recall any discussions with the Russian Ambassador... regarding the political campaign'' and he had not been in contact with any Russian official ``about the 2016 election,'' the evidence concerning the nature of Sessions's interactions with Kislyak makes it plausible that Sessions did not recall discussing the campaign with Kislyak at the time of his statements.
Similarly, while Sessions stated in his January 2017 oral testimony that he ``did not have communications with Russians,'' he did so in response to a question that had linked such communications to an alleged ``continuing exchange of information'' between the Trump Campaign and Russian government intermediaries.
Sessions later explained to the Senate and to the Office that he understood the question as narrowly calling for disclosure of interactions with Russians that involved the exchange of campaign information, as distinguished from more routine contacts with Russian nationals.
Given the context in which the question was asked, that understanding is plausible.

Accordingly, the Office concluded that the evidence was insufficient to prove that Sessions was willfully untruthful in his answers and thus insufficient to obtain or sustain a conviction for perjury or false statements.
Consistent with the Principles of Federal Prosecution, the Office therefore determined not to pursue charges against Sessions and informed his counsel of that decision in March 2018.

\subparagraph{Others Interviewed During the Investigation}
The Office considered whether, during the course of the investigation, other individuals interviewed either omitted material information or provided information determined to be false.
Applying the Principles of Federal Prosecution, the Office did not seek criminal charges against any individuals other than those listed above.
In some instances, that decision was due to evidentiary hurdles to proving falsity.
In others, the Office determined that the witness ultimately provided truthful information and that considerations of culpability, deterrence, and resource-preservation weighed against prosecution.
\textit{See} Justice Manual \S\S 9-27.220, 9-27.230.
\blackout{Personal Privacy}

\blackout{Personal Privacy}

\blackout{Personal Privacy}
\blackout{Grand Jury}
\blackout{Personal Privacy}
\blackout{Grand Jury}
\blackout{Personal Privacy}

\blackout{Personal Privacy}

\blackout{Grand Jury}
\blackout{Personal Privacy}

\blackout{Personal Privacy}

\blackout{Personal Privacy}
