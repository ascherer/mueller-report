\section*{Introduction to Volume II}
\label{sec:introduction-2}
\addcontentsline{toc}{section}{\nameref{sec:introduction-2}}
\markboth{Introduction to Volume II}{Introduction to Volume II}

This report is submitted to the Attorney General pursuant to 28 C.F.R. \S 600.8(c), which states that, “[a]t the conclusion of the Special Counsel’s work, he ... shall provide the Attorney General a confidential report explaining the prosecution or declination decisions [the Special Counsel] reached.”

Beginning in 2017, the President of the United States took a variety of actions towards the ongoing FBI investigation into Russia’s interference in the 2016 presidential election and related matters that raised questions about whether he had obstructed justice.
The Order appointing the Special Counsel gave this Office jurisdiction to investigate matters that arose directly from the FBI’s Russia investigation, including whether the President had obstructed justice in connection with Russia-related investigations.
The Special Counsel’s jurisdiction also covered potentially obstructive acts related to the Special Counsel’s investigation itself.
This Volume of our report summarizes our obstruction-of-justice investigation of the President.

We first describe the considerations that guided our obstruction-of-justice investigation, and then provide an overview of this Volume:

First, a traditional prosecution or declination decision entails a binary determination to initiate or decline a prosecution, but we determined not to make a traditional prosecutorial judgment.
The Office of Legal Counsel (OLC) has issued an opinion finding that “the indictment or criminal prosecution of a sitting President would impermissibly undermine the capacity of the executive branch to perform its constitutionally assigned functions” in violation of “the constitutional separation of powers.”% 1
\footnote{\textit{A Sitting President’s Amenability to Indictment and Criminal Prosecution}, 24 Op.\ O.L.C. 222, 222, 260 (2000) (OLC Op.).}
Given the role of the Special Counsel as an attorney in the Department of Justice and the framework of the Special Counsel regulations, see 28 U.S.C. \S 515; 28 C.F.R. \S 600.7(a), this Office accepted OLC’s legal conclusion for the purpose of exercising prosecutorial jurisdiction.
And apart from OLC’s constitutional view, we recognized that a federal criminal accusation against a sitting President would place burdens on the President’s capacity to govern and potentially preempt constitutional processes for addressing presidential misconduct.% 2
\footnote{\textit{See} \textsc{U.S. Const.} Art.~I \S 2, cl.~5; \S 3, cl.~6; \textit{cf.} OLC Op.\ at 257--258 (discussing relationship between impeachment and criminal prosecution of a sitting President).}

Second, while the OLC opinion concludes that a sitting President may not be prosecuted, it recognizes that a criminal investigation during the President’s term is permissible.% 3
\footnote{OLC Op.\ at 257 n.36 (“A grand jury could continue to gather evidence throughout the period of immunity”).}
The OLC opinion also recognizes that a President does not have immunity after he leaves office.% 4
\footnote{OLC Op.\ at 255 (“Recognizing an immunity from prosecution for a sitting President would not preclude such prosecution once the President’s term is over or he is otherwise removed from office by resignation or impeachment”).}
And if individuals other than the President committed an obstruction offense, they may be prosecuted at this time.
Given those considerations, the facts known to us, and the strong public interest in safeguarding the integrity of the criminal justice system, we conducted a thorough factual investigation in order to preserve the evidence when memories were fresh and documentary materials were available.

Third, we considered whether to evaluate the conduct we investigated under the Justice Manual standards governing prosecution and declination decisions, but we determined not to apply an approach that could potentially result in a judgment that the President committed crimes.
The threshold step under the Justice Manual standards is to assess whether a person’s conduct
“constitutes a federal offense.”
U.S. Dep’t of Justice, Justice Manual \S 9-27.220 (2018) (Justice Manual).
Fairness concerns counseled against potentially reaching that judgment when no charges can be brought.
The ordinary means for an individual to respond to an accusation is through a speedy and public trial, with all the procedural protections that surround a criminal case.
An individual who believes he was wrongly accused can use that process to seek to clear his name.
In contrast, a prosecutor’s judgment that crimes were committed, but that no charges will be brought, affords no such adversarial opportunity for public name-clearing before an impartial adjudicator.% 5
\footnote{For that reason, criticisms have been lodged against the practice of naming unindicted co-conspirators in an indictment.
\textit{See United States v.\ Briggs}, 514 F.2d 794, 802 (5th Cir.~1975) (“The courts have struck down with strong language efforts by grand juries to accuse persons of crime while affording them no forum in which to vindicate themselves.”);
\textit{see also} Justice Manual § 9-11.130.}

The concerns about the fairness of such a determination would be heightened in the case of a sitting President, where a federal prosecutor’s accusation of a crime, even in an internal report, could carry consequences that extend beyond the realm of criminal justice.
OLC noted similar concerns about sealed indictments.
Even if an indictment were sealed during the President’s term, OLC reasoned, “it would be very difficult to preserve [an indictment’s] secrecy,” and if an indictment became public, “[t]he stigma and opprobrium” could imperil the President’s ability to govern.”% 6
\footnote{OLC Op.\ at 259 \& n.38 (citation omitted).}
Although a prosecutor’s internal report would not represent a formal public accusation akin to an indictment, the possibility of the report’s public disclosure and the absence of a neutral adjudicatory forum to review its findings counseled against potentially determining “that the person’s conduct constitutes a federal offense.”
Justice Manual \S 9-27.220.

Fourth, if we had confidence after a thorough investigation of the facts that the President clearly did not commit obstruction of justice, we would so state.
Based on the facts and the applicable legal standards, however, we are unable to reach that judgment.
The evidence we obtained about the President’s actions and intent presents difficult issues that prevent us from
conclusively determining that no criminal conduct occurred.
Accordingly, while this report does not conclude that the President committed a crime, it also does not exonerate him.

\hr

This report on our investigation consists of four parts.
Section I provides an overview of obstruction-of-justice principles and summarizes certain investigatory and evidentiary considerations.
Section II sets forth the factual results of our obstruction investigation and analyzes the evidence.
Section III addresses statutory and constitutional defenses.
Section IV states our conclusion.
