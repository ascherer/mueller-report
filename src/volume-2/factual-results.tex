\section{Factual Results of the Obstruction Investigation}
This section of the report details the evidence we obtained.
We first provide an overview of how Russia became an issue in the 2016 presidential campaign, and how candidate Trump responded.
We then turn to the key events that we investigated: the President's conduct concerning the FBI investigation of Michael Flynn;
the President's reaction to public confirmation of the FBI's Russia investigation; events leading up to and surrounding the termination of FBI Director Comey;
efforts to terminate the Special Counsel; efforts to curtail the scope of the Special Counsel's investigation;
efforts to prevent disclosure of information about the June 9, 2016 Trump Tower meeting between Russians and senior campaign officials; efforts to have the Attorney General unrecuse;
and conduct towards McGahn, Cohen, and other witnesses.

We summarize the evidence we found and then analyze it by reference to the three statutory obstruction-of-justice elements: obstructive act, nexus to a proceeding, and intent.
We focus on elements because, by regulation, the Special Counsel has ``jurisdiction \dots to investigate \dots federal crimes committed in the course of, and with intent to interfere with, the Special Counsel's investigation, such as perjury, obstruction of justice, destruction of evidence, and intimidation of witnesses.''
28 C.F.R. \S 600.4(a).
Consistent with our jurisdiction to investigate federal obstruction crimes, we gathered evidence that is relevant to the elements of those crimes and analyzed them within an elements framework--while refraining from reaching ultimate conclusions about whether crimes were committed, for the reasons explained above.
This section also does not address legal and constitutional defenses raised by counsel for the President; those defenses are analyzed in Volume II, Section III, \textit{infra}.

\subsection{The Campaign's Response to Reports About Russian Support for Trump}
During the 2016 campaign, the media raised questions about a possible connection between the Trump Campaign and Russia.% 7
\footnote{This section summarizes and cites various news stories not for the truth of the information contained in the stories, but rather to place candidate Trump's response to those stories in context.
Volume I of this report analyzes the underlying facts of several relevant events that were reported on by the media during the campaign.}
The questions intensified after WikiLeaks released politically damaging Democratic Party emails that were reported to have been hacked by Russia.
Trump responded to questions about possible connections to Russia by denying any business involvement in Russia--even though the Trump Organization had pursued a business project in Russia as late as June 2016.
Trump also expressed skepticism that Russia had hacked the emails at the same time as he and other Campaign advisors privately sought information about any further planned WikiLeaks releases.
After the election, when questions persisted about possible links between Russia and the Trump Campaign, the President-Elect continued to deny any connections to Russia and privately expressed concerns that reports of Russian election interference might lead the public to question the legitimacy of his election.% 8
\footnote{As discussed in Volume I, while the investigation identified numerous links between individuals with ties to the Russian government and individuals associated with the Trump Campaign, the evidence was not sufficient to charge that any member of the Trump Campaign conspired or coordinated with representatives of the Russian government to interfere in the 2016 election.}

\subsubsection{Press Reports Allege Links Between the Trump Campaign and Russia}
On June 16, 2015, Donald J. Trump declared his intent to seek nomination as the Republican candidate for President.% 9
\footnote{\@realDonaldTrump 6/16/15 (11:57 a.m.~ET) Tweet.}
By early 2016, he distinguished himself among Republican candidates by speaking of closer ties with Russia,% 10
\footnote{\textit{See, e.g.}, Meet the Press Interview with Donald J. Trump, NBC (Dec.~20, 2015) (Trump: "I think it would be a positive thing if Russia and the United States actually got along");
\textit{Presidential Candidate Donald Trump News Conference, Hanahan, South Carolina}, C-SPAN (Feb.~15, 2016) ("You want to make a good deal for the country, you want to deal with Russia.").}
saying he would get along well with Russian President Vladimir Putin,% 11
\footnote{\textit{See, e.g., Anderson Cooper 360 Degrees}, CNN (July 8, 2015) ("I think I get along with [Putin] fine.");
Andrew Rafferty, \textit{Trump Says He Would "Get Along Very Well" With Putin}, NBC (July 30, 2015), (quoting Trump as saying, "I think T would get along very well with Vladimir Putin.").}
questioning whether the NATO alliance was obsolete,% 12
\footnote{\textit{See, e.g.}, \@realDonaldTrump Tweet 3/24/16 (7:47 a.m.~ET);
\@realDonaldTrump Tweet 3/24/16 (7:59 a.m.~ET).}
and praising Putin as a ``strong leader.''% 13
\footnote{\textit{See, e.g.}, Meet the Press Interview with Donald J. Trump, NBC (Dec.~20, 2015) ("[Putin] is a strong leader.
What am I gonna say, he's a weak leader?
He's making mince meat out of our President.");
\textit{Donald Trump Campaign Rally in Vandalia, Ohio}, C-SPAN (Mar.~12, 2016) ("I said [Putin] was a strong leader, which he is.
I mean, he might be bad, he might be good.
But he's a strong leader.").}
The press reported that Russian political analysts and commentators perceived Trump as favorable to Russia.% 14
\footnote{\textit{See, e.g.}, Andrew Osborn, \textit{From Russia with love: why the Kremlin backs Trump}, Reuters (Mar.~24, 2016);
Robert Zubrin, \textit{Trump: The Kremlin's Candidate}, National Review (Apr.~4, 2016).}

Beginning in February 2016 and continuing through the summer, the media reported that several Trump campaign advisors appeared to have ties to Russia.
For example, the press reported that campaign advisor Michael Flynn was seated next to Vladimir Putin at an RT gala in Moscow in December 2015 and that Flynn had appeared regularly on RT as an analyst.% 15
\footnote{See, e.g., Mark Hosenball \& Steve Holland, \textit{Trump being advised by ex-U.S. Lieutenant General who favors closer Russia ties}, Reuters (Feb.~26, 2016);
Tom Hamburger et al., \textit{Inside Trump's financial ties to Russia and his unusual flattery of Vladimir Putin}, Washington Post (June 17, 2016).
Certain matters pertaining to Flynn are described in Volume I, Section IV.B.7, \textit{supra}.}
The press also reported that foreign policy advisor Carter Page had ties to a Russian state-run gas company,% 16
\footnote{\textit{See, e.g.}, Zachary Mider, \textit{Trump's New Russia Advisor Has Deep Ties to Kremlin's Gazprom}, Bloomberg (Mar.~30, 2016);
Julia Iofee, \textit{Who is Carter Page?}, Politico (Sep.~23, 2016).
Certain matters pertaining to Page are described in Volume I, Section IV.A.3, \textit{supra}.}
and that campaign chairman Paul Manafort had done work for the ``Russian-backed former Ukrainian president Viktor Yanukovych.''% 17
\footnote{Tracy Wilkinson, \textit{In a shift, Republican platform doesn't call for arming Ukraine against Russia, spurring outrage}, Los Angeles Times (July 21, 2016);
Josh Rogin, \textit{Trump campaign guts GOP's anti-Russia stance on Ukraine}, Washington Post (July 18, 2016).}
In addition, the press raised questions during the Republican National Convention about the Trump Campaign's involvement in changing the Republican platform's stance on giving ``weapons to Ukraine to fight Russian and rebel forces.''% 18
\footnote{Josh Rogin, \textit{Trump campaign guts GOP's anti-Russia stance on Ukraine}, Washington Post, Opinions (July 18, 2016).
The Republican Platform events are described in Volume I, Section IV.A.6, \textit{supra}.}

\subsubsection{The Trump Campaign Reacts to WikiLeaks's Release of Hacked Emails}
On June 14, 2016, a cybersecurity firm that had conducted in-house analysis for the Democratic National Committee (DNC) posted an announcement that Russian government hackers had infiltrated the DNC's computer and obtained access to documents.% 19
\footnote{\textit{Bears in the Midst: Intrusion into the Democratic National Committee}, CrowdStrike (June 15, 2016) (post originally appearing on June 14, 2016, according to records of the timing provided by CrowdStrike);
Ellen Nakashima, \textit{Russian government hackers penetrated DNC, stole opposition research on Trump}, Washington Post (June 14, 2016).}
On July 22, 2016, the day before the Democratic National Convention, WikiLeaks posted thousands of hacked DNC documents revealing sensitive internal deliberations.% 20
\footnote{Tom Hamburger and Karen Tumulty, \textit{WikiLeaks releases thousands of documents about Clinton and internal deliberations}, Washington Post (July 22, 2016).}
Soon thereafter, Hillary Clinton's campaign manager publicly contended that Russia had hacked the DNC emails and arranged their release in order to help candidate Trump.% 21
\footnote{Amber Phillips, \textit{Clinton campaign manager: Russians leaked Democrats' emails to help Donald Trump}, Washington Post (July 24, 2016).}
On July 26, 2016, the New York Times reported that U.S. ``intelligence agencies ha[d] told the White House they now have `high confidence' that the Russian government was behind the theft of emails and documents from the Democratic National Committee.''% 22
\footnote{David E. Sanger and Eric Schmitt, \textit{Spy Agency Consensus Grows That Russia Hacked D.N.C.}, New York Times (July 26, 2016).}
Within the Trump Campaign, aides reacted with enthusiasm to reports of the hacks.% 23
\footnote{Gates 4/10/18 302, at 5;
Newman 8/23/18 302, at 1.}
\blackout{Harm to Ongoing Matter} discussed with Campaign officials that WikiLeaks would release the hacked material.% 24
\footnote{Gates 4/11/18 302, at 2-3 (SM-2180998);
Gates 10/25/18 302, at 2;
\textit{see also} Volume I, Section III.D.1, \textit{supra}.}
Some witnesses said that Trump himself discussed the possibility of upcoming releases \blackout{Harm to Ongoing Matter}.
Michael Cohen, then-executive vice president of the Trump Organization and special counsel to Trump, recalled hearing \blackout{Harm to Ongoing Matter}.% 25
\footnote{Cohen 8/7/18 302, at 8;
\textit{see also} Volume I, Section III.D.1, \textit{supra}.
According to Cohen, after WikiLeaks's subsequent release of stolen DNC emails on July 22, 2016, Trump said to Cohen words to the effect of, \blackout{Harm to Ongoing Matter}
Cohen 9/18/18 302, at 10.
Cohen's role in the candidate's and later President's activities, and his own criminal conduct, is described in Volume II, Section II.K, \textit{infra}, and in Volume I, Section IV.A.1, \textit{supra}.}
Cohen recalled that Trump responded, ``oh good, alright,'' and \blackout{Harm to Ongoing Matter}.% 26
\footnote{Cohen 8/7/18 302, at 8.}
Manafort said that shortly after WikiLeaks's July 22, 2016 release of hacked documents, he spoke to Trump \blackout{Harm to Ongoing Matter}; Manafort recalled that Trump responded that Manafort should \blackout{Harm to Ongoing Matter} keep Trump updated.% 27
\footnote{\blackout{Grand Jury}.
As explained in footnote 197 of Volume I, Section III.D.1.b, \textit{supra}, this Office has included Manafort's account of these events because it aligns with those of other witnesses and is corroborated to that extent.}
Deputy campaign manager Rick Gates said that Manafort was getting pressure about \blackout{Harm to Ongoing Matter} information and that Manafort instructed Gates \blackout{Harm to Ongoing Matter} status updates on upcoming releases.% 28
\footnote{Gates 10/25/18 302, at 4.}
Around the same time, Gates was with Trump on a trip to an airport \blackout{Harm to Ongoing Matter}, and shortly after the call ended, Trump told Gates that more releases of damaging information would be coming.% 29
\footnote{Gates 10/25/18 302, at 4.}
\blackout{Harm to Ongoing Matter} were discussed within the Campaign,% 30
\footnote{Bannon 1/18/19 302, at 3.}
and in the summer of 2016, the Campaign was planning a communications strategy based on the possible release of Clinton emails by WikiLeaks.% 31
\footnote{Gates 4/11/18 302, at 1-2 (SM-2180998);
Gates 10/25/18 302, at 2 (messaging strategy was being formed in June/July timeframe based on claims by Assange on June 12, 2016, \blackout{Harm to Ongoing Matter}}

\subsubsection{The Trump Campaign Reacts to Allegations That Russia was Seeking to Aid Candidate Trump}
In the days that followed WikiLeaks's July 22, 2016 release of hacked DNC emails, the Trump Campaign publicly rejected suggestions that Russia was seeking to aid candidate Trump.
On July 26, 2016, Trump tweeted that it was "[c]razy" to suggest that Russia was ``dealing with Trump''% 32
\footnote{\@realDonaldTrump 7/26/16 (6:47 p.m.~ET) Tweet.}
and that ``[f]or the record,'' he had ``ZERO investments in Russia.''% 33
\footnote{\@realDonaldTrump 7/26/16 (6:50 p.m.~ET) Tweet.}

In a press conference the next day, July 27, 2016, Trump characterized ``this whole thing with Russia'' as ``a total deflection'' and stated that it was ``far fetched'' and ``ridiculous.''% 34
\footnote{\textit{Donald Trump News Conference, Doral, Florida}, C-SPAN (July 27, 2016).}
Trump said that the assertion that Russia had hacked the emails was unproven, but stated that it would give him ``no pause'' if Russia had Clinton's emails.% 35
\footnote{\textit{Donald Trump News Conference, Doral, Florida}, C-SPAN (July 27, 2016).}
Trump added, ``Russia, if you're listening, I hope you're able to find the 30,000 emails that are missing. I think you will probably be rewarded mightily by our press.''% 36
\footnote{\textit{Donald Trump News Conference, Doral, Florida}, C-SPAN (July 27, 2016).
Within five hours of Trump's remark, a Russian intelligence service began targeting email accounts associated with Hillary Clinton for possible hacks.
\textit{See} Volume I, Section III, \textit{supra}.
In written answers submitted in this investigation, the President stated that he made the "Russia, if you're listening" statement "in jest and sarcastically, as was apparent to any objective observer."
Written Responses of Donald J. Trump (Nov.~20, 2018), at 13 (Response to Question II, Part (d)).}
Trump also said that ``there's nothing that I can think of that I'd rather do than have Russia friendly as opposed to the way they are right now,'' and in response to a question about whether he would recognize Crimea as Russian territory and consider lifting sanctions, Trump replied, ``We'll be looking at that. Yeah, we'll be looking.''% 37
\footnote{\textit{Donald Trump News Conference, Doral, Florida}, C-SPAN (July 27, 2016).
In his written answers submitted in this investigation, the President said that his statement that "we'll be looking" at Crimea and sanctions "did not communicate any position."
Written Responses of Donald J. Trump (Nov.~20, 2018), at 17 (Response to Question IV, Part (g)).}

During the press conference, Trump repeated ``I have nothing to do with Russia'' five times.% 38
\footnote{\textit{Donald Trump News Conference, Doral, Florida}, C-SPAN (July 27, 2016).}
He stated that ``the closest [he] came to Russia'' was that Russians may have purchased a home or condos from him.% 39
\footnote{\textit{Donald Trump News Conference, Doral, Florida}, C-SPAN (July 27, 2016).}
He said that after he held the Miss Universe pageant in Moscow in 2013 he had been interested in working with Russian companies that ``wanted to put a lot of money into developments in Russia'' but ``it never worked out.''% 40
\footnote{\textit{Donald Trump News Conference, Doral, Florida}, C-SPAN (July 27, 2016).}
He explained, "[f]rankly, I didn't want to do it for a couple of different reasons.
But we had a major developer \dots that wanted to develop property in Moscow and other places.
But we decided not to do it.''% 41
\footnote{\textit{Donald Trump News Conference, Doral, Florida}, C-SPAN (July 27, 2016).}
The Trump Organization, however, had been pursuing a building project in Moscow--the Trump Tower Moscow project--from approximately September 2015 through June 2016, and the candidate was regularly updated on developments, including possible trips by Michael Cohen to Moscow to promote the deal and by Trump himself to finalize it.% 42
\footnote{The Trump Tower Moscow project and Trump's involvement in it is discussed in detail in Volume I, Section IV.A.1, \textit{supra}, and Volume II, Section II.K, \textit{infra}.}

Cohen recalled speaking with Trump after the press conference about Trump's denial of any business dealings in Russia, which Cohen regarded as untrue.% 43
\footnote{Cohen 9/18/18 302, at 4.}
Trump told Cohen that Trump Tower Moscow was not a deal yet and said, ``Why mention it  if it is not a deal?''% 44
\footnote{Cohen 9/18/18 302, at 4-5.}
According to Cohen, at around this time, in response to Trump's disavowal of connections to Russia, campaign advisors had developed a ``party line'' that Trump had no business with Russia and no connections to Russia.% 45
\footnote{Cohen 11/20/18 302, at 1;
Cohen 9/18/18 302, at 3-5.
The formation of the "party line" is described in greater detail in Volume II, Section II.K, \textit{infra}.}

In addition to denying any connections with Russia, the Trump Campaign reacted to reports of Russian election interference in aid of the Campaign by seeking to distance itself from Russian contacts.
For example, in August 2016, foreign policy advisor J.D. Gordon declined an invitation to Russian Ambassador Sergey Kislyak's residence because the timing was ``not optimal'' in view of media reports about Russian interference.% 46
\footnote{DJTFP00004953 (8/8/16 Email, Gordon to Pchelyakov) (stating that "[t]hese days are not optimal for us, as we are busily knocking down a stream of false media stories").
The invitation and Gordon's response are discussed in Volume I, Section IV.A.7.a, \textit{supra}.}
On August 19, 2016, Manafort was asked to resign amid media coverage scrutinizing his ties to a pro-Russian political party in Ukraine and links to Russian business.% 47
\footnote{\textit{See, e.g.}, Amber Phillips, \textit{Paul Manafort's complicated ties to Ukraine, explained}, Washington Post (Aug.~19, 2016) ("There were also a wave of fresh headlines dealing with investigations into [Manafort's] ties to a pro-Russian political party in Ukraine.");
Tom Winter \& Ken Dilanian, \textit{Donald Trump Aide Paul Manafort Scrutinized for Russian Business Ties}, NBC (Aug.~18, 2016).
Relevant events involving Manafort are discussed in Volume I, Section IV.A.8, supra.}
And when the media published stories about Page's connections to Russia in September 2016, Trump Campaign officials terminated Page's association with the Campaign and told the press that he had played ``no role'' in the Campaign.% 48
\footnote{Michael Isikoff, \textit{U.S. intel officials probe ties between Trump adviser and Kremlin}, Yahoo News (Sep.~23, 2016);
\textit{see, e.g.}, 9/25/16 Email, Hicks to Conway \& Bannon;
9/23/16 Email, J. Miller to Bannon \& S. Miller;
Page 3/16/17 302, at 2.}

On October 7, 2016, WikiLeaks released the first set of emails stolen by a Russian intelligence agency from Clinton Campaign chairman John Podesta.% 49
\footnote{\@WikiLeaks 10/7/16 (4:32 p.m.~ET) Tweet.}
The same day, the federal government announced that ``the Russian Government directed the recent compromises of e-mails from US persons and institutions, including from US political organizations.''% 50
\footnote{Joint Statement from the Department Of Homeland Security and Office of the Director of National Intelligence on Election Security, DHS (Oct.~7, 2016).}
The government statement directly linked Russian hacking to the releases on WikiLeaks, with the goal of interfering with the presidential election, and concluded ``that only Russia's senior-most officials could have authorized these activities'' based on their ``scope and sensitivity.''% 51
\footnote{Joint Statement from the Department Of Homeland Security and Office of the Director of National Intelligence on Election Security, DHS (Oct.~7, 2016).}

On October 11, 2016, Podesta stated publicly that the FBI was investigating Russia's hacking and said that candidate Trump might have known in advance that the hacked emails were going to be released.% 52
\footnote{John Wagner \& Anne Gearan, \textit{Clinton campaign chairman ties email hack to Russians, suggests Trump had early warning}, Washington Post (Oct.~11, 2016).}
Vice Presidential Candidate Mike Pence was asked whether the Trump Campaign was ``in cahoots'' with WikiLeaks in releasing damaging Clinton-related information and responded, ``Nothing could be further from the truth.''% 53
\footnote{Louis Nelson, \textit{Pence denies Trump camp in cahoots with WikiLeaks}, Politico (Oct.~14, 2016).}

\subsubsection{After the Election, Trump Continues to Deny Any Contacts or Connections with Russia or That Russia Aided his Election}

On November 8, 2016, Trump was elected President.
Two days later, Russian officials told the press that the Russian government had maintained contacts with Trump's "immediate entourage" during the campaign.% 54
\footnote{Ivan Nechepurenko, \textit{Russian Officials Were in Contact With Trump Allies, Diplomat Says}, New York Times (Nov.~10, 2016) (quoting Russian Deputy Foreign Minister Sergey Ryabkov saying, "[t]here were contacts" and "I cannot say that all, but a number of them maintained contacts with Russian representatives");
Jim Heintz \& Matthew Lee, \textit{Russia eyes better ties with Trump; says contacts underway}, Associated Press (Nov.~11, 2016) (quoting Ryabkov saying, "I don't say that all of them, but a whole array of them supported contacts with Russian representatives").}
In response, Hope Hicks, who had been the Trump Campaign spokesperson, said, "We are not aware of any campaign representatives that were in touch with any foreign entities before yesterday, when Mr.~Trump spoke with many world leaders."% 55
\footnote{Ivan Nechepurenko, \textit{Russian Officials Were in Contact With Trump Allies, Diplomat Says}, New York Times (Nov.~11, 2016) (quoting Hicks).}
Hicks gave an additional statement denying any contacts between the Campaign and Russia:
"It never happened.
There was no communication between the campaign and any foreign entity during the campaign."% 56
\footnote{Jim Heintz \& Matthew Lee, \textit{Russia eyes better ties with Trump; says contacts underway}, Associated Press (Nov.~10, 2016) (quoting Hicks).
Hicks recalled that after she made that statement, she spoke with Campaign advisors Kellyanne Conway, Stephen Miller, Jason Miller, and probably Kushner and Bannon to ensure it was accurate, and there was no hesitation or pushback from any of them.
Hicks 12/8/17 302, at 4.}

On December 10, 2016, the press reported that U.S.intelligence agencies had "concluded that Russia interfered in last month's presidential election to boost Donald Trump's bid for the White House."% 57
\footnote{Damien Gayle, \textit{CIA concludes Russia interfered to help Trump win election, say reports}, Guardian (Dec.~10, 2016).}
Reacting to the story the next day, President-Elect Trump stated, "I think it's ridiculous.
I think it's just another excuse."% 58
\footnote{\textit{Chris Wallace Hosts "Fox News Sunday," Interview with President-Elect Donald Trump}, CQ Newsmaker Transcripts (Dec.~11, 2016).}
He continued that no one really knew who was responsible for the hacking, suggesting that the intelligence community had "no idea if it's Russia or China or somebody.
It could be somebody sitting in a bed some place."% 59
\footnote{\textit{Chris Wallace Hosts "Fox News Sunday," Interview with President-Elect Donald Trump}, CQ Newsmaker Transcripts (Dec.~11, 2016).}
The President-Elect also said that Democrats were "putting [] out" the story of Russian interference "because they suffered one of the greatest defeats in the history of politics."% 60
\footnote{\textit{Chris Wallace Hosts "Fox News Sunday," Interview with President-Elect Donald Trump}, CQ Newsmaker Transcripts (Dec.~11, 2016).}

On December 18, 2016, Podesta told the press that the election was "distorted by the Russian intervention" and questioned whether Trump Campaign officials had been "in touch with the Russians."% 61
\footnote{David Morgan, \textit{Clinton campaign: It's an 'open question' if Trump team colluded with Russia}, Reuters Business Insider (Dec.~18, 2016).}
The same day, incoming Chief of Staff Reince Priebus appeared on Fox News Sunday and declined to say whether the President-Elect accepted the intelligence community's determination that Russia intervened in the election.% 62
\footnote{\textit{Chris Wallace Hosts "Fox News Sunday," Interview with President-Elect Donald Trump}, CQ Newsmaker Transcripts (Dec.~11, 2016).}
When asked about any contact or coordination between the Campaign and Russia, Priebus said, "Even this question is insane.
Of course we didn't interface with the Russians."% 63
\footnote{\textit{Chris Wallace Hosts "Fox News Sunday," Interview with President-Elect Donald Trump}, CQ Newsmaker Transcripts (Dec.~11, 2016).}
Priebus added that "this whole thing is a spin job" and said, "the real question is, why the Democrats ... are doing everything they can to delegitimize the outcome of the election?"% 64
\footnote{\textit{Chris Wallace Hosts "Fox News Sunday," Interview with President-Elect Donald Trump}, CQ Newsmaker Transcripts (Dec.~11, 2016).}

On December 29, 2016, the Obama Administration announced that in response to Russian cyber operations aimed at the U.S. election, it was imposing sanctions and other measures on several Russian individuals and entities.% 65
\footnote{\textit{Statement by the President on Actions in Response to Russian Malicious Cyber Activity and Harassment}, White House (Dec.~29, 2016);
\textit{see also} Missy Ryan et al., \textit{Obama administration announces measures to punish Russia for 2016 election interference}, Washington Post (Dec.~29, 2016).
}
When first asked about the sanctions, President-Elect Trump said, "I think we ought to get on with our lives."% 66
\footnote{John Wagner, \textit{Trump on alleged election interference by Russia: 'Get on with our lives,'} Washington Post (Dec.~29, 2016).}
He then put out a statement that said "It's time for our country to move on to bigger and better things," but indicated that he would meet with intelligence community leaders the following week for a briefing on Russian interference.% 67
\footnote{Missy Ryan et al., \textit{Obama administration announces measures to punish Russia for 2016 election interference}, Washington Post (Dec.~29, 2016).}
The briefing occurred on January 6, 2017.% 68
\footnote{Comey 11/15/17 302, at 3.}
Following the briefing, the intelligence community released the public version of its assessment, which concluded with high confidence that Russia had intervened in the election through a variety of means with the goal of harming Clinton's electability.% 69
\footnote{Office of the Director of National Intelligence, \textit{Russia's Influence Campaign Targeting the 2016 US Presidential Election}, at 1 (Jan.~6, 2017).}
The assessment further concluded with high confidence that Putin and the Russian government had developed a clear preference for Trump.% 70
\footnote{Office of the Director of National Intelligence, \textit{Russia's Influence Campaign Targeting the 2016 US Presidential Election}, at 1 (Jan.~6, 2017).}

Several days later, BuzzFeed published unverified allegations compiled by former British intelligence officer Christopher Steele during the campaign about candidate Trump's Russia connections under the headline "These Reports Allege Trump Has Deep Ties To Russia."% 71
\footnote{Ken Bensinger et al., \textit{These Reports Allege Trump Has Deep Ties To Russia}, BuzzFeed (Jan.~10, 2017).}
Ina press conference the next day, the President-Elect called the release "an absolute disgrace" and said, "I have no dealings with Russia.
I have no deals that could happen in Russia, because we've stayed away....
So I have no deals, I have no loans and I have no dealings.
We could make deals in Russia very easily if we wanted to, I just don't want to because I think that would be a conflict."% 72
\footnote{\textit{Donald Trump's News Conference: Full Transcript and Video}, New York Times (Jan.~11, 2017), available at \url{https://www.nytimes.com/2017/01/11/us/politics/trump-press-conference-transcript.html}.}

Several advisors recalled that the President-Elect viewed stories about his Russian connections, the Russia investigations, and the intelligence community assessment of Russian interference as a threat to the legitimacy of his electoral victory.% 73
\footnote{Priebus 10/13/17 302, at 7;
Hicks 3/13/18 302, at 18;
Spicer 10/16/17 302, at 6;
Bannon 2/14/18 302, at 2;
Gates 4/18/18 302, at 3;
\textit{see} Pompeo 6/28/17 302, at 2 (the President believed that the purpose of the Russia investigation was to delegitimize his presidency).}
Hicks, for example, said that the President-Elect viewed the intelligence community assessment as his "Achilles heel" because, even if Russia had no impact on the election, people would think Russia helped him win, taking away from what he had accomplished.% 74
\footnote{Hicks 3/13/18 302, at 18.}
Sean Spicer, the first White House communications director, recalled that the President thought the Russia story was developed to undermine the legitimacy of his election.% 75
\footnote{Spicer 10/17/17 302, at 6.}
Gates said the President viewed the Russia investigation as an attack on the legitimacy of his win.% 76
\footnote{Gates 4/18/18 302, at 3.}
And Priebus recalled that when the intelligence assessment came out, the President-Elect was concerned people would question the legitimacy of his win.% 77
\footnote{Priebus 10/13/17 302, at 7.}

\subsection{The President's Conduct Concerning the Investigation of Michael Flynn}

\begin{center}
\textbf{Overview}
\end{center}

During the presidential transition, incoming National Security Advisor Michael Flynn had two phone calls with the Russian Ambassador to the United States about the Russian response to U.S. sanctions imposed because of Russia's election interference.
After the press reported on Flynn's contacts with the Russian Ambassador, Flynn lied to incoming Administration officials by saying he had not discussed sanctions on the calls.
The officials publicly repeated those lies in press interviews.
The FBI, which previously was investigating Flynn for other matters, interviewed him about the calls in the first week after the inauguration, and Flynn told similar lies to the FBI.
On January 26, 2017, Department of Justice (DOJ) officials notified the White House that Flynn and the Russian Ambassador had discussed sanctions and that Flynn had been interviewed by the FBI.
The next night, the President had a private dinner with FBI Director James Comey in which he asked for Comey's loyalty.
On February 13, 2017, the President asked Flynn to resign.
The following day, the President had a one-on-one conversation with Comey in which he said, "I hope you can see your way clear to letting this go, to letting Flynn go."

\begin{center}
\textbf{Evidence}
\end{center}

\subsubsection{Incoming National Security Advisor Flynn Discusses Sanctions on Russia with Russian Ambassador Sergey Kislyak}

Shortly after the election, President-Elect Trump announced he would appoint Michael Flynn as his National Security Advisor.% 78
\footnote{Flynn 11/16/17 302, at 7;
\textit{President-Elect Donald J. Trump Selects U.S. Senator Jeff Sessions for Attorney General, Lt.~Gen.\ Michael Flynn as Assistant to the President for National Security Affairs and U.S. Rep.~Mike Pompeo as Director of the Central Intelligence Agency}, President-Elect Donald J. Trump Press Release (Nov.~18, 2016);
\textit{see also}, e.g., Bryan Bender, \textit{Trump names Mike Flynn national security adviser}, Politico, (Nov.~17, 2016).}
For the next two months, Flynn played an active role on the Presidential Transition Team (PTT) coordinating policy positions and communicating with foreign government officials, including Russian Ambassador to the United States Sergey Kislyak.% 79
\footnote{Flynn 11/16/17 302, at 8-14;
Priebus 10/13/17 302, at 3-5.}

On December 29, 2016, as noted in Volume II, Section II.A.4, supra, the Obama Administration announced that it was imposing sanctions and other measures on several Russian individuals and entities.% 80
\footnote{\textit{Statement by the President on Actions in Response to Russian Malicious Cyber Activity and Harassment}, The White House, Office of the Press Secretary (Dec.~29, 2016).}
That day, multiple members of the PTT exchanged emails about the sanctions and the impact they would have on the incoming Administration, and Flynn informed members of the PTT that he would be speaking to the Russian Ambassador later in the day.% 81
\footnote{12/29/16 Email, O'Brien to McFarland et al.;
12/29/16 Email, Bossert to Flynn et al.;
12/29/16 Email, McFarland to Flynn et al.;
SF000001 (12/29/16 Text Message, Flynn to Flaherty) ("Tit for tat w Russia not good. Russian AMBO reaching out to me today.");
Flynn 1/19/18 302, at 2.}
Flynn, who was in the Dominican Republic at the time, and K.T. McFarland, who was slated to become the Deputy National Security Advisor and was at the Mar-a-Lago resort in Florida with the President-Elect and other senior staff, talked by phone about what, if anything, Flynn should communicate to Kislyak about the sanctions.% 82
\footnote{Statement of Offense at 2-3, \textit{United States v.\ Michael T. Flynn}, 1:17-cr-232 (D.D.C. Dec.~1, 2017), Doc.~4 (Flynn Statement of Offense);
Flynn 11/17/17 302, at 3-4;
Flynn 11/20/17 302, at 3;
McFarland 12/22/17 302, at 6-7.}
McFarland had spoken with incoming Administration officials about the sanctions and Russia's possible responses and thought she had mentioned in those conversations that Flynn was scheduled to speak with Kislyak.% 83
\footnote{McFarland 12/22/17 302, at 4-7 (recalling discussions about this issue with Bannon and Priebus).}
Based on those conversations, McFarland informed Flynn that incoming Administration officials at Mar-a-Lago did not want Russia to escalate the situation.% 84
\footnote{Flynn Statement of Offense, at 3;
Flynn 11/17/17 302, at 3-4;
McFarland 12/22/17 302, at 6-7.}
At 4:43 p.m.\ that afternoon, McFarland sent an email to several officials about the sanctions and informed the group that "Gen [F]lynn is talking to russian ambassador this evening."% 85
\footnote{12/29/16 Email, McFarland to Flynn et al.}

Approximately one hour later, McFarland met with the President-Elect and senior officials and briefed them on the sanctions and Russia's possible responses.% 86
\footnote{McFarland 12/22/17 302, at 7.}
Incoming Chief of Staff Reince Priebus recalled that McFarland may have mentioned at the meeting that the sanctions situation could be "cooled down" and not escalated.% 87
\footnote{Priebus 1/18/18 302, at 3.}
McFarland recalled that at the end of the meeting, someone may have mentioned to the President-Elect that Flynn was speaking to the Russian Ambassador that evening.% 88
\footnote{McFarland 12/22/17 302, at 7.
Priebus thought it was possible that McFarland had mentioned Flynn's scheduled call with Kislyak at this meeting, although he was not certain.
Priebus 1/18/18 302, at 3.}
McFarland did not recall any response by the President-Elect.% 89
\footnote{McFarland 12/22/17 302, at 7.}
Priebus recalled that the President-Elect viewed the sanctions as an attempt by the Obama Administration to embarrass him by delegitimizing his election.% 90
\footnote{Priebus 1/18/18 302, at 3.}

Immediately after discussing the sanctions with McFarland on December 29, 2016, Flynn called Kislyak and requested that Russia respond to the sanctions only in a reciprocal manner, without escalating the situation.% 91
\footnote{\textit{Flynn} Statement of Offense, at 3;
Flynn 11/17/17 302, at 3-4.}
After the call, Flynn briefed McFarland on its substance.% 92
\footnote{\textit{Flynn} Statement of Offense, at 3;
McFarland 12/22/17 302, at 7-8;
Flynn 11/17/17 302, at 4.}
Flynn told McFarland that the Russian response to the sanctions was not going to be escalatory because Russia wanted a good relationship with the Trump Administration.% 93
\footnote{McFarland 12/22/17 302, at 8.}
On December 30, 2016, Russian President Vladimir Putin announced that Russia would not take retaliatory measures in response to the sanctions at that time and would instead "plan ... further steps to restore Russian-US relations based on the policies of the Trump Administration."% 94
\footnote{\textit{Statement by the President of Russia}, President of Russia (Dec.~30, 2016) 12/30/16.}
Following that announcement, the President-Elect tweeted, "Great move on delay (by V. Putin) - I always knew he was very smart!"% 95
\footnote{\@realDonaldTrump 12/30/16 (2:41 p.m.~ET) Tweet.}

On December 31, 2016, Kislyak called Flynn and told him that Flynn's request had been received at the highest levels and Russia had chosen not to retaliate in response to the request.% 96
\footnote{Flynn 1/19/18 302, at 3;
\textit{Flynn} Statement of Offense, at 3.}
Later that day, Flynn told McFarland about this follow-up conversation with Kislyak and Russia's decision not to escalate the sanctions situation based on Flynn's request.% 97
\footnote{Flynn 1/19/18 302, at 3;
Flynn 11/17/17 302, at 6;
McFarland 12/22/17 302, at 10;
\textit{Flynn} Statement of Offense, at 3.}
McFarland recalled that Flynn thought his phone call had made a difference.% 98
\footnote{McFarland 12/22/17 302, at 10;
\textit{see} Flynn 1/19/18 302, at 4.}
Flynn spoke with other incoming Administration officials that day, but does not recall whether they discussed the sanctions.% 99
\footnote{Flynn 11/17/17 302, at 5-6.}

Flynn recalled discussing the sanctions issue with incoming Administration official Stephen Bannon the next day.% 100
\footnote{Flynn 1/19/18 302, at 4-5.
Bannon recalled meeting with Flynn that day, but said he did not remember discussing sanctions with him.
Bannon 2/12/18 302, at 9.}
Flynn said that Bannon appeared to know about Flynn's conversations with Kislyak, and he and Bannon agreed that they had "stopped the train on Russia's response "to the sanctions.% 101
\footnote{Flynn 11/21/17 302, at 1;
Flynn 1/19/18 302, at 5.}
On January 3, 2017, Flynn saw the President-Elect in person and thought they discussed the Russian reaction to the sanctions, but Flynn did not have a specific recollection of telling the President-Elect about the substance of his calls with Kislyak.% 102
\footnote{Flynn 1/19/18 302, at 6;
Flynn 11/17/17 302, at 6.}

Members of the intelligence community were surprised by Russia's decision not to retaliate in response to the sanctions.% 103
\footnote{McCord 7/17/17 302, at 2.}
When analyzing Russia's response, they became aware of Flynn's discussion of sanctions with Kislyak.% 104
\footnote{McCord 7/17/17 302, at 2.}
Previously, the FBI had opened an investigation of Flynn based on his relationship with the Russian government.% 105
\footnote{McCord 7/17/17 302, at 2-3;
Comey 11/15/17 302, at 5.}
Flynn's contacts with Kislyak became a key component of that investigation.% 106
\footnote{McCord 7/17/17 302, at 2-3.}

\subsubsection{President-Elect Trump is Briefed on the Intelligence Community's Assessment of Russian Interference in the Election and Congress Opens Election-Interference Investigations}

On January 6, 2017, as noted in Volume II, Section II.A.4, supra, intelligence officials briefed President-Elect Trump and the incoming Administration on the intelligence community's assessment that Russia had interfered in the 2016 presidential election.% 107
\footnote{\textit{Hearing on Russian Election Interference Before the Senate Select Intelligence Committee}, 115th Cong.\ (June 8, 2017) (Statement for the Record of James B. Comey, former Director of the FBI, at 1-2).}
When the briefing concluded, Comey spoke with the President-Elect privately to brief him on unverified, personally sensitive allegations compiled by Steele.% 108
\footnote{Comey 11/15/17 302, at 3;
\textit{Hearing on Russian Election Interference Before the Senate Select Intelligence Committee}, 115th Cong.\ (June 8, 2017) (Statement for the Record of James B. Comey, former Director of the FBI, at 1-2).}
According to a memorandum Comey drafted immediately after their private discussion, the President-Elect began the meeting by telling Comey he had conducted himself honorably over the prior year and had a great reputation.% 109
\footnote{Comey 1/7/17 Memorandum, at 1.
Comey began drafting the memorandum summarizing the meeting immediately after it occurred.
Comey 11/15/17 302, at 4.
He finished the memorandum that evening and finalized it the following morning.
Comey 11/15/17 302, at 4.}
The President-Elect stated that he thought highly of Comey, looked forward to working with him, and hoped that he planned to stay on as FBI director.% 110
\footnote{Comey 1/7/17 Memorandum, at 1;
Comey 11/15/17 302, at 3.
Comey identified several other occasions in January 2017 when the President reiterated that he hoped Comey would stay on as FBI director.
On January 11, President-Elect Trump called Comey to discuss the Steele reports and stated that he thought Comey was doing great and the President-Elect hoped he would remain in his position as FBI director.
Comey 11/15/17 302, at 4;
\textit{Hearing on Russian Election Interference Before the Senate Select Intelligence Committee}, 115th Cong.\ (June 8, 2017) (testimony of James B. Comey, former Director of the FBI), CQ Cong.\ Transcripts, at 90.
("[D]uring that call, he asked me again, 'Hope you're going to stay, you're doing a great job.' And told him that I intended to.").
On January 22, at a White House reception honoring law enforcement, the President greeted Comey and said he looked forward to working with him.
\textit{Hearing on Russian Election Interference Before the Senate Select Intelligence Committee}, 115th Cong.\ (June 8, 2017) (testimony of James B. Comey, former Director of the FBI), CQ Cong.\ Transcripts, at 22.
And as discussed in greater detail in Volume II, Section II.D, infra, on January 27, the President invited Comey to dinner at the White House and said he was glad Comey wanted to stay on as FBI Director.}
Comey responded that he intended to continue serving in that role.% 111
\footnote{Comey 1/7/17 Memorandum, at 1;
Comey 11/15/17 302, at 3.}
Comey then briefed the President-Elect on the sensitive material in the Steele reporting.% 112
\footnote{Comey 1/7/17 Memorandum, at 1-2;
Comey 11/15/17 302, at 3.
Comey's briefing included the Steele reporting's unverified allegation that the Russians had compromising tapes of the President involving conduct when he was a private citizen during a 2013 trip to Moscow for the Miss Universe Pageant.
During the 2016 presidential campaign, a similar claim may have reached candidate Trump.
On October 30, 2016, Michael Cohen received text from Russian businessman Giorgi Rtskhiladze that said, "Stopped flow of tapes from Russia but not sure if there's anything else.
Just so you know...." 10/30/16 Text Message, Rtskhiladze to Cohen.
Rtskhiladze said "tapes" referred to compromising tapes of Trump rumored to be held by persons associated with the Russian real estate conglomerate Crocus Group, which had helped host the 2013 Miss Universe Pageant in Russia.
Rtskhiladze 4/4/18 302, at 12.
Cohen said he spoke to Trump about the issue after receiving the texts from Rtskhiladze.
Cohen 9/12/18 302, at 13.
Rtskhiladze said he was told the tapes were fake, but he did not communicate that to Cohen.
Rtskhiladze 5/10/18 302, at 7.}
Comey recalled that the President-Elect seemed defensive, so Comey decided to assure him that the FBI was not investigating him personally.% 113
\footnote{Comey 11/15/17 302, at 3-4;
\textit{Hearing on Russian Election Interference Before the Senate Select Intelligence Committee}, 115th Cong.\ (June 8, 2017) (Statement for the Record of James B. Comey, former Director of the FBI, at 2).}
Comey recalled he did not want the President-Elect to think of the conversation as a "J. Edgar Hoover move."% 114
\footnote{Comey 11/15/17 302, at 3.}

On January 10, 2017, the media reported that Comey had briefed the President-Elect on the Steele reporting,% 115
\footnote{\textit{See, e.g.}, Evan Perez et al., \textit{Intel chiefs presented Trump with claims of Russian efforts to compromise him}, CNN (Jan.~10, 2017;
updated Jan.~12, 2017).}
and BuzzFeed News published information compiled by Steele online, stating that the information included "specific, unverified, and potentially unverifiable allegations of contact between Trump aides and Russian operatives."% 116
\footnote{Ken Bensinger et al., \textit{These Reports Allege Trump Has Deep Ties To Russia}, BuzzFeed News (Jan.~10, 2017).}
The next day, the President-Elect expressed concern to intelligence community leaders about the fact that the information had leaked and asked whether they could make public statements refuting the allegations in the Steele reports.% 117
\footnote{See 1/11/17 Email, Clapper to Comey ("He asked if I could put out a statement.
He would prefer of course that I say the documents are bogus, which, of course, I can't do.");
1/12/17 Email, Comey to Clapper ("He called me at 5 yesterday and we had a very similar conversation.");
Comey 11/15/17 302, at 4-5.}

In the following weeks, three Congressional committees opened investigations to examine Russia's interference in the election and whether the Trump Campaign had colluded with Russia.% 118
\footnote{\textit{See 2016 Presidential Election Investigation Fast Facts}, CNN (first published Oct.~12, 2017;
updated Mar.~1, 2019) (summarizing starting dates of Russia-related investigations).}
On January 13, 2017, the Senate Select Committee on Intelligence (SSCI) announced that it would conduct a bipartisan inquiry into Russian interference in the election, including any "links between Russia and individuals associated with political campaigns."% 119
\footnote{\textit{Joint Statement on Committee Inquiry into Russian Intelligence Activities}, SSCI (Jan.~13, 2017).}
On January 25, 2017, the House Permanent Select Committee on Intelligence (HPSCI) announced that it had been conducting an investigation into Russian election interference and possible coordination with the political campaigns.% 120
\footnote{\textit{Joint Statement on Progress of Bipartisan HPSCI Inquiry into Russian Active Measures}, HPSCI (Jan.~25, 2017).}
And on February 2, 2017, the Senate Judiciary Committee announced that it too would investigate Russian efforts to intervene in the election.% 121
\footnote{\textit{Joint Statement from Senators Graham and Whitehouse on Investigation into Russian Influence on Democratic Nations' Elections} (Feb.~2, 2017).}

\subsubsection{Flynn Makes False Statements About his Communications with Kislyak to Incoming Administration Officials, the Media, and the FBI}

On January 12, 2017, a Washington Post columnist reported that Flynn and Kislyak communicated on the day the Obama Administration announced the Russia sanctions.% 122
\footnote{David Ignatius, \textit{Why did Obama dawdle on Russia's hacking?}, Washington Post (Jan.~12, 2017).}
The column questioned whether Flynn had said something to "undercut the U.S. sanctions" and whether Flynn's communications had violated the letter or spirit of the Logan Act.% 123
\footnote{David Ignatius, \textit{Why did Obama dawdle on Russia's hacking?}, Washington Post (Jan.~12, 2017).
The Logan Act makes it a crime for "[a]ny citizen of the United States, wherever he maybe" to "without authority of the United States, directly or indirectly commence[] or carr[y] on any correspondence or intercourse with any foreign government or any officer or agent thereof, in relation to any disputes or controversies with the United States, or to defeat the measures of the United States." 18 U.S.C. \S 953.}

President-Elect Trump called Priebus after the story was published and expressed anger about it.% 124
\footnote{Priebus 1/18/18 302, at 6.}
Priebus recalled that the President-Elect asked, "What the hell is this all about?"% 125
\footnote{Priebus 1/18/18 302, at 6.}
Priebus called Flynn and told him that the President-Elect was angry about the reporting on Flynn's conversations with Kislyak.% 126
\footnote{Priebus 1/18/18 302, at 6.}
Flynn recalled that he felt a lot of pressure because Priebus had spoken to the "boss" and said Flynn needed to "kill the story."% 127
\footnote{Flynn 11/21/17 302, at 1;
Flynn 11/20/17 302, at 6.}
Flynn directed McFarland to call the Washington Post columnist and inform him that no discussion of sanctions had occurred.% 128
\footnote{McFarland 12/22/17 302, at 12-13.}
McFarland recalled that Flynn said words to the effect of, "I want to kill the story."% 129
\footnote{McFarland 12/22/17 302, at 12.}
McFarland made the call as Flynn had requested although she knew she was providing false information, and the Washington Post updated the column to reflect that a "Trump official" had denied that Flynn and Kislyak discussed sanctions.% 130
\footnote{McFarland 12/22/17 302, at 12-13;
McFarland 8/29/17 302, at 8;
\textit{see} David Ignatius, \textit{Why did Obama dawdle on Russia's hacking?}, Washington Post (Jan.~12, 2017).}

When Priebus and other incoming Administration officials questioned Flynn internally about the Washington Post column, Flynn maintained that he had not discussed sanctions with Kislyak.% 131
\footnote{Flynn 11/17/17 302, at 1, 8;
Flynn 1/19/18 302, at 7;
Priebus 10/13/17 302, at 7-8;
S. Miller 8/31/17 302, at 8-11.}
Flynn repeated that claim to Vice President-Elect Michael Pence and to incoming press secretary Sean Spicer.% 132
\footnote{Flynn 11/17/17 302, at 1, 8;
Flynn 1/19/18 302, at 7;
S. Miller 8/31/17 302, at 10-11.}
In subsequent media interviews in mid-January, Pence, Priebus, and Spicer denied that Flynn and Kislyak had discussed sanctions, basing those denials on their conversations with Flynn.% 133
\footnote{\textit{Face the Nation Interview with Vice President-Elect Pence}, CBS (Jan.~15, 2017);
Julie Hirschfield Davis et al., \textit{Trump National Security Advisor Called Russian Envoy Day Before Sanctions Were Imposed}, Washington Post (Jan.~13, 2017);
\textit{Meet the Press Interview with Reince Priebus}, NBC (Jan.~15, 2017).}

The public statements of incoming Administration officials denying that Flynn and Kislyak had discussed sanctions alarmed senior DOJ officials, who were aware that the statements were not true.% 134
\footnote{Yates 8/15/17 302, at 2-3;
McCord 7/17/17 302, at 3-4;
McCabe 8/17/17 302, at 5 (DOJ officials were "really freaked out about it").}
Those officials were concerned that Flynn had lied to his colleagues - who in turn had unwittingly misled the American public - creating a compromise situation for Flynn because the Department of Justice assessed that the Russian government could prove Flynn lied.% 135
\footnote{Yates 8/15/17 302, at 3;
McCord 7/17/17 302, at 4.}
The FBI investigative team also believed that Flynn's calls with Kislyak and subsequent denials about discussing sanctions raised potential Logan Act issues and were relevant to the FBI's broader Russia investigation.% 136
\footnote{McCord 7/17/17 302, at 4;
McCabe 8/17/17 302, at 5-6.}

On January 20, 2017, President Trump was inaugurated and Flynn was sworn in as National Security Advisor.
On January 23, 2017, Spicer delivered his first press briefing and stated that he had spoken with Flynn the night before, who confirmed that the calls with Kislyak were about topics unrelated to sanctions.% 137
\footnote{Sean Spicer, \textit{White House Daily Briefing}, C-SPAN (Jan.~23, 2017).}
Spicer's statements added to the Department of Justice's concerns that Russia had leverage over Flynn based on his lies and could use that derogatory information to compromise him.% 138
\footnote{Yates 8/15/17 302, at 4;
Axelrod 7/20/17 302, at 5.}

On January 24, 2017, Flynn agreed to be interviewed by agents from the FBI.% 139
\footnote{\textit{Flynn} Statement of Offense, at 2.}
During the interview, which took place at the White House, Flynn falsely stated that he did not ask Kislyak to refrain from escalating the situation in response to the sanctions on Russia imposed by the Obama Administration.% 140
\footnote{\textit{Flynn} Statement of Offense, at 2.}
Flynn also falsely stated that he did not remember a follow-up conversation in which Kislyak stated that Russia had chosen to moderate its response to those sanctions as a result of Flynn's request.% 141
\footnote{\textit{Flynn} Statement of Offense, at 2.
On December 1, 2017, Flynn admitted to making these false statements and pleaded guilty to violating 18 U.S.C. \S 1001, which makes it a crime to knowingly and willfully "make[] any materially false, fictitious, or fraudulent statement or representation" to federal law enforcement officials.
See Volume I, Section IV.A.7, \textit{supra}.}

\subsubsection{DOJ Officials Notify the White House of Their Concerns About Flynn}

On January 26, 2017, Acting Attorney General Sally Yates contacted White House Counsel Donald McGahn and informed him that she needed to discuss a sensitive matter with him in person.% 142
\footnote{Yates 8/15/17 302, at 6.}
Later that day, Yates and Mary McCord, a senior national security official at the Department of Justice, met at the White House with McGahn and White House Counsel's Office attorney James Burnham.% 143
\footnote{Yates 8/15/17 302, at 6;
McCord 7/17/17 302, at 6;
SCR015\_000198 (2/15/17 Draft Memorandum to file from the Office of the Counsel to the President).}
Yates said that the public statements made by the Vice President denying that Flynn and Kislyak discussed sanctions were not true and put Flynn in a potentially compromised position because the Russians would know he had lied.% 144
\footnote{Yates 8/15/17 302, at 6-8;
McCord 7/17/17 302, at 6-7;
Burnham 11/3/17 302, at 4;
SCR015\_000198 (2/15/17 Draft Memorandum to file from the Office of the Counsel to the President).}
Yates disclosed that Flynn had been interviewed by the FBI.% 145
\footnote{McGahn 11/30/17 302, at 5;
Yates 8/15/17 302, at 7;
McCord 7/17/17 302, at 7;
Burnham 11/3/17 302, at 4.}
She declined to answer a specific question about how Flynn had performed during that interview,% 146
\footnote{Yates 8/15/17 302, at 7;
McCord 7/17/17 302, at 7.}
but she indicated that Flynn's statements to the FBI were similar to the statements he had made to Pence and Spicer denying that he had discussed sanctions.% 147
\footnote{SCR015\_000198 (2/15/17 Draft Memorandum to file from the Office of the Counsel to the President);
Burnham 11/3/17 302, at 4.}
McGahn came away from the meeting with the impression that the FBI had not pinned Flynn down in lies,% 148
\footnote{McGahn 11/30/17 302, at 5.}
but he asked John Eisenberg, who served as legal advisor to the National Security Council, to examine potential legal issues raised by Flynn's FBI interview and his contacts with Kislyak.% 149
\footnote{SCR015\_000198 (2/15/17 Draft Memorandum to file from the Office of the Counsel to the President);
McGahn 11/30/17 302, at 6, 8.}

That afternoon, McGahn notified the President that Yates had come to the White House to discuss concerns about Flynn.% 150
\footnote{McGahn 11/30/17 302, at 6;
SCR015\_000278 (White House Counsel's Office Memorandum re: "Flynn Tick Tock") (on January 26, "McGahn IMMEDIATELY advises POTUS");
SCR015\_000198 (2/15/17 Draft Memorandum to file from the Office of the Counsel to the President).}
McGahn described what Yates had told him, and the President asked him to repeat it, so he did.% 151
\footnote{McGahn 11/30/17 302, at 6.}
McGahn recalled that when he described the FBI interview of Flynn, he said that Flynn did not disclose having discussed sanctions with Kislyak, but that there may not have been a clear violation of 18 U.S.C. \S 1001.% 152
\footnote{McGahn 11/30/17 302, at 7.}
The President asked about Section 1001, and McGahn explained the law to him, and also explained the Logan Act.% 153
\footnote{McGahn 11/30/17 302, at 7.}
The President instructed McGahn to work with Priebus and Bannon to look into the matter further and directed that they not discuss it with any other officials.% 154
\footnote{McGahn 11/30/17 302, at 7;
SCR015\_000198-99 (2/15/17 Draft Memorandum to file from the Office of the Counsel to the President).}
Priebus recalled that the President was angry with Flynn in light of what Yates had told the White House and said, "not again, this guy, this stuff."% 155
\footnote{Priebus 10/13/17 302, at 8.
Several witnesses said that the President was unhappy with Flynn for other reasons at this time.
Bannon said that Flynn's standing with the President was not good by December 2016.
Bannon 2/12/18 302, at 12.
The President-Elect had concerns because President Obama had warned him about Flynn shortly after the election.
Bannon 2/12/18 302, at 4-5;
Hicks 12/8/17 302, at 7 (President Obama's comments at with President-Elect Trump more than Hicks expected).
Priebus said that the President had become unhappy with Flynn even before the story of his calls with Kislyak broke and had become so upset with Flynn that he would not look at him during intelligence briefings.
Priebus 1/18/18 302, at 8.
Hicks said that the President thought Flynn had bad judgment and was angered by tweets sent by Flynn and his son, and she described Flynn as "being on thin ice" by early February 2017.
Hicks 12/8/17 302, at 7, 10.}

That evening, the President dined with several senior advisors and asked the group what they thought about FBI Director Comey.% 156
\footnote{Coats 6/14/17 302, at 2.}
According to Director of National Intelligence Dan Coats, who was at the dinner, no one openly advocated terminating Comey but the consensus on him was not positive.% 157
\footnote{Coats 6/14/17 302, at 2.}
Coats told the group that he thought Comey was a good director.% 158
\footnote{Coats 6/14/17 302, at 2.}
Coats encouraged the President to meet Comey face-to-face and spend time with him before making a decision about whether to retain him.% 159
\footnote{Coats 6/14/17 302, at 2.}

\subsubsection{McGahn has a Follow-Up Meeting About Flynn with Yates; President Trump has Dinner with FBI Director Comey}

The next day, January 27, 2017, McGahn and Eisenberg discussed the results of Eisenberg's initial legal research into Flynn's conduct, and specifically whether Flynn may have violated the Espionage Act, the Logan Act, or 18 U.S.C. \S 1001.% 160
\footnote{SCR015\_000199 (2/15/17 Draft Memorandum to file from the Office of the Counsel to the President);
McGahn 11/30/17 302, at 8.}
Based on his preliminary research, Eisenberg informed McGahn that there was a possibility that Flynn had violated 18 U.S.C. \S 1001 and the Logan Act.% 161
\footnote{SCR015\_000199 (2/15/17 Draft Memorandum to file from the Office of the Counsel to the President);
McGahn 11/30/17 302, at 9.}
Eisenberg noted that the United States had never successfully prosecuted an individual under the Logan Act and that Flynn could have possible defenses, and told McGahn that he believed it was unlikely that a prosecutor would pursue a Logan Act charge under the circumstances.% 162
\footnote{SCR015\_000199 (2/15/17 Draft Memorandum to file from the Office of the Counsel to the President);
Eisenberg 11/29/17 302, at 9.}

That same morning, McGahn asked Yates to return to the White House to discuss Flynn again.% 163
\footnote{SCR015\_000199 (2/15/17 Draft Memorandum to file from the Office of the Counsel to the President);
McGahn 11/30/17 302, at 8;
Yates 8/15/17 302, at 8.}
In that second meeting, McGahn expressed doubts that the Department of Justice would bring a Logan Act prosecution against Flynn, but stated that the White House did not want to take action that would interfere with an ongoing FBI investigation of Flynn.% 164
\footnote{Yates 8/15/17 302, at 9;
McGahn 11/30/17 302, at 8.}
Yates responded that Department of Justice had notified the White House so that it could take action in response to the information provided.% 165
\footnote{Yates 8/15/17 302, at 9;
Burnham 11/3/17 302, at 5;
\textit{see} SCR015\_00199 (2/15/17 Draft Memorandum to file from the Office of the Counsel to the President) ("Yates was unwilling to confirm or deny that there was an ongoing investigation but did indicate that the Department of Justice would not object to the White House taking action against Flynn.").}
McGahn ended the meeting by asking Yates for access to the underlying information the Department of Justice possessed pertaining to Flynn's discussions with Kislyak.% 166
\footnote{Yates 9/15/17 302, at 9;
Burnham 11/3/17 302, at 5.
In accordance with McGahn's request, the Department of Justice made the underlying information available and Eisenberg viewed the information in early February.
Eisenberg 11/29/17 302, at 5;
FBI 2/7/17 Electronic Communication, at 1 (documenting 2/2/17 meeting with Eisenberg).}

Also on January 27, the President called FBI Director Comey and invited him to dinner that evening.% 167
\footnote{Comey 11/15/17 302, at 6;
SCR012b\_000001 (President's Daily Diary, 1/27/17);
\textit{Hearing on Russian Election Interference Before the Senate Select Intelligence Committee}, 115th Cong.\ (June 8, 2017) (Statement for the Record of James B. Comey, former Director of the FBI, at 2-3).}
Priebus recalled that before the dinner, he told the President something like, "don't talk about Russia, whatever you do," and the President promised he would not talk about Russia at the dinner.% 168
\footnote{Priebus 10/13/17 302, at 17.}
McGahn had previously advised the President that he should not communicate directly with the Department of Justice to avoid the perception or reality of political interference in law enforcement.% 169
\footnote{\textit{See} McGahn 11/30/17 302, at 9;
Dhillon 11/21/17 302, at 2;
Bannon 2/12/18 302, at 17.}
When Bannon learned about the President's planned dinner with Comey, he suggested that he or Priebus also attend, but the President stated that he wanted to dine with Comey alone.% 170
\footnote{Bannon 2/12/18 302, at 17.}
Comey said that when he arrived for the dinner that evening, he was surprised and concerned to see that no one else had been invited.% 171
\footnote{\textit{Hearing on Russian Election Interference Before the Senate Select Intelligence Committee}, 115th Cong.\ (June 8, 2017) (Statement for the Record of James B. Comey, former Director of the FBI, at 3);
\textit{see} Comey 11/15/17 302, at 6.}

Comey provided an account of the dinner in a contemporaneous memo, an interview with this Office, and congressional testimony.
According to Comey's account of the dinner, the President repeatedly brought up Comey's future, asking whether he wanted to stay on as FBI director.% 172
\footnote{Comey 11/15/17 302, at 7;
Comey 1/28/17 Memorandum, at 1, 3;
\textit{Hearing on Russian Election Interference Before the Senate Select Intelligence Committee}, 115th Cong.\ (June 8, 2017) (Statement for the Record of James B. Comey, former Director of the FBI, at 3).}
Because the President had previously said he wanted Comey to stay on as FBI director, Comey interpreted the President's comments as an effort to create a patronage relationship by having Comey ask for his job.% 173
\footnote{Comey 11/15/17 302, at 7;
\textit{Hearing on Russian Election Interference Before the Senate Select Intelligence Committee}, 115th Cong.\ (June 8, 2017) (Statement for the Record of James B. Comey, former Director of the FBI, at 3).}
The President also brought up the Steele reporting that Comey had raised in the January 6, 2017 briefing and stated that he was thinking about ordering the FBI to investigate the allegations to prove they were false.% 174
\footnote{Comey 1/28/17 Memorandum, at 3;
\textit{Hearing on Russian Election Interference Before the Senate Select Intelligence Committee}, 115th Cong.\ (June 8, 2017) (Statement for the Record of James B. Comey, former Director of the FBI, at 4).}
Comey responded that the President should think carefully about issuing such an order because it could create a narrative that the FBI was investigating him personally, which was incorrect.% 175
\footnote{\textit{Hearing on Russian Election Interference Before the Senate Select Intelligence Committee}, 115th Cong.\ (June 8, 2017) (Statement for the Record of James B. Comey, former Director of the FBI, at 4).}
Later in the dinner, the President brought up Flynn and said, "the guy has serious judgment issues."% 176
\footnote{Comey 1/28/17 Memorandum, at 4;
Comey 11/15/17 302, at 7.}
Comey did not comment on Flynn and the President did not acknowledge any FBI interest in or contact with Flynn.% 177
\footnote{Comey 1/28/17 Memorandum, at 4;
Comey 11/15/17 302, at 7.}

According to Comey's account, at one point during the dinner the President stated, "I need loyalty, I expect loyalty."% 178
\footnote{Comey 1/28/18 Memorandum, at 2;
Comey 11/15/17 302, at 7;
\textit{Hearing on Russian Election Interference Before the Senate Select Intelligence Committee}, 115th Cong.\ (June 8, 2017) (Statement for the Record of James B. Comey, former Director of the FBI, at 3).}
Comey did not respond and the conversation moved on to other topics, but the President returned to the subject of Comey's job at the end of the dinner and repeated, "I need loyalty."% 179
\footnote{Comey 1/28/17 Memorandum, at 3;
Comey 11/15/17 302, at 7;
\textit{Hearing on Russian Election Interference Before the Senate Select Intelligence Committee}, 115th Cong.\ (June 8, 2017) (Statement for the Record of James B. Comey, former Director of the FBI, at 3 - 4).}
Comey responded, "You will always get honesty from me."% 180
\footnote{Comey 1/28/17 Memorandum, at 3;
Comey 11/15/17 302, at 7;
\textit{Hearing on Russian Election Interference Before the Senate Select Intelligence Committee}, 115th Cong.\ (June 8, 2017) (Statement for the Record of James B. Comey, former Director of the FBI, at 4).}
The President said, "That's what I want, honest loyalty."% 181
\footnote{\textit{Hearing on Russian Election Interference Before the Senate Select Intelligence Committee}, 115th Cong.\ (June 8, 2017) (Statement for the Record of James B. Comey, former Director of the FBI, at 4).}
Comey said, "You will get that from me."% 182
\footnote{\textit{Hearing on Russian Election Interference Before the Senate Select Intelligence Committee}, 115th Cong.\ (June 8, 2017) (Statement for the Record of James B. Comey, former Director of the FBI, at 4).}

After Comey's account of the dinner became public, the President and his advisors disputed that he had asked for Comey's loyalty.% 183
\footnote{See, e.g., Michael S. Schmidt, \textit{In a Private Dinner, Trump Demanded Loyalty. Comey Demurred.}, New York Times (May 11, 2017) (quoting Sarah Sanders as saying, "[The President] would never even suggest the expectation of personal loyalty");
Ali Vitali, \textit{Trump Never Asked for Comey's Loyalty, President's Personal Lawyer Says}, NBC (June 8, 2017) (quoting the President's personal counsel as saying, "The president also never told Mr.~Comey, 'I need loyalty, I expect loyalty,' in form or substance.");
Remarks by President Trump in Press Conference, White House (June 9, 2017) ("I hardly know the man.
I'm not going to say 'I want you to pledge allegiance.'
Who would do that?
Who would ask a man to pledge allegiance under oath?").
In a private conversation with Spicer, the President stated that he had never asked for Comey's loyalty, but added that if he had asked for loyalty, "Who cares?" Spicer 10/16/17 302, at 4.
The President also told McGahn that he never said what Comey said he had. McGahn 12/12/17 302, at 17.}
The President also indicated that he had not invited Comey to dinner, telling a reporter that he thought Comey had "asked for the dinner" because "he wanted to stay on."% 184
\footnote{\textit{Interview of Donald J. Trump}, NBC (May 11, 2017).}
But substantial evidence corroborates Comey's account of the dinner invitation and the request for loyalty.
The President's Daily Diary confirms that the President "extend[ed] a dinner invitation" to Comey on January 27.% 185
\footnote{SCRO12b\_000001 (President's Daily Diary, 1/27/17) (reflecting that the President called Comey in the morning on January 27 and "[t]he purpose of the call was to extend a dinner invitation").
In addition, two witnesses corroborate Comey's account that the President reached out to schedule the dinner, without Comey having asked for it.
Priebus 10/13/17 302, at 17 (the President asked to schedule the January 27 dinner because he did not know much about Comey and intended to ask him whether he wanted to stay on as FBI Director);
Rybicki 11/21/18 302, at 3 (recalling that Comey told him about the President's dinner invitation on the day of the dinner).}
With respect to the substance of the dinner conversation, Comey documented the President's request for loyalty in a memorandum he began drafting the night of the dinner;% 186
\footnote{Comey 11/15/17 302, at 8;
\textit{Hearing on Russian Election Interference Before the Senate Select Intelligence Committee}, 115th Cong.\ (June 8, 2017) (Statement for the Record of James B. Comey, former Director of the FBI, at 4).}
senior FBI officials recall that Comey told them about the loyalty request shortly after the dinner occurred;% 187
\footnote{McCabe 8/17/17 302, at 9-10;
Rybicki 11/21/18 302, at 3.
After leaving the White House, Comey called Deputy Director of the FBI Andrew McCabe, summarized what he and the President had discussed, including the President's request for loyalty, and expressed shock over the President's request.
McCabe 8/17/17 302, at 9.
Comey also convened a meeting with his senior leadership team to discuss what the President had asked of him during the dinner and whether he had handled the request for loyalty properly.
McCabe 8/17/17 302, at 10;
Rybicki 11/21/18 302, at 3.
In addition, Comey distributed his memorandum documenting the dinner to his senior leadership team, and McCabe confirmed that the memorandum captured what Comey said on the telephone call immediately following the dinner.
McCabe 8/17/17 302, at 9-10.}
and Comey described the request while under oath in congressional proceedings and in a subsequent interview with investigators subject to penalties for lying under 18 U.S.C. \S 1001.
Comey's memory of the details of the dinner, including that the President requested loyalty, has remained consistent throughout.% 188
\footnote{There also is evidence that corroborates other aspects of the memoranda Comey wrote documenting his interactions with the President.
For example, Comey recalled, and his memoranda reflect, that he told the President in his January 6, 2017 meeting, and on phone calls on March 30 and April 11, 2017, that the FBI was not investigating the President personally.
On May 8, 2017, during White House discussions about firing Comey, the President told Rosenstein and others that Comey had told him three times that he was not under investigation, including once in person and twice on the phone.
Gauhar-000058 (Gauhar 5/16/17 Notes).}

\subsubsection{Flynn's Resignation}

On February 2, 2017, Eisenberg reviewed the underlying information relating to Flynn's calls with Kislyak.% 189
\footnote{Eisenberg 11/29/17 302, at 5;
FBI 2/7/17 Electronic Communication, at 1 (documenting 2/2/17 meeting with Eisenberg).}
Eisenberg recalled that he prepared a memorandum about criminal statutes that could apply to Flynn's conduct, but he did not believe the White House had enough information to make a definitive recommendation to the President.% 190
\footnote{Eisenberg 11/29/17 302, at 6.}
Eisenberg and McGahn discussed that Eisenberg's review of the underlying information confirmed his preliminary conclusion that Flynn was unlikely to be prosecuted for violating the Logan Act.% 191
\footnote{Risenberg 11/29/17 302, at 9;
SCR015\_000200 (2/15/17 Draft Memorandum to file from the Office of the Counsel to the President).}
Because White House officials were uncertain what Flynn had told the FBI, however, they could not assess his exposure to prosecution for violating 18 U.S.C. \S 1001.% 192
\footnote{Eisenberg 11/29/17 302, at 9.}

The week of February 6, Flynn had a one-on-one conversation with the President in the Oval Office about the negative media coverage of his contacts with Kislyak.% 193
\footnote{Flynn 11/21/17 302, at 2.}
Flynn recalled that the President was upset and asked him for information on the conversations.% 194
\footnote{Flynn 11/21/17 302, at 2.}
Flynn listed the specific dates on which he remembered speaking with Kislyak, but the President corrected one of the dates he listed.% 195
\footnote{Flynn 11/21/17 302, at 2.}
The President asked Flynn what he and Kislyak discussed and Flynn responded that he might have talked about sanctions.% 196
\footnote{Flynn 11/21/17 302, at 2-3.}

On February 9, 2017, the Washington Post reported that Flynn discussed sanctions with Kislyak the month before the President took office.% 197
\footnote{Greg Miller et al., \textit{National security adviser Flynn discussed sanctions with Russian ambassador, despite denials, officials say}, Washington Post (Feb.~9, 2017).}
After the publication of that story, Vice President Pence learned of the Department of Justice's notification to the White House about the content of Flynn's calls.% 198
\footnote{SCR015\_000202 (2/15/17 Draft Memorandum to file from the Office of the Counsel to the President); McGahn 11/30/17 302, at 12.}
He and other advisors then sought access to and reviewed the underlying information about Flynn's contacts with Kislyak.% 199
\footnote{SCR015\_000202 (2/15/17 Draft Memorandum to file from the Office of the Counsel to the President);
McCabe 8/17/17 302, at 11-13;
Priebus 10/13/17 302, at 10;
McGahn 11/30/17 302, at 12.}
FBI Deputy Director Andrew McCabe, who provided the White House officials access to the information and was present when they reviewed it, recalled the officials asking him whether Flynn's conduct violated the Logan Act.% 200
\footnote{McCabe 8/17/17 302, at 13.}
McCabe responded that he did not know, but the FBI was investigating the matter because it was a possibility.% 201
\footnote{McCabe 8/17/17 302, at 13.}
Based on the evidence of Flynn's contacts with Kislyak, McGahn and Priebus concluded that Flynn could not have forgotten the details of the discussions of sanctions and had instead been lying about what he discussed with Kislyak.% 202
\footnote{McGahn 11/30/17 302, at 12;
Priebus 1/18/18 302, at 8;
Priebus 10/13/17 302, at 10;
SCR015\_000202 (2/15/17 Draft Memorandum to file from the Office of the Counsel to the President).}
Flynn had also told White House officials that the FBI had told him that the FBI was closing out its investigation of him,% 203
\footnote{McGahn 11/30/17 302, at 11;
Eisenberg 11/29/17 302, at 9;
Priebus 10/13/17 302, at 11.}
but Eisenberg did not believe him.% 204
\footnote{Eisenberg 11/29/17 302, at 9.}
After reviewing the materials and speaking with Flynn, McGahn and Priebus concluded that Flynn should be terminated and recommended that course of action to the President.% 205
\footnote{SCR015\_000202 (2/15/17 Draft Memorandum to file from the Office of the Counsel to the President);
Priebus 10/13/17 302, at 10;
McGahn 11/30/17 302, at 12.}

That weekend, Flynn accompanied the President to Mar-a-Lago.% 206
\footnote{Flynn 11/17/17 302, at 8.}
Flynn recalled that on February 12, 2017, on the return flight to D.C. on Air Force One, the President asked him whether he had lied to the Vice President.% 207
\footnote{Flynn 1/19/18 302, at 9;
Flynn 11/17/17 302, at 8.}
Flynn responded that he may have forgotten details of his calls, but he did not think he lied.% 208
\footnote{Flynn 11/17/17 302, at 8;
Flynn 1/19/18 302, at 9.}
The President responded, "Okay.
That's fine.
I got it."% 209
\footnote{Flynn 1/19/18 302, at 9.}

On February 13, 2017, Priebus told Flynn he had to resign.% 210
\footnote{Priebus 1/18/18 302, at 9.}
Flynn said he wanted to say goodbye to the President, so Priebus brought him to the Oval Office.% 211
\footnote{Priebus 1/18/18 302, at 9;
Flynn 11/17/17 302, at 10.}
Priebus recalled that the President hugged Flynn, shook his hand, and said, "We'll give you a good recommendation.
You're a good guy.
We'll take care of you."% 212
\footnote{Priebus 1/18/18 302, at 9;
Flynn 11/17/17 302, at 10.}

Talking points on the resignation prepared by the White House Counsel's Office and distributed to the White House communications team stated that McGahn had advised the President that Flynn was unlikely to be prosecuted, and the President had determined that the issue with Flynn was one of trust.% 213
\footnote{SCRO04\_00600 (2/16/17 Email, Burnham to Donaldson).}
Spicer told the press the next day that Flynn was forced to resign "not based on legal issue, but based on a trust issue, [where] a level of trust between the President and General Flynn had eroded to the point where [the President] felt he had to make a change."% 214
\footnote{Sean Spicer, \textit{White House Daily Briefing}, C-SPAN (Feb.~14, 2017).
After Flynn pleaded guilty to violating 18U.S.C. \S 1001 in December 2017, the President tweeted, "I had to fire General Flynn because he lied to the Vice President and the FBI."
\@realDonaldTrump 12/2/17 (12:14 p.m.~ET) Tweet.
The next day, the President's personal counsel told the press that he had drafted the tweet.
Maegan Vazquezetal., \textit{Trump's lawyer says he was behind President's tweet about firing Flynn}, CNN (Dec.~3, 2017).}

\subsubsection{The President Discusses Flynn with FBI Director Comey}

On February 14, 2017, the day after Flynn's resignation, the President had lunch at the White House with New Jersey Governor Chris Christie.% 215
\footnote{Christie 2/13/19 302, at 2-3;
SCRO12b\_000022(Presidents Daily Diary, 2/14/17).}
According to Christie, at one point during the lunch the President said, "Now that we fired Flynn, the Russia thing is over."% 216
\footnote{Christie 2/13/19 302, at 3.}
Christie laughed and responded, "No way."% 217
\footnote{Christie 2/13/19 302, at 3.}
He said, "this Russia thing is far from over" and "[w]e'll be here on Valentine's Day 2018 talking about this."% 218
\footnote{Christie 2/13/19 302, at 3.
Christie said he thought when the President said "the Russia thing" he was referring to not just the investigations but also press coverage about Russia.
Christie thought the more important thing was that there was an investigation.
Christie 2/13/19 302, at 4.}
The President said, "[w]hat do you mean?
Flynn met with the Russians.
That was the problem.
I fired Flynn.
It's over."% 219
\footnote{Christie 2/13/19 302, at 3.}
Christie recalled responding that based on his experience both as a prosecutor and as someone who had been investigated, firing Flynn would not end the investigation.% 220
\footnote{Christie 2/13/19 302, at 3.}
Christie said there was no way to make an investigation shorter, but a lot of ways to make it longer.% 221
\footnote{Christie 2/13/19 302, at 3.}
The President asked Christie what he meant, and Christie told the President not to talk about the investigation even if he was frustrated at times.% 222
\footnote{Christie 2/13/19 302, at 3 - 4.}
Christie also told the President that he would never be able to get rid of Flynn, "like gum on the bottom of your shoe."% 223
\footnote{Christie 2/13/19 302, at 3.
Christie also recalled that during the lunch, Flynn called Kushner, who was at the lunch, and complained about what Spicer had said about Flynn in his press briefing that day.
Kushner told Flynn words to the effect of, "You know the President respects you.
The President cares about you.
I'll get the President to send out a positive tweet about you later."
Kushner looked at the President when he mentioned the tweet, and the President nodded his assent.
Christie 2/13/19 302, at 3.
Flynn recalled getting upset at Spicer's comments in the press conference and calling Kushner to say he did not appreciate the comments.
Flynn 1/19/18 302, at 9.}

Towards the end of the lunch, the President brought up Comey and asked if Christie was still friendly with him.% 224
\footnote{Christie 2/13/19 302, at 4.}
Christie said he was.% 225
\footnote{Christie 2/13/19 302, at 4.}
The President told Christie to call Comey and tell him that the President "really like[s] him.
Tell him he's part of the team."% 226
\footnote{Christie 2/13/19 302, at 4 - 5.}
At the end of the lunch, the President repeated his request that Christie reach out to Comey.% 227
\footnote{Christie 2/13/19 302, at 5.}
Christie had no intention of complying with the President's request that he contact Comey.% 228
\footnote{Christie 2/13/19 302, at 5.}
He thought the President's request was "nonsensical" and Christie did not want to put Comey in the position of having to receive such a phone call.% 229
\footnote{Christie 2/13/19 302, at 5.}
Christie thought it would have been uncomfortable to pass on that message.% 230
\footnote{Christie 2/13/19 302, at 5.}

At 4 p.m.\ that afternoon, the President met with Comey, Sessions, and other officials for a homeland security briefing.% 231
\footnote{SCRO12b\_000022 (President's Daily Diary, 2/14/17);
Comey 11/15/17 302, at 9.}
At the end of the briefing, the President dismissed the other attendees and stated that he wanted to speak to Comey alone.% 232
\footnote{Comey 11/15/17 302, at 10;
2/14/17 Comey Memorandum, at 1;
\textit{Hearing on Russian Election Interference Before the Senate Select Intelligence Committee}, 115th Cong.\ (June 8, 2017) (Statement for the Record of James B. Comey, former Director of the FBI, at 4);
Priebus 10/13/17 302, at 18 (confirming that everyone was shooed out "like Comey said" in his June testimony).}
Sessions and senior advisor to the President Jared Kushner remained in the Oval Office as other participants left, but the President excused them, repeating that he wanted to speak only with Comey.% 233
\footnote{Comey 11/15/17 302, at 10;
Comey 2/14/17 Memorandum, at 1;
\textit{Hearing on Russian Election Interference Before the Senate Select Intelligence Committee}, 115th Cong.\ (June 8, 2017) (Statement for the Record of James B. Comey, former Director of the FBI, at 4).
Sessions recalled that the President asked to speak to Comey alone and that Sessions was one of the last to leave the room;
he described Comey's testimony about the events leading up to the private meeting with the President as "pretty accurate."
Sessions 1/17/18 302, at 6.
Kushner had no recollection of whether the President asked Comey to stay behind.
Kushner 4/11/18 302, at 24.}
At some point after others had left the Oval Office, Priebus opened the door, but the President sent him away.% 234
\footnote{Comey 2/14/17 Memorandum, at 2;
Priebus 10/13/17 302, at 18.}

According to Comey's account of the meeting, once they were alone, the President began the conversation by saying, "I want to talk about Mike Flynn."% 235
\footnote{Comey 11/15/17 302, at 10;
Comey 2/14/17 Memorandum, at 1;
\textit{Hearing on Russian Election Interference Before the Senate Select Intelligence Committee}, 115th Cong.\ (June 8, 2017) (Statement for the Record of James B. Comey, former Director of the FBI, at 4).}
The President stated that Flynn had not done anything wrong in speaking with the Russians, but had to be terminated because he had misled the Vice President.% 236
\footnote{Comey 2/14/17 Memorandum, at 1;
\textit{Hearing on Russian Election Interference Before the Senate Select Intelligence Committee}, 115th Cong.\ (June 8, 2017) (Statement for the Record of James B. Comey, former Director of the FBI, at 5).}
The conversation turned to the topic of leaks of classified information, but the President returned to Flynn, saying "the is a good guy and has been through a lot."% 237
\footnote{Comey 11/15/17 302, at 10;
Comey 2/14/17 Memorandum, at 2;
\textit{Hearing on Russian Election Interference Before the Senate Select Intelligence Committee}, 115th Cong.\ (June 8, 2017) (Statement for the Record of James B. Comey, former Director of the FBI, at 5).}
The President stated, "I hope you can see your way clear to letting this go, to letting Flynn go.
He is a good guy.
I hope you can let this go."% 238
\footnote{\textit{Hearing on Russian Election Interference Before the Senate Select Intelligence Committee}, 115th Cong.\ (June 8, 2017) (Statement for the Record of James B. Comey, former Director of the FBI, at 5);
Comey 2/14/17 Memorandum, at 2.
Comey said he was highly confident that the words in quotations in his Memorandum documenting this meeting were the exact words used by the President.
He said he knew from the outset of the meeting that he was about to have a conversation of consequence, and he remembered the words used by the President and wrote them down soon after the meeting.
Comey 11/15/17 302, at 10-11.}
Comey agreed that Flynn "is a good guy," but did not commit to ending the investigation of Flynn.% 239
\footnote{Comey 11/15/17 302, at 10;
Comey 2/14/17 Memorandum, at 2.}
Comey testified under oath that he took the President's statement "as a direction" because of the President's position and the circumstances of the one-on-one meeting.% 240
\footnote{\textit{Hearing on Russian Election Interference Before the Senate Select Intelligence Committee}, 115th Cong.\ (June 8, 2017) (CQ Cong.\ Transcripts, at 31) (testimony of James B. Comey, former Director of the FBI).
Comey further stated, "I mean, this is the president of the United States, with me alone, saying, 'I hope' this.
I took it as, this is what he wants me to do." \textit{Id.};
\textit{see also} Comey 11/15/17 302, at 10 (Comey took the statement as an order to shut down the Flynn investigation).}

Shortly after meeting with the President, Comey began drafting a memorandum documenting their conversation.% 241
\footnote{Comey 11/15/17 302, at 11;
\textit{Hearing on Russian Election Interference Before the Senate Select Intelligence Committee}, 115th Cong.\ (June 8, 2017) (Statement for the record of James B. Comey, former Director of the FBI, at 5).}
Comey also met with his senior leadership team to discuss the President's request, and they agreed not to inform FBI officials working on the Flynn case of the President's statements so the officials would not be influenced by the request.% 242
\footnote{Comey 11/15/17 302, at 11;
Rybicki 6/9/17 302, at 4;
Rybicki 6/22/17 302, at 1;
\textit{Hearing on Russian Election Interference Before the Senate Select Intelligence Committee}, 115th Cong.\ (June 8, 2017) (Statement for the record of James B. Comey, former Director of the FBI, at 5-6).}
Comey also asked for a meeting with Sessions and requested that Sessions not leave Comey alone with the President again.% 243
\footnote{Comey 11/15/17 302, at 11;
Rybicki 6/9/17 302, at 4-5;
Rybicki 6/22/17 302, at 1-2;
Sessions 1/17/18 302, at 6 (confirming that later in the week following Comey's one-on-one meeting with the President in the Oval Office, Comey told the Attorney General that he did not want to be alone with the President);
Hunt 2/1/18 302, at 6 (within days of the February 14 Oval Office meeting, Comey told Sessions he did not think it was appropriate for the FBI Director to meet alone with the President);
Rybicki 11/21/18 302, at 4 (Rybicki helped to schedule the meeting with Sessions because Comey wanted to talk about his concerns about meeting with the President alone);
\textit{Hearing on Russian Election Interference Before the Senate Select Intelligence Committee}, 115th Cong.\ (June 8, 2017) (Statement for the record of James B. Comey, former Director of the FBI, at 6).}

\subsubsection{The Media Raises Questions About the President's Delay in Terminating Flynn}

After Flynn was forced to resign, the press raised questions about why the President waited more than two weeks after the DOJ notification to remove Flynn and whether the President had known about Flynn's contacts with Kislyak before the DOJ notification.% 244
\footnote{\textit{See, e.g.}, Sean Spicer, \textit{White House Daily Briefing}, C-SPAN (Feb.~14, 2017) (questions from the press included, "if [the President] was notified 17 days ago that Flynn had misled the Vice President, other officials here, and that he was a potential threat to blackmail by the Russians, why would he be kept on for almost three weeks?" and "Did the President instruct [Flynn] to talk about sanctions with the [Russian ambassador]?").
Priebus recalled that the President initially equivocated on whether to fire Flynn because it would generate negative press to lose his National Security Advisor so early in his term.
Priebus 1/18/18 302, at 8.}
The press also continued to raise questions about connections between Russia and the President's campaign.% 245
\footnote{\textit{Eg.}, Sean Sullivan et al., \textit{Senators from both parties pledge to deepen probe of Russia and the 2016 election}, Washington Post (Feb.~14, 2017);
Aaron Blake, \textit{5 times Donald Trump's team denied contact with Russia}, Washington Post (Feb.~15, 2017);
Oren Dorell, \textit{Donald Trump's ties to Russia go back 30 years}, USA Today (Feb.~15, 2017);
Pamela Brown et al., \textit{Trump aides were in constant touch with senior Russian officials during campaign}, CNN (Feb.~15, 2017);
Austin Wright, \textit{Comey briefs senators amid furor over Trump-Russia ties}, Politico (Feb.~17, 2017);
Megan Twohey \& Scott Shane, \textit{A Back-Channel Plan for Ukraine and Russia, Courtesy of Trump Associates}, New York Times (Feb.~19, 2017).}
On February 15, 2017, the President told reporters, "General Flynn is a wonderful man.
I think he's been treated very, very unfairly by the media."% 246
\footnote{Remarks by President Trump and Prime Minister Netanyahu of Israel in Joint Press Conference, White House (Feb.~15, 2017).}
On February 16, 2017, the President held a press conference and said that he removed Flynn because Flynn "didn't tell the Vice President of the United States the facts, and then he didn't remember.
And that just wasn't acceptable to me."% 247
\footnote{Remarks by President Trump in Press Conference, White House (Feb.~16, 2017).}
The President said he did not direct Flynn to discuss sanctions with Kislyak, but "it certainly would have been okay with me if he did.
I would have directed him to do it if I thought he wasn't doing it.
I didn't direct him, but I would have directed him because that's his job."% 248
\footnote{Remarks by President Trump in Press Conference, White House (Feb.~16, 2017).
The President also said that Flynn's conduct "wasn't wrong - what he did in terms of the information he saw."
The President said that Flynn was just "doing the job," and "if anything, he did something right."}
In listing the reasons for terminating Flynn, the President did not say that Flynn had lied to him.% 249
\footnote{Remarks by President Trump in Press Conference, White House (Feb.~16, 2017);
Priebus 1/18/18 302, at 9.}
The President also denied having any connection to Russia, stating, "I have nothing to do with Russia.
I told you, I have no deals there.
I have no anything."% 250
\footnote{Remarks by President Trump in Press Conference, White House (Feb.~16, 2017).}
The President also said he "had nothing to do with" WikiLeaks's publication of information hacked from the Clinton campaign.% 251
\footnote{Remarks by President Trump in Press Conference, White House (Feb.~16, 2017).}

\subsubsection{The President Attempts to Have K.T. McFarland Create a Witness Statement Denying that he Directed Flynn's Discussions with Kislyak}

On February 22, 2017, Priebus and Bannon told McFarland that the President wanted her to resign as Deputy National Security Advisor, but they suggested to her that the Administration could make her the ambassador to Singapore.% 252
\footnote{KTMF\_00000047 (McFarland 2/26/17 Memorandum for the Record);
McFarland 12/22/17 302, at 16-17.}
The next day, the President asked Priebus to have McFarland draft an internal email that would confirm that the President did not direct Flynn to call the Russian Ambassador about sanctions.% 253
\footnote{\textit{See} Priebus 1/18/18 302, at 11;
\textit{see also} KTMF\_00000048 (McFarland 2/26/17 Memorandum for the Record);
McFarland 12/22/17 302, at 17.}
Priebus said he told the President he would only direct McFarland to write such letter if she were comfortable with it.% 254
\footnote{Priebus 1/18/18 302, at 11.}
Priebus called McFarland into his office to convey the President's request that she memorialize in writing that the President did not direct Flynn to talk to Kislyak.% 255
\footnote{KTMF\_00000048 (McFarland 2/26/17 Memorandum for the Record);
McFarland 12/22/17 302, at 17.}
McFarland told Priebus she did not know whether the President had directed Flynn to talk to Kislyak about sanctions, and she declined to say yes or no to the request.% 256
\footnote{KTMF\_00000047 (McFarland 2/26/17 Memorandum for the Record) ("I said I did not know whether he did or didn't, but was in Mar-a-Lago the week between Christmas and New Year's (while Flynn was on vacation in Caribbean) and I was not aware of any Flynn-Trump, or Trump-Russian phone calls");
McFarland 12/22/17 302, at 17.}
Priebus understood that McFarland was not comfortable with the President's request, and he recommended that she talk to attorneys in the White House Counsel's Office.% 257
\footnote{Priebus 1/18/18 302, at 11.}

McFarland then reached out to Eisenberg.% 258
\footnote{McFarland 12/22/17 302, at 17.}
McFarland told him that she had been fired from her job as Deputy National Security Advisor and offered the ambassadorship in Singapore but that the President and Priebus wanted letter from her denying that the President directed Flynn to discuss sanctions with Kislyak.% 259
\footnote{McFarland 12/22/17 302, at 17.}
Eisenberg advised McFarland not to write the requested letter.% 260
\footnote{KTMF\_00000048 (McFarland 2/26/17 Memorandum for the Record);
McFarland 12/22/17 302, at 17.}
As documented by McFarland in a contemporaneous "Memorandum for the Record" that she wrote because she was concerned by the President's request:
"Eisenberg ... thought the requested email and letter would be a bad idea - from my side because the email would be awkward.
Why would I be emailing Priebus to make a statement for the record?
But it would also be a bad idea for the President because it looked as if my ambassadorial appointment was in some way a quid pro quo."% 261
\footnote{KTMF\_00000048 (McFarland 2/26/17 Memorandum for the Record);
\textit{see} McFarland 12/22/17 302, at 17.}
Later that evening, Priebus stopped by McFarland's office and told her not to write the email and to forget he even mentioned it.% 262
\footnote{McFarland 12/22/17 302, at 17;
KTMF\_00000048 (McFarland 2/26/17 Memorandum for the Record).}

Around the same time, the President asked Priebus to reach out to Flynn and let him know that the President still cared about him.% 263
\footnote{Priebus 1/18/18 302, at 9.}
Priebus called Flynn and said that he was checking in and that Flynn was an American hero.% 264
\footnote{Priebus 1/18/18 302, at 9;
Flynn 1/19/18 302, at 9.}
Priebus thought the President did not want Flynn saying bad things about him.% 265
\footnote{Priebus 1/18/18 302, at 9-10.}

On March 31, 2017, following news that Flynn had offered to testify before the FBI and congressional investigators in exchange for immunity, the President tweeted,
"Mike Flynn should ask for immunity in that this is a witch hunt (excuse for big election loss), by media \& Dems, of historic proportion!"% 266
\footnote{\@realDonaldTrump 3/31/17 (7:04 a.m.~ET) Tweet;
\textit{see} Shane Harris at al., \textit{Mike Flynn Offers to Testify in Exchange for Immunity}, Wall Street Journal (Mar.~30, 2017).}
Tn late March or early April, the President asked McFarland to pass a message to Flynn telling him the President felt bad for him and that he should stay strong.% 267
\footnote{McFarland 12/22/17 302, at 18.}

\begin{center}
\textbf{Analysis}
\end{center}

In analyzing the President's conduct related to the Flynn investigation, the following evidence is relevant to the elements of obstruction of justice:

\underline{Obstructive act.}
According to Comey's account of his February 14, 2017 meeting in the Oval Office, the President told him, "I hope you can see your way clear to letting this go, to letting Flynn go. ... I hope you can let this go."
In analyzing whether these statements constitute an obstructive act, a threshold question is whether Comey's account of the interaction is accurate, and, if so, whether the President's statements had the tendency to impede the administration of
justice by shutting down an inquiry that could result in a grand jury investigation and a criminal charge.

After Comey's account of the President's request to "let[] Flynn go" became public, the President publicly disputed several aspects of the story.
The President told the New York Times that he did not "shoo other people out of the room" when he talked to Comey and that he did not remember having a one-on-one conversation with Comey.% 268
\footnote{\textit{Excerpts From The Times's Interview With Trump}, New York Times (July 19, 2017).
Hicks recalled that the President told her he had never asked Comey to stay behind in his office.
Hicks 12/8/17 302, at 12.}
The President also publicly denied that he had asked Comey to "let[] Flynn go" or otherwise communicated that Comey should drop the investigation of Flynn.% 269
\footnote{In a statement on May 16, 2017, the White House said:
"While the President has repeatedly expressed his view that General Flynn is a decent man who served and protected our country, the President has never asked Mr.~Comey or anyone else to end any investigation, including any investigation involving General Flynn....
This is not a truthful or accurate portrayal of the conversation between the President and Mr.~Comey."
\textit{See} Michael S. Schmidt, \textit{Comey Memorandum Says Trump Asked Him to End Flynn Investigation}, New York Times (May 16, 2017) (quoting White House statement);
\@realDonaldTrump 12/3/17 (6:15 a.m.~ET) Tweet ("I never asked Comey to stop investigating Flynn.
Just more Fake News covering another Comey lie!").}
In private, the President denied aspects of Comey's account to White House advisors, but acknowledged to Priebus that he brought Flynn up in the meeting with Comey and stated that Flynn was a good guy.% 270
\footnote{Priebus recalled that the President acknowledged telling Comey that Flynn was a good guy and he hoped "everything worked out for him."
Priebus 10/13/17 302, at 19.
McGahn recalled that the President denied saying to Comey that he hoped Comey would let Flynn go, but added that he was "allowed to hope."
The President told McGahn he did not think he had crossed any lines.
McGahn 12/14/17 302, at 8.}
Despite those denials, substantial evidence corroborates Comey's account.

First, Comey wrote a detailed memorandum of his encounter with the President on the same day it occurred.
Comey also told senior FBI officials about the meeting with the President that day, and their recollections of what Comey told them at the time are consistent with Comey's account.% 271
\footnote{Rybicki 11/21/18 302, at 4;
McCabe 8/17/17 302, at 13-14.}

Second, Comey provided testimony about the President's request that he "let[] Flynn go" under oath in congressional proceedings and in interviews with federal investigators subject to penalties for lying under 18 U.S.C. \S 1001.
Comey's recollections of the encounter have remained consistent over time.

Third, the objective, corroborated circumstances of how the one-on-one meeting came to occur support Comey's description of the event.
Comey recalled that the President cleared the room to speak with Comey alone after a homeland security briefing in the Oval Office, that Kushner and Sessions lingered and had to be shooed out by the President, and that Priebus briefly opened the door during the meeting, prompting the President to wave him away.
While the President has publicly denied those details, other Administration officials who were present have confirmed Comey's account of how he ended up in a one-on-one meeting with the President.% 272
\footnote{\textit{See} Priebus 10/13/17 302, at 18;
Sessions 1/17/18 302, at 6.}
And the President acknowledged to Priebus and McGahn that he in fact spoke to Comey about Flynn in their one-on-one meeting.

Fourth, the President's decision to clear the room and, in particular, to exclude the Attorney General from the meeting signals that the President wanted to be alone with Comey, which is consistent with the delivery of a message of the type that Comey recalls, rather than a more innocuous conversation that could have occurred in the presence of the Attorney General.

Finally, Comey's reaction to the President's statements is consistent with the President having asked him to "let[] Flynn go."
Comey met with the FBI leadership team, which agreed to keep the President's statements closely held and not to inform the team working on the Flynn investigation so that they would not be influenced by the President's request.
Comey also promptly met with the Attorney General to ask him not to be left alone with the President again, an account verified by Sessions, FBI Chief of Staff James Rybicki, and Jody Hunt, who was then the Attorney General's chief of staff.

A second question is whether the President's statements, which were not phrased as a direct order to Comey, could impede or interfere with the FBI's investigation of Flynn.
While the President said he "hope[d]" Comey could "let[] Flynn go," rather than affirmatively directing him to do so, the circumstances of the conversation show that the President was asking Comey to close the FBI's investigation into Flynn.
First, the President arranged the meeting with Comey sot hat they would be alone and purposely excluded the Attorney General, which suggests that the President meant to make a request to Comey that he did not want anyone else to hear.
Second, because the President is the head of the Executive Branch, when he says that he "hopes" a subordinate will do something, it is reasonable to expect that the subordinate will do what the President wants.
Indeed, the President repeated a version of "let this go" three times, and Comey testified that he understood the President's statements as a directive, which is corroborated by the way Comey reacted at the time.

\underline{Nexus to a proceeding.}
To establish a nexus to a proceeding, it would be necessary to show that the President could reasonably foresee and actually contemplated that the investigation of Flynn was likely to lead to a grand jury investigation or prosecution.

At the time of the President's one-on-one meeting with Comey, no grand jury subpoenas had been issued as part of the FBI's investigation into Flynn.
But Flynn's lies to the FBI violated federal criminal law, \blackout{Harm to Ongoing Matter}, and resulted in Flynn's prosecution for violating 18 U.S.C. \S 1001.
By the time the President spoke to Comey about Flynn, DOJ officials had informed McGahn, who informed the President, that Flynn's statements to senior White House officials about his contacts with Kislyak were not true and that Flynn had told the same version of events to the FBI.
McGahn also informed the President that Flynn's conduct could violate 18 U.S.C. \S 1001.
After the Vice President and senior White House officials reviewed the underlying information about Flynn's calls on February 10, 2017, they believed that Flynn could not have forgotten his conversations with Kislyak and concluded that he had been lying.
In addition, the President's instruction to the FBI Director to "let[] Flynn go" suggests his awareness that Flynn could face criminal exposure for his conduct and was at risk of prosecution.

\underline{Intent.}
As part of our investigation, we examined whether the President had a personal stake in the outcome of an investigation into Flynn - for example, whether the President was aware of Flynn's communications with Kislyak close in time to when they occurred, such that the President knew that Flynn had lied to senior White House officials and that those lies had been passed on to the public.
Some evidence suggests that the President knew about the existence and content of Flynn's calls when they occurred, but the evidence is inconclusive and could not be relied upon to establish the President's knowledge.
In advance of Flynn's initial call with Kislyak, the President attended a meeting where the sanctions were discussed and an advisor may have mentioned that Flynn was scheduled to talk to Kislyak.
Flynn told McFarland about the substance of his calls with Kislyak and said they may have made a difference in Russia's response, and Flynn recalled talking to Bannon in early January 2017 about how they had successfully "stopped the train on Russia's response" to the sanctions.
It would have been reasonable for Flynn to have wanted the President to know of his communications with Kislyak because Kislyak told Flynn his request had been received at the highest levels in Russia and that Russia had chosen not to retaliate in response to the request, and the President was pleased by the Russian response, calling it a "[g]reat move."
And the President never said publicly or internally that Flynn had lied to him about the calls with Kislyak.

But McFarland did not recall providing the President-Elect with Flynn's read-out of his calls with Kislyak, and Flynn does not have a specific recollection of telling the President-Elect directly about the calls.
Bannon also said he did not recall hearing about the calls from Flynn.
And in February 2017, the President asked Flynn what was discussed on the calls and whether he had lied to the Vice President, suggesting that he did not already know.
Our investigation accordingly did not produce evidence that established that the President knew about Flynn's discussions of sanctions before the Department of Justice notified the White House of those discussions in late January 2017.
The evidence also does not establish that Flynn otherwise possessed information damaging to the President that would give the President a personal incentive to end the FBI's inquiry into Flynn's conduct.

Evidence does establish that the President connected the Flynn investigation to the FBI's broader Russia investigation and that he believed, as he told Christie, that terminating Flynn would end "the whole Russia thing."
Flynn's firing occurred at a time when the media and Congress were raising questions about Russia's interference in the election and whether members of the President's campaign had colluded with Russia.
Multiple witnesses recalled that the President viewed the Russia investigations as a challenge to the legitimacy of his election.
The President paid careful attention to negative coverage of Flynn and reacted with annoyance and anger when the story broke disclosing that Flynn had discussed sanctions with Kislyak.
Just hours before meeting one-on-one with Comey, the President told Christie that firing Flynn would put an end to the Russia inquiries.
And after Christie pushed back, telling the President that firing Flynn would not end the Russia investigation, the President asked Christie to reach out to Comey and convey that the President liked him and he was part of "the team."
That afternoon, the President cleared the room and asked Comey to "let[] Flynn go."

We also sought evidence relevant to assessing whether the President's direction to Comey was motivated by sympathy towards Flynn.
In public statements the President repeatedly described Flynn as a good person who had been harmed by the Russia investigation, and the President directed advisors to reach out to Flynn to tell him the President "care[d]" about him and felt bad for him.
At the same time, multiple senior advisors, including Bannon, Priebus, and Hicks, said that the President had become unhappy with Flynn well before Flynn was forced to resign and that the President was frequently irritated with Flynn.
Priebus said he believed the President's initial reluctance to fire Flynn stemmed not from personal regard, but from concern about the negative press that would be generated by firing the National Security Advisor so early in the Administration.
And Priebus indicated that the President's post-firing expressions of support for Flynn were motivated by the President's desire to keep Flynn from saying negative things about him.

The way in which the President communicated the request to Comey also is relevant to understanding the President's intent.
When the President first learned about the FBI investigation into Flynn, he told McGahn, Bannon, and Priebus not to discuss the matter with anyone else in the White House.
The next day, the President invited Comey for a one-on-one dinner against the advice of an aide who recommended that other White House officials also attend.
At the dinner, the President asked Comey for "loyalty" and, at a different point in the conversation, mentioned that Flynn had judgment issues.
When the President met with Comey the day after Flynn's termination - shortly after being told by Christie that firing Flynn would not end the Russia investigation - the President cleared the room, even excluding the Attorney General, so that he could again speak to Comey alone.
The President's decision to meet one-on-one with Comey contravened the advice of the White House Counsel that the President should not communicate directly with the Department of Justice to avoid any appearance of interfering in law enforcement activities.
And the President later denied that he cleared the room and asked Comey to "let[] Flynn go" - a denial that would have been unnecessary if he believed his request was a proper exercise of prosecutorial discretion.

Finally, the President's effort to have McFarland write an internal email denying that the President had directed Flynn to discuss sanctions with Kislyak highlights the President's concern about being associated with Flynn's conduct.
The evidence does not establish that the President was trying to have McFarland lie.
The President's request, however, was sufficiently irregular that McFarland - who did not know the full extent of Flynn's communications with the President and thus could not make the representation the President wanted - felt the need to draft an internal memorandum documenting the President's request, and Eisenberg was concerned that the request would look like a quid pro quo in exchange for an ambassadorship.

\subsection{The President's Reaction to Public Confirmation of the FBI's Russia Investigation}

\begin{center}
\textbf{Overview}
\end{center}

In early March 2017, the President learned that Sessions was considering recusing from the Russia investigation and tried to prevent the recusal.
After Sessions announced his recusal on March 2, the President expressed anger at Sessions for the decision and then privately asked Sessions to "unrecuse."
On March 20, 2017, Comey publicly disclosed the existence of the FBI's Russia investigation.
In the days that followed, the President contacted Comey and other intelligence agency leaders and asked them to push back publicly on the suggestion that the President had any connection to the Russian election-interference effort in order to "lift the cloud" of the ongoing investigation.

\begin{center}
\textbf{Evidence}
\end{center}

\subsubsection{Attorney General Sessions Recuses From the Russia Investigation}

In late February 2017, the Department of Justice began an internal analysis of whether Sessions should recuse from the Russia investigation based on his role in the 2016 Trump Campaign.% 273
\footnote{Sessions 1/17/18 302, at 1;
Hunt 2/1/18 302, at 3.}
On March 1, 2017, the press reported that, in his January confirmation hearing to become Attorney General, Senator Sessions had not disclosed two meetings he had with Russian Ambassador Kislyak before the presidential election, leading to congressional calls for Sessions to recuse or for a special counsel to investigate Russia's interference in the presidential election.% 274
\footnote{\textit{E.g.}, Adam Entous et al., \textit{Sessions met with Russian envoy twice last year, encounters he later did not disclose}, Washington Post (Mar.~1, 2017).}

Also on March 1, the President called Comey and said he wanted to check in and see how Comey was doing.% 275
\footnote{3/1/17 Email, Comey to Rybicki;
SCR012b\_000030 (President's Daily Diary, 3/1/17, reflecting call with Comey at 11:55 am.)}
According to an email Comey sent to his chief of staff after the call, the President "talked about Sessions a bit," said that he had heard Comey was "doing great," and said that he hoped Comey would come by to say hello when he was at the White House.% 276
\footnote{3/1/17 Email, Comey to Rybicki;
\textit{see Hearing on Russian Election Interference Before the Senate Select Intelligence Committee}, 115th Cong.\ (June 8, 2017) (CQ Cong.\ Transcripts, at 86) (testimony of James B. Comey, former Director of the FBI) ("[H]e called me one day.... [H]e just called to check in and tell me I was doing an awesome job, and wanted to see how I was doing.").}
Comey interpreted the call as an effort by the President to "pull [him] in," but he did not perceive the call as an attempt by the President to find out what Comey was doing with the Flynn investigation.% 277
\footnote{Comey 11/15/17 302, at 17-18.}

The next morning, the President called McGahn and urged him to contact Sessions to tell him not to recuse himself from the Russia investigation.% 278
\footnote{McGahn 11/30/17 302, at 16.}
McGahn understood the President to be concerned that a recusal would make Sessions look guilty for omitting details in his confirmation hearing;
leave the President unprotected from an investigation that could hobble the presidency and derail his policy objectives;
and detract from favorable press coverage of a Presidential Address to Congress the President had delivered earlier in the week.% 279
\footnote{McGahn 11/30/17 302, at 16-17;
\textit{see} SC\_AD\_00123 (Donaldson 3/2/17 Notes) ("Just in the middle of another Russia Fiasco.").}
McGahn reached out to Sessions and reported that the President was not happy about the possibility of recusal.% 280
\footnote{Sessions 1/17/18 302, at 3.}
Sessions replied that he intended to follow the rules on recusal.% 281
\footnote{McGahn 11/30/17 302, at 17.}
McGahn reported back to the President about the call with Sessions, and the President reiterated that he did not want Sessions to recuse.% 282
\footnote{McGahn 11/30/17 302, at 17.}
Throughout the day, McGahn continued trying on behalf of the President to avert Sessions's recusal by speaking to Sessions's personal counsel, Sessions's chief of staff, and Senate Majority Leader Mitch McConnell, and by contacting Sessions himself two more times.% 283
\footnote{McGahn 11/30/17 302, at 18-19;
Sessions 1/17/18 302, at 3;
Hunt 2/1/18 302, at 4;
Donaldson 11/6/17 302, at 8-10;
\textit{see} Hunt-000017;
SC\_AD\_00121 (Donaldson 3/2/17 Notes).}
Sessions recalled that other White House advisors also called him that day to argue against his recusal.% 284
\footnote{Sessions 1/17/18 302, at 3.}

That afternoon, Sessions announced his decision to recuse "from any existing or future investigations of any matters related in any way to the campaigns for President of the United States."% 285
\footnote{Attorney General Sessions Statement on Recusal, Department of Justice Press Release (Mar.~2, 2017) ("During the course of the last several weeks, I have met with the relevant senior career Department officials to discuss whether I should recuse myself from any matters arising from the campaigns for President of the United States.
Having concluded those meetings today, I have decided to recuse myself from any existing or future investigations of any matters related in any way to the campaigns for President of the United States.").
At the time of Sessions's recusal, Dana Boente, then the Acting Deputy Attorney General and U.S. Attorney for the Eastern District of Virginia, became the Acting Attorney General for campaign-related matters pursuant to an executive order specifying the order of succession at the Department of Justice.
\textit{Id}. ("Consistent with the succession order for the Department of Justice, ... Dana Boente shall act as and perform the functions of the Attorney General with respect to any matters from which I have recused myself to the extent they exist.");
\textit{see} Exec.\ Order No.~13775, 82 Fed.\ Reg.\ 10697 (Feb.~14, 2017).}
Sessions believed the decision to recuse was not a close call, given the applicable language in the Code of Federal Regulations (CFR), which Sessions considered to be clear and decisive.% 286
\footnote{Sessions 1/17/18 302, at 1-2. 28 C.F.R. \S 45.2 provides that "no employee shall participate in a criminal investigation or prosecution if he has a personal or political relationship with ... [any person or organization substantially involved in the conduct that is the subject of the investigation or prosecution," and defines "political relationship" as "a close identification with an elected official, a candidate (whether or not successful) for elective, public office, a political party, or a campaign organization, arising from service as a principal adviser thereto or a principal official thereof."}
Sessions thought that any argument that the CFR did not apply to him was "very thin."% 287
\footnote{Sessions 1/17/18 302, at 2.}
Sessions got the impression, based on calls he received from White House officials, that the President was very upset with him and did not think he had done his duty as Attorney General.% 288
\footnote{Sessions 1/17/18 302, at 3.}

Shortly after Sessions announced his recusal, the White House Counsel's Office directed that Sessions should not be contacted about the matter.% 289
\footnote{Donaldson 11/6/17 302, at 11;
SC\_AD\_00123 (Donaldson 3/2/17 Notes).
It is not clear whether the President was aware of the White House Counsel's Office direction not to contact Sessions about his recusal.}
Internal White House Counsel's Office notes from March 2, 2017, state "No contact w/ Sessions" and "No comms/Serious concerns about obstruction."% 290
\footnote{SC\_AD\_00123 (Donaldson 3/2/17 Notes).
McGahn said he believed the note "No comms / Serious concerns about obstruction" may have referred to concerns McGahn had about the press team saying "crazy things" and trying to spin Sessions's recusal in a way that would raise concerns about obstruction.
McGahn 11/30/17 302, at 19.
Donaldson recalled that "No comms" referred to the order that no one should contact Sessions.
Donaldson 11/6/17 302, at 11.}

On March 3, the day after Sessions's recusal, McGahn was called into the Oval Office.% 291
\footnote{McGahn 12/12/17 302, at 2.}
Other advisors were there, including Priebus and Bannon.% 292
\footnote{McGahn 12/12/17 302, at 2.}
The President opened the conversation by saying, "I don't have a lawyer."% 293
\footnote{McGahn 12/12/17 302, at 2.}
The President expressed anger at McGahn about the recusal and brought up Roy Cohn, stating that he wished Cohn was his attorney.% 294
\footnote{McGahn 12/12/17 302, at 2.
Cohn had previously served as a lawyer for the President during his career as a private businessman.
Priebus recalled that when the President talked about Cohn, he said Cohn would win cases for him that had no chance, and that Cohn had done incredible things for him.
Priebus 4/3/18 302, at 5.
Bannon recalled the President describing Cohn as a winner and a fixer, someone who got things done.
Bannon 2/14/18 302, at 6.}
McGahn interpreted this comment as directed at him, suggesting that Cohn would fight for the President whereas McGahn would not.% 295
\footnote{McGahn 12/12/17 302, at 2.}
The President wanted McGahn to talk to Sessions about the recusal, but McGahn told the President that DOJ ethics officials had weighed in on Sessions's decision to recuse.% 296
\footnote{McGahn 12/12/17 302, at 2.}
The President then brought up former Attorneys General Robert Kennedy and Eric Holder and said that they had protected their presidents.% 297
\footnote{McGahn 12/12/17 302, at 3.
Bannon said the President saw Robert Kennedy and Eric Holder as Attorneys General who protected the presidents they served.
The President thought Holder always stood up for President Obama and even took a contempt charge for him, and Robert Kennedy always had his brother's back.
Bannon 2/14/18 302, at 5.
Priebus recalled that the President said he had been told his entire life he needed to have a great lawyer, a "bulldog," and added that Holder had been willing to take a contempt-of-Congress charge for President Obama.
Priebus 4/3/18 302, at 5.}
The President also pushed back on the DOJ contacts policy, and said words to the effect of, "You're telling me that Bobby and Jack didn't talk about investigations?
Or Obama didn't tell Eric Holder who to investigate?"% 298
\footnote{McGahn 12/12/17 302, at 3.}
Bannon recalled that the President was as mad as Bannon had ever seen him and that he screamed at McGahn about how weak Sessions was.% 299
\footnote{Bannon 2/14/18 302, at 5.}
Bannon recalled telling the President that Sessions's recusal was not a surprise and that before the inauguration they had discussed that Sessions would have to recuse from campaign-related investigations because of his work on the Trump Campaign.% 300
\footnote{Bannon 2/14/18 302, at 5.}

That weekend, Sessions and McGahn flew to Mar-a-Lago to meet with the President.% 301
\footnote{Sessions 1/17/18 302, at 3;
Hunt 2/1/18 302, at 5;
McGahn 12/12/17 302, at 3.}
Sessions recalled that the President pulled him aside to speak to him alone and suggested that Sessions should "unrecuse" from the Russia investigation.% 302
\footnote{Sessions 1/17/18 302, at 3-4.}
The President contrasted Sessions with Attorneys General Holder and Kennedy, who had developed a strategy to help their presidents where Sessions had not.% 303
\footnote{Sessions 1/17/18 302, at 3-4.}
Sessions said he had the impression that the President feared that the investigation could spin out of control and disrupt his ability to govern, which Sessions could have helped avert if he were still overseeing it.% 304
\footnote{Sessions 1/17/18 302, at 3-4.
Hicks recalled that after Sessions recused, the President was angry and scolded Sessions in her presence, but she could not remember exactly when that conversation occurred.
Hicks 12/8/17 302, at 13.}

On March 5, 2017, the White House Counsel's Office was informed that the FBI was asking for transition-period records relating to Flynn - indicating that the FBI was still actively investigating him.% 305
\footnote{SC\_AD\_000137 (Donaldson 3/5/17 Notes);
\textit{see} Donaldson 11/6/17 302, at 13.}
On March 6, the President told advisors he wanted to call the Acting Attorney General to find out whether the White House or the President was being investigated, although it is not clear whether the President knew at that time of the FBI's recent request concerning Flynn.% 306
\footnote{Donaldson 11/6/17 302, at 14;
\textit{see} SC\_AD\_000168 (Donaldson 3/6/17 Notes) ("POTUS wants to call Dana [then the Acting Attorney General for campaign-related investigations] / Is investigation / No / We know something on Flynn / GSA got contacted by FBI / There's something hot").}

\subsubsection{FBI Director Comey Publicly Confirms the Existence of the Russia Investigation in Testimony Before HPSCI}

On March 9, 2017, Comey briefed the "Gang of Eight" congressional leaders about the FBI's investigation of Russian interference, including an identification of the principal U.S. subjects of the investigation.% 307
\footnote{Comey 11/15/17 302, at 13-14;
SNS-Classified-0000140-44 (3/8/17 Email, Gauhar to Page et al.).}
Although it is unclear whether the President knew of that briefing at the time, notes taken by Annie Donaldson, then McGahn's chief of staff, on March 12, 2017, state, "POTUS in panic/chaos ... Need binders to put in front of POTUS. (1) All things related to Russia."% 308
\footnote{SC\_AD\_00188 (Donaldson 3/12/18 Notes).
Donaldson said she was not part of the conversation that led to these notes, and must have been told about it from others.
Donaldson 11/6/17 302, at 13.}
The week after Comey's briefing, the White House Counsel's Office was in contact with SSCI Chairman Senator Richard Burr about the Russia investigations and appears to have received information about the status of the FBI investigation.% 309
\footnote{Donaldson 11/6/17 302, at 14-15.
On March 16, 2017, the White House Counsel's Office was briefed by Senator Burr on the existence of "4-5 targets."
Donaldson 11/6/17 302, at 15.
The "targets" were identified in notes taken by Donaldson as "Flynn (FBI was in - wrapping up) $\longrightarrow$ DOJ looking for phone records";
"Comey $\longrightarrow$ Manafort (Ukr + Russia, not campaign)";
\blackout{Harm to Ongoing Matter}
"Carter Page (\$ game)";
and "Greek Guy" (potentially referring to George Papadopoulos, later charged with violating 18 U.S.C. \S 1001 for lying to the FBI).
SC\_AD\_00198 (Donaldson 3/16/17 Notes).
Donaldson and McGahn both said they believed these were targets of SSCI.
Donaldson 11/6/17 302, at 15;
McGahn 12/12/17 302, at 4.
But SSCI does not formally investigate individuals as "targets";
the notes on their face reference the FBI, the Department of Justice, and Comey;
and the notes track the background materials prepared by the FBI for Comey's briefing to the Gang of 8 on March 9.
\textit{See} SNS-Classified-0000140-44 (3/8/17 Email, Gauhar to Page et al.);
\textit{see also} Donaldson 1 1/6/17 302, at 15 (Donaldson could not rule out that Burr had told McGahn those individuals were the FBI's targets).}

On March 20, 2017, Comey was scheduled to testify before HPSCI.% 310
\footnote{\textit{Hearing on Russian Election Tampering Before the House Permanent Select Intelligence Committee}, 115th Cong.\ (Mar.~20, 2017).}
In advance of Comey's testimony, congressional officials made clear that they wanted Comey to provide information about the ongoing FBI investigation.% 311
\footnote{Comey 11/15/17 302, at 16;
McCabe 8/17/17, at 15;
McGahn 12/14/17 302, at 1.}
Dana Boente, who at that time was the Acting Attorney General for the Russia investigation, authorized Comey to confirm the existence of the Russia investigation and agreed that Comey should decline to comment on whether any particular individuals, including the President, were being investigated.% 312
\footnote{Boente 1/31/18 302, at 5;
Comey 11/15/17 302, at 16-17.}

In his opening remarks at the HPSCI hearing, which were drafted in consultation with the Department of Justice, Comey stated that he had "been authorized by the Department of Justice to confirm that the FBI, as part of [its] counterintelligence mission, is investigating the Russian government's efforts to interfere in the 2016 presidential election and that includes investigating the nature of any links between individuals associated with the Trump campaign and the Russian government and whether there was any coordination between the campaign and Russia's efforts.
As with any counterintelligence investigation, this will also include an assessment of whether any crimes were committed."% 313
\footnote{\textit{Hearing on Russian Election Tampering Before the House Permanent Select Intelligence Committee}, 115th Cong.\ (Mar.~20, 2017) (CQ Cong.\ Transcripts, at 11) (testimony by FBI Director James B.Comey);
Comey 11/15/17 302, at 17;
Boente 1/31/18 302, at 5 (confirming that the Department of Justice authorized Comey's remarks).}
Comey added that he would not comment further on what the FBI was "doing and whose conduct[it] [was] examining" because the investigation was ongoing and classified - but he observed that he had "taken the extraordinary step in consultation with the Department of Justice of briefing this Congress's leaders ... in a classified setting in detail about the investigation."% 314
\footnote{\textit{Hearing on Russian Election Tampering Before the House Permanent Select Intelligence Committee}, 115th Cong.\ (Mar.~20, 2017) (CQ Cong.\ Transcripts, at 11) (testimony by FBI Director James B. Comey).}
Comey was specifically asked whether President Trump was "under investigation during the campaign" or "under investigation now."% 315
\footnote{\textit{Hearing on Russian Election Tampering Before the House Permanent Select Intelligence Committee}, 115th Cong.\ (Mar.~20, 2017) (CQ Cong.\ Transcripts, at 130) (question by Rep.~Swalwell).}
Comey declined to answer, stating, "Please don't over interpret what I've said as - as the chair and ranking know, we have briefed him in great detail on the subjects of the investigation and what we're doing, but I'm not gonna answer about anybody in this forum."% 316
\footnote{\textit{Hearing on Russian Election Tampering Before the House Permanent Select Intelligence Committee}, 115th Cong.\ (Mar.~20, 2017) (CQ Cong.\ Transcripts, at 130) (testimony by FBI Director James B. Comey).}
Comey was also asked whether the FBI was investigating the information contained in the Steele reporting, and he declined to answer.% 317
\footnote{\textit{Hearing on Russian Election Tampering Before the House Permanent Select Intelligence Committee}, 115th Cong.\ (Mar.~20, 2017) (CQ Cong.\ Transcripts, at 143) (testimony by FBI Director James B. Comey).}

According to McGahn and Donaldson, the President had expressed frustration with Comey before his March 20 testimony, and the testimony made matters worse.% 318
\footnote{Donaldson 11/6/17 302, at 21;
McGahn 12/12/17 302, at 7.}
The President had previously criticized Comey for too frequently making headlines and for not attending intelligence briefings at the White House, and the President suspected Comey of leaking certain information to the media.% 319
\footnote{Donaldson 11/6/17 302, at 21;
McGahn 12/12/17 302, at 6-9.}
McGahn said the President thought Comey was acting like "his own branch of government."% 320
\footnote{McGahn 12/12/17 302, at 7.}

Press reports following Comey's March 20 testimony suggested that the FBI was investigating the President, contrary to what Comey had told the President at the end of the January 6, 2017 intelligence assessment briefing.% 321
\footnote{\textit{E.g.}, Matt Apuzzo et al., \textit{F.B.I. Is Investigating Trump's Russia Ties, Comey Confirms}, New York Times (Mar.~20, 2017);
Andy Greenberg, \textit{The FBI Has Been Investigating Trump's Russia Ties Since July}, Wired (Mar.~20, 2017);
Julie Borger \& Spencer Ackerman, \textit{Trump-Russia collusion is being investigated by FBI, Comey confirms}, Guardian (Mar.~20, 2017);
\textit{see} Comey 1/6/17 Memorandum, at 2.}
McGahn, Donaldson, and senior advisor Stephen Miller recalled that the President was upset with Comey's testimony and the press coverage that followed because of the suggestion that the President was under investigation.% 322
\footnote{Donaldson 11/6/17 302, at 16-17;
S. Miller 10/31/17 302, at 4;
McGahn 12/12/17 302, at 5-7.}
Notes from the White House Counsel's Office dated March 21, 2017, indicate that the President was "beside himself" over Comey's testimony.% 323
\footnote{SC\_AD\_00213 (Donaldson 3/21/17 Notes).
The notes from that day also indicate that the President referred to the "Comey bombshell" which "made [him] look like a fool."
SC\_AD\_00206 (Donaldson 3/21/17 Notes).}
The President called McGahn repeatedly that day to ask him to intervene with the Department of Justice, and, according to the notes, the President was "getting hotter and hotter, get rid?"% 324
\footnote{SC\_AD\_00210 (Donaldson 3/21/17 Notes).}
Officials in the White House Counsel's Office became so concerned that the President would fire Comey that they began drafting a memorandum that examined whether the President needed cause to terminate the FBI director.% 325
\footnote{SCR016\_000002-05 (White House Counsel's Office Memorandum).
White House Counsel's Office attorney Uttam Dhillon did not recall a triggering event causing the White House Counsel's Office to begin this research.
Dhillon 11/21/17 302, at 5.
Metadata from the document, which was provided by the White House, establishes that it was created on March 21, 2017.}

At the President's urging, McGahn contacted Boente several times on March 21, 2017, to seek Boente's assistance in having Comey or the Department of Justice correct the misperception that the President was under investigation.% 326
\footnote{Donaldson 11/6/17 302, at 16-21;
McGahn 12/12/17 302, at 5-7.}
Boente did not specifically recall the conversations, although he did remember one conversation with McGahn around this time where McGahn asked if there was a way to speed up or end the Russia investigation as quickly as possible.% 327
\footnote{Boente 1/31/18 302, at 5.}
Boente said McGahn told him the President was under a cloud and it made it hard for him to govern.% 328
\footnote{Boente 1/31/18 302, at 5.}
Boente recalled telling McGahn that there was no good way to shorten the investigation and attempting to do so could erode confidence in the investigation's conclusions.% 329
\footnote{Boente 1/31/18 302, at 5.}
Boente said McGahn agreed and dropped the issue.% 330
\footnote{Boente 1/31/18 302, at 5.}
The President also sought to speak with Boente directly, but McGahn told the President that Boente did not want to talk to the President about the request to intervene with Comey.% 331
\footnote{SC\_AD\_00210 (Donaldson 3/21/17 Notes);
McGahn 12/12/17 302, at 7;
Donaldson 11/6/17 302, at 19.}
McGahn recalled Boente telling him in calls that day that he did not think it was sustainable for Comey to stay on as FBI director for the next four years, which McGahn said he conveyed to the President.% 332
\footnote{McGahn 12/12/17 302, at 7;
Burnham 11/03/17 302, at 11.}
Boente did not recall discussing with McGahn or anyone else the idea that Comey should not continue as FBI director.% 333
\footnote{Boente 1/31/18 302, at 3.}

\subsubsection{The President Asks Intelligence Community Leaders to Make Public Statements that he had No Connection to Russia}

In the weeks following Comey's March 20, 2017 testimony, the President repeatedly asked intelligence community officials to push back publicly on any suggestion that the President had a connection to the Russian election-interference effort.

On March 22, 2017, the President asked Director of National Intelligence Daniel Coats and CIA Director Michael Pompeo to stay behind in the Oval Office after a Presidential Daily Briefing.% 334
\footnote{Coats 6/14/17 302, at 3;
Culver 6/14/17 302, at 2.}
According to Coats, the President asked them whether they could say publicly that no link existed between him and Russia.% 335
\footnote{Coats 6/14/17 302, at 3.}
Coats responded that the Office of the Director of National Intelligence (ODNI) has nothing to do with investigations and it was not his role to make a public statement on the Russia investigation.% 336
\footnote{Coats 6/14/17 302, at 3.}
Pompeo had no recollection of being asked to stay behind after the March 22 briefing, but he recalled that the President regularly urged officials to get the word out that he had not done anything wrong related to Russia.% 337
\footnote{Pompeo 6/28/17 302, at 1-3.}

Coats told this Office that the President never asked him to speak to Comey about the FBI investigation.% 338
\footnote{Coats 6/14/17 302, at 3.}
Some ODNI staffers, however, had a different recollection of how Coats described the meeting immediately after it occurred.
According to senior ODNI official Michael Dempsey, Coats said after the meeting that the President had brought up the Russia investigation and asked him to contact Comey to see if there was a way to get past the investigation, get it over with, end it, or words to that effect.% 339
\footnote{Dempsey 6/14/17 302, at 2.}
Dempsey said that Coats described the President's comments as falling "somewhere between musing about hating the investigation" and wanting Coats to "do something to stop it."% 340
\footnote{Dempsey 6/14/17 302, at 2 - 3.}
Dempsey said Coats made it clear that he would not get involved with an ongoing FBI investigation.% 341
\footnote{Dempsey 6/14/17 302, at 3.}
Edward Gistaro, another ODNI official, recalled that right after Coats's meeting with the President, on the walk from the Oval Office back to the Eisenhower Executive Office Building, Coats said that the President had kept him behind to ask him what he could do to "help with the investigation."% 342
\footnote{Gistaro 6/14/17 302, at 2.}
Another ODNI staffer who had been waiting for Coats outside the Oval Office talked to Gistaro a few minutes later and recalled Gistaro reporting that Coats was upset because the President had asked him to contact Comey to convince him there was nothing to the Russia investigation.% 343
\footnote{Culver 6/14/17 302, at 2-3.}

On Saturday, March 25, 2017, three days after the meeting in the Oval Office, the President called Coats and again complained about the Russia investigations, saying words to the effect of, "I can't do anything with Russia, there's things I'd like to do with Russia, with trade, with ISIS, they're all over me with this."% 344
\footnote{Coats 6/14/17 302, at 4.}
Coats told the President that the investigations were going to go on and the best thing to do was to let them run their course.% 345
\footnote{Coats 6/14/17 302, at 4;
Dempsey 6/14/17 302, at 3 (Coats relayed that the President had asked several times what Coats could do to help "get [the investigation] done," and Coats had repeatedly told the President that fastest way to "get it done" was to let it run its course).}
Coats later testified in a congressional hearing that he had "never felt pressure to intervene or interfere in any way and shape - with shaping intelligence in a political way, or in relationship ... to an ongoing investigation."% 346
\footnote{\textit{Hearing on Foreign Intelligence Surveillance Act Before the Senate Select Intelligence Committee}, 115 Cong.\ (June 7, 2017) (CQ Cong.\ Transcripts, at 25) (testimony by Daniel Coats, Director of National Intelligence).}

On March 26, 2017, the day after the President called Coats, the President called NSA Director Admiral Michael Rogers.% 347
\footnote{Rogers 6/12/17 302, at 3-4.}
The President expressed frustration with the Russia investigation, saying that it made relations with the Russians difficult.% 348
\footnote{Rogers 6/12/17 302, at 4.}
The President told Rogers "the thing with the Russians [wa]s messing up" his ability to get things done with Russia.% 349
\footnote{Ledgett 6/13/17 302, at 1-2;
\textit{see} Rogers 6/12/17 302, at 4.}
The President also said that the news stories linking him with Russia were not true and asked Rogers if he could do anything to refute the stories.% 350
\footnote{Rogers 6/12/17 302, at 4-5;
Ledgett 6/13/17 302, at 2.}
Deputy Director of the NSA Richard Ledgett, who was present for the call, said it was the most unusual thing he had experienced in 40 years of government service.% 351
\footnote{Ledgett 6/13/17 302, at 2.}
After the call concluded, Ledgett prepared a memorandum that he and Rogers both signed documenting the content of the conversation and the President's request, and they placed the memorandum in a safe.% 352
\footnote{Ledgett 6/13/17 302, at 2-3;
Rogers 6/12/17 302, at 4.}
But Rogers did not perceive the President's request to be an order, and the President did not ask Rogers to push back on the Russia investigation itself.% 353
\footnote{Rogers 6/12/17 302, at 5;
Ledgett 6/13/17 302, at 2.}
Rogers later testified in a congressional hearing that as NSA Director he had "never been directed to do anything [he] believe[d] to be illegal, immoral, unethical or inappropriate" and did "not recall ever feeling pressured to do so."% 354
\footnote{\textit{Hearing on Foreign Intelligence Surveillance Act Before the Senate Select Intelligence Committee}, 115th Cong.\ (June 7, 2017) (CQ Cong.\ Transcripts, at 20) (testimony by Admiral Michael Rogers, Director of the National Security Agency).}

In addition to the specific comments made to Coats, Pompeo, and Rogers, the President spoke on other occasions in the presence of intelligence community officials about the Russia investigation and stated that it interfered with his ability to conduct foreign relations.% 355
\footnote{Gistaro 6/14/17 302, at 1, 3;
Pompeo 6/28/17 302, at 2-3.}
On at least two occasions, the President began Presidential Daily Briefings by stating that there was no collusion with Russia and he hoped press statement to that effect could be issued.% 356
\footnote{Gistaro 6/14/17 302, at 1.}
Pompeo recalled that the President vented about the investigation on multiple occasions, complaining that there was no evidence against him and that nobody would publicly defend him.% 357
\footnote{Pompeo 6/28/17 302, at 2.}
Rogers recalled a private conversation with the President in which he "vent[ed]" about the investigation, said he had done nothing wrong, and said something like the "Russia thing has got to go away."% 358
\footnote{Rogers 6/12/17 302, at 6.}
Coats recalled the President bringing up the Russia investigation several times, and Coats said he finally told the President that Coats's job was to provide intelligence and not get involved in investigations.% 359
\footnote{Coats 6/14/17 302, at 3-4.}

\subsubsection{The President Asks Comey to "Lift the Cloud" Created by the Russia Investigation}

On the morning of March 30, 2017, the President reached out to Comey directly about the Russia investigation.% 360
\footnote{SCR012b\_000044 (President's Daily Diary, 3/30/17, reflecting call to Comey from 8:14 - 8:24 a.m.);
Comey 3/30/17 Memorandum, at 1 ("The President called me on my CMS phone at 8:13 am today....
The call lasted 11 minutes (about 10 minutes when he was connected).";
\textit{Hearing on Russian Election Interference Before the Senate Select Intelligence Committee}, 115th Cong.\ (June 8, 2017) (Statement for the Record of James B. Comey, former Director of the FBI, at 6).}
According to Comey's contemporaneous record of the conversation, the President said "he was trying to run the country and the cloud of this Russia business was making that difficult."% 361
\footnote{Comey 3/30/17 Memorandum, at 1.
Comey subsequently testified before Congress about this conversation and described it to our Office;
his recollections were consistent with his memorandum.
\textit{Hearing on Russian Election Interference Before the Senate Select Intelligence Committee}, 115th Cong.\ (June 8, 2017) (Statement for the Record of James B. Comey, former Director of the FBI, at 6);
Comey 11/15/17 302, at 18.}
The President asked Comey what could be done to "lift the cloud."% 362
\footnote{Comey 3/30/17 Memorandum, at 1;
Comey 11/15/17 302, at 18.}
Comey explained "that we were running it down as quickly as possible and that there would be great benefit, if we didn't find anything, to our Good Housekeeping seal of approval, but we had to do our work."% 363
\footnote{Comey 3/30/17 Memorandum, at 1;
Comey 11/15/17 302, at 18.}
Comey also told the President that congressional leaders were aware that the FBI was not investigating the President personally.% 364
\footnote{Comey 3/30/17 Memorandum, at 1;
\textit{Hearing on Russian Election Interference Before the Senate Select Intelligence Committee}, 115th Cong.\ (June 8, 2017) (Statement for the Record of James B. Comey, former Director of the FBI, at 6).}
The President said several times, "We need to get that fact out."% 365
\footnote{Comey 3/30/17 Memorandum, at 1;
\textit{Hearing on Russian Election Interference Before the Senate Select Intelligence Committee}, 115th Cong.\ (June 8, 2017) (Statement for the Record of James B. Comey, former Director of the FBI, at 6).}
The President commented that if there was "some satellite" (which Comey took to mean an associate of the President's or the campaign) that did something, "it would be good to find that out" but that he himself had not done anything wrong and he hoped Comey "would find a way to get out that we weren't investigating him."% 366
\footnote{Comey 3/30/17 Memorandum, at 1;
\textit{Hearing on Russian Election Interference Before the Senate Select Intelligence Committee}, 115th Cong.\ (June 8, 2017) (Statement for the Record of James B. Comey, former Director of the FBI, at 6 - 7).}
After the call ended, Comey called Boente and told him about the conversation, asked for guidance on how to respond, and said he was uncomfortable with direct contact from the President about the investigation.% 367
\footnote{Comey 3/30/17 Memorandum, at 2;
Boente 1/31/18 302, at 6-7;
\textit{Hearing on Russian Election Interference Before the Senate Select Intelligence Committee}, 115th Cong.\ (June 8, 2017) (Statement for the Record of James B. Comey, former Director of the FBI, at 7).}

On the morning of April 11, 2017, the President called Comey again.% 368
\footnote{SCR012b\_000053 (President's Daily Diary, 4/11/17, reflecting call to Comey from 8:27-8:31 a.m.);
Comey 4/11/17 Memorandum, at 1 ("I returned the president's call this morning at 8:26 am EDT.
We spoke for about four minutes.").}
According to Comey's contemporaneous record of the conversation, the President said he was "following up to see if [Comey] did what [the President] had asked last time - getting out that he personally is not under investigation."% 369
\footnote{Comey 4/11/17 Memorandum, at 1.
Comey subsequently testified before Congress about this conversation and his recollections were consistent with his memo.
\textit{Hearing on Russian Election Interference Before the Senate Select Intelligence Committee}, 115th Cong.\ (June 8, 2017) (Statement for the Record of James B. Comey, former Director of the FBI, at 7).}
Comey responded that he had passed the request to Boente but not heard back, and he informed the President that the traditional channel for such a request would be to have the White House Counsel contact DOJ leadership.% 370
\footnote{Comey 4/11/17 Memorandum, at 1.}
The President said he would take that step.% 371
\footnote{Comey 4/11/17 Memorandum, at 1.}
The President then added, "Because I have been very loyal to you, very loyal, we had that thing, you know."% 372
\footnote{Comey 4/11/17 Memorandum, at 1.
In a footnote to this statement in his memorandum, Comey wrote, "His use of these words did not fit with the flow of the call, which at that point had moved away from any request of me, but I have recorded it here as it happened."}
In a televised interview that was taped early that afternoon, the President was asked if it was too late for him to ask Comey to step down;
the President responded, "No, it's not too late, but you know, I have confidence in him.
We'll see what happens.
You know, it's going to be interesting."% 373
\footnote{Maria Bartiromo, \textit{Interview with President Trump}, Fox Business Network (Apr.~12, 2017); SCR012b\_000054 (President's Daily Diary, 4/11/17, reflecting Bartiromo interview from 12:30 - 12:55 p.m.).}
After the interview, Hicks told the President she thought the President's comment about Comey should be removed from the broadcast of the interview, but the President wanted to keep it in, which Hicks thought was unusual.% 374
\footnote{Hicks 12/8/17 302, at 13.}

Later that day, the President told senior advisors, including McGahn and Priebus, that he had reached out to Comey twice in recent weeks.% 375
\footnote{Priebus 10/13/17 302, at 23;
McGahn 12/12/17 302, at 9.}
The President acknowledged that McGahn would not approve of the outreach to Comey because McGahn had previously cautioned the President that he should not talk to Comey directly to prevent any perception that the White House was interfering with investigations.% 376
\footnote{Priebus 10/13/17 302, at 23;
McGahn 12/12/17 302, at 9;
\textit{see} McGahn 11/30/17 302, at 9;
Dhillon 11/21/17 302, at 2 (stating that White House Counsel attorneys had advised the President not to contact the FBI Director directly because it could create a perception he was interfering with investigations).
Later in April, the President told other attorneys in the White House Counsel's Office that he had called Comey even though he knew they had advised against direct contact.
Dhillon 11/21/17 302, at 2 (recalling that the President said, "I know you told me not to, but I called Comey anyway.").}
The President told McGahn that Comey had indicated the FBI could make a public statement that the President was not under investigation if the Department of Justice approved that action.% 377
\footnote{McGahn 12/12/17 302, at 9.}
After speaking with the President, McGahn followed up with Boente to relay the President's understanding that the FBI could make a public announcement if the Department of Justice cleared it.% 378
\footnote{McGahn 12/12/17 302, at 9.}
McGahn recalled that Boente said Comey had told him there was nothing obstructive about the calls from the President, but they made Comey uncomfortable.% 379
\footnote{McGahn 12/12/17 302, at 9;
\textit{see} Boente 1/31/18 302, at 6 (recalling that Comey told him after the March 30, 2017 call that it was not obstructive).}
According to McGahn, Boente responded that he did not want to issue a statement about the President not being under investigation because of the potential political ramifications and did not want to order Comey to do it because that action could prompt the appointment of a Special Counsel.% 380
\footnote{McGahn 12/12/17 302, at 9-10.}
Boente did not recall that aspect of his conversation with McGahn, but did recall telling McGahn that the direct outreaches from the President to Comey were a problem.% 381
\footnote{Boente 1/31/18 302, at 7;
McGahn 12/12/17 302, at 9.}
Boente recalled that McGahn agreed and said he would do what he could to address that issue.% 382
\footnote{Boente 1/31/18 302, at 7.}

\begin{center}
\textbf{Analysis}
\end{center}

In analyzing the President's reaction to Sessions's recusal and the requests he made to Coats, Pompeo, Rogers, and Comey, the following evidence is relevant to the elements of obstruction of justice:

\underline{Obstructive act.}
The evidence shows that, after Comey's March 20, 2017 testimony, the President repeatedly reached out to intelligence agency leaders to discuss the FBI's investigation.
But witnesses had different recollections of the precise content of those outreaches.
Some ODNI officials recalled that Coats told them immediately after the March 22 Oval Office meeting that the President asked Coats to intervene with Comey and "stop" the investigation.
But the first-hand witnesses to the encounter remember the conversation differently.
Pompeo had no memory of the specific meeting, but generally recalled the President urging officials to get the word out that the President had not done anything wrong related to Russia.
Coats recalled that the President asked that Coats state publicly that no link existed between the President and Russia, but did not ask him to speak with Comey or to help end the investigation.
The other outreaches by the President during this period were similar in nature.
The President asked Rogers if he could do anything to refute the stories linking the President to Russia, and the President asked Comey to make a public statement that would "lift the cloud" of the ongoing investigation by making clear that the President was not personally under investigation.
These requests, while significant enough that Rogers thought it important to document the encounter in a written memorandum, were not
interpreted by the officials who received them as directives to improperly interfere with the investigation.

\underline{Nexus to a proceeding.}
At the time of the President's outreaches to leaders of the intelligence agencies in late March and early April 2017, the FBI's Russia investigation did not yet involve grand jury proceedings.
The outreaches, however, came after and were in response to Comey's March 20, 2017 announcement that the FBI, as a part of its counterintelligence mission, was conducting an investigation into Russian interference in the 2016 presidential election.
Comey testified that the investigation included any links or coordination with Trump campaign officials and would "include an assessment of whether any crimes were committed."

\underline{Intent.}
As described above, the evidence does not establish that the President asked or directed intelligence agency leaders to stop or interfere with the FBI's Russia investigation - and the President affirmatively told Comey that if "some satellite" was involved in Russian election interference "it would be good to find that out."
But the President's intent in trying to prevent Sessions's recusal, and in reaching out to Coats, Pompeo, Rogers, and Comey following Comey's public announcement of the FBI's Russia investigation, is nevertheless relevant to understanding what motivated the President's other actions towards the investigation.

The evidence shows that the President was focused on the Russia investigation's implications for his presidency - and, specifically, on dispelling any suggestion that he was under investigation or had links to Russia.
In early March, the President attempted to prevent Sessions's recusal, even after being told that Sessions was following DOJ conflict-of-interest rules.
After Sessions recused, the White House Counsel's Office tried to cut off further contact with Sessions about the matter, although it is not clear whether that direction was conveyed to the President.
The President continued to raise the issue of Sessions's recusal and, when he had the opportunity, he pulled Sessions aside and urged him to unrecuse.
The President also told advisors that he wanted an Attorney General who would protect him, the way he perceived Robert Kennedy and Eric Holder to have protected their presidents.
The President made statements about being able to direct the course of criminal investigations, saying words to the effect of, "You're telling me that Bobby and Jack didn't talk about investigations?
Or Obama didn't tell Eric Holder who to investigate?"

After Comey publicly confirmed the existence of the FBI's Russia investigation on March 20, 2017, the President was "beside himself' and expressed anger that Comey did not issue a statement correcting any misperception that the President himself was under investigation.
The President sought to speak with Acting Attorney General Boente directly and told McGahn to contact Boente to request that Comey make a clarifying statement.
The President then asked other intelligence community leaders to make public statements to refute the suggestion that the President had links to Russia, but the leaders told him they could not publicly comment on the investigation.
On March 30 and April 11, against the advice of White House advisors who had informed him that any direct contact with the FBI could be perceived as improper interference in an ongoing investigation, the President made personal outreaches to Comey asking him to "lift the cloud" of the Russia investigation by making public the fact that the President was not personally under investigation.

Evidence indicates that the President was angered by both the existence of the Russia investigation and the public reporting that he was under investigation, which he knew was not true based on Comey's representations.
The President complained to advisors that if people thought Russia helped him with the election, it would detract from what he had accomplished.

Other evidence indicates that the President was concerned about the impact of the Russia investigation on his ability to govern.
The President complained that the perception that he was under investigation was hurting his ability to conduct foreign relations, particularly with Russia.
The President told Coats he "can't do anything with Russia," he told Rogers that "the thing with the Russians" was interfering with his ability to conduct foreign affairs, and he told Comey that "he was trying to run the country and the cloud of this Russia business was making that difficult."

\subsection{Events Leading Up To and Surrounding the Termination of FBI Director Comey}

\begin{center}
\textbf{Overview}
\end{center}

Comey was scheduled to testify before Congress on May 3, 2017.
Leading up to that testimony, the President continued to tell advisors that he wanted Comey to make public that the President was not under investigation.
At the hearing, Comey declined to answer questions about the scope or subjects of the Russia investigation and did not state publicly that the President was not under investigation.
Two days later, on May 5, 2017, the President told close aides he was going to fire Comey, and on May 9, he did so, using his official termination letter to make public that Comey had on three occasions informed the President that he was not under investigation.
The President decided to fire Comey before receiving advice or a recommendation from the Department of Justice, but he approved an initial public account of the termination that attributed it to a recommendation from the Department of Justice based on Comey's handling of the Clinton email investigation.
After Deputy Attorney General Rod Rosenstein resisted attributing the firing to his recommendation, the President acknowledged that he intended to fire Comey regardless of the DOJ recommendation and was thinking of the Russia investigation when he made the decision.
The President also told the Russian Foreign Minister, "I just fired the head of the F.B.I.
He was crazy, a real nut job.
I faced great pressure because of Russia.
That's taken off.....
I'm not under investigation."

\begin{center}
\textbf{Evidence}
\end{center}

\subsubsection{Comey Testifies Before the Senate Judiciary Committee and Declines to Answer Questions About Whether the President is Under Investigation}

On May 3, 2017, Comey was scheduled to testify at an FBI oversight hearing before the Senate Judiciary Committee.% 383
\footnote{\textit{Hearing on Oversight of the FBI before the Senate Judiciary Committee}, 115th Cong.\ (May~3, 2017).}
McGahn recalled that in the week leading up to the hearing, the President said that it would be the last straw if Comey did not take the opportunity to set the record straight by publicly announcing that the President was not under investigation.% 384
\footnote{McGahn 12/12/17 302, at 10-11.}
The President had previously told McGahn that the perception that the President was under investigation was hurting his ability to carry out his presidential duties and deal with foreign leaders.% 385
\footnote{McGahn 12/12/17 302, at 7, 10-11 (McGahn believed that two foreign leaders had expressed sympathy to the President for being under investigation);
SC\_AD\_00265 (Donaldson 4/11/17 Notes) ("P Called Comey - Day we told him not to? 'You are not under investigation' NK/China/Sapping Credibility").}
At the hearing, Comey declined to answer questions about the status of the Russia investigation, stating "{t]he Department of Justice ha[d] authorized [him] to confirm that [the Russia investigation] exists," but that he was "not going to say another word about it" until the investigation was completed.% 386
\footnote{\textit{Hearing on FBI Oversight Before the Senate Judiciary Committee}, 115th Cong.\ (CQ Cong.\ Transcripts, at 70) (May 3, 2017) (testimony by FBI Director James Comey).
Comey repeated this point several times during his testimony.
\textit{See id}. at 26 (explaining that he was "not going to say another peep about[the investigation] until we're done");
\textit{id}. at 90 (stating that he would not provide any updates about the status of investigation "before the matter is concluded").}
Comey also declined to answer questions about whether investigators had "ruled out anyone in the Trump campaign as potentially a target of th[e] criminal investigation," including whether the FBI had "ruled out the president of the United States."% 387
\footnote{\textit{Hearing on FBI Oversight Before the Senate Judiciary Committee}, 115th Cong.\ (May 3, 2017) (CQ Cong.\ Transcripts, at 87-88) (questions by Sen.~Blumenthal and testimony by FBI Director James B. Comey).}

Comey was also asked at the hearing about his decision to announce 11 days before the presidential election that the FBI was reopening the Clinton email investigation.% 388
\footnote{\textit{Hearing on FBI Oversight Before the Senate Judiciary Committee}, 115th Cong.\ (May 3, 2017) (CQ Cong.\ Transcripts, at 15) (question by Sen.~Feinstein).}
Comey stated that it made him "mildly nauseous to think that we might have had some impact on the election," but added that "even in hindsight" he "would make the same decision."% 389
\footnote{\textit{Hearing on FBI Oversight Before the Senate Judiciary Committee}, 115th Cong.\ (May 3, 2017) (CQ Cong.\ Transcripts, at 17) (testimony by FBI Director James B. Comey).}
He later repeated that he had no regrets about how he had handled the email investigation and believed he had "done the right thing at each turn."% 390
\footnote{\textit{Hearing on FBI Oversight Before the Senate Judiciary Committee}, 115th Cong.\ (May 3, 2017) (CQ Cong.\ Transcripts, at 92) (testimony by FBI Director James B. Comey).}

In the afternoon following Comey's testimony, the President met with McGahn, Sessions, and Sessions's Chief of Staff Jody Hunt.% 391
\footnote{Sessions 1/17/18 302, at 8;
Hunt 2/1/18 302, at 8.}
At that meeting, the President asked McGahn how Comey had done in his testimony and McGahn relayed that Comey had declined to answer questions about whether the President was under investigation.% 392
\footnote{Sessions 1/17/18 302, at 8;
Hunt-000021 (Hunt 5/3/17 Notes);
McGahn 3/8/18 302, at 6.}
The President became very upset and directed his anger at Sessions.% 393
\footnote{Sessions 1/17/18 302, at 8-9.}
According to notes written by Hunt, the President said, "This is terrible Jeff.
It's all because you recused.
AG is supposed to be most important appointment.
Kennedy appointed his brother.
Obama appointed Holder. T appointed you and you recused yourself.
You left me on an island.
I can't do anything."% 394
\footnote{Hunt-000021 (Hunt 5/3/17 Notes).
Hunt said that he wrote down notes describing this meeting and others with the President after the events occurred.
Hunt 2/1/17 302, at 2.}
The President said that the recusal was unfair and that it was interfering with his ability to govern and undermining his authority with foreign leaders.% 395
\footnote{Hunt-00002 1-22 (Hunt 5/3/17 Notes) ("I have foreign leaders saying they are sorry I am being investigated.");
Sessions 1/17/18 302, at 8 (Sessions recalled that a Chinese leader had said to the President that he was sorry the President was under investigation, which the President interpreted as undermining his authority);
Hunt 2/1/18 302, at 8.}
Sessions responded that he had had no choice but to recuse, and it was a mandatory rather than discretionary decision.% 396
\footnote{Sessions 1/17/18 302, at 8;
Hunt-000022 (Hunt 5/3/17 Notes).}
Hunt recalled that Sessions also stated at some point during the conversation that a new start at the FBI would be appropriate and the President should consider replacing Comey as FBI director.% 397
\footnote{Hunt-000022 (Hunt 5/3/17 Notes).}
According to Sessions, when the meeting concluded, it was clear that the President was unhappy with Comey, but Sessions did not think the President had made the decision to terminate Comey.% 398
\footnote{Sessions 1/17/18 302, at 9.}

Bannon recalled that the President brought Comey up with him at least eight times on May 3 and May 4, 2017.% 399
\footnote{Bannon 2/12/18 302, at 20.}
According to Bannon, the President said the same thing each time:
"He told me three times I'm not under investigation.
He's a showboater.
He's a grandstander.
I don't know any Russians.
There was no collusion."% 400
\footnote{Bannon 2/12/18 302, at 20.}
Bannon told the President that he could not fire Comey because "that ship had sailed."% 401
\footnote{Bannon 2/12/18 302, at 20.}
Bannon also told the President that firing Comey was not going to stop the investigation, cautioning him that he could fire the FBI director but could not fire the FBI.% 402
\footnote{Bannon 2/12/18 302, at 20-21;
\textit{see} Priebus 10/13/17 302, at 28.}

\subsubsection{The President Makes the Decision to Terminate Comey}

The weekend following Comey's May 3, 2017 testimony, the President traveled to his resort in Bedminster, New Jersey.% 403
\footnote{S. Miller 10/31/17 302, at 4-5;
SCR025\_000019 (President's Daily Diary, 5/4/17).}
At a dinner on Friday, May 5, attended by the President and various advisors and family members, including Jared Kushner and senior advisor Stephen Miller, the President stated that he wanted to remove Comey and had ideas for a letter that would be used to make the announcement.% 404
\footnote{S. Miller 10/31/17 302, at 5.}
The President dictated arguments and specific language for the letter, and Miller took notes.% 405
\footnote{S. Miller 10/31/17 302, at 5-6.}
As reflected in the notes, the President told Miller that the letter should start, "While I greatly appreciate you informing me that I am not under investigation concerning what I have often stated is a fabricated story on a Trump-Russia relationship - pertaining to the 2016 presidential election, please be informed that I, and I believe the American public - including Ds and Rs - have lost faith in you as Director of the FBI."% 406
\footnote{S. Miller 5/5/17 Notes, at 1;
\textit{see} S. Miller 10/31/17 302, at 8.}
Following the dinner, Miller prepared a termination letter based on those notes and research he conducted to support the President's arguments.% 407
\footnote{S. Miller 10/31/17 302, at 6.}
Over the weekend, the President provided several rounds of edits on the draft letter.% 408
\footnote{S. Miller 10/31/17 302, at 6-8.}
Miller said the President was adamant that he not tell anyone at the White House what they were preparing because the President was worried about leaks.% 409
\footnote{S. Miller 10/31/17 302, at 7.
Miller said he did not want Priebus to be blindsided, so on Sunday night he called Priebus to tell him that the President had been thinking about the "Comey situation" and there would be an important discussion on Monday.
S. Miller 10/31/17 302, at 7.}

In his discussions with Miller, the President made clear that he wanted the letter to open with a reference to him not being under investigation.% 410
\footnote{S. Miller 10/31/17 302, at 8.}
Miller said he believed that fact was important to the President to show that Comey was not being terminated based on any such investigation.% 411
\footnote{S. Miller 10/31/17 302, at 8.}
According to Miller, the President wanted to establish as a factual matter that Comey had been under a "review period" and did not have assurance from the President that he would be permitted to keep his job.% 412
\footnote{S. Miller 10/31/17 302, at 10.}

The final version of the termination letter prepared by Miller and the President began in a way that closely tracked what the President had dictated to Miller at the May 5 dinner:
"Dear Director Comey, While I greatly appreciate your informing me, on three separate occasions, that I am not under investigation concerning the fabricated and politically-motivated allegations of a Trump-Russia relationship with respect to the 2016 Presidential Election, please be informed that I, along with members of both political parties and, most importantly, the American Public, have lost faith in you as the Director of the FBI and you are hereby terminated."% 413
\footnote{SCR013c\_000003-06 (Draft Termination Letter to FBI Director Comey).}
The four-page letter went onto critique Comey's judgment and conduct, including his May 3 testimony before the
Senate Judiciary Committee, his handling of the Clinton email investigation, and his failure to hold leakers accountable.% 414
\footnote{SCR013c\_000003-06 (Draft Termination Letter to FBI Director Comey).
Kushner said that the termination letter reflected the reasons the President wanted to fire Comey and was the truest representation of what the President had said during the May 5 dinner.
Kushner 4/11/18 302, at 25.}
The letter stated that Comey had "asked [the President] at dinner shortly after inauguration to let [Comey] stay on in the Director's role, and [the President] said that [he] would consider it," but the President had "concluded that [he] ha[d] no alternative but to find new leadership for the Bureau - leader that restores confidence and trust."% 415
\footnote{SCR013c\_000003 (Draft Termination Letter to FBI Director Comey).}

In the morning of Monday, May 8, 2017, the President met in the Oval Office with senior advisors, including McGahn, Priebus, and Miller, and informed them he had decided to terminate Comey.% 416
\footnote{McGahn 12/12/17 302, at 11;
Priebus 10/13/17 302, at 24;
S. Miller 10/31/17 302, at 11;
Dhillon 11/21/17 302, at 6;
Eisenberg 11/29/17 302, at 13.}
The President read aloud the first paragraphs of the termination letter he wrote with Miller and conveyed that the decision had been made and was not up for discussion.% 417
\footnote{S. Miller 10/31/17 302, at 11 (observing that the President started the meeting by saying, "I'm going to read you a letter.
Don't talk me out of this.
I've made my decision.");
Dhillon 11/21/17 302, at 6 (the President announced in an irreversible way that he was firing Comey);
Eisenberg 11/29/17 302, at 13 (the President did not leave whether or not to fire Comey up for discussion);
Priebus 10/13/17 302, at 25;
McGahn 12/12/17 302, at 11-12.}
The President told the group that Miller had researched the issue and determined the President had the authority to terminate Comey without cause.% 418
\footnote{Dhillon 302 11/21/17, at 6;
Eisenberg 11/29/17 302, at 13;
McGahn 12/12/17 302, at 11.}
In an effort to slow down the decision-making process, McGahn told the President that DOJ leadership was currently discussing Comey's status and suggested that White House Counsel's Office attorneys should talk with Sessions and Rod Rosenstein, who had recently been confirmed as the Deputy Attorney General.% 419
\footnote{McGahn 12/12/17 302, at 12, 13;
S. Miller 10/31/17 302, at 11;
Dhillon 11/21/17 302, at 7.
Because of the Attorney General's recusal, Rosenstein became the Acting Attorney General for the Russia investigation upon his confirmation as Deputy Attorney General.
\textit{See} 28 U.S.C. \S 508(a) ("In case of a vacancy in the office of Attorney General, or of his absence or disability, the Deputy Attorney General may exercise all the duties of that office').}
McGahn said that previously scheduled meetings with Sessions and Rosenstein that day would be an opportunity to find out what they thought about firing Comey.% 420
\footnote{McGahn 12/12/17 302, at 12.}

At noon, Sessions, Rosenstein, and Hunt met with McGahn and White House Counsel's Office attorney Uttam Dhillon at the White House.% 421
\footnote{Dhillon 11/21/17 302, at 7;
McGahn 12/12/17 302, at 13;
Gauhar-000056 (Gauhar 5/16/17 Notes);
\textit{see} Gauhar-000056-72 (2/11/19 Memorandum to File attaching Gauhar handwritten notes) ("Ms.~Gauhar determined that she likely recorded all these notes during one or more meetings on Tuesday, May 16, 2017.").}
McGahn said that the President had decided to fire Comey and asked for Sessions's and Rosenstein's views.% 422
\footnote{McGahn 12/12/17 302, at 13;
\textit{see} Gauhar-000056 (Gauhar 5/16/17 Notes).}
Sessions and Rosenstein criticized Comey and did not raise concerns about replacing him.% 423
\footnote{Dhillon 11/21/17 302, at 7-9;
Sessions 1/17/18 302, at 9;
McGahn 12/12/17 302, at 13.}
McGahn and Dhillon said the fact that neither Sessions nor Rosenstein objected to replacing Comey gave them peace of mind that the President's decision to fire Comey was not an attempt to obstruct justice.% 424
\footnote{McGahn 12/12/17 302, at 13;
Dhillon 11/21/17 302, at 9.}
An Oval Office meeting was scheduled later that day so that Sessions and Rosenstein could discuss the issue with the President.% 425
\footnote{Hunt-000026 (Hunt 5/8/17 Notes);
\textit{see} Gauhar-000057 (Gauhar 5/16/17 Notes).}

At around 5 p.m., the President and several White House officials met with Sessions and Rosenstein to discuss Comey.% 426
\footnote{Rosenstein 5/23/17 302, at 2;
McGahn 12/12/17 302, at 14;
\textit{see} Gauhar-000057 (Gauhar 5/16/17 Notes).}
The President told the group that he had watched Comey's May 3 testimony over the weekend and thought that something was "not right" with Comey.% 427
\footnote{Hunt-000026-27 (Hunt 5/8/17 Notes).}
The President said that Comey should be removed and asked Sessions and Rosenstein for their views.% 428
\footnote{Sessions 1/17/18 302, at 10;
\textit{see} Gauhar-000058 (Gauhar 5/16/17 Notes) ("POTUS to AG: What is your rec?"').}
Hunt, who was in the room, recalled that Sessions responded that he had previously recommended that Comey be replaced.% 429
\footnote{Hunt-000027 (Hunt 5/8/17 Notes).}
McGahn and Dhillon said Rosenstein described his concerns about Comey's handling of the Clinton email investigation.% 430
\footnote{McGahn 12/12/17 302, at 14;
Dhillon 11/21/17 302, at 7.}

The President then distributed copies of the termination letter he had drafted with Miller, and the discussion turned to the mechanics of how to fire Comey and whether the President's letter should be used.% 431
\footnote{Hunt-000028 (Hunt 5/8/17 Notes).}
McGahn and Dhillon urged the President to permit Comey to resign, but the President was adamant that he be fired.% 432
\footnote{McGahn 12/12/17 302, at 13.}
The group discussed the possibility that Rosenstein and Sessions could provide a recommendation in writing that Comey should be removed.% 433
\footnote{Hunt-000028-29 (Hunt 5/8/17 Notes).}
The President agreed and told Rosenstein to draft a memorandum, but said he wanted to receive it first thing the next morning.% 434
\footnote{McCabe 9/26/17 302, at 13;
Rosenstein 5/23/17 302, at 2;
\textit{see} Gauhar-000059 (Gauhar 5/16/17 Notes) ("POTUS tells DAG to write a memo").}
Hunt's notes reflect that the President told Rosenstein to include in his recommendation the fact that Comey had refused to confirm that the President was not personally under investigation.% 435
\footnote{Hunt-000028-29 (Hunt 5/8/17 Notes) ("POTUS asked if Rod's recommendation would include the fact that although Comey talks about the investigation he refuses to say that the President is not under investigation.
...So it would be good if your recommendation would make mention of the fact that Comey refuses to say public[ly] what he said privately 3 times.").}
According to notes taken by a senior DOJ official of Rosenstein's description of his meeting with the President, the President said, "Put the Russia stuff in the memo."% 436
\footnote{Gauhar-000059 (Gauhar 5/16/17 Notes).}
Rosenstein responded that the Russia investigation was not the basis of his recommendation, so he did not think Russia should be mentioned.% 437
\footnote{Sessions 1/17/18 302 at 10;
McCabe 9/26/17 302, at 13;
\textit{see} Gauhar-000059 (Gauhar 5/16/17 Notes).}
The President told Rosenstein he would appreciate it if Rosenstein put it in his letter anyway.% 438
\footnote{Gauhar-000059 (Gauhar 5/16/17 Notes);
McCabe 5/16/17 Memorandum 1;
McCabe 9/26/17 302, at 13.}
When Rosenstein left the meeting, he knew that Comey would be terminated, and he told DOJ colleagues that his own reasons for replacing Comey were "not[the President's] reasons."% 439
\footnote{Rosenstein 5/23/17 302, at 2;
Gauhar-000059 (Gauhar 5/16/17 Notes) ("DAG reasons not their reasons [POTUS]");
Gauhar-000060 (Gauhar 5/16/17 Notes) ("1st draft had a recommendation.
Took it out b/c knew decision had already been made.").}

On May 9, Hunt delivered to the White House letter from Sessions recommending Comey's removal and a memorandum from Rosenstein, addressed to the Attorney General, titled "Restoring Public Confidence in the FBI."% 440
\footnote{Rosenstein 5/23/17 302, at 4;
McGahn 12/12/17 302, at 15;
5/9/17 Letter, Sessions to President Trump ("Based on my evaluation, and for the reasons expressed by the Deputy Attorney General in the attached memorandum, I have concluded that a fresh start is needed at the leadership of the FBI.");
5/9/17 Memorandum, Rosenstein to Sessions (concluding with, "The way the Director handled the conclusion of the email investigation was wrong.
As a result, the FBI is unlikely to regain public and congressional trust until it has a Director who understands the gravity of the mistakes and pledges never to repeat them.
Having refused to admit his errors, the Director cannot be expected to implement the necessary corrective actions.").}
McGahn recalled that the President liked the DOJ letters and agreed that they should provide the foundation for a new cover letter from the President accepting the recommendation to terminate Comey.% 441
\footnote{S. Miller 10/31/17 302, at 12;
McGahn 12/12/17 302, at 15;
Hunt-000031 (Hunt 5/9/17 Notes).}
Notes taken by Donaldson on May 9 reflected the view of the White House Counsel's Office that the President's original termination letter should "[n]ot [see the] light of day" and that it would be better to offer "[n]o other rationales" for the firing than what was in Rosenstein's and Sessions's memoranda.% 442
\footnote{SC\_AD\_00342 (Donaldson 5/9/17 Notes).
Donaldson also wrote "[i]s this the beginning of the end?" because she was worried that the decision to terminate Comey and the manner in which it was carried out would be the end of the presidency.
Donaldson 11/6/17 302, at 25.}
The President asked Miller to draft a new termination letter and directed Miller to say in the letter that Comey had informed the President three times that he was not under investigation.% 443
\footnote{S. Miller 10/31/17 302, at 12;
McGahn 12/12/17 302, at 15;
Hunt-000032 (Hunt 5/9/17 Notes).}
McGahn, Priebus, and Dhillon objected to including that language, but the President insisted that it be included.% 444
\footnote{McGahn 12/12/17 302, at 15;
S. Miller 10/31/17 302, at 12;
Dhillon 11/21/17 302, at 8, 10;
Priebus 10/13/17 302, at 27;
Hunt 2/1/18 302, at 14-15;
Hunt-000032 (Hunt 5/9/17 Notes).}
McGahn, Priebus, and others perceived that language to be the most important part of the letter to the President.% 445
\footnote{Dhillon 11/21/17 302, at 10;
Eisenberg 11/29/17 302, at 15 (providing the view that the President's desire to include the language about not being under investigation was the "driving animus of the whole thing");
Burnham 11/3/17 302, at 16 (Burnham knew the only line the President cared about was the line that said Comey advised the President on three separate occasions that the President was not under investigation).
According to Hunt's notes, the reference to Comey's statement would indicate that "notwithstanding" Comey's having informed the President that he was not under investigation, the President was terminating Comey.
Hunt-000032 (Hunt 5/9/17 Notes).
McGahn said he believed the President wanted the language included so that people would not think that the President had terminated Comey because the President was under investigation.
McGahn 12/12/17 302, at 15.}
Dhillon made a final pitch to the President that Comey should be permitted to resign, but the President refused.% 446
\footnote{McGahn 12/12/17 302, at 15;
Donaldson 11/6/17 302, at 25;
\textit{see} SC\_AD\_00342 (Donaldson 5/9/17 Notes) ("Resign vs.\ Removal. - POTUS/removal.").}

Around the time the President's letter was finalized, Priebus summoned Spicer and the press team to the Oval Office, where they were told that Comey had been terminated for the reasons stated in the letters by Rosenstein and Sessions.% 447
\footnote{Spicer 10/16/17 302, at 9;
McGahn 12/12/17 302, at 16.}
To announce Comey's termination, the White House released a statement, which Priebus thought had been dictated by the President.% 448
\footnote{Priebus 10/13/17 302, at 28.}
In full, the statement read: "Today, President Donald J. Trump informed FBI Director James Comey that he has been terminated and removed from office.
President Trump acted based on the clear recommendations of both Deputy Attorney General Rod Rosenstein and Attorney General Jeff Sessions."% 449
\footnote{\textit{Statement of the Press Secretary}, The White House, Office of the Press Secretary (May 9, 2017).}

That evening, FBI Deputy Director Andrew McCabe was summoned to meet with the President at the White House.% 450
\footnote{McCabe 9/26/17 302, at 4;
SCR025\_000044 (President's Daily Diary, 5/9/17);
McCabe 5/10/17 Memorandum, at 1.}
The President told McCabe that he had fired Comey because of the decisions Comey had made in the Clinton email investigation and for many other reasons.% 451
\footnote{McCabe 9/26/17 302, at 5;
McCabe 5/10/17 Memorandum, at 1.}
The President asked McCabe if he was aware that Comey had told the President three times that he was not under investigation.% 452
\footnote{McCabe 9/26/17 302, at 5;
McCabe 5/10/17 Memorandum, at 1-2.}
The President also asked McCabe whether many people in the FBI disliked Comey and whether McCabe was part of the "resistance" that had disagreed with Comey's decisions in the Clinton investigation.% 453
\footnote{McCabe 9/26/17 302, at 5;
McCabe 5/10/17 Memorandum, at 1-2.}
McCabe told the President that he knew Comey had told the President he was not under investigation, that most people in the FBI felt positively about Comey, and that McCabe worked "very closely" with Comey and was part of all the decisions that had been made in the Clinton investigation.% 454
\footnote{McCabe 9/26/17 302, at 5;
McCabe 5/10/17 Memorandum, at 1-2.}

Later that evening, the President told his communications team he was unhappy with the press coverage of Comey's termination and ordered them to go out and defend him.% 455
\footnote{Spicer 10/16/17 302, at 11;
Hicks 12/8/17, at 18;
Sanders 7/3/18 302, at 2.}
The President also called Chris Christie and, according to Christie, said he was getting "killed" in the press over Comey's termination.% 456
\footnote{Christie 2/13/19 302, at 6.}
The President asked what he should do.% 457
\footnote{Christie 2/13/19 302, at 6.}
Christie asked, "Did you fire [Comey] because of what Rod wrote in the memo?", and the President responded, "Yes.% 458
\footnote{Christie 2/13/19 302, at 6.}
Christie said that the President should "get Rod out there" and have him defend the decision.% 459
\footnote{Christie 2/13/19 302, at 6.}
The President told Christie that this was a "good idea" and said he was going to call Rosenstein right away.% 460
\footnote{Christie 2/13/19 302, at 6.}

That night, the White House Press Office called the Department of Justice and said the White House wanted to put out a statement saying that it was Rosenstein's idea to fire Comey.% 461
\footnote{Gauhar-000071 (Gauhar 5/16/17 Notes);
Page Memorandum, at 3 (recording events of 5/16/17);
McCabe 9/26/17 302, at 14.}
Rosenstein told other DOJ officials that he would not participate in putting out a "false story."% 462
\footnote{Rosenstein 5/23/17 302, at 4-5;
Gauhar-000059 (Gauhar 5/16/17 Notes).}
The President then called Rosenstein directly and said he was watching Fox News, that the coverage had been great, and that he wanted Rosenstein to do a press conference.% 463
\footnote{Rosenstein 5/23/17 302, at 4-5;
Gauhar-000071 (Gauhar 5/16/17 Notes).}
Rosenstein responded that this was not a good idea because if the press asked him, he would tell the truth that Comey's firing was not his idea.% 464
\footnote{Gauhar-000071 (Gauhar 5/16/17 Notes).
DOJ notes from the week of Comey's firing indicate that Priebus was "screaming" at the DOJ public affairs office trying to get Rosenstein to do a press conference, and the DOJ public affairs office told Priebus that Rosenstein had told the President he was not doing it.
Gauhar-000071-72 (Gauhar 5/16/17 Notes).}
Sessions also informed the White House Counsel's Office that evening that Rosenstein was upset that his memorandum was being portrayed as the reason for Comey's termination.% 465
\footnote{McGahn 12/12/17 302, at 16-17;
Donaldson 11/6/17 302, at 26-27;
Dhillon 11/21/17 302, at 11.}

In an unplanned press conference late in the evening of May 9, 2017, Spicer told reporters, "It was all [Rosenstein].
No one from the White House.
It was a DOJ decision."% 466
\footnote{Jenna Johnson, \textit{After Trump fired Comey, White House staff scrambled to explain why}, Washington Post (May 10, 2017) (quoting Spicer).}
That evening and the next morning, White House officials and spokespeople continued to maintain that the President's decision to terminate Comey was driven by the recommendations the President received from Rosenstein and Sessions.% 467
\footnote{\textit{See, e.g.}, Sarah Sanders, \textit{White House Daily Briefing}, C-SPAN (May 10, 2017);
SCR013\_001088 (5/10/17 Email, Hemming to Cheung et al.) (internal White House email describing comments on the Comey termination by Vice President Pence).}

In the morning on May 10, 2017, President Trump met with Russian Foreign Minister Sergey Lavrov and Russian Ambassador Sergey Kislyak in the Oval Office.% 468
\footnote{SCR08\_000353 (5/9/17 White House Document, "Working Visit with Foreign Minister Sergey Lavrov of Russia");
SCR08\_001274 (5/10/17 Email, Ciaramella to Kelly et al.).
The meeting had been planned on May 2, 2017, during a telephone call between the President and Russian President Vladimir Putin, and the meeting date was confirmed on May 5, 2017, the same day the President dictated ideas for the Comey termination letter to Stephen Miller.
SCR08\_001274 (5/10/17 Email, Ciaramella to Kelly et al.).}
The media subsequently reported that during the May 10 meeting the President brought up his decision the prior day to terminate Comey, telling Lavrov and Kislyak:
"I just fired the head of the F.B.I.
He was crazy, a real nut job.
I faced great pressure because of Russia.
That's taken off....
I'm not under investigation."% 469
\footnote{Matt Apuzzo et al., \textit{Trump Told Russians That Firing "Nut Job" Comey Eased Pressure From Investigation}, New York Times (May 19, 2017).}
The President never denied making those statements, and the White House did not dispute the account, instead issuing a statement that said:
"By grandstanding and politicizing the investigation into Russia's actions, James Comey created unnecessary pressure on our ability to engage and negotiate with Russia.
The investigation would have always continued, and obviously, the termination of Comey would not have ended it.
Once again, the real story is that our national security has been undermined by the leaking of private and highly classified information."% 470
\footnote{SCR08\_002117 (5/19/17 Email, Walters to Farhi (CBS News));
\textit{see} Spicer 10/16/17 302, at 13 (noting he would have been told to "clean it up" if the reporting on the meeting with the Russian Foreign Minister was inaccurate, but he was never told to correct the reporting);
Hicks 12/8/17 302, at 19 (recalling that the President never denied making the statements attributed to him in the Lavrov meeting and that the President had said similar things about Comey in an off-the-record meeting with reporters on May 18, 2017, calling Comey a "nut job" and "crazy").}
Hicks said that when she told the President about the reports on his meeting with Lavrov, he did not look concerned and said of Comey, "he \textit{is} crazy."% 471
\footnote{Hicks 12/8/17 302, at 19.}
When McGahn asked the President about his comments to Lavrov, the President said it was good that Comey was fired because that took the pressure off by making it clear that he was not under investigation so he could get more work done.% 472
\footnote{McGahn 12/12/17 302, at 18.}

That same morning, on May 10, 2017, the President called McCabe.% 473
\footnote{SCR025\_000046 (President's Daily Diary, 5/10/17);
McCabe 5/10/17 Memorandum, at 1.}
According to a memorandum McCabe wrote following the call, the President asked McCabe to come over to the White House to discuss whether the President should visit FBI headquarters and make a speech to employees.% 474
\footnote{McCabe 5/10/17 Memorandum, at 1.}
The President said he had received "hundreds" of messages from FBI employees indicating their support for terminating Comey.% 475
\footnote{McCabe 5/10/17 Memorandum, at 1.}
The President also told McCabe that Comey should not have been permitted to travel back to Washington, D.C. on the FBI's airplane after he had been terminated and that he did not want Comey "in the building again," even to collect his belongings.% 476
\footnote{McCabe 5/10/17 Memorandum, at 1;
Rybicki 6/13/17 302, at 2.
Comey had been visiting the FBI's Los Angeles office when he found out he had been terminated.
Comey 11/15/17 302, at 22.}
When McCabe met with the President that afternoon, the President, without prompting, told McCabe that people in the FBI loved the President, estimated that at least 80\% of the FBI had voted for him, and asked McCabe who he had voted for in the 2016 presidential election.% 477
\footnote{McCabe 5/10/17 Memorandum, at 1-2.
McCabe's memorandum documenting his meeting with the President is consistent with notes taken by the White House Counsel's Office.
\textit{See} SC\_AD\_00347 (Donaldson 5/10/17 Notes).}

In the afternoon of May 10, 2017, deputy press secretary Sarah Sanders spoke to the President about his decision to fire Comey and then spoke to reporters in a televised press conference.% 478
\footnote{Sanders 7/3/18 302, at 4;
Sarah Sanders, \textit{White House Daily Briefing}, C-SPAN (May 10, 2017).}
Sanders told reporters that the President, the Department of Justice, and bipartisan members of Congress had lost confidence in Comey, "[a]nd most importantly, the rank and file of the FBI had lost confidence in their director.
Accordingly, the President accepted the recommendation of his Deputy Attorney General to remove James Comey from his position."% 479
\footnote{Sarah Sanders, \textit{White House Daily Briefing}, C-SPAN (May 10, 2017);
Sanders 7/3/18 302, at 4.}
In response to questions from reporters, Sanders said that Rosenstein decided "on his own" to review Comey's performance and that Rosenstein decided "on his own" to come to the President on Monday, May 8 to express his concerns about Comey.
When a reporter indicated that the "vast majority" of FBI agents supported Comey, Sanders said, "Look, we've heard from countless members of the FBI that say very different things."% 480
\footnote{Sarah Sanders, \textit{White House Daily Briefing}, C-SPAN (May 10, 2017).}
Following the press conference, Sanders spoke to the President, who told her she did a good job and did not point out any inaccuracies in her comments.% 481
\footnote{Sanders 7/3/18 302, at 4.}
Sanders told this Office that her reference to hearing from "countless members of the FBI" was a "slip of the tongue."% 482
\footnote{Sanders 7/3/18 302, at 4.}
She also recalled that her statement in a separate press interview that rank-and-file FBI agents had lost confidence in Comey was a comment she made "in the heat of the moment" that was not founded on anything.% 483
\footnote{Sanders 7/3/18 302, at 3.}

Also on May 10, 2017, Sessions and Rosenstein each spoke to McGahn and expressed concern that the White House was creating a narrative that Rosenstein had initiated the decision to fire Comey.% 484
\footnote{McGahn 12/12/17 302, at 16-17;
Donaldson 11/6/17 302, at 26;
\textit{see} Dhillon 11/21/17 302, at 11.}
The White House Counsel's Office agreed that it was factually wrong to say that the Department of Justice had initiated Comey's termination,% 485
\footnote{Donaldson 11/6/17 302, at 27.}
and McGahn asked attorneys in the White House Counsel's Office to work with the press office to correct the narrative.% 486
\footnote{McGahn 12/12/17 302, at 17.}

The next day, on May 11, 2017, the President participated in an interview with Lester Holt.
The President told White House Counsel's Office attorneys in advance of the interview that the communications team could not get the story right, so he was going on Lester Holt to say what really happened.% 487
\footnote{Dhillon 11/21/17 302, at 11.}
During the interview, the President stated that he had made the decision to fire Comey before the President met with Rosenstein and Sessions.
The President told Holt, "I was going to fire regardless of recommendation .... [Rosenstein] made a recommendation.
But regardless of recommendation, I was going to fire Comey knowing there was no good time to do it."% 488
\footnote{\textit{Interview with President Donald Trump}, NBC (May 11, 2017) Transcript, at 2.}
The President continued, "And in fact, when I decided to just do it, I said to myself - I said, you know, this Russia thing with Trump and Russia is a made-up story.
It's an excuse by the Democrats for having lost an election that they should've won."% 489
\footnote{\textit{Interview with President Donald Trump}, NBC (May 11, 2017) Transcript, at 2.}

In response to a question about whether he was angry with Comey about the Russia investigation, the President said, "As far as I'm concerned, I want that thing to be absolutely done properly."% 490
\footnote{\textit{Interview with President Donald Trump}, NBC (May 11, 2017) Transcript, at 3.}
The President added that he realized his termination of Comey "probably maybe will confuse people" with the result that it "might even lengthen out the investigation," but he "ha[d] to do the right thing for the American people" and Comey was "the wrong man for that position."% 491
\footnote{\textit{Interview with President Donald Trump}, NBC (May 11, 2017) Transcript, at 3.}
The President described Comey as "a showboat" and "a grandstander," said that "[t]he FBI has been in turmoil," and said he wanted "to have a really competent, capable director."% 492
\footnote{\textit{Interview with President Donald Trump}, NBC (May 11, 2017) Transcript, at 1, 5.}
The President affirmed that he expected the new FBI director to continue the Russia investigation.% 493
\footnote{\textit{Interview with President Donald Trump}, NBC (May 11, 2017) Transcript, at 7.}

On the evening of May 11, 2017, following the Lester Holt interview, the President tweeted, "Russia must be laughing up their sleeves watching as the U.S. tears itself apart over a Democrat EXCUSE for losing the election."% 494
\footnote{\@realDonaldTrump 5/11/17 (4:34 p.m.~ET) Tweet.}
The same day, the media reported that the President had demanded that Comey pledge his loyalty to the President in a private dinner shortly after being sworn in.% 495
\footnote{Michael S. Schmidt, \textit{In a Private Dinner, Trump Demanded Loyalty. Comey Demurred.}, New York Times (May 11, 2017).}
Late in the morning of May 12, 2017, the President tweeted, "Again, the story that there was collusion between the Russians \& Trump campaign was fabricated by Dems as an excuse for losing the election."% 496
\footnote{\@realDonaldTrump 5/12/17 (7:51 a.m.~ET) Tweet.}
The President also tweeted, "James Comey better hope that there are no 'tapes' of our conversations before he starts leaking to the press!" and "When James Clapper himself, and virtually everyone else with knowledge of the witch hunt, says there is no collusion, when does it end?"% 497
\footnote{\@realDonaldTrump 5/12/17 (8:26 a.m.~ET) Tweet;
\@realDonaldTrump 5/12/17 (8:54 a.m.~ET) Tweet.}

\begin{center}
\textbf{Analysis}
\end{center}

In analyzing the President's decision to fire Comey, the following evidence is relevant to the elements of obstruction of justice:

\underline{Obstructive act.}
The act of firing Comey removed the individual overseeing the FBI's Russia investigation.
The President knew that Comey was personally involved in the investigation based on Comey's briefing of the Gang of Eight, Comey's March 20, 2017 public testimony about the investigation, and the President's one-on-one conversations with Comey.

Firing Comey would qualify as an obstructive act if it had the natural and probable effect of interfering with or impeding the investigation - for example, if the termination would have the effect of delaying or disrupting the investigation or providing the President with the opportunity to appoint a director who would take a different approach to the investigation that the President perceived as more protective of his personal interests.
Relevant circumstances bearing on that issue include whether the President's actions had the potential to discourage a successor director or other law enforcement officials in their conduct of the Russia investigation.
The President fired Comey abruptly without offering him an opportunity to resign, banned him from the FBI building, and criticized him publicly, calling him a "showboat" and claiming that the FBI was "in turmoil" under his leadership.
And the President followed the termination with public statements that were highly critical of the investigation;
for example, three days after firing Comey, the President referred to the investigation as a "witch hunt" and asked, "when does it end?"
Those actions had the potential to affect a successor director's conduct of the investigation.

The anticipated effect of removing the FBI director, however, would not necessarily be to prevent or impede the FBI from continuing its investigation.
As a general matter, FBI investigations run under the operational direction of FBI personnel levels below the FBI director.
Bannon made a similar point when he told the President that he could fire the FBI director, but could not fire the FBI.
The White House issued a press statement the day after Comey was fired that said, "The investigation would have always continued, and obviously, the termination of Comey would not have ended it."
In addition, in his May 11 interview with Lester Holt, the President stated that he understood when he made the decision to fire Comey that the action might prolong the investigation.
And the President chose McCabe to serve as interim director, even though McCabe told the President he had worked "very closely" with Comey and was part of all the decisions made in the Clinton investigation.

\underline{Nexus to a proceeding.}
The nexus element would be satisfied by evidence showing that a grand jury proceeding or criminal prosecution arising from an FBI investigation was objectively foreseeable and actually contemplated by the President when he terminated Comey.

Several facts would be relevant to such as showing.
At the time the President fired Comey, a grand jury had not begun to hear evidence related to the Russia investigation and no grand jury subpoenas had been issued.
On March 20, 2017, however, Comey had announced that the FBI was investigating Russia's interference in the election, including "an assessment of whether any crimes were committed."
It was widely known that the FBI, as part of the Russia investigation, was investigating the hacking of the DNC's computers - a clear criminal offense.

In addition, at the time the President fired Comey, evidence indicates the President knew that Flynn was still under criminal investigation and could potentially be prosecuted, despite the President's February 14, 2017 request that Comey "let[] Flynn go."
On March 5, 2017, the White House Counsel's Office was informed that the FBI was asking for transition-period records relating to Flynn - indicating that the FBI was still actively investigating him.
The same day, the President told advisors he wanted to call Dana Boente, then the Acting Attorney General for the Russia investigation, to find out whether the White House or the President was being investigated.
On March 31, 2017, the President signaled his awareness that Flynn remained in legal jeopardy by tweeting that "Mike Flynn should ask for immunity" before he agreed to provide testimony to the FBI or Congress.
And in late March or early April, the President asked McFarland to pass a message to Flynn telling him that the President felt bad for him and that he should stay strong, further demonstrating the President's awareness of Flynn's criminal exposure.

\underline{Intent.}
Substantial evidence indicates that the catalyst for the President's decision to fire Comey was Comey's unwillingness to publicly state that the President was not personally under investigation, despite the President's repeated requests that Comey make such an announcement.
In the week leading up to Comey's May 3, 2017 Senate Judiciary Committee testimony, the President told McGahn that it would be the last straw if Comey did not set the record straight and publicly announce that the President was not under investigation.
But during his May 3 testimony, Comey refused to answer questions about whether the President was being investigated.
Comey's refusal angered the President, who criticized Sessions for leaving him isolated and exposed, saying "You left me on an island."
Two days later, the President told advisors he had decided to fire Comey and dictated a letter to Stephen Miller that began with a reference to the fact that the President was not being investigated:
"While I greatly appreciate you informing me that I am not under investigation concerning what I have often stated is a fabricated story on a Trump-Russia relationship...."
The President later asked Rosenstein to include "Russia" in his memorandum and to say that Comey had told the President that he was not under investigation.
And the President's final termination letter included a sentence, at the President's insistence and against McGahn's advice, stating that Comey had told the President on three separate occasions that he was not under investigation.

The President's other stated rationales for why he fired Comey are not similarly supported by the evidence.
The termination letter the President and Stephen Miller prepared in Bedminster cited Comey's handling of the Clinton email investigation, and the President told McCabe he fired Comey for that reason.
But the facts surrounding Comey's handling of the Clinton email investigation were well known to the President at the time he assumed office, and the President had made it clear to both Comey and the President's senior staff in early 2017 that he wanted Comey to stay on as director.
And Rosenstein articulated his criticism of Comey's handling of the Clinton investigation after the President had already decided to fire Comey.
The President's draft termination letter also stated that morale in the FBI was at an all-time low and Sanders told the press after Comey's termination that the White House had heard from "countless" FBI agents who had lost confidence in Comey.
But the evidence does not support those claims.
The President told Comey at their January 27 dinner that "the people of the FBI really like [him]," no evidence suggests that the President heard otherwise before deciding to terminate Comey, and Sanders acknowledged to investigators that her comments were not founded on anything.

We also considered why it was important to the President that Comey announce publicly that he was not under investigation.
Some evidence indicates that the President believed that the erroneous perception he was under investigation harmed his ability to manage domestic and foreign affairs, particularly in dealings with Russia.
The President told Comey that the "cloud" of "this Russia business" was making it difficult to run the country.
The President told Sessions and McGahn that foreign leaders had expressed sympathy to him for being under investigation and that the perception he was under investigation was hurting his ability to address foreign relations issues.
The President complained to Rogers that "the thing with the Russians [was] messing up" his ability to get things done with Russia, and told Coats, "I can't do anything with Russia, there's things I'd
like to do with Russia, with trade, with ISIS, they're all over me with this."
The President also may have viewed Comey as insubordinate for his failure to make clear in the May 3 testimony that the President was not under investigation.

Other evidence, however, indicates that the President wanted to protect himself from an investigation into his campaign.
The day after learning about the FBI's interview of Flynn, the President had a one-on-one dinner with Comey, against the advice of senior aides, and told Comey he needed Comey's "loyalty."
When the President later asked Comey for a second time to make public that he was not under investigation, he brought up loyalty again, saying "Because I have been very loyal to you, very loyal, we had that thing, you know."
After the President learned of Sessions's recusal from the Russia investigation, the President was furious and said he wanted an Attorney General who would protect him the way he perceived Robert Kennedy and Eric Holder to have protected their presidents.
The President also said he wanted to be able to tell his Attorney General "who to investigate."

In addition, the President had a motive to put the FBI's Russia investigation behind him.
The evidence does not establish that the termination of Comey was designed to cover up a conspiracy between the Trump Campaign and Russia: As described in Volume I, the evidence uncovered in the investigation did not establish that the President or those close to him were involved in the charged Russian computer-hacking or active-measure conspiracies, or that the President otherwise had an unlawful relationship with any Russian official.
But the evidence does indicate that a thorough FBI investigation would uncover facts about the campaign and the President personally that the President could have understood to be crimes or that would give rise to personal and political concerns.
Although the President publicly stated during and after the election that he had no connection to Russia, the Trump Organization, through Michael Cohen, was pursuing the proposed Trump Tower Moscow project through June 2016 and candidate Trump was repeatedly briefed on the progress of those efforts.% 498
\footnote{\textit{See} Volume II, Section II.K.1, \textit{infra}.}
In addition, some witnesses said that Trump was aware that \blackout{Harm to Ongoing Matter}
at a time when public reports stated that Russian intelligence officials were behind the hacks, and that Trump privately sought information about future WikiLeaks releases.% 499
\footnote{\textit{See} Volume I, Section III.D.1, \textit{supra}.}
More broadly, multiple witnesses described the President's preoccupation with press coverage of the Russia investigation and his persistent concern that it raised questions about the legitimacy of his election.% 500
\footnote{In addition to whether the President had a motive related to Russia-related matters that an FBI investigation could uncover, we considered whether the President's intent in firing Comey was connected to other conduct that could come to light as a result of the FBI's Russian-interference investigation.
In particular, Michael Cohen was a potential subject of investigation because of his pursuit of the Trump Tower Moscow project and involvement in other activities.
And facts uncovered in the Russia investigation, which our Office referred to the U.S. Attorney's Office for the Southern District of New York, ultimately led to the conviction of Cohen in the Southern District of New York for campaign-finance offenses related to payments he said he made at the direction of the President.
\textit{See} Volume II, Section II.K.5, \textit{infra}.
The investigation, however, did not establish that when the President fired Comey, he was considering the possibility that the FBI's investigation would uncover these payments or that the President's intent in firing Comey was otherwise connected to a concern about these matters coming to light.}

Finally, the President and White House aides initially advanced a pretextual reason to the press and the public for Comey's termination.
In the immediate aftermath of the firing, the President dictated a press statement suggesting that he had acted based on the DOJ recommendations, and White House press officials repeated that story.
But the President had decided to fire Comey before the White House solicited those recommendations.
Although the President ultimately acknowledged that he was going to fire Comey regardless of the Department of Justice's recommendations, he did so only after DOJ officials made clear to him that they would resist the White House's suggestion that they had prompted the process that led to Comey's termination.
The initial reliance on a pretextual justification could support an inference that the President had concerns about providing the real reason for the firing, although the evidence does not resolve whether those concerns were personal, political, or both.

\subsection{The President's Efforts to Remove the Special Counsel}

\begin{center}
\textbf{Overview}
\end{center}

The Acting Attorney General appointed a Special Counsel on May 17, 2017, prompting the President to state that it was the end of his presidency and that Attorney General Sessions had failed to protect him and should resign.
Sessions submitted his resignation, which the President ultimately did not accept.
The President told senior advisors that the Special Counsel had conflicts of interest, but they responded that those claims were "ridiculous" and posed no obstacle to the Special Counsel's service.
Department of Justice ethics officials similarly cleared the Special Counsel's service.
On June 14, 2017, the press reported that the President was being personally investigated for obstruction of justice and the President responded with a series of tweets criticizing the Special Counsel's investigation.
That weekend, the President called McGahn and directed him to have the Special Counsel removed because of asserted conflicts of interest.
McGahn did not carry out the instruction for fear of being seen as triggering another Saturday Night Massacre and instead prepared to resign.
McGahn ultimately did not quit and the President did not follow up with McGahn on his request to have the Special Counsel removed.

\begin{center}
\textbf{Evidence}
\end{center}

\subsubsection{The Appointment of the Special Counsel and the President's Reaction}

On May 17, 2017, Acting Attorney General Rosenstein appointed Robert S. Mueller, III as Special Counsel and authorized him to conduct the Russia investigation and matters that arose from the investigation.% 501
\footnote{Office of the Deputy Attorney General, Order No.~3915-2017, \textit{Appointment of Special Counsel to Investigate Russian Interference with the 2016 Presidential Election and Related Matters} (May 17, 2017).}
The President learned of the Special Counsel's appointment from Sessions, who was with the President, Hunt, and McGahn conducting interviews for a new FBI Director.% 502
\footnote{Sessions 1/17/18 302, at 13;
Hunt 2/1/18 302, at 18;
McGahn 12/14/17 302, at 4;
Hunt-000039 (Hunt 5/17/17 Notes).}
Sessions stepped out of the Oval Office to take a call from Rosenstein, who told him about the Special Counsel appointment, and Sessions then returned to inform the President of the news.% 503
\footnote{Sessions 1/17/18 302, at 13;
Hunt 2/1/18 302, at 18;
McGahn 12/14/17 302, at 4;
Hunt-000039 (Hunt 5/17/17 Notes).}
According to notes written by Hunt, when Sessions told the President that a Special Counsel had been appointed, the President slumped back in his chair and said, "Oh my God.
This is terrible.
This is the end of my Presidency.
I'm fucked."% 504
\footnote{Hunt-000039 (Hunt 5/17/17 Notes).}
The President became angry and lambasted the Attorney General for his decision to recuse from the investigation, stating, "How could you let this happen, Jeff?"% 505
\footnote{Hunt-000039 (Hunt 5/17/17 Notes);
Sessions 1/17/18 302, at 13-14.}
The President said the position of Attorney General was his most important appointment and that Sessions had "let [him] down," contrasting him to Eric Holder and Robert Kennedy.% 506
\footnote{Hunt-000040;
\textit{see} Sessions 1/17/18 302, at 14.}
Sessions recalled that the President said to him, "you were supposed to protect me," or words to that effect.% 507
\footnote{Sessions 1/17/18 302, at 14.}
The President returned to the consequences of the appointment and said, "Everyone tells me if you get one of these independent counsels it ruins your presidency.
It takes years and years and I won't be able to do anything.
This is the worst thing that ever happened to me."% 508
\footnote{Hunt-000040 (Hunt 5/17/17 Notes);
\textit{see} Sessions 1/17/18 302, at 14.
Early the next morning, the President tweeted, "This is the single greatest witch hunt of a politician in American history!"
\@realDonaldTrump 5/18/17 (7:52 a.m.~ET) Tweet.}

The President then told Sessions he should resign as Attorney General.% 509
\footnote{Hunt-000041 (Hunt 5/17/17 Notes);
Sessions 1/17/18 302, at 14.}
Sessions agreed to submit his resignation and left the Oval Office.% 510
\footnote{Hunt-000041 (Hunt 5/17/17 Notes);
Sessions 1/17/18 302, at 14.}
Hicks saw the President shortly after Sessions departed and described the President as being extremely upset by the Special Counsel's appointment.% 511
\footnote{Hicks 12/8/17 302, at 21.}
Hicks said that she had only seen the President like that one other time, when the Access Hollywood tape came out during the campaign.% 512
\footnote{Hicks 12/8/17 302, at 21.
The Access Hollywood tape was released on October 7, 2016, as discussed in Volume I, Section III.D.1, \textit{supra}.}

The next day, May 18, 2017, FBI agents delivered to McGahn preservation notice that discussed an investigation related to Comey's termination and directed the White House to preserve all relevant documents.% 513
\footnote{McGahn 12/14/17 302, at 9;
SCR015\_000175-82 (Undated Draft Memoranda to White House Staff).}
When he received the letter, McGahn issued a document hold to White House staff and instructed them not to send out any burn bags over the weekend while he sorted things out.% 514
\footnote{McGahn 12/14/17 302, at 9;
SCR015\_000175-82 (Undated Draft Memoranda to White House Staff).
The White House Counsel's Office had previously issued a document hold on February 27, 2017.
SCR015\_000171 (2/17/17 Memorandum from McGahn to Executive Office of the President Staff).}

Also on May 18, Sessions finalized a resignation letter that stated, "Pursuant to our conversation of yesterday, and at your request, I hereby offer my resignation."% 515
\footnote{Hunt-000047 (Hunt 5/18/17 Notes); 5/18/17 Letter, Sessions to President Trump (resigning as Attorney General).}
Sessions, accompanied by Hunt, brought the letter to the White House and handed it to the President.% 516
\footnote{Hunt-000047-49 (Hunt 5/18/17 Notes);
Sessions 1/17/18 302, at 14.}
The President put the resignation letter in his pocket and asked Sessions several times whether he wanted to continue serving as Attorney General.% 517
\footnote{Hunt-000047-49 (Hunt 5/18/17 Notes);
Sessions 1/17/18 302, at 14.}
Sessions ultimately told the President he wanted to stay, but it was up to the President.% 518
\footnote{Hunt-000048-49 (Hunt 5/18/17 Notes);
Sessions 1/17/18 302, at 14.}
The President said he wanted Sessions to stay.% 519
\footnote{Sessions 1/17/18 302, at 14.}
At the conclusion of the meeting, the President shook Sessions's hand but did not return the resignation letter.% 520
\footnote{Hunt-000049 (Hunt 5/18/17 Notes).}

When Priebus and Bannon learned that the President was holding onto Sessions's resignation letter, they became concerned that it could be used to influence the Department of Justice.% 521
\footnote{Hunt-000050-51 (Hunt 5/18/17 Notes).}
Priebus told Sessions it was not good for the President to have the letter because it would function as a kind of "shock collar" that the President could use any time he wanted;
Priebus said the President had "DOJ by the throat."% 522
\footnote{Hunt-000050 (Hunt 5/18/17 Notes);
Priebus 10/13/17 302, at 21;
Hunt 2/1/18 302, at 21.}
Priebus and Bannon told Sessions they would attempt to get the letter back from the President with a notation that he was not accepting Sessions's resignation.% 523
\footnote{Hunt-000051 (Hunt 5/18/17 Notes).}

On May 19, 2017, the President left for trip to the Middle East.% 524
\footnote{SCR026\_000110 (President's Daily Diary, 5/19/17).}
Hicks recalled that on the President's flight from Saudi Arabia to Tel Aviv, the President pulled Sessions's resignation letter from his pocket, showed it to a group of senior advisors, and asked them what he should do about it.% 525
\footnote{Hicks 12/8/17 302, at 22.}
During the trip, Priebus asked about the resignation letter so he could return it to Sessions, but the President told him that the letter was back at the White House, somewhere in the residence.% 526
\footnote{Priebus 10/13/17 302, at 21.
Hunt's notes state that when Priebus returned from the trip, Priebus told Hunt that the President was supposed to have given him the letter, but when he asked for it, the President "slapped the desk" and said he had forgotten it back at the hotel.
Hunt-000052 (Hunt Notes, undated).}
It was not until May 30, three days after the President returned from the trip, that the President returned the letter to Sessions with a notation saying, "Not accepted."% 527
\footnote{Hunt-000052-53 (Hunt 5/30/17 Notes);
5/18/17 Letter, Sessions to President Trump (resignation letter).
Robert Porter, who was the White House Staff Secretary at the time, said that in the days after the President returned from the Middle East trip, the President took Sessions's letter out of a drawer in the Oval Office and showed it to Porter.
Porter 4/13/18 302, at 8.
\blackout{Personal Privacy}}

\subsubsection{The President Asserts that the Special Counsel has Conflicts of Interest}

In the days following the Special Counsel's appointment, the President repeatedly told advisors, including Priebus, Bannon, and McGahn, that Special Counsel Mueller had conflicts of interest.% 528
\footnote{Priebus 1/18/18 302, at 12;
Bannon 2/14/18 302, at 10;
McGahn 3/8/18 302, at 1;
McGahn 12/14/17 302, at 10;
Bannon 10/26/18 302, at 12.}
The President cited as conflicts that Mueller had interviewed for the FBI Director position shortly before being appointed as Special Counsel, that he had worked for a law firm that represented people affiliated with the President, and that Mueller had disputed certain fees relating to his membership in a Trump golf course in Northern Virginia.% 529
\footnote{Priebus 1/18/18 302, at 12;
Bannon 2/14/18 302, at 10.
In October 2011, Mueller resigned his family's membership from Trump National Golf Club in Sterling, Virginia, in a letter that noted that "we live in the District and find that we are unable to make full use of the Club" and that inquired "whether we would be entitled to a refund of a portion of our initial membership fee," which was paid in 1994.
10/12/11 Letter, Muellers to Trump National Golf Club.
About two weeks later, the controller of the club responded that the Muellers' resignation would be effective October 31, 2011, and that they would be "placed on a wait list to be refunded on a first resigned / first refunded basis" in accordance with the club's legal documents.
10/27/11 Letter, Muellers to Trump National Golf Club.
The Muellers have not had further contact with the club.}
The President's advisors pushed back on his assertion of conflicts, telling the President they did not count as true conflicts.% 530
\footnote{Priebus 4/3/18 302, at 3;
Bannon 10/26/18 302, at 13 (confirming that he, Priebus, and McGahn pushed back on the asserted conflicts).}
Bannon recalled telling the President that the purported conflicts were "ridiculous" and that none of them was real or could come close to justifying precluding Mueller from serving as Special Counsel.% 531
\footnote{Bannon 10/26/18 302, at 12-13.}
As for Mueller's interview for FBI Director, Bannon recalled that the White House had invited Mueller to speak to the President to offer a perspective on the institution of the FBI.% 532
\footnote{Bannon 10/26/18 302, at 12.}
Bannon said that, although the White House thought about beseeching Mueller to become Director again, he did not come in looking for the job.% 533
\footnote{Bannon 10/26/18 302, at 12.}
Bannon also told the President that the law firm position did not amount to a conflict in the legal community.% 534
\footnote{Bannon 10/26/18 302, at 12.}
And Bannon told the President that the golf course dispute did not rise to the level of a conflict and claiming one was "ridiculous and petty."% 535
\footnote{Bannon 10/26/18 302, at 13.}
The President did not respond when Bannon pushed back on the stated conflicts of interest.% 536
\footnote{Bannon 10/26/18 302, at 12.}

On May 23, 2017, the Department of Justice announced that ethics officials had determined that the Special Counsel's prior law firm position did not bar his service, generating media reports that Mueller had been cleared to serve.% 537
\footnote{Matt Zapotosky \& Matea Gold, \textit{Justice Department ethics experts clear Mueller to lead Russia probe}, Washington Post (May 23, 2017).}
McGahn recalled that around the same time, the President complained about the asserted conflicts and prodded McGahn to reach out to Rosenstein about the issue.% 538
\footnote{McGahn 3/8/18 302, at 1;
McGahn 12/14/17 302, at 10;
Priebus 1/18/18 302, at 12.}
McGahn said he responded that he could not make such call and that the President should instead consult his personal lawyer because it was not a White House issue.% 539
\footnote{McGahn 3/8/18 302, at 1.
McGahn and Donaldson said that after the appointment of the Special Counsel, they considered themselves potential fact witnesses and accordingly told the President that inquiries related to the investigation should be brought to his personal counsel.
McGahn 12/14/17 302, at 7;
Donaldson 4/2/18 302, at 5.}
Contemporaneous notes of a May 23, 2017 conversation between McGahn and the President reflect that McGahn told the President that he would not call Rosenstein and that he would suggest that the President not make such call either.% 540
\footnote{SC\_AD\_00361 (Donaldson 5/31/17 Notes).}
McGahn advised that the President could discuss the issue with his personal attorney but it would "look like still trying to meddle in [the] investigation" and "knocking out Mueller" would be "[a]nother fact used to claim obst[ruction] of just[ice]."% 541
\footnote{SC\_AD\_00361 (Donaldson 5/31/17 Notes).}
McGahn told the President that his "biggest exposure" was not his act of firing Comey but his "other contacts" and "calls," and his "ask re: Flynn."% 542
\footnote{SC\_AD\_00361 (Donaldson 5/31/17 Notes).}
By the time McGahn provided this advice to the President, there had been widespread reporting on the President's request for Comey's loyalty, which the President publicly denied;
his request that Comey "let[] Flynn go," which the President also denied;
and the President's statement to the Russian Foreign Minister that the termination of Comey had relieved "great pressure" related to Russia, which the President did not deny.% 543
\footnote{\textit{See, e.g}, Michael S. Schmidt, \textit{In a Private Dinner, Trump Demanded Loyalty. Comey Demurred.}, New York Times (May 11, 2017);
Michael S. Schmidt, \textit{Comey Memorandum Says Trump Asked Him to End Flynn Investigation}, New York Times (May 16, 2017);
Matt Apuzzo et al., \textit{Trump Told Russians That Firing 'Nut Job' Comey Eased Pressure From Investigation}, New York Times (May 19, 2017).}

On June 8, 2017, Comey testified before Congress about his interactions with the President before his termination, including the request for loyalty, the request that Comey "let[] Flynn go," and the request that Comey "lift the cloud" over the presidency caused by the ongoing investigation.% 544
\footnote{\textit{Hearing on Russian Election Interference Before the Senate Select Intelligence Committee}, 115th Cong.\ (June 8, 2017) (Statement for the Record of James B. Comey, former Director of the FBI, at 5-6).
Comey testified that he deliberately caused his memorandum documenting the February 14, 2017 meeting to be leaked to the New York Times in response to a tweet from the President, sent on May 12, 2017, that stated "James Comey better hope that there are no 'tapes' of our conversations before he starts leaking to the press!," and because he thought sharing the memorandum with a reporter "might prompt the appointment of a special counsel."
\textit{Hearing on Russian Election Interference Before the Senate Select Intelligence Committee}, 115th Cong.\ (June 8, 2017) (CQ Cong.\ Transcripts, at 55) (testimony by James B. Comey, former Director of the FBI).}
Comey's testimony led to a series of news reports about whether the President had obstructed justice.% 545
\footnote{\textit{See, e.g.}, Matt Zapotosky, \textit{Comey lays out the case that Trump obstructed justice}, Washington Post (June 8, 2017) ("Legal analysts said Comey's testimony clarified and bolstered the case that the president obstructed justice.").}
On June 9, 2017, the Special Counsel's Office informed the White House Counsel's Office that investigators intended to interview intelligence community officials who had allegedly been asked by the President to push back against the Russia investigation.% 546
\footnote{6/9/17 Email, Special Counsel's Office to the White House Counsel's Office.
This Office made the notification to give the White House an opportunity to invoke executive privilege in advance of the interviews.
On June 12, 2017, the Special Counsel's Office interviewed Admiral Rogers in the presence of agency counsel.
Rogers 6/12/17 302, at 1.
On June 13, the Special Counsel's Office interviewed Ledgett.
Ledgett 6/13/17 302, at 1.
On June 14, the Office interviewed Coats and other personnel from his office.
Coats 6/14/17 302, at 1;
Gistaro 6/14/17 302, at 1;
Culver 6/14/17 302, at 1.}

On Monday, June 12, 2017, Christopher Ruddy, the chief executive of Newsmax Media and a longtime friend of the President's, met at the White House with Priebus and Bannon.% 547
\footnote{Ruddy 6/6/18 302, at 5.}
Ruddy recalled that they told him the President was strongly considering firing the Special Counsel and that he would do so precipitously, without vetting the decision through Administration officials.% 548
\footnote{Ruddy 6/6/18 302, at 5-6.}
Ruddy asked Priebus if Ruddy could talk publicly about the discussion they had about the Special Counsel, and Priebus said he could.% 549
\footnote{Ruddy 6/6/18 302, at 6.}
Priebus told Ruddy he hoped another blow up like the one that followed the termination of Comey did not happen.% 550
\footnote{Ruddy 6/6/18 302, at 6.}
Later that day, Ruddy stated in a televised interview that the President was "considering perhaps terminating the Special Counsel "based on purported conflicts of interest.% 551
\footnote{\textit{Trump Confidant Christopher Ruddy says Mueller has "real conflicts" as special counsel}, PBS (June 12, 2017);
Michael D. Shear \& Maggie Haberman, \textit{Friend Says Trump Is Considering Firing Mueller as Special Counsel}, New York Times (June 12, 2017).}
Ruddy later told another news outlet that "Trump is definitely considering" terminating the Special Counsel and "it's not something that's being dismissed."% 552
\footnote{Katherine Faulders \& Veronica Stracqualursi, \textit{Trump friend Chris Ruddy says Spicer's 'bizarre' statement doesn't deny claim Trump seeking Mueller firing}, ABC (June 13, 2017).}
Ruddy's comments led to extensive coverage in the media that the President was considering firing the Special Counsel.% 553
\footnote{\textit{See, e.g.}, Michael D. Shear \& Maggie Haberman, \textit{Friend Says Trump Is Considering Firing Mueller as Special Counsel}, New York Times (June 12, 2017).}

White House officials were unhappy with that press coverage and Ruddy heard from friends that the President was upset with him.% 554
\footnote{Ruddy 6/6/18 302, at 6-7.}
On June 13, 2017, Sanders asked the President for guidance on how to respond to press inquiries about the possible firing of the Special Counsel.% 555
\footnote{Sanders 7/3/18 302, at 6-7.}
The President dictated an answer, which Sanders delivered, saying that "[w]hile the president has every right to" fire the Special Counsel, "he has no intention to do so."% 556
\footnote{Glenn Thrush et al., \textit{Trump Stews, Staff Steps In, and Mueller Is Safe for Now}, New York Times (June 13, 2017);
\textit{see} Sanders 7/3/18 302, at 6 (Sanders spoke with the President directly before speaking to the press on Air Force One and the answer she gave is the answer the President told her to give).}

Also on June 13, 2017, the President's personal counsel contacted the Special Counsel's Office and raised concerns about possible conflicts.% 557
\footnote{Special Counsel's Office Attorney 6/13/17 Notes.}
The President's counsel cited Mueller's previous partnership in his law firm, his interview for the FBI Director position, and an asserted personal relationship he had with Comey.% 558
\footnote{Special Counsel's Office Attorney 6/13/17 Notes.}
That same day, Rosenstein had testified publicly before Congress and said he saw no evidence of good cause to terminate the Special Counsel, including for conflicts of interest.% 559
\footnote{\textit{Hearing on Fiscal 2018 Justice Department Budget before the Senate Appropriations Subcommittee on Commerce, Justice, and Science}, 115th Cong.\ (June 13, 2017) (CQ Cong.\ Transcripts, at 14) (testimony by Rod Rosenstein, Deputy Attorney General).}
Two days later, on June 15, 2017, the Special Counsel's Office informed the Acting Attorney General's office about the areas of concern raised by the President's counsel and told the President's counsel that their concerns had been communicated to Rosenstein so that the Department of Justice could take any appropriate action.% 560
\footnote{Special Counsel's Office Attorney 6/15/17 Notes.}

\subsubsection{The Press Reports that the President is Being Investigated for Obstruction of Justice and the President Directs the White House Counsel to Have the Special Counsel Removed}

On the evening of June 14, 2017, the Washington Post published an article stating that the Special Counsel was investigating whether the President had attempted to obstruct justice.% 561
\footnote{Devlin Barrett et al., \textit{Special counsel is investigating Trump for possible obstruction of justice, officials say}, Washington Post (June 14, 2017).}
This was the first public report that the President himself was under investigation by the Special Counsel's Office, and cable news networks quickly picked up on the report.% 562
\footnote{CNN, for example, began running a chyron at 6:55 p.m.\ that stated: "WASH POST: MUELLER INVESTIGATING TRUMP FOR OBSTRUCTION OF JUSTICE." CNN, (June 14, 2017, published online at 7:15 p.m.~ET).}
The Post story stated that the Special Counsel was interviewing intelligence community leaders, including Coats and Rogers, about what the President had asked them to do in response to Comey's March 20, 2017 testimony;
that the inquiry into obstruction marked "a major turning point" in the investigation;
and that while "Trump had received private assurances from then-FBI Director James B. Comey starting in January that he was not personally under investigation," "[o]fficials say that changed shortly after Comey's firing."% 563
\footnote{Devlin Barrett et al., \textit{Special counsel is investigating Trump for possible obstruction of justice}, officials say, Washington Post (June 14, 2017).}
That evening, at approximately 10:31 p.m., the President called McGahn on McGahn's personal cell phone and they spoke for about 15 minutes.% 564
\footnote{SCR026\_000183 (President's Daily Diary, 6/14/17) (reflecting call from the President to McGahn on 6/14/17 with start time 10:31 p.m.\ and end time 10:46 p.m.);
Call Records of Don McGahn.}
McGahn did not have a clear memory of the call but thought they might have discussed the stories reporting that the President was under investigation.% 565
\footnote{McGahn 2/28/19 302, at 1-2.
McGahn thought he and the President also probably talked about the investiture ceremony for Supreme Court Justice Neil Gorsuch, which was scheduled for the following day.
McGahn 2/28/18 302, at 2.}

Beginning early the next day, June 15, 2017, the President issued a series of tweets acknowledging the existence of the obstruction investigation and criticizing it.
He wrote: "They made up a phony collusion with the Russians story, found zero proof, so now they go for obstruction of justice on the phony story.
Nice";% 566
\footnote{\@realDonaldTrump 6/15/17 (6:55 a.m.~ET) Tweet.}
"You are witnessing the single greatest WITCH HUNT in American political history - led by some very bad and conflicted people!";% 567
\footnote{\@realDonaldTrump6/15/17 (7:57 a.m.~ET) Tweet.}
and "Crooked H destroyed phones w/ hammer, 'bleached' emails, \& had husband meet w/AG days before she was cleared - \& they talk about obstruction?"% 568
\footnote{\@realDonaldTrump6/15/17 (3:56 p.m.~ET) Tweet.}
The next day, June 16, 2017, the President wrote additional tweets criticizing the investigation:
"After 7 months of investigations \& committee hearings about my 'collusion with the Russians,' nobody has been able to show any proof, Sad!";% 569
\footnote{\@realDonaldTrump 6/16/17 (7:53 a.m.~ET) Tweet.}
and "I am being investigated for firing the FBI Director by the man who told me to fire the FBI Director!
Witch Hunt."% 570
\footnote{\@realDonaldTrump6/16/17 (9:07 a.m.~ET) Tweet.}

On Saturday, June 17, 2017, the President called McGahn and directed him to have the Special Counsel removed.% 571
\footnote{McGahn 3/8/18 302, at 1-2;
McGahn 12/14/17 302, at 10.}
McGahn was at home and the President was at Camp David.% 572
\footnote{McGahn 3/8/18 302, at 1, 3;
SCR026\_000196 (President's Daily Diary, 6/17/17) (records showing President departed the White House at 11:07 a.m.\ on June 17, 2017, and arrived at Camp David at 11:37 a.m.).}
In interviews with this Office, McGahn recalled that the President called him at home twice and on both occasions directed him to call Rosenstein and say that Mueller had conflicts that precluded him from serving as Special Counsel.% 573
\footnote{McGahn 3/8/18 302, at 1-2;
McGahn 12/14/17 302, at 10.
Phone records show that the President called McGahn in the afternoon on June 17, 2017, and they spoke for approximately 23 minutes.
SCR026\_000196 (President's Daily Diary, 6/17/17) (reflecting call from the President to McGahn on 6/17/17 with start time 2:23 p.m.\ and end time 2:46 p.m.);
(Call Records of Don McGahn).
Phone records do not show another call between McGahn and the President that day.
Although McGahn recalled receiving multiple calls from the President on the same day, in light of the phone records he thought it was possible that the first call instead occurred on June 14, 2017, shortly after the press reported that the President was under investigation for obstruction of justice.
McGahn 2/28/19 302, at 1-3.
While McGahn was not certain of the specific dates of the calls, McGahn was confident that he had at least two phone conversations with the President in which the President directed him to call the Acting Attorney General to have the Special Counsel removed.
McGahn 2/28/19 302, at 1-3.}

On the first call, McGahn recalled that the President said something like, "You gotta do this.
You gotta call Rod."% 574
\footnote{McGahn 3/8/18 302, at 1.}
McGahn said he told the President that he would see what he could do.% 575
\footnote{McGahn 3/8/18 302, at 1.}
McGahn was perturbed by the call and did not intend to act on the request.% 576
\footnote{McGahn 3/8/18 302, at 1.}
He and other advisors believed the asserted conflicts were "silly" and "not real," and they had previously communicated that view to the President.% 577
\footnote{McGahn 3/8/18 302, at 1-2.}
McGahn also had made clear to the President that the White House Counsel's Office should not be involved in any effort to press the issue of conflicts.% 578
\footnote{McGahn 3/8/18 302, at 1-2.}
McGahn was concerned about having any role in asking the Acting Attorney General to fire the Special Counsel because he had grown up in the Reagan era and wanted to be more like Judge Robert Bork and not "Saturday Night Massacre Bork."% 579
\footnote{McGahn 3/8/18 302, at 2.}
McGahn considered the President's request to be an inflection point and he wanted to hit the brakes.% 580
\footnote{McGahn 3/8/18 302, at 2.}

When the President called McGahn a second time to follow up on the order to call the Department of Justice, McGahn recalled that the President was more direct, saying something like, "Call Rod, tell Rod that Mueller has conflicts and can't be the Special Counsel."% 581
\footnote{McGahn 3/8/18 302, at 5.}
McGahn recalled the President telling him "Mueller has to go" and "Call me back when you do it."% 582
\footnote{McGahn 3/8/18 302, at 2, 5;
McGahn 2/28/19 302, at 3.}
McGahn understood the President to be saying that the Special Counsel had to be removed by Rosenstein?% 583
\footnote{McGahn 3/8/18 302, at 1-2, 5.}
To end the conversation with the President, McGahn left the President with the impression that McGahn would call Rosenstein.% 584
\footnote{McGahn 3/8/18 302, at 2.}
McGahn recalled that he had already said no to the President's request and he was worn down, so he just wanted to get off the phone.% 585
\footnote{McGahn 2/28/19 302, at 3;
McGahn 3/8/18 302, at 2.}

McGahn recalled feeling trapped because he did not plan to follow the President's directive but did not know what he would say the next time the President called.% 586
\footnote{McGahn 3/8/18 302, at 2.}
McGahn decided he had to resign.% 587
\footnote{McGahn 3/8/18 302, at 2.}
He called his personal lawyer and then called his chief of staff, Annie Donaldson, to inform her of his decision.% 588
\footnote{McGahn 3/8/18 302, at 2-3;
McGahn 2/28/19 302, at 3;
Donaldson 4/2/18 302, at 4;
Call Records of Don McGahn.}
He then drove to the office to pack his belongings and submit his resignation letter.% 589
\footnote{McGahn 3/8/18 302, at 2;
Donaldson 4/2/18 302, at 4.}
Donaldson recalled that McGahn told her the President had called and demanded he contact the Department of Justice and that the President wanted him to do something that McGahn did not want to do.% 590
\footnote{Donaldson 4/2/18 302, at 4.}
McGahn told Donaldson that the President had called at least twice and in one of the calls asked "have you done it?"% 591
\footnote{Donaldson 4/2/18 302, at 4.}
McGahn did not tell Donaldson the specifics of the President's request because he was consciously trying not to involve her in the investigation, but Donaldson inferred that the President's directive was related to the Russia investigation.% 592
\footnote{McGahn 2/28/19 302, at 3-4;
Donaldson 4/2/18 302, at 4-5.
Donaldson said she believed McGahn consciously did not share details with her because he did not want to drag her into the investigation.
Donaldson 4/2/18 302, at 5;
\textit{see} McGahn 2/28/19 302, at 3.}
Donaldson prepared to resign along with McGahn.% 593
\footnote{Donaldson 4/2/18 302, at 5.}

That evening, McGahn called both Priebus and Bannon and told them that he intended to resign.% 594
\footnote{McGahn 12/14/17 302, at 10;
Call Records of Don McGahn;
McGahn 2/28/19 302, at 3-4;
Priebus 4/3/18 302, at 6-7.}
McGahn recalled that, after speaking with his attorney and given the nature of the President's request, he decided not to share details of the President's request with other White House staff.% 595
\footnote{McGahn 2/28/19 302, at 4.
Priebus and Bannon confirmed that McGahn did not tell them the specific details of the President's request.
Priebus 4/3/18 302, at 7;
Bannon 2/14/18 302, at 10.}
Priebus recalled that McGahn said that the President had asked him to "do crazy shit," but he thought McGahn did not tell him the specifics of the President's request because McGahn was trying to protect Priebus from what he did not need to know.% 596
\footnote{Priebus 4/3/18 302, at 7.}
Priebus and Bannon both urged McGahn not to quit, and McGahn ultimately returned to work that Monday and remained in his position.% 597
\footnote{McGahn 3/8/18 302, at 3;
McGahn 2/28/19 302, at 3-4.}
He had not told the President directly that he planned to resign, and when they next saw each other the President did not ask McGahn whether he had followed through with calling Rosenstein.% 598
\footnote{McGahn 3/8/18 302, at 3.}

Around the same time, Chris Christie recalled a telephone call with the President in which the President asked what Christie thought about the President firing the Special Counsel.% 599
\footnote{Christie 2/13/19 302, at 7.
Christie did not recall the precise date of this call, but believed it was after Christopher Wray was announced as the nominee to be the new FBI director, which was on June 7, 2017.
Christie 2/13/19 302, at 7.
Telephone records show that the President called Christie twice after that time period, on July 4, 2017, and July 14, 2017.
Call Records of Chris Christie.}
Christie advised against doing so because there was no substantive basis for the President to fire the Special Counsel, and because the President would lose support from Republicans in Congress if he did so.% 600
\footnote{Christie 2/13/19 302, at 7.}

\begin{center}
\textbf{Analysis}
\end{center}

In analyzing the President's direction to McGahn to have the Special Counsel removed, the following evidence is relevant to the elements of obstruction of justice:

\underline{Obstructive act.}
As with the President's firing of Comey, the attempt to remove the Special Counsel would qualify as an obstructive act if it would naturally obstruct the investigation and any grand jury proceedings that might flow from the inquiry.
Even if the removal of the lead prosecutor would not prevent the investigation from continuing under a new appointee, a fact finder would need to consider whether the act had the potential to delay further action in the investigation, chill the actions of any replacement Special Counsel, or otherwise impede the investigation.

A threshold question is whether the President in fact directed McGahn to have the Special Counsel removed.
After news organizations reported that in June 2017 the President had ordered McGahn to have the
Special Counsel removed, the President publicly disputed these accounts, and privately told McGahn that he had simply wanted McGahn to bring conflicts of interest to the Department of Justice's attention.
See Volume II, Section III, infra.
Some of the President's specific language that McGahn recalled from the calls is consistent with that explanation.
Substantial evidence, however, supports the conclusion that the President went further and in fact directed McGahn to call Rosenstein to have the Special Counsel removed.

First, McGahn's clear recollection was that the President directed him to tell Rosenstein not only that conflicts existed but also that "Mueller has to go."
McGahn is a credible witness with no motive to lie or exaggerate given the position he held in the White House.% 601
\footnote{When this Office first interviewed McGahn about this topic, he was reluctant to share detailed information about what had occurred and only did so after continued questioning.
See McGahn 12/14/17 302 (agent notes).}
McGahn spoke with the President twice and understood the directive the same way both times, making it unlikely that he misheard or misinterpreted the President's request.
In response to that request, McGahn decided to quit because he did not want to participate in events that he described as akin to the Saturday Night Massacre.
He called his lawyer, drove to the White House, packed up his office, prepared to submit a resignation letter with his chief of staff, told Priebus that the President had asked him to "do crazy shit," and informed Priebus and Bannon that he was leaving.
Those acts would be a highly unusual reaction to a request to convey information to the Department of Justice.

Second, in the days before the calls to McGahn, the President, through his counsel, had already brought the asserted conflicts to the attention of the Department of Justice.
Accordingly, the President had no reason to have McGahn call Rosenstein that weekend to raise conflicts issues that already had been raised.

Third, the President's sense of urgency and repeated requests to McGahn to take immediate action on a weekend - "You gotta do this.
You gotta call Rod."
- support McGahn's recollection that the President wanted the Department of Justice to take action to remove the Special Counsel.
Had the President instead sought only to have the Department of Justice re-examine asserted conflicts to evaluate whether they posed an ethical bar, it would have been unnecessary to set the process in motion on a Saturday and to make repeated calls to McGahn.

Finally, the President had discussed "knocking out Mueller" and raised conflicts of interest in a May 23, 2017 call with McGahn, reflecting that the President connected the conflicts to a plan to remove the Special Counsel.
And in the days leading up to June 17, 2017, the President made clear to Priebus and Bannon, who then told Ruddy, that the President was considering terminating the Special Counsel.
Also during this time period, the President reached out to Christie to get his thoughts on firing the Special Counsel.
This evidence shows that the President was not just seeking an examination of whether conflicts existed but instead was looking to use asserted conflicts as a way to terminate the Special Counsel.

\underline{Nexus to an official proceeding.}
To satisfy the proceeding requirement, it would be necessary to establish a nexus between the President's act of seeking to terminate the Special Counsel and a pending or foreseeable grand jury proceeding.

Substantial evidence indicates that by June 17, 2017, the President knew his conduct was under investigation by a federal prosecutor who could present any evidence of federal crimes to a grand jury.
On May 23, 2017, McGahn explicitly warned the President that his "biggest exposure" was not his act of firing Comey but his "other contacts" and "calls," and his "ask re: Flynn."
By early June, it was widely reported in the media that federal prosecutors had issued grand jury subpoenas in the Flynn inquiry and that the Special Counsel had taken over the Flynn investigation.% 602
\footnote{\textit{See, e.g.}, Evan Perez et al., \textit{CNN exclusive: Grand jury subpoenas issued in FBI's Russia investigation}, CNN (May 9, 2017);
Matt Ford, \textit{Why Mueller Is Taking Over the Michael Flynn Grand Jury}, The Atlantic (June 2, 2017).}
On June 9, 2017, the Special Counsel's Office informed the White House that investigators would be interviewing intelligence agency officials who allegedly had been asked by the President to push back against the Russia investigation.
On June 14, 2017, news outlets began reporting that the President was himself being investigated for obstruction of justice.
Based on widespread reporting, the President knew that such an investigation could include his request for Comey's loyalty;
his request that Comey "let[] Flynn go";
his outreach to Coats and Rogers;
and his termination of Comey and statement to the Russian Foreign Minister that the termination had relieved "great pressure" related to Russia.
And on June 16, 2017, the day before he directed McGahn to have the Special Counsel removed, the President publicly acknowledged that his conduct was under investigation by a federal prosecutor, tweeting, "I am being investigated for firing the FBI Director by the man who told me to fire the FBI Director!"

\underline{Intent.}
Substantial evidence indicates that the President's attempts to remove the Special Counsel were linked to the Special Counsel's oversight of investigations that involved the President's conduct - and, most immediately, to reports that the President was being investigated for potential obstruction of justice.

Before the President terminated Comey, the President considered it critically important that he was not under investigation and that the public not erroneously think he was being investigated.
As described in Volume II, Section II.D, supra, advisors perceived the President, while he was drafting the Comey termination letter, to be concerned more than anything else about getting out that he was not personally under investigation.
When the President learned of the appointment of the Special Counsel on May 17, 2017, he expressed further concern about the investigation, saying "[t]his is the end of my Presidency."
The President also faulted Sessions for recusing, saying "you were supposed to protect me."

On June 14, 2017, when the Washington Post reported that the Special Counsel was investigating the President for obstruction of justice, the President was facing what he had wanted to avoid: a criminal investigation into his own conduct that was the subject of widespread media attention.
The evidence indicates that news of the obstruction investigation prompted the President to call McGahn and seek to have the Special Counsel removed.
By mid-June, the Department of Justice had already cleared the Special Counsel's service and the President's advisors had told him that the claimed conflicts of interest were "silly" and did not provide a basis to remove the Special Counsel.
On June 13, 2017, the Acting Attorney General testified before Congress that no good cause for removing the Special Counsel existed, and the President dictated a press statement to Sanders saying he had no intention of firing the Special Counsel.
But the next day, the media reported that the President was under investigation for obstruction of justice and the Special Counsel was interviewing witnesses about events related to possible obstruction - spurring the President to write critical tweets about the Special Counsel's investigation.
The President called McGahn at home that night and then called him on Saturday from Camp David.
The evidence accordingly indicates that news that an obstruction investigation had been opened is what led the President to call McGahn to have the Special Counsel terminated.

There also is evidence that the President knew that he should not have made those calls to McGahn.
The President made the calls to McGahn after McGahn had specifically told the President that the White House Counsel's Office - and McGahn himself - could not be involved in pressing conflicts claims and that the President should consult with his personal counsel if he wished to raise conflicts.
Instead of relying on his personal counsel to submit the conflicts claims, the President sought to use his official powers to remove the Special Counsel.
And after the media reported on the President's actions, he denied that he ever ordered McGahn to have the Special Counsel terminated and made repeated efforts to have McGahn deny the story, as discussed in Volume II, Section III, infra.
Those denials are contrary to the evidence and suggest the President's awareness that the direction to McGahn could be seen as improper.

\subsection{The President's Efforts to Curtail the Special Counsel Investigation}

\begin{center}
\textbf{Overview}
\end{center}

Two days after the President directed McGahn to have the Special Counsel removed, the President made another attempt to affect the course of the Russia investigation.
On June 19, 2017, the President met one-on-one with Corey Lewandowski in the Oval Office and dictated a message to be delivered to Attorney General Sessions that would have had the effect of limiting the Russia investigation to future election interference only.
One month later, the President met again with Lewandowski and followed up on the request to have Sessions limit the scope of the Russia investigation.
Lewandowski told the President the message would be delivered soon.
Hours later, the President publicly criticized Sessions in an unplanned press interview, raising questions about Sessions's job security.

\subsubsection{The President Asks Corey Lewandowski to Deliver a Message to Sessions to Curtail the Special Counsel Investigation}

On June 19, 2017, two days after the President directed McGahn to have the Special Counsel removed, the President met one-on-one in the Oval Office with his former campaign manager Corey Lewandowski.% 603
\footnote{Lewandowski 4/6/18 302, at 2;
SCR026\_000201 (President's Daily Diary, 6/19/17.
\blackout{Personal Privacy}}
Senior White House advisors described Lewandowski as a "devotee" of the President and said the relationship between the President and Lewandowski was "close."% 604
\footnote{Kelly 8/2/18 302, at 7;
Dearborn 6/20/18 302, at 1 (describing Lewandowski as a "comfort to the President" whose loyalty was appreciated).
Kelly said that when he was Chief of Staff and the President had meetings with friends like Lewandowski, Kelly tried not to be there and to push the meetings to the residence to create distance from the West Wing.
Kelly 8/2/18 302, at 7.}

During the June 19 meeting, Lewandowski recalled that, after some small talk, the President brought up Sessions and criticized his recusal from the Russia investigation.% 605
\footnote{Lewandowski 4/6/18 302, at 2.}
The President told Lewandowski that Sessions was weak and that if the President had known about the likelihood of recusal in advance, he would not have appointed Sessions.% 606
\footnote{Lewandowski 4/6/18 302, at 2.}
The President then asked Lewandowski to deliver a message to Sessions and said "write this down."% 607
\footnote{Lewandowski 4/6/18 302, at 2.}
This was the first time the President had asked Lewandowski to take dictation, and Lewandowski wrote as fast as possible to make sure he captured the content correctly.% 608
\footnote{Lewandowski 4/6/18 302, at 3.}

The President directed that Sessions should give a speech publicly announcing:

\begin{quote}
I know that I recused myself from certain things having to do with specific areas.
But our POTUS... is being treated very unfairly.
He shouldn't have a Special Prosecutor/Counsel b/c he hasn't done anything wrong.
I was on the campaign w/ him for nine months, there were no Russians involved with him.
I know it for a fact b/c I was there.
He didn't do anything wrong except he ran the greatest campaign in American history.% 609
\footnote{Lewandowski 4/6/18 302, at 2-3;
Lewandowski 6/19/17 Notes, at 1-2.}
\end{quote}

The dictated message went on to state that Sessions would meet with the Special Counsel to limit his jurisdiction to future election interference:

\begin{quote}
Now a group of people want to subvert the Constitution of the United States.
I am going to meet with the Special Prosecutor to explain this is very unfair and let the Special Prosecutor move forward with investigating election meddling for future elections so that nothing can happen in future elections.% 610
\footnote{Lewandowski 4/6/18 302, at 3;
Lewandowski 6/19/17 Notes, at 3.}
\end{quote}

The President said that if Sessions delivered that statement he would be the "most popular guy in the country."% 611
\footnote{Lewandowski 4/6/18 302, at 3;
Lewandowski 6/19/17 Notes, at 4.}
Lewandowski told the President he understood what the President wanted Sessions to do.% 612
\footnote{Lewandowski 4/6/18 302, at 3.}

Lewandowski wanted to pass the message to Sessions in person rather than over the phone.% 613
\footnote{Lewandowski 4/6/18 302, at 3-4.}
He did not want to meet at the Department of Justice because he did not want a public log of his visit and did not want Sessions to have an advantage over him by meeting on what Lewandowski described as Sessions's turf.% 614
\footnote{Lewandowski 4/6/18 302, at 4.}
Lewandowski called Sessions and arranged a meeting for the following evening at Lewandowski's office, but Sessions had to cancel due to a last minute conflict.% 615
\footnote{Lewandowski 4/6/18 302, at 4.}
Shortly thereafter, Lewandowski left Washington, D.C., without having had an opportunity to meet with Sessions to convey the President's message.% 616
\footnote{Lewandowski 4/6/18 302, at 4.}
Lewandowski stored the notes in a safe at his home, which he stated was his standard procedure with sensitive items.% 617
\footnote{Lewandowski 4/6/18 302, at 4.}

\subsubsection{The President Follows Up with Lewandowski}

Following his June meeting with the President, Lewandowski contacted Rick Dearborn, then a senior White House official, and asked if Dearborn could pass a message to Sessions.% 618
\footnote{Lewandowski 4/6/18 302, at 4;
\textit{see} Dearborn 6/20/18 302, at 3.}
Dearborn agreed without knowing what the message was, and Lewandowski later confirmed that Dearborn would meet with Sessions for dinner in late July and could deliver the message then.% 619
\footnote{Lewandowski 4/6/18 302, at 4-5.}
Lewandowski recalled thinking that the President had asked him to pass the message because the President knew Lewandowski could be trusted, but Lewandowski believed Dearborn would be a better messenger because he had a longstanding relationship with Sessions and because Dearborn was in the government while Lewandowski was not.% 620
\footnote{Lewandowski 4/6/18 302, at 4, 6.}

On July 19, 2017, the President again met with Lewandowski alone in the Oval Office.% 621
\footnote{Lewandowski 4/6/18 302, at 5;
SCR029b\_000002-03 (6/5/18 Additional Response to Special Counsel Request for Certain Visitor Log Information).}
In the preceding days, as described in Volume II, Section II.G, infra, emails and other information about the June 9, 2016 meeting between several Russians and Donald Trump Jr., Jared Kushner, and Paul Manafort had been publicly disclosed.
In the July 19 meeting with Lewandowski, the President raised his previous request and asked if Lewandowski had talked to Sessions.% 622
\footnote{Lewandowski 4/6/18 302, at 5.}
Lewandowski told the President that the message would be delivered soon.% 623
\footnote{Lewandowski 4/6/18 302, at 5.}
Lewandowski recalled that the President told him that if Sessions did not meet with him, Lewandowski should tell Sessions he was fired.% 624
\footnote{Lewandowski 4/6/18 302, at 6.
Priebus vaguely recalled Lewandowski telling him that in approximately May or June 2017 the President had asked Lewandowski to get Sessions's resignation.
Priebus recalled that Lewandowski described his reaction as something like, "What can I do?
I'm not an employee of the administration.
I'm a nobody."
Priebus 4/3/18 302, at 6.}

Immediately following the meeting with the President, Lewandowski saw Dearborn in the anteroom outside the Oval Office and gave him a typewritten version of the message the President had dictated to be delivered to Sessions.% 625
\footnote{Lewandowski 4/6/18 302, at 5.
Lewandowski said he asked Hope Hicks to type the notes when he went in to the Oval Office, and he then retrieved the notes from her partway through his meeting with the President.
Lewandowski 4/6/18 302, at 5.}
Lewandowski told Dearborn that the notes were the message they had discussed, but Dearborn did not recall whether Lewandowski said the message was from the President.% 626
\footnote{Lewandowski 4/6/18 302, at 5;
Dearborn 6/20/18 302, at 3.}
The message "definitely raised an eyebrow" for Dearborn, and he recalled not wanting to ask where it came from or think further about doing anything with it.% 627
\footnote{Dearborn 6/20/18 302, at 3.}
Dearborn also said that being asked to serve as a messenger to Sessions made him uncomfortable.% 628
\footnote{Dearborn 6/20/18 302, at 3.}
He recalled later telling Lewandowski that he had handled the situation, but he did not actually follow through with delivering the message to Sessions, and he did not keep a copy of the typewritten notes Lewandowski had given him.% 629
\footnote{Dearborn 6/20/18 302, at 3-4.}

\subsubsection{The President Publicly Criticizes Sessions in a New York Times Interview}

Within hours of the President's meeting with Lewandowski on July 19, 2017, the President gave an unplanned interview to the New York Times in which he criticized Sessions's decision to recuse from the Russia investigation.% 630
\footnote{Peter Baker et al., \textit{Excerpts From The Times's Interview With Trump}, New York Times (July 19, 2017).}
The President said that "Sessions should have never recused himself, and if he was going to recuse himself, he should have told me before he took the job, and I would have picked somebody else."% 631
\footnote{Peter Baker et al., \textit{Excerpts From The Times's Interview With Trump}, New York Times (July 19, 2017).}
Sessions's recusal, the President said, was "very unfair to the president.
How do you take a job and then recuse yourself?
If he would have recused himself before the job, I would have said, 'Thanks, Jeff, but I can't, you know, I'm not going to take you.'
It's extremely unfair, and that's a mild word, to the president."% 632
\footnote{Deter Baker et al., \textit{Excerpts From The Times's Interview With Trump}, New York Times (July 19, 2017).}
Hicks, who was present for the interview, recalled trying to "throw [herself] between the reporters and [the President]" to stop parts of the interview, but the President "loved the interview."% 633
\footnote{Hicks 12/8/17 302, at 23.}

Later that day, Lewandowski met with Hicks and they discussed the President's New York Times interview.% 634
\footnote{Hicks 3/13/18 302, at 10;
Lewandowski 4/6/18 302, at 6.}
Lewandowski recalled telling Hicks about the President's request that he meet with Sessions and joking with her about the idea of firing Sessions as a private citizen if Sessions would not meet with him.% 635
\footnote{Lewandowski 4/6/18 302, at 6.}
As Hicks remembered the conversation, Lewandowski told her the President had recently asked him to meet with Sessions and deliver a message that he needed to do the "right thing" and resign.% 636
\footnote{Hicks 3/13/18 302, at 10.
Hicks thought that the President might be able to make a recess appointment of a new Attorney General because the Senate was about to go on recess.
Hicks 3/13/18 302, at 10.
Lewandowski recalled that in the afternoon of July 19, 2017, following his meeting with the President, he conducted research on recess appointments but did not share his research with the President.
Lewandowski 4/6/18 302, at 7.}
While Hicks and Lewandowski were together, the President called Hicks and told her he was happy with how coverage of his New York Times interview criticizing Sessions was playing out.% 637
\footnote{Lewandowski 4/6/18 302, at 6.}

\subsubsection{The President Orders Priebus to Demand Sessions's Resignation}

Three days later, on July 21, 2017, the Washington Post reported that U.S. intelligence intercepts showed that Sessions had discussed campaign-related matters with the Russian ambassador, contrary to what Sessions had said publicly.% 638
\footnote{Adam Entous et al., \textit{Sessions discussed Trump campaign-related matters with Russian ambassador, U.S. intelligence intercepts show}, Washington Post (July 21, 2017).
The underlying events concerning the Sessions-Kislyak contacts are discussed in Volume I, Section IV.A.4.c, \textit{supra}.}
That evening, Priebus called Hunt to talk about whether Sessions might be fired or might resign.% 639
\footnote{Hunt 2/1/18 302, at 23.}
Priebus had previously talked to Hunt when the media had reported on tensions between Sessions and the President, and, after speaking to Sessions, Hunt had told Priebus that the President would have to fire Sessions if he wanted to remove Sessions because Sessions was not going to quit.% 640
\footnote{Hunt 2/1/18 302, at 23.}
According to Hunt, who took contemporaneous notes of the July 21 call, Hunt told Priebus that, as they had previously discussed, Sessions had no intention of resigning.% 641
\footnote{Hunt 2/1/18 302, at 23-24;
Hunt 7/21/17 Notes, at 1.}
Hunt asked Priebus what the President would accomplish by firing Sessions, pointing out there was an investigation before and there would be an investigation after.% 642
\footnote{Hunt 2/1/18 302, at 23-24;
Hunt 7/21/17 Notes, at 1-2.}

Early the following morning, July 22, 2017, the President tweeted, "A new INTELLIGENCE LEAK from the Amazon Washington Post, this time against A. G. Jeff Sessions.
These illegal leaks, like Comey's, must stop!"% 643
\footnote{\@realDonaldTrump 7/22/17 (6:33 a.m.~ET) Tweet.}
Approximately one hour later, the President tweeted, "So many people are asking why isn't the A.G. or Special Council looking at the many Hillary Clinton or Comey crimes.
33,000 e-mails deleted?"% 644
\footnote{\@realDonaldTrump 7/22/17 (7:44 a.m.~ET) Tweet.
Three minutes later, the President tweeted, "What about all of the Clinton ties to Russia, including Podesta Company, Uranium deal, Russian Reset, big dollar speeches etc."
\@realDonaldTrump 7/22/17 (7:47 a.m.~ET) Tweet.}
Later that morning, while aboard Marine One on the way to Norfolk, Virginia, the President told Priebus that he had to get Sessions to resign immediately.% 645
\footnote{Priebus 1/18/18 302, at 13-14.}
The President said that the country had lost confidence in Sessions and the negative publicity was not tolerable.% 646
\footnote{Priebus 1/18/18 302, at 14;
Priebus 4/3/18 302, at 4-5;
\textit{see} RP\_000073 (Priebus 7/22/17 Notes).}
According to contemporaneous notes taken by Priebus, the President told Priebus to say that he "need[ed]a letter of resignation on [his] desk immediately" and that Sessions had "no choice" but "must immediately resign."% 647
\footnote{RP\_000073 (Priebus 7/22/17 Notes).}
Priebus replied that if they fired Sessions, they would never get a new Attorney General confirmed and that the Department of Justice and Congress would turn their backs on the President, but the President suggested he could make a recess appointment to replace Sessions.% 648
\footnote{Priebus 4/3/18 302, at 5.}

Priebus believed that the President's request was a problem, so he called McGahn and asked for advice, explaining that he did not want to pull the trigger on something that was "all wrong."% 649
\footnote{Priebus 1/18/18 302, at 14;
Priebus 4/3/18 302, at 4-5.}
Although the President tied his desire for Sessions to resign to Sessions's negative press and poor performance in congressional testimony, Priebus believed that the President's desire to replace Sessions was driven by the President's hatred of Sessions's recusal from the Russia investigation.% 650
\footnote{Priebus 4/3/18 302, at 5.}
McGahn told Priebus not to follow the President's order and said they should consult their personal counsel, with whom they had attorney-client privilege.% 651
\footnote{RP\_000074 (Priebus 7/22/17 Notes);
McGahn 12/14/17 302, at 11;
Priebus 1/18/18 302, at 14.
Priebus followed McGahn's advice and called his personal attorney to discuss the President's request because he thought it was the type of thing about which one would need to consult an attorney.
Priebus 1/18/18 302, at 14.}
McGahn and Priebus discussed the possibility that they would both have to resign rather than carry out the President's order to fire Sessions.% 652
\footnote{McGahn 12/14/17 302, at 11;
RP\_000074 (Priebus 7/22/17 Notes) ("discuss resigning together').}

That afternoon, the President followed up with Priebus about demanding Sessions's resignation, using words to the effect of, "Did you get it?
Are you working on it?"% 653
\footnote{Priebus 1/18/18 302, at 14;
Priebus 4/3/18 302, at 4.}
Priebus said that he believed that his job depended on whether he followed the order to remove Sessions, although the President did not directly say so.% 654
\footnote{Priebus 4/3/18 302, at 4.}
Even though Priebus did not intend to carry out the President's directive, he told the President he would get Sessions to resign.% 655
\footnote{Priebus 1/18/18 302, at 15.}
Later in the day, Priebus called the President and explained that it would be a calamity if Sessions resigned because Priebus expected that Rosenstein and Associate Attorney General Rachel Brand would also resign and the President would be unable to get anyone else confirmed.% 656
\footnote{Priebus 1/18/18 302, at 15.}
The President agreed to hold off on demanding Sessions's resignation until after the Sunday shows the next day, to prevent the shows from focusing on the firing.% 657
\footnote{Priebus 1/18/18 302, at 15.}

By the end of that weekend, Priebus recalled that the President relented and agreed not to ask Sessions to resign.% 658
\footnote{Priebus 1/18/18 302, at 15.}
Over the next several days, the President tweeted about Sessions.
On the morning of Monday, July 24, 2017, the President criticized Sessions for neglecting to investigate Clinton and called him "beleaguered."% 659
\footnote{\@realDonaldTrump 7/24/17 (8:49 a.m.~ET) Tweet ("So why aren't the Committees and investigators, and of course our beleaguered A.G., looking into Crooked Hillarys crimes \& Russia relations?").}
On July 25, the President tweeted, "Attorney General Jeff Sessions has taken a VERY weak position on Hillary Clinton crimes (where are E-mails \& DNC server) \& Intel leakers!"% 660
\footnote{\@realDonaldTrump 7/25/17 (6:12 a.m.~ET) Tweet.
The President sent another tweet shortly before this one asking "where is the investigation A.G."
\@realDonaldTrump 7/25/17 (6:03 a.m.~ET) Tweet.}
The following day, July 26, the President tweeted, "Why didn't A.G. Sessions replace Acting FBI Director Andrew McCabe, a Comey friend who was in charge of Clinton investigation."% 661
\footnote{\@realDonaldTrump 7/26/17 (9:48 a.m.~ET) Tweet.}
According to Hunt, in light of the President's frequent public attacks, Sessions prepared another resignation letter and for the rest of the year carried it with him in his pocket every time he went to the White House.% 662
\footnote{Hunt 2/1/18 302, at 24-25.}

\begin{center}
\textbf{Analysis}
\end{center}

In analyzing the President's efforts to have Lewandowski deliver a message directing Sessions to publicly announce that the Special Counsel investigation would be confined to future election interference, the following evidence is relevant to the elements of obstruction of justice:

\underline{Obstructive act.}
The President's effort to send Sessions a message through Lewandowski would qualify as an obstructive act if it would naturally obstruct the investigation and any grand jury proceedings that might flow from the inquiry.

The President sought to have Sessions announce that the President "shouldn't have a Special Prosecutor/Counsel" and that Sessions was going to "meet with the Special Prosecutor to explain this is very unfair and let the Special Prosecutor move forward with investigating election meddling for future elections so that nothing can happen in future elections."
The President wanted Sessions to disregard his recusal from the investigation, which had followed from a formal DOJ ethics review, and have Sessions declare that he knew "for a fact" that "there were no Russians involved with the campaign" because he "was there."
The President further directed that Sessions should explain that the President should not be subject to an investigation "because he hasn't done anything wrong."
Taken together, the President's directives indicate that Sessions was being instructed to tell the Special Counsel to end the existing investigation into the President and his campaign, with the Special Counsel being permitted to "move forward with investigating election meddling for future elections."

\underline{Nexus to an official proceeding.}
As described above, by the time of the President's initial one-on-one meeting with Lewandowski on June 19, 2017, the existence of a grand jury investigation supervised by the Special Counsel was public knowledge.
By the time of the President's follow-up meeting with Lewandowski, \blackout{Harm to Ongoing Matter}
See Volume II, Section II.G, infra.
To satisfy the nexus requirement, it would be necessary to show that limiting the Special Counsel's investigation would have the natural and probable effect of impeding that grand jury proceeding.

\underline{Intent.}
Substantial evidence indicates that the President's effort to have Sessions limit the scope of the Special Counsel's investigation to future election interference was intended to prevent further investigative scrutiny of the President's and his campaign's conduct.

As previously described, see Volume II, Section II.B, supra, the President knew that the Russia investigation was focused in part on his campaign, and he perceived allegations of Russian interference to cast doubt on the legitimacy of his election.
The President further knew that the investigation had broadened to include his own conduct and whether he had obstructed justice.
Those investigations would not proceed if the Special Counsel's jurisdiction were limited to future election interference only.

The timing and circumstances of the President's actions support the conclusion that he sought that result.
The President's initial direction that Sessions should limit the Special Counsel's investigation came just two days after the President had ordered McGahn to have the Special Counsel removed, which itself followed public reports that the President was personally under investigation for obstruction of justice.
The sequence of those events raises an inference that after seeking to terminate the Special Counsel, the President sought to exclude his and his campaign's conduct from the investigation's scope.
The President raised the matter with Lewandowski again on July 19, 2017, just days after emails and information about the June 9, 2016 meeting between Russians and senior campaign officials had been publicly disclosed, generating substantial media coverage and investigative interest.

The manner in which the President acted provides additional evidence of his intent.
Rather than rely on official channels, the President met with Lewandowski alone in the Oval Office.
The President selected a loyal "devotee" outside the White House to deliver the message, supporting an inference that he was working outside White House channels, including McGahn, who had previously resisted contacting the Department of Justice about the Special Counsel.
The President also did not contact the Acting Attorney General, who had just testified publicly that there was no cause to remove the Special Counsel.
Instead, the President tried to use Sessions to restrict and redirect the Special Counsel's investigation when Sessions was recused and could not properly take any action on it.

The July 19, 2017 events provide further evidence of the President's intent.
The President followed up with Lewandowski in a separate one-on-one meeting one month after he first dictated the message for Sessions, demonstrating he still sought to pursue the request.
And just hours after Lewandowski assured the President that the message would soon be delivered to Sessions, the President gave an unplanned interview to the New York Times in which he publicly attacked Sessions and raised questions about his job security.
Four days later, on July 22, 2017, the President directed Priebus to obtain Sessions's resignation.
That evidence could raise an inference that the President wanted Sessions to realize that his job might be on the line as he evaluated whether to comply with the President's direction that Sessions publicly announce that, notwithstanding his recusal, he was going to confine the Special Counsel's investigation to future election interference.

\subsection{The President's Efforts to Prevent Disclosure of Emails About the June 9, 2016 Meeting Between Russians and Senior Campaign Officials}

\begin{center}
\textbf{Overview}
\end{center}

By June 2017, the President became aware of emails setting up the June 9, 2016 meeting between senior campaign officials and Russians who offered derogatory information on Hillary Clinton as "part of Russia and its government's support for Mr. Trump."
On multiple occasions in late June and early July 2017, the President directed aides not to publicly disclose the emails, and he then dictated a statement about the meeting to be issued by Donald Trump Jr.\ describing the meeting as about adoption.

\subsubsection{The President Learns About the Existence of Emails Concerning the June 9; 2016 Trump Tower Meeting}

In mid-June 2017 - the same week that the President first asked Lewandowski to pass a message to Sessions - senior Administration officials became aware of emails exchanged during the campaign arranging a meeting between Donald Trump Jr., Paul Manafort, Jared Kushner, and a Russian attorney."% 663
\footnote{Hicks 3/13/18 302, at 1;
Raffel 2/8/18 302, at 2.}
As described in Volume I, Section IV.A.5, supra, the emails stated that the "Crown [P]rosecutor of Russia" had offered "to provide the Trump campaign with some official documents and information that would incriminate Hillary and her dealings with Russia" as part of "Russia and its government's support for Mr.~Trump."% 664
\footnote{RG000061 (6/3/16 Email, Goldstone to Trump~Jr.);
\@DonaldJTrumpJR 7/11/17 (11:01 a.m.~ET) Tweet.}
Trump Jr.responded,"[I]f it's what you say I love it,"% 665
\footnote{RG000061 (6/3/16 Email, Trump Jr.\ to Goldstone);
\@DonaldJTrumpJR 7/11/17 (11:01 am. ET) Tweet.}
and he, Kushner, and Manafort met with the Russian attorney and several other Russian individuals at Trump Tower on June 9, 2016.% 666
\footnote{Samochornov 7/12/17 302, at 4.}
At the meeting, the Russian attorney claimed that funds derived from illegal activities in Russia were provided to Hillary Clinton and other Democrats, and the Russian attorney then spoke about the Magnitsky Act, a 2012 U.S. statute that imposed financial and travel sanctions on Russian officials and that had resulted in a retaliatory ban in Russia on U.S. adoptions of Russian children.% 667
\footnote{\textit{See} Volume I, Section IV.A.5, \textit{supra} (describing meeting in detail).}

According to written answers submitted by the President in response to questions from this Office, the President had no recollection of learning of the meeting or the emails setting it up at the time the meeting occurred or at any other time before the election.% 668
\footnote{Written Responses of Donald J. Trump (Nov.~20, 2018), at 8 (Response to Question I, Parts (a) through (c)).
The President declined to answer questions about his knowledge of the June 9 meeting or other events after the election.}

The Trump Campaign had previously received a document request from SSCI that called for the production of various information, including, "[a] list and a description of all meetings" between any "individual affiliated with the Trump campaign" and "any individual formally or informally affiliated with the Russian government or Russian business interests which took place between June 16, 2015, and 12 pm on January 20, 2017," and associated records.% 669
\footnote{DJTFP\_SCO\_PDF\_00000001-02 (5/17/17 Letter, SSCI to Donald J. Trump for President, Inc.).}
Trump Organization attorneys became aware of the June 9 meeting no later than the first week of June 2017, when they began interviewing the meeting participants, and the Trump Organization attorneys provided the emails setting up the meeting to the President's personal counsel.% 670
\footnote{Goldstone 2/8/18 302, at 12;
6/2/17 and 6/5/17 Emails, Goldstone \& Garten;
Raffel 2/8/18 302, at 3;
Hicks 3/13/18 302, at 2.}
Mark Corallo, who had been hired as a spokesman for the President's personal legal team, recalled that he learned about the June 9 meeting around June 21 or 22, 2017.% 671
\footnote{Corallo 2/15/18 302, at 3.}
Priebus recalled learning about the June 9 meeting from Fox News host Sean Hannity in late June 2017.% 672
\footnote{Priebus 4/3/18 302, at 7.}
Priebus notified one of the President's personal attorneys, who told Priebus he was already working on it.% 673
\footnote{Priebus 4/3/18 302, at 7.}
By late June, several advisors recalled receiving media inquiries that could relate to the June 9 meeting.% 674
\footnote{Corallo 2/15/18 302, at 3;
Hicks 12/7/17 302, at 8;
Raffel 2/8/18 302, at 3.}

\subsubsection{The President Directs Communications Staff Not to Publicly Disclose Information About the June 9 Meeting}

Communications advisors Hope Hicks and Josh Raffel recalled discussing with Jared Kushner and Ivanka Trump that the emails were damaging and would inevitably be leaked.% 675
\footnote{Raffel 2/8/18 302, at 2-3;
Hicks 3/13/18 302, at 2.}
Hicks and Raffel advised that the best strategy was to proactively release the emails to the press.% 676
\footnote{Raffel 2/8/18 302, at 2-3, 5;
Hicks 3/13/18 302, at 2;
Hicks 12/7/17 302, at 8.}
On or about June 22, 2017, Hicks attended a meeting in the White House residence with the President, Kushner, and Ivanka Trump.% 677
\footnote{Hicks 12/7/17 302, at 6-7;
Hicks 3/13/18 302, at 1.}
According to Hicks, Kushner said that he wanted to fill the President in on something that had been discovered in the documents he was to provide to the congressional committees involving a meeting with him, Manafort, and Trump Jr.% 678
\footnote{Hicks 12/7/17 302, at 7;
Hicks 3/13/18 302, at 1.}
Kushner brought a folder of documents to the meeting and tried to show them to the President, but the President stopped Kushner and said he did not want to know about it, shutting the conversation down.% 679
\footnote{Hicks 12/7/17 302, at 7;
Hicks 3/13/18 302, at 1.
Counsel for Ivanka Trump provided an attorney proffer that is consistent with Hicks's account and with the other events involving Ivanka Trump set forth in this section of the report.
Kushner said that he did not recall talking to the President at this time about the June 9 meeting or the underlying emails.
Kushner 4/11/18 302, at 30.}

On June 28, 2017, Hicks viewed the emails at Kushner's attorney's office.% 680
\footnote{Hicks 3/13/18 302, at 1-2.}
She recalled being shocked by the emails because they looked "really bad."% 681
\footnote{Hicks 3/13/18 302, at 2.}
The next day, Hicks spoke privately with the President to mention her concern about the emails, which she understood were soon going to be shared with Congress.% 682
\footnote{Hicks 12/7/17 302, at 8.}
The President seemed upset because too many people knew about the emails and he told Hicks that just one lawyer should deal with the matter.% 683
\footnote{Hicks 3/13/18 302, at 2-3;
Hicks 12/7/17 302, at 8.}
The President indicated that he did not think the emails would leak, but said they would leak if everyone had access to them.% 684
\footnote{Hicks 12/7/17 302, at 8.}

Later that day, Hicks, Kushner, and Ivanka Trump went together to talk to the President.% 685
\footnote{Hicks 12/7/17 302, at 8;
Hicks 3/13/18 302, at 2.}
Hicks recalled that Kushner told the President the June 9 meeting was not a big deal and was about Russian adoption, but that emails existed setting up the meeting.% 686
\footnote{Hicks 3/13/18 302, at 2;
Hicks 12/7/17 302, at 9.}
Hicks said she wanted to get in front of the story and have Trump Jr.\ release the emails as part of an interview with "softball questions."% 687
\footnote{Hicks 3/13/18 302, at 2-3.}
The President said he did not want to know about it and they should not go to the press.% 688
\footnote{Hicks 3/13/18 302, at 2-3;
Hicks 12/7/17 302, at 9.}
Hicks warned the President that the emails were "really bad" and the story would be "massive" when it broke, but the President was insistent that he did not want to talk about it and said he did not want details.% 689
\footnote{Hicks 3/13/18 302, at 3;
Hicks 12/7/17 302, at 9.}
Hicks recalled that the President asked Kushner when his document production was due.% 690
\footnote{Hicks 3/13/18 302, at 3.}
Kushner responded that it would be a couple of weeks and the President said, "then leave it alone."% 691
\footnote{Hicks 3/13/18 302, at 3.}
Hicks also recalled that the President said Kushner's attorney should give the emails to whomever he needed to give them to, but the President did not think they would be leaked to the press.% 692
\footnote{Hicks 12/7/17 302, at 9.}
Raffel later heard from Hicks that the President had directed the group not to be proactive in disclosing the emails because the President believed they would not leak.% 693
\footnote{Raffel 2/8/18 302, at 5.}

\subsubsection{The President Directs Trump Jr.'s Response to Press Inquiries About the June 9 Meeting}

The following week, the President departed on an overseas trip for the G20 summit in Hamburg, Germany, accompanied by Hicks, Raffel, Kushner, and Ivanka Trump, among others.% 694
\footnote{Raffel 2/8/18 302, at 6.}
On July 7, 2017, while the President was overseas, Hicks and Raffel learned that the New York Times was working on a story about the June 9 meeting.% 695
\footnote{Raffel 2/8/18 302, at 6-7;
Hicks 3/13/18 302, at 3.}
The next day, Hicks told the President about the story and he directed her not to comment.% 696
\footnote{Hicks 12/7/17 302, at 10;
Hicks 3/13/18 302, at 3.}
Hicks thought the President's reaction was odd because he usually considered not responding to the press to be the ultimate sin.% 697
\footnote{Hicks 12/7/17 302, at 10.}
Later that day, Hicks and the President again spoke about the story.% 698
\footnote{Hicks 3/13/18 302, at 3.}
Hicks recalled that the President asked her what the meeting had been about, and she said that she had been told the meeting was about Russian adoption.% 699
\footnote{Hicks 3/13/18 302, at 3;
Hicks 12/7/17 302, at 10.}
The President responded, "then just say that."% 700
\footnote{Hicks 3/13/18 302, at 3;
\textit{see} Hicks 12/7/17 302, at 10.}

On the flight home from the G20 on July 8, 2017, Hicks obtained a draft statement about the meeting to be released by Trump Jr.\ and brought it to the President.% 701
\footnote{Hicks 3/13/18 302, at 4.}
The draft statement began with a reference to the information that was offered by the Russians in setting up the meeting: "I was asked to have a meeting by an acquaintance I knew from the 2013 Miss Universe pageant with an individual who I was told might have information helpful to the campaign."% 702
\footnote{Hicks 7/8/17 Notes.}
Hicks again wanted to disclose the entire story, but the President directed that the statement not be
issued because it said too much.% 703
\footnote{Hicks 3/13/18 302, at 4-5;
Hicks 12/7/17 302, at 11.}
The President told Hicks to say only that Trump Jr.\ took a brief meeting and it was about Russian adoption.% 704
\footnote{Hicks 12/7/17 302, at 11.}
After speaking with the President, Hicks texted Trump Jr.\ a revised statement on the June 9 meeting that read:

\begin{quote}
It was a short meeting.
I asked Jared and Paul to stop by.
We discussed a program about the adoption of Russian children that was active and popular with American families years ago and was since ended by the Russian government, but it was not a campaign issue at that time and there was no follow up.% 705
\footnote{SCRO11a\_000004 (7/8/17 Text Message, Hicks to Trump Jr.).}
\end{quote}

Hicks's text concluded, "Are you ok with this? Attributed to you."% 706
\footnote{SCRO11a\_000004 (7/8/17 Text Message, Hicks to Trump Jr.).}
Trump Jr.\ responded by text message that he wanted to add the word "primarily" before "discussed" so that the statement would read, "We primarily discussed a program about the adoption of Russian children."% 707
\footnote{SCRO11a\_000005 (7/8/17 Text Message, Trump Jr.\ to Hicks).}
Trump Jr.\ texted that he wanted the change because "[t]hey started with some Hillary thing which was bs and some other nonsense which we shot down fast."% 708
\footnote{SCRO11a\_000005 (7/8/17 Text Message, Trump Jr.\ to Hicks).}
Hicks texted back, "I think that's right too but boss man worried it invites a lot of questions[.]
[U]ltimately [d]efer to you and [your attorney] on that word Be I know it's important and I think the mention of a campaign issue adds something to it in case we have to go further."% 709
\footnote{SCRO11a\_000005 (7/8/17 Text Message, Trump Jr.\ to Hicks).}
Trump Jr.\ responded, "If I don't have it in there it appears as though I'm lying later when they inevitably leak something."% 710
\footnote{SCRO11a\_000006 (7/8/17 Text Message, Trump Jr.\ to Hicks).}
Trump Jr.'s statement - adding the word "primarily" and making other minor additions - was then provided to the New York Times.% 711
\footnote{Hicks 3/13/18 302, at 6;
\textit{see} Jo Becker et al., \textit{Trump Team Met With Lawyer Linked to Kremlin During Campaign}, New York Times (July 8, 2017).}
The full statement provided to the Times stated:

\begin{quote}
It was a short introductory meeting.
I asked Jared and Paul to stop by.
We primarily discussed a program about the adoption of Russian children that was active and popular with American families years ago and was since ended by the Russian government, but it was not a campaign issue at the time and there was no follow up.
I was asked to attend the meeting by an acquaintance, but was not told the name of the person I would be meeting with beforehand.% 712
\footnote{See Jo Becker et al., \textit{Trump Team Met With Lawyer Linked to Kremlin During Campaign}, New York Times (July 8, 2017).}
\end{quote}

The statement did not mention the offer of derogatory information about Clinton or any discussion of the Magnitsky Act or U.S. sanctions, which were the principal subjects of the meeting, as described in Volume I, Section IV.A.5, \textit{supra}.

A short while later, while still on Air Force One, Hicks learned that Priebus knew about the emails, which further convinced her that additional information about the June 9 meeting would leak and the White House should be proactive and get in front of the story.% 713
\footnote{Hicks 3/13/18 302, at 6;
Raffel 2/8/18 302, at 9-10.}
Hicks recalled again going to the President to urge him that they should be fully transparent about the June 9 meeting, but he again said no, telling Hicks, "You've given a statement.
We're done."% 714
\footnote{Hicks 12/7/17 302, at 12;
Raffel 2/8/18 302, at 10.}

Later on the flight home, Hicks went to the President's cabin, where the President was on the phone with one of his personal attorneys.% 715
\footnote{Hicks 3/13/18 302, at 7.}
At one point the President handed the phone to Hicks, and the attorney told Hicks that he had been working with Circa News on a separate story, and that she should not talk to the New York Times.% 716
\footnote{Hicks 3/13/18 302, at 7.}

\subsubsection{The Media Reports on the June 9, 2016 Meeting}

Before the President's flight home from the G20 landed, the New York Times published its story about the June 9, 2016 meeting.% 717
\footnote{\textit{See} Jo Becker et al., \textit{Trump Team Met With Lawyer Linked to Kremlin During Campaign}, New York Times (July 8, 2017);
Raffel 2/8/18 302, at 10.}
In addition to the statement from Trump Jr., the Times story also quoted a statement from Corallo on behalf of the President's legal team suggesting that the meeting might have been a setup by individuals working with the firm that produced the Steele reporting.% 718
\footnote{See Jo Becker et al., \textit{Trump Team Met With Lawyer Linked to Kremlin During Campaign}, New York Times (July 8, 2017).}
Corallo also worked with Circa News on a story published an hour later that questioned whether Democratic operatives had arranged the June 9 meeting to create the appearance of improper connections between Russia and Trump family members.% 719
\footnote{\textit{See Donald Trump Jr.\ gathered members of campaign for meeting with Russian lawyer before election}, Circa News (July 8, 2017).}
Hicks was upset about Corallo's public statement and called him that evening to say the President had not approved the statement.% 720
\footnote{Hicks 3/13/18 302, at 8;
Corallo 2/15/18 302, at 6-7.}

The next day, July 9, 2017, Hicks and the President called Corallo together and the President criticized Corallo for the statement he had released.% 721
\footnote{Corallo 2/15/18 302, at 7.}
Corallo told the President the statement had been authorized and further observed that Trump Jr.'s statement was inaccurate and that a document existed that would contradict it.% 722
\footnote{Corallo 2/15/18 302, at 7.}
Corallo said that he purposely used the term "document" to refer to the emails setting up the June 9 meeting because he did not know what the President knew about the emails.% 723
\footnote{Corallo 2/15/18 302, at 7-9.}
Corallo recalled that when he referred to the "document" on the call with the President, Hicks responded that only a few people had access to it and said "it will never get out."% 724
\footnote{Corallo 2/15/18 302, at 8.}
Corallo took contemporaneous notes of the call that say:
"Also mention existence of doc.
Hope says 'only a few people have it.
It will never get out."% 725
\footnote{Corallo 2/15/18 302, at 8;
Corallo 7/9/17 Notes ("Sunday 9th - Hope calls w/ POTUS on line").
Corallo said he is "100\% confident" that Hicks said "It will never get out" on the call.
Corallo 2/15/18 302, at 9.}
Hicks later told investigators that she had no memory of making that comment and had always believed the emails would eventually be leaked, but she might have been channeling the President on the phone call because it was clear to her throughout her conversations with the President that he did not think the emails would leak.% 726
\footnote{Hicks 3/13/18 302, at 9.}

On July 11, 2017, Trump Jr.\ posted redacted images of the emails setting up the June 9 meeting on Twitter; the New York Times reported that he did so "[a]fter being told that The Times was about to publish the content of the emails."% 727
\footnote{\@DonaldJTrumpJR 7/11/17 (11:01 a.m.~ET) Tweet;
Jo Becker et al., \textit{Russian Dirt on Clinton? `I Love It,' Donald Trump Jr.\ Said}, New York Times (July 11, 2017).}
Later that day, the media reported that the President had been personally involved in preparing Trump Jr.'s initial statement to the New York Times that had claimed the meeting "primarily" concerned "a program about the adoption of Russian children."% 728
\footnote{\textit{See, e.g.}, Peter Baker \& Maggie Haberman, \textit{Rancor at White House as Russia Story Refuses to Let the Page Turn}, New York Times (July 11, 2017) (reporting that the President ``signed off'' on Trump Jr.'s statement).}
Over the next several days, the President's personal counsel repeatedly and inaccurately denied that the President played any role in drafting Trump Jr.'s statement.% 729
\footnote{\textit{See, e.g.}, David Wright, \textit{Trump lawyer: President was aware of "nothing"}, CNN (July 12, 2017) (quoting the President's personal attorney as saying, "I wasn't involved in the statement drafting at all nor was the President.");
\textit{see also} Good Morning America, ABC (July 12, 2017) ("The President didn't sign off on anything... The President wasn't involved in that.");
Meet the Press, NBC (July 16, 2017) ("I do want to be clear - the President was not involved in the drafting of the statement.").}
After consulting with the President on the issue, White House Press Secretary Sarah Sanders told the media that the President "certainly didn't dictate" the statement, but that "he weighed in, offered suggestions like any father would do."% 730
\footnote{Sarah Sanders, \textit{White House Daily Briefing}, C-SPAN (Aug.~1, 2017);
Sanders 7/3/18 302, at 9 (the President told Sanders he "weighed in, as any father would" and knew she intended to tell the press what he said).}
Several months later, the President's personal counsel stated in a private communication to the Special Counsel's Office that "the President dictated a short but accurate response to the New York Times article on behalf of his son, Donald Trump, Jr."% 731
\footnote{1/29/18 Letter, President's Personal Counsel to Special Counsel's Office, at 18.}
The President later told the press that it was "irrelevant" whether he dictated the statement and said, "It's a statement to the New York Times.... judges."% 732
\footnote{Remarks by President Trump in Press Gaggle (June 15, 2018).}
That's not a statement to a high tribunal of

\begin{quote}
On July 12, 2017, the Special Counsel's Office \blackout{Grand Jury} Trump Jr.\
\blackout{Grand Jury} related to the June 9 meeting and those who attended the June 9 meeting.% 733
\footnote{\blackout{Grand Jury}}
\end{quote}

On July 19, 2017, the President had his follow-up meeting with Lewandowski and then met with reporters for the New York Times.
In addition to criticizing Sessions in his Times interview, the President addressed the June 9, 2016 meeting and said he "didn't know anything about the meeting "at the time."% 734
\footnote{Peter Baker et al., \textit{Excerpts From The Times's Interview With Trump}, New York Times (July 19, 2017).}
The President added, "As I've said - most other people, you know, when they call up and say, 'By the way, we have information on your opponent,' I think most politicians - I was just with a lot of people, they said... ,
'Who wouldn't have taken a meeting like that?"% 735
\footnote{Peter Baker et al., \textit{Excerpts From The Times's Interview With Trump}, New York Times (July 19, 2017).}

\begin{center}
\textbf{Analysis}
\end{center}

In analyzing the President's actions regarding the disclosure of information about the June 9 meeting, the following evidence is relevant to the elements of obstruction of justice:

\underline{Obstructive act.}
On at least three occasions between June 29, 2017, and July 9, 2017, the President directed Hicks and others not to publicly disclose information about the June 9, 2016 meeting between senior campaign officials and a Russian attorney.
On June 29, Hicks warned the President that the emails setting up the June 9 meeting were "really bad" and the story would be "massive" when it broke, but the President told her and Kushner to "leave it alone."
Early on July 8, after Hicks told the President the New York Times was working on a story about the June 9 meeting, the President directed her not to comment, even though Hicks said that the President usually considered not responding to the press to be the ultimate sin.
Later that day, the President rejected Trump Jr.'s draft statement that would have acknowledged that the meeting was with "an individual who I was told might have information helpful to the campaign." The President then dictated a statement to Hicks that said the meeting was about Russian adoption (which the President had twice been told was discussed at the meeting).
The statement dictated by the President did not mention the offer of derogatory information about Clinton.

Each of these efforts by the President involved his communications team and was directed at the press.
They would amount to obstructive acts only if the President, by taking these actions, sought to withhold information from or mislead congressional investigators or the Special Counsel.
On May 17, 2017, the President's campaign received a document request from SSCI that clearly covered the June 9 meeting and underlying emails, and those documents also plainly would have been relevant to the Special Counsel's investigation.

But the evidence does not establish that the President took steps to prevent the emails or other information about the June 9 meeting from being provided to Congress or the Special Counsel.
The series of discussions in which the President sought to limit access to the emails and prevent their public release occurred in the context of developing a press strategy.
The only evidence we have of the President discussing the production of documents to Congress or the Special Counsel is the conversation on June 29, 2017, when Hicks recalled the President acknowledging that Kushner's attorney should provide emails related to the June 9 meeting to whomever he needed to give them to.
We do not have evidence of what the President discussed with his own lawyers at that time.

\underline{Nexus to an official proceeding.}
As described above, by the time of the President's attempts to prevent the public release of the emails regarding the June 9 meeting, the existence of a grand jury investigation supervised by the Special Counsel was public knowledge, and the President had been told that the emails were responsive to congressional inquiries.
To satisfy the nexus requirement, however, it would be necessary to show that preventing the release of the emails to the public would have the natural and probable effect of impeding the grand jury proceeding or congressional inquiries.
As noted above, the evidence does not establish that the President sought to prevent disclosure of the emails in those official proceedings.

\underline{Intent.}
The evidence establishes the President's substantial involvement in the communications strategy related to information about his campaign's connections to Russia and his desire to minimize public disclosures about those connections.
The President became aware of the emails no later than June 29, 2017, when he discussed them with Hicks and Kushner, and he could have been aware of them as early as June 2, 2017, when lawyers for the Trump Organization began interviewing witnesses who participated in the June 9 meeting.
The President thereafter repeatedly rejected the advice of Hicks and other staffers to publicly release information about the June 9 meeting.
The President expressed concern that multiple people had access to the emails and instructed Hicks that only one lawyer should deal with the matter.
And the President dictated a statement to be released by Trump Jr.\ in response to the first press accounts of the June 9 meeting that said the meeting was about adoption.

But as described above, the evidence does not establish that the President intended to prevent the Special Counsel's Office or Congress from obtaining the emails setting up the June 9 meeting or other information about that meeting.
The statement recorded by Corallo - that the emails "will never get out" - can be explained as reflecting a belief that the emails would not be made public if the President's press strategy were followed, even if the emails were provided to Congress and the Special Counsel.

\subsection{The President's Further Efforts to Have the Attorney General Take Over the Investigation}

\begin{center}
\textbf{Overview}
\end{center}

From summer 2017 through 2018, the President attempted to have Attorney General Sessions reverse his recusal, take control of the Special Counsel's investigation, and order an investigation of Hillary Clinton.

\begin{center}
\textbf{Evidence}
\end{center}

\subsubsection{The President Again Seeks to Have Sessions Reverse his Recusal}

After returning Sessions's resignation letter at the end of May 2017, but before the President's July 19, 2017 New York Times interview in which he publicly criticized Sessions for recusing from the Russia investigation, the President took additional steps to have Sessions reverse his recusal.
In particular, at some point after the May 17, 2017 appointment of the Special Counsel, Sessions recalled, the President called him at home and asked if Sessions would "unrecuse" himself.% 736
\footnote{Sessions 1/17/18 302, at 15.
That was the second time that the President asked Sessions to reverse his recusal from campaign-related investigations.
\textit{See} Volume II, Section II.C.1, \textit{supra} (describing President's March 2017 request at Mar-a-Lago for Sessions to unrecuse).
}
According to Sessions, the President asked him to reverse his recusal so that Sessions could direct the Department of Justice to investigate and prosecute Hillary Clinton, and the "gist" of the conversation was that the President wanted Sessions to unrecuse from "all of it," including the Special Counsel's Russia investigation.% 737
\footnote{Sessions 1/17/18 302, at 15.}
Sessions listened but did not respond, and he did not reverse his recusal or order an investigation of Clinton.% 738
\footnote{Sessions 1/17/18 302, at 15.}

In early July 2017, the President asked Staff Secretary Rob Porter what he thought of Associate Attorney General Rachel Brand.% 739
\footnote{Porter 4/13/18 302, at 11;
Porter 5/8/18 302, at 6.}
Porter recalled that the President asked him if Brand was good, tough, and "on the team."% 740
\footnote{Porter 4/13/18 302, at 11;
Porter 5/8/18 302, at 6.}
The President also asked if Porter thought Brand was interested in being responsible for the Special Counsel's investigation and whether she would want to be Attorney General one day.% 741
\footnote{Porter 4/13/18 302, at 11;
Porter 5/8/18 302, at 6.
Because of Sessions's recusal, if Rosenstein were no longer in his position, Brand would, by default, become the DOJ official in charge of supervising the Special Counsel's investigation, and if both Sessions and Rosenstein were removed, Brand would be next in line to become Acting Attorney General for all DOJ matters.
\textit{See} 28 U.S.C. \S 508.}
Because Porter knew Brand, the President asked him to sound her out about taking responsibility for the investigation and being Attorney General.% 742
\footnote{Porter 4/13/18 302, at 11;
Porter 5/8/18 302, at 6.}
Contemporaneous notes taken by Porter show that the President told Porter to "Keep in touch with your friend," in reference to Brand.% 743
\footnote{SC\_RRP000020 (Porter 7/10/17 Notes).}
Later, the President asked Porter a few times in passing whether he had spoken to Brand, but Porter did not reach out to her because he was uncomfortable with the task.% 744
\footnote{Porter 4/13/18 302, at 11-12.}
In asking him to reach out to Brand, Porter understood the President to want to find someone to end the Russia investigation or fire the Special Counsel, although the President never said so explicitly.% 745
\footnote{Porter 4/13/18 302, at 11-12.}
Porter did not contact Brand because he was sensitive to the implications of that action and did not want to be involved in a chain of events associated with an effort to end the investigation or fire the Special Counsel.% 746
\footnote{Porter 4/13/18 302, at 11-12.
Brand confirmed that no one ever raised with her the prospect of taking over the Russia investigation or becoming Attorney General.
Brand 1/29/19 302, at 2.}

McGahn recalled that during the summer of 2017, he and the President discussed the fact that if Sessions were no longer in his position the Special Counsel would report directly to a non-recused Attorney General.% 747
\footnote{McGahn 12/14/17 302, at 11.}
McGahn told the President that things might not change much under a new Attorney General.% 748
\footnote{McGahn 12/14/17 302, at 11.}
McGahn also recalled that in or around July 2017, the President frequently brought up his displeasure with Sessions.% 749
\footnote{McGahn 12/14/17 302, at 9.}
Hicks recalled that the President viewed Sessions's recusal from the Russia investigation as an act of disloyalty.% 750
\footnote{Hicks 3/13/18 302, at 10.}
In addition to criticizing Sessions's recusal, the President raised other concerns about Sessions and his job performance with McGahn and Hicks.% 751
\footnote{McGahn 12/14/17 302, at 9;
Hicks 3/13/18 302, at 10.}

\subsubsection{Additional Efforts to Have Sessions Unrecuse or Direct Investigations Covered by his Recusal}

Later in 2017, the President continued to urge Sessions to reverse his recusal from campaign-related investigations and considered replacing Sessions with an Attorney General who would not be recused.

On October 16, 2017, the President met privately with Sessions and said that the Department of Justice was not investigating individuals and events that the President thought the Department should be investigating.% 752
\footnote{Porter 5/8/18 302, at 10.}
According to contemporaneous notes taken by Porter, who was at the meeting, the President mentioned Clinton's emails and said, "Don't have to tell us, just take [a] look."% 753
\footnote{SC\_RRP000024 (Porter 10/16/17 Notes);
\textit{see} Porter 5/8/18 302, at 10.}
Sessions did not offer any assurances or promises to the President that the Department of Justice would comply with that request.% 754
\footnote{Porter 5/8/18 302, at 10.}
Two days later, on October 18, 2017, the President tweeted, "Wow, FBI confirms report that James Comey drafted letter exonerating Crooked Hillary Clinton long before investigation was complete.
Many people not interviewed, including Clinton herself.
Comey stated under oath that he didn't do this-obviously a fix?
Where is Justice Dept?"% 755
\footnote{\@realDonaldTrump 10/18/17 (6:21 a.m.~ET) Tweet;
\@realDonaldTrump 10/18/17 (6:27 a.m.~ET) Tweet.}
On October 29, 2017, the President tweeted that there was "ANGER \& UNITY" over a "lack of investigation" of Clinton and "the Comey fix," and concluded: "DO SOMETHING!"% 756
\footnote{\@realDonaldTrump 10/29/17 (9:53 a.m.~ET) Tweet;
\@realDonaldTrump 10/29/17 (10:02 a.m.~ET) Tweet;
\@realDonaldTrump 10/29/17 (10:17 a.m.~ET) Tweet.}

On December 6, 2017, five days after Flynn pleaded guilty to lying about his contacts with the Russian government, the President asked to speak with Sessions in the Oval Office at the end of a cabinet meeting.% 757
\footnote{Porter 4/13/18 302, at 5-6;
see SC\_RRP000031 (Porter 12/6/17 Notes) ("12:45pm With the President, Gen.~Kelly, and Sessions (who I pulled in after the Cabinet meeting)");
SC\_RRP000033 (Porter 12/6/17 Notes) ("Post-cabinet meeting - POTUS asked me to get AG Sessions.
Asked me to stay.
Also COS Kelly.").}
During that Oval Office meeting, which Porter attended, the President again suggested that Sessions could "unrecuse," which Porter linked to taking back supervision of the Russia investigation and directing an investigation of Hillary Clinton.% 758
\footnote{Porter 5/8/18 302, at 12;
Porter 4/13/18 302, at 5-6.}
According to contemporaneous notes taken by Porter, the President said, "I don't know if you could un-recuse yourself.
You'd be a hero.
Not telling you to do anything.
Dershowitz says POTUS can get involved.
Can order AG to investigate.
I don't want to get involved.
I'm not going to get involved.
I'm not going to do anything or direct you to do anything.
I just want to be treated fairly."% 759
\footnote{SC\_RRP000033 (Porter 12/6/17 Notes);
\textit{see} Porter 4/13/18 302, at 6;
Porter 5/8/18 302, at 12.}
According to Porter's notes, Sessions responded, "We are taking steps; whole new leadership team.
Professionals; will operate according to the law."% 760
\footnote{SC\_RRP0000033 (Porter 12/6/17 Notes);
\textit{see} Porter 4/13/18 302, at 6.}
Sessions also said, "I never saw anything that was improper," which Porter thought was noteworthy because it did not fit with the previous discussion about Clinton."% 761
\footnote{SC\_RRP000033 (Porter 12/6/17 Notes);
Porter 4/13/18 302, at 6.}
Porter understood Sessions to be reassuring the President that he was on the President's team.% 762
\footnote{Porter 4/13/18 302, at 6-7.}

At the end of December, the President told the New York Times it was "too bad" that Sessions had recused himself from the Russia investigation.% 763
\footnote{Michael S. Schmidt \& Michael D. Shear, \textit{Trump Says Russia Inquiry Makes U.S. ``Look Very Bad''}, New York Times (Dec.~28, 2017).}
When asked whether Holder had been a more loyal Attorney General to President Obama than Sessions was to him, the President said, "I don't want to get into loyalty, but I will tell you that, I will say this: Holder protected President Obama.
Totally protected him.
When you look at the things that they did, and Holder protected the president.
And I have great respect for that, I'll be honest."% 764
\footnote{Michael S. Schmidt \& Michael D. Shear, \textit{Trump Says Russia Inquiry Makes U.S. ``Look Very Bad''}, New York Times (Dec.~28, 2017).}
Later in January, the President brought up the idea of replacing Sessions and told Porter that he wanted to "clean house" at the Department of Justice.% 765
\footnote{Porter 4/13/18 302, at 14.}
In a meeting in the White House residence that Porter attended on January 27, 2018, Porter recalled that the President talked about the great attorney she had in the past with successful win records, such as Roy Cohn and Jay Goldberg, and said that one of his biggest failings as President was that he had not surrounded himself with good attorneys, citing Sessions as an example.% 766
\footnote{Porter 5/8/18 302, at 15.
Contemporaneous notes Porter took of the conversation state, "Roy Cohn (14-0) / Jay Goldberg (12-0)."
SC\_RRP000047 (Porter 1/27/18 Notes).}
The President raised Sessions's recusal and brought up and criticized the Special Counsel's investigation.% 767
\footnote{Porter 5/8/18 302, at 15-16.}

Over the next several months, the President continued to criticize Sessions in tweets and media interviews and on several occasions appeared to publicly encourage him to take action in the Russia investigation despite his recusal.% 768
\footnote{\textit{See, e.g.}, \@realDonaldTrump 2/28/18 (9:34 a.m.~ET) Tweet ("Why is A.G. Jeff Sessions asking the Inspector General to investigate potentially massive FISA abuse.
Will take forever, has no prosecutorial power and already late with reports on Comey etc.
Isn't the I.G. an Obama guy?
Why not use Justice Department lawyers?
DISGRACEFUL!");
\@realDonaldTrump 4/7/18 (4:52 p.m.~ET) Tweet ("Lawmakers of the House Judiciary Committee are angrily accusing the Department of Justice of missing the Thursday Deadline for turning over UNREDACTED Documents relating to FISA abuse, FBI, Comey, Lynch, McCabe, Clinton Emails and much more.
Slow walking - what is going on? BAD!");
\@realDonaldTrump 4/22/18 (8:22 a.m.~ET) Tweet (""GOP Lawmakers asking Sessions to Investigate Comey and Hillary Clinton.' \@FoxNews Good luck with that request!");
\@realDonaldTrump 12/16/18 (3:37 p.m.~ET) Tweet ("Jeff Sessions should be ashamed of himself for allowing this total HOAX to get started in the first place!").}
On June 5, 2018, for example, the President tweeted, "The Russian Witch Hunt Hoax continues, all because Jeff Sessions didn't tell me he was going to recuse himself.... I would have quickly picked someone else.
So much time and money wasted, so many lives ruined ... and Sessions knew better than most that there was No Collusion!"% 769
\footnote{\@realDonaldTrump 6/5/18 (7:31 a.m.~ET) Tweet.}
On August 1, 2018, the President tweeted that "Attorney General Jeff Sessions should stop this Rigged Witch Hunt right now."% 770
\footnote{\@realDonaldTrump 8/1/18 (9:24 a.m.~ET) Tweet.}
On August 23, 2018, the President publicly criticized Sessions in a press interview and suggested that prosecutions at the Department of Justice were politically motivated because Paul Manafort had been prosecuted but Democrats had not.% 771
\footnote{Fox \& Friends Interview of President Trump, Fox News (Aug.~23, 2018).}
The President said, "I put in an Attorney General that never took control of the Justice Department, Jeff Sessions."% 772
\footnote{Fox \& Friends Interview of President Trump, Fox News (Aug.~23, 2018).}
That day, Sessions issued a press statement that said, "I took control of the Department of Justice the day I was sworn in....
While I am Attorney General, the actions of the Department of Justice will not be improperly influenced by political considerations."% 773
\footnote{Sessions 8/23/18 Press Statement.}
The next day, the President tweeted a response: "'Department of Justice will not be improperly influenced by political considerations.'
Jeff, this is GREAT, what everyone wants, so look into all of the corruption on the 'other side' including deleted Emails, Comey lies \& leaks, Mueller conflicts, McCabe, Strzok, Page, Ohr, FISA abuse, Christopher Steele \& his phony and corrupt Dossier, the Clinton Foundation, illegal surveillance of Trump campaign, Russian collusion by Dems - and so much more.
Open up the papers \& documents without redaction? Come on Jeff, you can do it, the country is waiting!"% 774
\footnote{\@realDonaldTrump8/24/18 (6:17 a.m.~ET) Tweet;
\@realDonaldTrump 8/24/18 (6:28 a.m.~ET) Tweet.}

On November7, 2018, the day after the midterm elections, the President replaced Sessions with Sessions's chief of staff as Acting Attorney General.% 775
\footnote{\@realDonaldTrump 11/7/18 (2:44 p.m.~ET) Tweet.}

\begin{center}
\textbf{Analysis}
\end{center}

In analyzing the President's efforts to have Sessions unrecuse himself and regain control of the Russia investigation, the following considerations and evidence are relevant to the elements of obstruction of justice:

\underline{Obstructive act.}
To determine if the President's efforts to have the Attorney General unrecuse could qualify as an obstructive act, it would be necessary to assess evidence on whether those actions would naturally impede the Russia investigation.
That inquiry would take into account the supervisory role that the Attorney General, if unrecused, would play in the Russia investigation.
It also would have to take into account that the Attorney General's recusal covered other campaign-related matters.
The inquiry would not turn on what Attorney General Sessions would actually do if unrecused, but on whether the efforts to reverse his recusal would naturally have had the effect of impeding the Russia investigation.

On multiple occasions in 2017, the President spoke with Sessions about reversing his recusal so that he could take over the Russia investigation and begin an investigation and prosecution of Hillary Clinton.
For example, in early summer 2017, Sessions recalled the President asking him to unrecuse, but Sessions did not take it as a directive.
When the President raised the issue again in December 2017, the President said, as recorded by Porter, "Not telling you to do anything....
I'm not going to get involved.
I'm not going to do anything or direct you to do anything.
I just want to be treated fairly."
The duration of the President's efforts - which spanned from March 2017 to August 2018 - and the fact that the President repeatedly criticized Sessions in public and in private for failing to tell the President that he would have to recuse is relevant to assessing whether the President's efforts to have Sessions unrecuse could qualify as obstructive acts.

\underline{Nexus to an official proceeding.}
As described above, by mid-June 2017, the existence of a grand jury investigation supervised by the Special Counsel was public knowledge.
In addition, in July 2017, a different grand jury supervised by the Special Counsel was empaneled in the District of Columbia, and the press reported on the existence of this grand jury in early August 2017.% 776
\footnote{Eg., Del Quentin Wilbur \& Byron Tau, \textit{Special Counsel Robert Mueller Impanels Washington Grand Jury in Russia Probe}, Wall Street Journal (Aug.~3, 2017);
Carol D. Leonnig et al., \textit{Special Counsel Mueller using grand jury in federal court in Washington as part of Russia investigation}, Washington Post (Aug.~3, 2017).}
Whether the conduct towards the Attorney General would have a foreseeable impact on those proceedings turns on much of the same evidence discussed above with respect to the obstructive-act element.

\underline{Intent.}
There is evidence that at least one purpose of the President's conduct toward Sessions was to have Sessions assume control over the Russia investigation and supervise it in a way that would restrict its scope.
By the summer of 2017, the President was aware that the Special Counsel was investigating him personally for obstruction of justice.
And in the wake of the disclosures of emails about the June 9 meeting between Russians and senior members of the campaign, see Volume II, Section II.G, supra, it was evident that the investigation into the campaign now included the President's son, son-in-law, and former campaign manager.
The President had previously and unsuccessfully sought to have Sessions publicly announce that the Special Counsel investigation would be confined to future election interference.
Yet Sessions remained recused.
In December 2017, shortly after Flynn pleaded guilty, the President spoke to Sessions in the Oval Office with only Porter present and told Sessions that he would be a hero if he unrecused.
Porter linked that request to the President's desire that Sessions take back supervision of the Russia investigation and direct an investigation of Hillary Clinton.
The President said in that meeting that he "just want[ed] to be treated fairly," which could reflect his perception that it was unfair that he was being investigated while Hillary Clinton was not.
But a principal effect of that act would be to restore supervision of the Russia investigation to the
Attorney General - a position that the President frequently suggested should be occupied by someone like Eric Holder and Bobby Kennedy, who the President described as protecting their presidents.
A reasonable inference from those statements and the President's actions is that the President believed that an unrecused Attorney General would play a protective role and could shield the President from the ongoing Russia investigation.

\subsection{The President Orders McGahn to Deny that the President Tried to Fire the Special Counsel}

\begin{center}
\textbf{Overview}
\end{center}

In late January 2018, the media reported that in June 2017 the President had ordered McGahn to have the Special Counsel fired based on purported conflicts of interest but McGahn had refused, saying he would quit instead.
After the story broke, the President, through his personal counsel and two aides, sought to have McGahn deny that he had been directed to remove the Special Counsel.
Each time he was approached, McGahn responded that he would not refute the press accounts because they were accurate in reporting on the President's effort to have the Special Counsel removed.
The President later personally met with McGahn in the Oval Office with only the Chief of Staff present and tried to get McGahn to say that the President never ordered him to fire the Special Counsel.
McGahn refused and insisted his memory of the President's direction to remove the Special Counsel was accurate.
In that same meeting, the President challenged McGahn for taking notes of his discussions with the President and asked why he had told Special Counsel investigators that he had been directed to have the Special Counsel removed.

\begin{center}
\textbf{Evidence}
\end{center}

\subsubsection{The Press Reports that the President Tried to Fire the Special Counsel}

On January 25, 2018, the New York Times reported that in June 2017, the President had ordered McGahn to have the Department of Justice fire the Special Counsel.% 777
\footnote{Michael S. Schmidt \& Maggie Haberman, \textit{Trump Ordered Mueller Fired, but Backed Off When White House Counsel Threatened to Quit}, New York Times (Jan.~25. 2018).}
According to the article, "[a]mid the first wave of news media reports that Mr.~Mueller was examining a possible obstruction case, the president began to argue that Mr.~Mueller had three conflicts of interest that disqualified him from overseeing the investigation."% 778
\footnote{Michael S. Schmidt \& Maggie Haberman, \textit{Trump Ordered Mueller Fired, but Backed Off When White House Counsel Threatened to Quit}, New York Times (Jan.~25. 2018).}
The article further reported that "[a]fter receiving the president's order to fire Mr.~Mueller, the White House counsel ...
Justice Department to dismiss the special counsel, saying he would quit instead."% 779
\footnote{Michael S. Schmidt \& Maggie Haberman, \textit{Trump Ordered Mueller Fired, but Backed Off When White House Counsel Threatened to Quit}, New York Times (Jan.~25. 2018).}
The article stated that the president "ultimately backed down after the White House counsel threatened to resign rather than carry out the directive."% 780
\footnote{Michael S. Schmidt \& Maggie Haberman, \textit{Trump Ordered Mueller Fired, but Backed Off When White House Counsel Threatened to Quit}, New York Times (Jan.~25. 2018).}
After the article was published, the President
refused to ask the
dismissed the story when asked about it by reporters, saying, "Fake news, folks.
Fake news.
A typical New York Times fake story."% 781
\footnote{Sophie Tatum \& Kara Scannell, \textit{Trump denies he called for Mueller's firing}, CNN (Jan.~26, 2018);
Michael S. Schmidt \& Maggie Haberman, \textit{Trump Ordered Mueller Fired, but Backed Off When White House Counsel Threatened to Quit}, New York Times (Jan.~25, 2018).}

The next day, the Washington Post reported on the same event but added that McGahn had not told the President directly that he intended to resign rather than carry out the directive to have the Special Counsel terminated.% 782
\footnote{The Post article stated, "Despite internal objections, Trump decided to assert that Mueller had unacceptable conflicts of interest and moved to remove him from his position....
In response, McGahn said he would not remain at the White House if Trump went through with the move....
McGahn did not deliver his resignation threat directly to Trump but was serious about his threat to leave."
Rosalind S. Helderman \& Josh Dawsey, \textit{Trump moved to fire Mueller in June, bringing White House counsel to the brink of leaving}, Washington Post (Jan.~26, 2018).}
In that respect, the Post story clarified the Times story, which could be read to suggest that McGahn had told the President of his intention to quit, causing the President to back down from the order to have the Special Counsel fired.% 783
\footnote{Rosalind S. Helderman \& Josh Dawsey, \textit{Trump moved to fire Mueller in June, bringing White House counsel to the brink of leaving}, Washington Post (Jan.~26, 2018);
\textit{see} McGahn 3/8/17 302, at 3-4.}

\subsubsection{The President Seeks to Have McGahn Dispute the Press Reports}

On January 26, 2018, the President's personal counsel called McGahn's attorney and said that the President wanted McGahn to put out a statement denying that he had been asked to fire the Special Counsel and that he had threatened to quit in protest.% 784
\footnote{McGahn 3/8/18 302, at 3 (agent note).}
McGahn's attorney spoke with McGahn about that request and then called the President's personal counsel to relay that McGahn would not make a statement.% 785
\footnote{McGahn 3/8/18 302, at 3 (agent note).}
McGahn's attorney informed the President's personal counsel that the Times story was accurate in reporting that the President wanted the Special Counsel removed.% 786
\footnote{McGahn 3/8/18 302, at 3-4 (agent note).}
Accordingly, McGahn's attorney said, although the article was inaccurate in some other respects, McGahn could not comply with the President's request to dispute the story.% 787
\footnote{McGahn 3/8/18 302, at 4 (agent note).}
Hicks recalled relaying to the President that one of his attorneys had spoken to McGahn's attorney about the issue.% 788
\footnote{Hicks 3/13/18 302, at 11.
Hicks also recalled that the President spoke on the phone that day with Chief of Staff John Kelly and that the President said Kelly told him that McGahn had totally refuted the story and was going to put out a statement.
Hicks 3/13/18 302, at 11.
But Kelly said that he did not speak to McGahn when the article came out and did not tell anyone he had done so.
Kelly 8/2/18 302, at 1-2.}

Also on January 26, 2017, Hicks recalled that the President asked Sanders to contact McGahn about the story.% 789
\footnote{Hicks 3/13/18 302, at 11.
Sanders did not recall whether the President asked her to speak to McGahn or if she did it on her own.
Sanders 7/23/18 302, at 2.}
McGahn told Sanders there was no need to respond and indicated that some of the article was accurate.% 790
\footnote{Sanders 7/23/18 302, at 1-2.}
Consistent with that position, McGahn did not correct the Times story.

On February 4, 2018, Priebus appeared on Meet the Press and said he had not heard the President say that he wanted the Special Counsel fired.% 791
\footnote{Meet the Press Interview with Reince Priebus, NBC (Feb.~4, 2018).}
After Priebus's appearance, the President called Priebus and said he did a great job on Meet the Press.% 792
\footnote{Priebus 4/3/18 302, at 10.}
The President also told Priebus that the President had "never said any of those things about" the Special Counsel.% 793
\footnote{Priebus 4/3/18 302, at 10.}

The next day, on February 5, 2018, the President complained about the Times article to Porter.% 794
\footnote{Porter 4/13/18 302, at 16-17.
Porter did not recall the timing of this discussion with the President.
Porter 4/13/18 302, at 17.
Evidence indicates it was February 5, 2018.
On the back of a pocket card dated February 5, 2018, Porter took notes that are consistent with his description of the discussion: "COS: (1) Letter from DM - Never threatened to quit - DJT never told him to fire M."
SC\_RRP000053 (Porter Undated Notes).
Porter said it was possible he took the notes on a day other than February 5.
Porter 4/13/18302, at17.
But Porter also said that "COS" referred to matters he wanted to discuss with Chief of Staff Kelly, Porter 4/13/18 302, at 17, and Kelly took notes dated February 5, 2018, that state "POTUS - Don McGahn letter - Mueller + resigning."
WH000017684 (Kelly 2/5/18 Notes).
Kelly said he did not recall what the notes meant, but thought the President may have "mused" about having McGahn write a letter.
Kelly 8/2/18 302, at 3.
McGahn recalled that Porter spoke with him about the President's request about two weeks after the New York Times story was published, which is consistent with the discussion taking place on or about February 5.
McGahn 3/8/18 302, at 4.}
The President told Porter that the article was "bullshit" and he had not sought to terminate the Special Counsel.% 795
\footnote{Porter 4/13/18 302, at 17.}
The President said that McGahn leaked to the media to make himself look good.% 796
\footnote{Porter 4/13/18 302, at 17.}
The President then directed Porter to tell McGahn to create a record to make clear that the President never directed McGahn to fire the Special Counsel.% 797
\footnote{Porter 4/13/18 302, at 17.}
Porter thought the matter should be handled by the White House communications office, but the President said he wanted McGahn to write a letter to the file "for our records" and wanted something beyond a press statement to demonstrate that the reporting was inaccurate.% 798
\footnote{Porter 4/13/18 302, at 17;
Porter 5/8/18 302, at 18.}
The President referred to McGahn as a "lying bastard" and said that he wanted a record from him.% 799
\footnote{Porter 4/13/18 302, at 17;
Porter 5/8/18 302, at 18.}
Porter recalled the President saying something to the effect of, "If he doesn't write a letter, then maybe I'll have to get rid of him.% 800
\footnote{Porter 4/13/18 302, at 17.}

Later that day, Porter spoke to McGahn to deliver the President's message.% 801
\footnote{Porter 4/13/18 302, at 17;
McGahn 3/8/18 302, at 4.}
Porter told McGahn that he had to write a letter to dispute that he was ever ordered to terminate the Special Counsel.% 802
\footnote{Porter 4/13/18 302, at 17;
McGahn 3/8/18 302, at 4.}
McGahn shrugged off the request, explaining that the media reports were true.% 803
\footnote{Porter 4/13/18 302, at 17;
McGahn 3/8/18 302, at 4.}
McGahn told Porter that the President had been insistent on firing the Special Counsel and that McGahn had planned to resign rather than carry out the order, although he had not personally told the President he intended to quit.% 804
\footnote{Porter 4/13/18 302, at 17;
McGahn 3/8/18 302, at 4.}
Porter told McGahn that the President suggested that McGahn would be fired if he did not write the letter.% 805
\footnote{Porter 4/13/18 302, at 17;
McGahn 3/8/18 302, at 4.}
McGahn dismissed the threat, saying that the optics would be terrible if the President followed through with firing him on that basis.% 806
\footnote{Porter 4/13/18 302, at 17-18;
McGahn 3/8/18 302, at 4.}
McGahn said he would not write the letter the President had requested.% 807
\footnote{McGahn 3/8/18 302, at 4.}
Porter said that to his knowledge the issue of McGahn's letter never came up with the President again, but Porter did recall telling Kelly about his conversation with McGahn.% 808
\footnote{Porter 4/13/18 302, at 18.}

The next day, on February 6, 2018, Kelly scheduled time for McGahn to meet with him and the President in the Oval Office to discuss the Times article.% 809
\footnote{McGahn 3/8/18 302, at 4;
WH000017685 (Kelly 2/6/18 Notes).
McGahn recalled that, before the Oval Office meeting, he told Kelly that he was not inclined to fix the article.
McGahn 3/8/18 302, at 4.}
The morning of the meeting, the President's personal counsel called McGahn's attorney and said that the President was going to be speaking with McGahn and McGahn could not resign no matter what happened in the meeting.% 810
\footnote{McGahn 3/8/18 302, at 5 (agent note);
2/26/19 Email, Counsel for Don McGahn to Special Counsel's Office (confirming February 6, 2018 date of call from the President's personal counsel).}

The President began the Oval Office meeting by telling McGahn that the New York Times story did not "look good" and McGahn needed to correct it.% 811
\footnote{McGahn 3/8/18 302, at 4; Kelly 8/2/18 302, at 2.}
McGahn recalled the President said, "I never said to fire Mueller.
I never said 'fire.'
This story doesn't look good.
You need to correct this.
You're the White House counsel."% 812
\footnote{McGahn 3/8/18 302, at 4; Kelly 8/2/18 302, at 2.}

In response, McGahn acknowledged that he had not told the President directly that he planned to resign, but said that the story was otherwise accurate.% 813
\footnote{McGahn 3/8/18 302, at 4.}
The President asked McGahn, "Did I say the word 'fire'?"% 814
\footnote{McGahn 3/8/18 302, at 4; Kelly 8/2/18 302, at 2.}
McGahn responded, "What you said is, 'Call Rod [Rosenstein], tell Rod that Mueller has conflicts and can't be the Special Counsel.'"% 815
\footnote{McGahn 3/8/18 302, at 5.}
The President responded, "I never said that.'% 816
\footnote{McGahn 3/8/18 302, at 5.}
The President said he merely wanted McGahn to raise the conflicts issue with Rosenstein and leave it to him to decide what to do.% 817
\footnote{McGahn 3/8/18 302, at 5.}
McGahn told the President he did not understand the conversation that way and instead had heard, "Call Rod.
There are conflicts.
Mueller has to go."% 818
\footnote{McGahn 3/8/18 302, at 5.}
The President asked McGahn whether he would "do a correction," and McGahn said no.% 819
\footnote{McGahn 3/8/18 302, at 5; Kelly 8/2/18 302, at 2.}
McGahn thought the President was testing his mettle to see how committed McGahn was to what happened.% 820
\footnote{McGahn 3/8/18 302, at 5.}
Kelly described the meeting as "a little tense."% 821
\footnote{Kelly 8/2/18 302, at 2.}

The President also asked McGahn in the meeting why he had told Special Counsel's Office investigators that the President had told him to have the Special Counsel removed.% 822
\footnote{McGahn 3/8/18 302, at 5.}
McGahn responded that he had to and that his conversations with the President were not protected by attorney-client privilege.% 823
\footnote{McGahn 3/8/18 302, at 5.}
The President then asked, "What about these notes? Why do you take notes? Lawyers don't take notes.
I never had a lawyer who took notes."% 824
\footnote{McGahn 3/8/18 302, at 5.
McGahn said the President was referring to Donaldson's notes, which the President thought of as McGahn's notes.
McGahn 3/8/18 302, at 5.}
McGahn responded that he keeps notes because he is a "real lawyer" and explained that notes create a record and are not a bad thing.% 825
\footnote{McGahn 3/8/18 302, at 5.}
The President said, "I've had lot of great lawyers, like Roy Cohn.
He did not take notes."% 826
\footnote{McGahn 3/8/18 302, at 5.}

After the Oval Office meeting concluded, Kelly recalled McGahn telling him that McGahn and the President "did have that conversation" about removing the Special Counsel.% 827
\footnote{Kelly 8/2/18 302, at 2.}
McGahn recalled that Kelly said that he had pointed out to the President after the Oval Office that McGahn had not backed down and would not budge.% 828
\footnote{McGahn 3/8/18 302, at 5.
Kelly did not recall discussing the Oval Office meeting with the President after the fact.
Kelly 8/2/18 302, at 2.
Handwritten notes taken by Kelly state, "Don[:] Mueller discussion in June. - Bannon Priebus - came out okay."
WH000017685 (Kelly 2/6/18 Notes).}
Following the Oval Office meeting, the President's personal counsel called McGahn's counsel and relayed that the President was "fine" with McGahn.% 829
\footnote{McGahn 3/8/18 302, at 5 (agent note).}

\begin{center}
\textbf{Analysis}
\end{center}

In analyzing the President's efforts to have McGahn deny that he had been ordered to have the Special Counsel removed, the following evidence is relevant to the elements of obstruction of justice:

\underline{Obstructive act.}
The President's repeated efforts to get McGahn to create a record denying that the President had directed him to remove the Special Counsel would qualify as an obstructive act if it had the natural tendency to constrain McGahn from testifying truthfully or to undermine his credibility as a potential witness if he testified consistently with his memory, rather than with what the record said.

There is some evidence that at the time the New York Times and Washington Post stories were published in late January 2018, the President believed the stories were wrong and that he had never told McGahn to have Rosenstein remove the Special Counsel.
The President correctly understood that McGahn had not told the President directly that he planned to resign.
In addition, the President told Priebus and Porter that he had not sought to terminate the Special Counsel, and in the Oval Office meeting with McGahn, the President said, "I never said to fire Mueller.
I never said 'fire.""
That evidence could indicate that the President was not attempting to persuade McGahn to change his story but was instead offering his own - but different - recollection of the substance of his June 2017 conversations with McGahn and McGahn's reaction to them.

Other evidence cuts against that understanding of the President's conduct.
As previously described, see Volume II, Section III, supra, substantial evidence supports McGahn's account that the President had directed him to have the Special Counsel removed, including the timing and context of the President's directive;
the manner in which McGahn reacted;
and the fact that the President had been told the conflicts were insubstantial, were being considered by the Department of Justice, and should be raised with the President's personal counsel rather than brought to McGahn.
In addition, the President's subsequent denials that he had told McGahn to have the Special Counsel removed were carefully worded.
When first asked about the New York Times story, the President said, "Fake news, folks.
Fake news.
A typical New York Times fake story." And when the President spoke with McGahn in the Oval Office, he focused on whether he had used the word "fire," saying, "I never said to fire Mueller.
I never said 'fire'" and "Did I say the word 'fire'?" The President's assertion in the Oval Office meeting that he had never directed McGahn to have the Special Counsel removed thus runs counter to the evidence.

In addition, even if the President sincerely disagreed with McGahn's memory of the June 17, 2017 events, the evidence indicates that the President knew by the time of the Oval Office
meeting that McGahn's account differed and that McGahn was firm in his views.
Shortly after the story broke, the President's counsel told McGahn's counsel that the President wanted McGahn to make a statement denying he had been asked to fire the Special Counsel, but McGahn responded through his counsel that that aspect of the story was accurate and he therefore could not comply with the President's request.
The President then directed Sanders to tell McGahn to correct the story, but McGahn told her he would not do so because the story was accurate in reporting on the President's order.
Consistent with that position, McGahn never issued a correction.
More than a week later, the President brought up the issue again with Porter, made comments indicating the President thought McGahn had leaked the story, and directed Porter to have McGahn create a record denying that the President had tried to fire the Special Counsel.
At that point, the President said he might "have to get rid of' McGahn if McGahn did not comply.
McGahn again refused and told Porter, as he had told Sanders and as his counsel had told the President's counsel, that the President had in fact ordered him to have Rosenstein remove the Special Counsel.
That evidence indicates that by the time of the Oval Office meeting the President was aware that McGahn did not think the story was false and did not want to issue a statement or create a written record denying facts that McGahn believed to be true.
The President nevertheless persisted and asked McGahn to repudiate facts that McGahn had repeatedly said were accurate.

\underline{Nexus to an official proceeding.}
By January 2018, the Special Counsel's use of a grand jury had been further confirmed by the return of several indictments.
The President also was aware that the Special Counsel was investigating obstruction-related events because, among other reasons, on January 8, 2018, the Special Counsel's Office provided his counsel with a detailed list of topics for a possible interview with the President.% 830
\footnote{1/29/18 Letter, President's Personal Counsel to Special Counsel's Office, at 1-2 ("In our conversation of January 8, your office identified the following topics as areas you desired to address with the President in order to complete your investigation on the subjects of alleged collusion and obstruction of justice";
listing 16 topics).}
The President knew that McGahn had personal knowledge of many of the events the Special Counsel was investigating and that McGahn had already been interviewed by Special Counsel investigators.
And in the Oval Office meeting, the President indicated he knew that McGahn had told the Special Counsel's Office about the President's effort to remove the Special Counsel.
The President challenged McGahn for disclosing that information and for taking notes that he viewed as creating unnecessary legal exposure.
That evidence indicates the President's awareness that the June 17, 2017 events were relevant to the Special Counsel's investigation and any grand jury investigation that might grow out of it.

To establish a nexus, it would be necessary to show that the President's actions would have the natural tendency to affect such a proceeding or that they would hinder, delay, or prevent the communication of information to investigators.
Because McGahn had spoken to Special Counsel investigators before January 2018, the President could not have been seeking to influence his prior statements in those interviews.
But because McGahn had repeatedly spoken to investigators and the obstruction inquiry was not complete, it was foreseeable that he would be interviewed again on obstruction-related topics.
If the President were focused solely on a press strategy in seeking to have McGahn refute the New York Times article, a nexus to a proceeding or to further investigative interviews would not be shown.
But the President's efforts to have McGahn write a letter "for our records" approximately ten days after the stories had come out - well past the typical time to issue a correction for a news story - indicates the President was not focused solely on a press strategy, but instead likely contemplated the ongoing investigation and any proceedings arising from it.

\underline{Intent.}
Substantial evidence indicates that in repeatedly urging McGahn to dispute that he was ordered to have the Special Counsel terminated, the President acted for the purpose of influencing McGahn's account in order to deflect or prevent further scrutiny of the President's conduct towards the investigation.

inquiry.
Several facts support that conclusion.
The President made repeated attempts to get McGahn to change his story.
As described above, by the time of the last attempt, the evidence suggests that the President had been told on multiple occasions that McGahn believed the President had ordered him to have the Special Counsel terminated.
McGahn interpreted his encounter with the President in the Oval Office as an attempt to test his mettle and see how committed he was to his memory of what had occurred.
The President had already laid the groundwork for pressing McGahn to alter his account by telling Porter that it might be necessary to fire McGahn if he did not deny the story, and Porter relayed that statement to McGahn.
Additional evidence of the President's intent may be gleaned from the fact that his counsel was sufficiently alarmed by the prospect of the President's meeting with McGahn that he called McGahn's counsel and said that McGahn could not resign no matter what happened in the Oval Office that day.
The President's counsel was well aware of McGahn's resolve not to issue what he believed to be a false account of events despite the President's request.
Finally, as noted above, the President brought up the Special Counsel investigation in his Oval Office meeting with McGahn and criticized him for telling this Office about the June 17, 2017 events.
The President's statements reflect his understanding - and his displeasure - that those events would be part of an obstruction-of-justice

\subsection{The President's Conduct Towards Flynn, Manafort, $\blacksquare\blacksquare\blacksquare\blacksquare\blacksquare\blacksquare\blacksquare\blacksquare$}

\begin{center}
\textbf{Overview}
\end{center}

In addition to the interactions with McGahn described above, the President has taken other actions directed at possible witnesses in the Special Counsel's investigation, including Flynn, Manafort, \blackout{Harm to Ongoing Matter} and as described in the next section, Cohen.
When Flynn withdrew from a joint defense agreement with the President, the President's personal counsel stated that Flynn's actions would be viewed as reflecting "hostility" towards the President.
During Manafort's prosecution and while the jury was deliberating, the President repeatedly stated that Manafort was being treated unfairly and made it known that Manafort could receive a pardon.
\blackout{Harm to Ongoing Matter}

\begin{center}
\textbf{Evidence}
\end{center}

\subsubsection{Conduct Directed at Michael Flynn}

As previously noted, see Volume II, Section II.B, supra, the President asked for Flynn's resignation on February 13, 2017.
Following Flynn's resignation, the President made positive public comments about Flynn, describing him as a "wonderful man,""a fine person," and a "very good person."% 831
\footnote{\textit{See, e.g., Remarks by President Trump in Press Conference}, White House (Feb.~16, 2018) (stating that "Flynn is a fine person" and "I don't think [Flynn] did anything wrong.
If anything, he did something right... You know, he was just doing his job");
Interview of Donald J. Trump, NBC (May 11, 2017) (stating that Flynn is a "very good person").}
The President also privately asked advisors to pass messages to Flynn conveying that the President still cared about him and encouraging him to stay strong.% 832
\footnote{\textit{See} Priebus 1/18/17 302, at 9-10 (the President asked Priebus to contact Flynn the week he was terminated to convey that the President still cared about him and felt bad about what happened to him;
Priebus thought the President did not want Flynn to have a problem with him);
McFarland 12/22/17 302, at 18 (about a month or two after Flynn was terminated, the President asked McFarland to get in touch with Flynn and tell him that he was a good guy, he should stay strong, and the President felt bad for him);
Flynn 1/19/18 302, at 9 (recalling the call from Priebus and an additional call from Hicks who said she wanted to relay on behalf of the President that the President hoped Flynn was okay);
Christie 2/13/19 302, at 3 (describing a phone conversation between Kushner and Flynn the day after Flynn was fired where Kushner said, "You know the President respects you.
The President cares about you.
I'll get the President to send out a positive tweet about you later," and the President nodded his assent to Kushner's comment promising a tweet).}

In late November 2017, Flynn began to cooperate with this Office.
On November 22, 2017, Flynn withdrew from a joint defense agreement he had with the President.% 833
\footnote{Counsel for Flynn 3/1/18 302, at 1.}
Flynn's counsel told the President's personal counsel and counsel for the White House that Flynn could no longer have confidential communications with the White House or the President.% 834
\footnote{Counsel for Flynn 3/1/18 302, at 1.}
Later that night, the President's personal counsel left a voicemail for Flynn's counsel that said:

\begin{quote}
I understand your situation, but let me see if I can't state it in starker terms....
[I]t wouldn't surprise me if you've gone on to make a deal with...
the government....
[I]f ... there's information that implicates the President, then we've got a national security issue, ...
so, you know, ... we need some kind of heads up.
Um, just for the sake of protecting all our interests if we can....
[R]emember what we've always said about the President and his feelings toward Flynn and, that still remains ... % 835
\footnote{1 1/22/17 Voicemail Transcript, President's Personal Counsel to Counsel for Michael Flynn.}
\end{quote}

On November 23, 2017, Flynn's attorneys returned the call from the President's personal counsel to acknowledge receipt of the voicemail.% 836
\footnote{Counsel for Flynn 3/1/18 302, at 1.}
Flynn's attorneys reiterated that they were no longer in a position to share information under any sort of privilege.% 837
\footnote{Counsel for Flynn 3/1/18 302, at 1.}
According to Flynn's attorneys, the President's personal counsel was indignant and vocal in his disagreement.% 838
\footnote{Counsel for Flynn 3/1/18 302, at 1.}
The President's personal counsel said that he interpreted what they said to him as a reflection of Flynn's hostility towards the President and that he planned to inform his client of that interpretation.% 839
\footnote{Counsel for Flynn 3/1/18 302, at 2.
Because of attorney-client privilege issues, we did not seek to interview the President's personal counsel about the extent to which he discussed his statements to Flynn's attorneys with the President.}
Flynn's attorneys understood that statement to be an attempt to make them reconsider their position because the President's personal counsel believed that Flynn would be disturbed to know that such a message would be conveyed to the President.% 840
\footnote{Counsel for Flynn 3/1/18 302, at 2.}

On December 1, 2017, Flynn pleaded guilty to making false statements pursuant to a cooperation agreement.% 841
\footnote{Information, \textit{United States v.\ Michael T. Flynn}, 1:17-cr-232 (D.D.C. Dec.~1, 2017), Doc.~1;
Plea Agreement, \textit{United States v.\ Michael T. Flynn}, 1:17-cr-232 (D.D.C. Dec.~1, 2017), Doc.~3.}
The next day, the President told the press that he was not concerned about what Flynn might tell the Special Counsel.% 842
\footnote{\textit{President Trump Remarks on Tax Reform and Michael Flynn's Guilty Plea}, C-SPAN (Dec.~2, 2017).}
In response to a question about whether the President still stood behind Flynn, the President responded, "We'll see what happens.% 843
\footnote{\textit{President Trump Remarks on Tax Reform and Michael Flynn's Guilty Plea}, C-SPAN (Dec.~2, 2017).}
Over the next several days, the President made public statements expressing sympathy for Flynn and indicating he had not been treated fairly.% 844
\footnote{\textit{See} \@realDonaldTrump12/2/17 (9:06 p.m.~ET) Tweet ("So General Flynn lies to the FBI and his life is destroyed, while Crooked Hillary Clinton, on that now famous FBI holiday 'interrogation' with no swearing in and no recording, lies many times ... and nothing happens to her?
Rigged system, or just a double standard?");
President Trump Departure Remarks, C-SPAN (Dec.~4, 2017) ("Well, I feel badly for General Flynn.
I feel very badly.
He's led a very strong life.
And I feel very badly.").}
On December 15, 2017, the President responded to a press inquiry about whether he was considering a pardon for Flynn by saying, "I don't want to talk about pardons for Michael Flynn yet.
We'll see what happens.
Let's see.
I can say this: When you look at what's gone on with the FBI and with the Justice Department, people are very, very angry.''% 845
\footnote{\textit{President Trump White House Departure}, C-SPAN (Dec.~15, 2017).}

\subsubsection{Conduct Directed at Paul Manafort}

On October27, 2017, a grand jury in the District of Columbia indicted Manafort and former deputy campaign manager Richard Gates on multiple felony counts, and on February 22, 2018, a grand jury in the Eastern District of Virginia indicted Manafort and Gates on additional felony counts.% 846
\footnote{Indictment, \textit{United States v.\ Paul J. Manafort, Jr.\ and Richard W. Gates III}, 1:17-cr-201 (D.D.C. Oct, 27, 2017), Doc.~13 ("Manafort and Gates D.D.C. Indictment");
Indictment, \textit{United States v.\ Paul J. Manafort, Jr.\ and Richard W. Gates III}, 1:18-cr-83 (E.D. Va.\ Feb.~22, 2018), Doc.~9 ("Manafort and Gates E.D. Va.\ Indictment')}
The charges in both cases alleged criminal conduct by Manafort that began as early as 2005 and continued through 2018.% 847
\footnote{\textit{Manafort and Gates} D.D.C. Indictment; \textit{Manafort and Gates} E.D. Va.\ Indictment.}

In January 2018, Manafort told Gates that he had talked to the President's personal counsel and they were "going to take care of us."% 848
\footnote{Gates 4/18/18 302, at 4.
In February 2018, Gates pleaded guilty, pursuant to a cooperation plea agreement, to a superseding criminal information charging him with conspiring to defraud and commit multiple offenses (i.e., tax fraud, failure to report foreign bank accounts, and acting as an unregistered agent of a foreign principal) against the United States, as well as making false statements to our Office.
Superseding Criminal Information, \textit{United States v.\ Richard W. Gates III}, 1:17-cr-201 (D.D.C. Feb.~23, 2018), Doc.~195;
Plea Agreement, \textit{United States v.\ Richard W. Gates III}, 1:17-cr-201 (D.D.C. Feb, 23, 2018), Doc.~205.
Gates has provided information and in-court testimony that the Office has deemed to be reliable.}
Manafort told Gates it was stupid to plead, saying that he had been in touch with the President's personal counsel and repeating that they should "sit tight" and "we'll be taken care of.'% 849
\footnote{Gates 4/18/18 302, at 4.}
Gates asked Manafort outright if anyone mentioned pardons and Manafort said no one used that word.% 850
\footnote{Gates 4/18/18 302, at 4.
Manafort told this Office that he never told Gates that he had talked to the President's personal counselor suggested that they would be taken care of.
Manafort also said he hoped for a pardon but never discussed one with the President, although he noticed the President's public comments about pardons.
Manafort 10/1/18 302, at 11.
As explained in Volume I, Section IV.A.8, \textit{supra}, Manafort entered into a plea agreement with our Office.
The U.S. District Court for the District of Columbia determined that he breached the agreement by being untruthful in proffer sessions and before the grand jury.
Order, \textit{United States v.\ Manafort}, 1:17-cr-201 (D.D.C. Feb.~13, 2019), Doc.~503.}

As the proceedings against Manafort progressed in court, the President told Porter that he never liked Manafort and that Manafort did not know what he was doing on the campaign.% 851
\footnote{Porter 5/8/18 302, at 11.
Priebus recalled that the President never really liked Manafort.
\textit{See} Priebus 4/3/18 302, at 11.
Hicks said that candidate Trump trusted Manafort's judgment while he worked on the Campaign, but she also once heard Trump tell Gates to keep an eye on Manafort.
Hicks 3/13/18 302, at 16.}
The President discussed with aides whether and in what way Manafort might be cooperating with the Special Counsel's investigation, and whether Manafort knew any information that would be harmful to the President.% 852
\footnote{Porter 5/8/18 302, at 11;
McGahn 12/14/17 302, at 14.}

In public, the President made statements criticizing the prosecution and suggesting that Manafort was being treated unfairly.
On June 15, 2018, before a scheduled court hearing that day on whether Manafort's bail should be revoked based on new charges that Manafort had tampered with witnesses while out on bail, the President told the press, "I feel badly about a lot of them because I think a lot of it is very unfair.
I mean, I look at some of them where they go back 12 years.
Like Manafort has nothing to do with our campaign.
But I feel so - I tell you, I feel little badly about it.
They went back 12 years to get things that he did 12 years ago? ...
I feel badly for some people, because they've gone back 12 years to find things about somebody, and I don't think it's right."% 853
\footnote{Remarks by President Trump in Press Gaggle, White House (June 15, 2018).}
In response to a question about whether he was considering a pardon for Manafort or other individuals involved in the Special Counsel's investigation, the President said, "T don't want to talk about that.
No, I don't want to talk about that....
But look, I do want to see people treated fairly.
That's what it's all about."% 854
\footnote{Remarks by President Trump in Press Gaggle, White House (June 15, 2018).}
Hours later, Manafort's bail was revoked and the President tweeted, "Wow, what a tough sentence for Paul Manafort, who has represented Ronald Reagan, Bob Dole and many other top political people and campaigns.
Didn't know Manafort was the head of the Mob.
What about Comey and Crooked Hillary and all the others?
Very unfair!"% 855
\footnote{\@realDonaldTrump 6/15/18 (1:41 p.m.~ET) Tweet.}

Immediately following the revocation of Manafort's bail, the President's personal lawyer, Rudolph Giuliani, gave a series of interviews in which he raised the possibility of a pardon for Manafort.
Giuliani told the New York Daily News that "[w]hen the whole thing is over, things might get cleaned up with some presidential pardons."% 856
\footnote{Chris Sommerfeldt, \textit{Rudy Giuliani says Mueller probe 'might get cleaned up' with 'presidential pardons' in light of Paul Manafort going to jail}, New York Daily News (June 15, 2018).}
Giuliani also said in an interview that, although the President should not pardon anyone while the Special Counsel's investigation was ongoing, "when the investigation is concluded, he's kind of on his own, right?"% 857
\footnote{Sharon LaFraniere, \textit{Judge Orders Paul Manafort Jailed Before Trial, Citing New Obstruction Charges}, New York Times (June 15, 2018) (quoting Giuliani).}
In a CNN interview two days later, Giuliani said, "I guess I should clarify this once and for all....
The president has issued no pardons in this investigation.
The president is not going to issue pardons in this investigation.... When it's over, hey, he's the president of the United States.
He retains his pardon power.
Nobody is taking that away from him."% 858
\footnote{\textit{State of the Union with Jake Tapper Transcript}, CNN (June 17, 2018);
\textit{see} Karoun Demirjian, \textit{Giuliani suggests Trump may pardon Manafort after Mueller's probe}, Washington Post (June 17, 2018).}
Giuliani rejected the suggestion that his and the President's comments could signal to defendants that they should not cooperate in a criminal prosecution because a pardon might follow, saying the comments were "certainly not intended that way."% 859
\footnote{\textit{State of the Union with Jake Tapper Transcript}, CNN (June 17, 2018).}
Giuliani said the comments only acknowledged that an individual involved in the investigation would not be "excluded from [a pardon], if in fact the president and his advisors come to the conclusion that you have been treated unfairly."% 860
\footnote{\textit{State of the Union with Jake Tapper Transcript}, CNN (June 17, 2018).}
Giuliani observed that pardons were not unusual in political investigations but said, "That doesn't mean they're going to happen here.
Doesn't mean that anybody should rely on it....
Big signal is, nobody has been pardoned yet.''% 861
\footnote{\textit{State of the Union with Jake Tapper Transcript}, CNN (June 17, 2018).}

On July 31, 2018, Manafort's criminal trial began in the Eastern District of Virginia, generating substantial news coverage.% 862
\footnote{\textit{See, e.g.}, Katelyn Polantz, \textit{Takeaways from day one of the Paul Manafort trial}, CNN(July 31, 2018);
Frank Bruni, \textit{Paul Manafort's Trial Is Donald Trump's, Too}, New York Times Opinion (July 31, 2018);
Rachel Weiner et al., \textit{Paul Manafort trial Day 2: Witnesses describe extravagant clothing purchases, home remodels, lavish cars paid with wire transfers}, Washington Post (Aug.~1, 2018).}
The next day, the President tweeted, "This is a terrible situation and Attorney General Jeff Sessions should stop this Rigged Witch Hunt right now, before it continues to stain our country any further.
Bob Mueller is totally conflicted, and his 17 Angry Democrats that are doing his dirty work are a disgrace to USA!"% 863
\footnote{\@realDonaldTrump 8/1/18 (9:24 a.m.~ET) Tweet.
Later that day, when Sanders was asked about the President's tweet, she told reporters, "It's not an order.
It's the President's opinion."
Sarah Sanders, \textit{White House Daily Briefing}, C-SPAN (Aug.~1, 2018).}
Minutes later, the President tweeted, "Paul Manafort worked for Ronald Reagan, Bob Dole and many other highly prominent and respected political leaders.
He worked for me for a very short time.
Why didn't government tell me that he was under investigation.
These old charges have nothing to do with Collusion - a Hoax!"% 864
\footnote{\@realDonaldTrump 8/1/18 (9:34 a.m.~ET) Tweet.}
Later in the day, the President tweeted, "Looking back on history, who was treated worse, Alfonse Capone, legendary mob boss, killer and 'Public Enemy Number One,' or Paul Manafort, political operative \& Reagan/Dole darling, now serving solitary confinement - although convicted of nothing?
Where is the Russian Collusion?"% 865
\footnote{\@realDonaldTrump 8/1/18 (11:35 a.m.~ET) Tweet.}
The President's tweets about the Manafort trial were widely covered by the press.% 866
\footnote{\textit{See, e.g.}, Carol D. Leonnig et al., \textit{Trump calls Manafort prosecution "a hoax," says Sessions should stop Mueller investigation "right now"}, Washington Post (Aug.~1, 2018);
Louis Nelson, \textit{Trump claims Manafort case has "nothing to do with collusion"}, Politico (Aug.~1. 2018).}
When asked about the President's tweets, Sanders told the press, "Certainly, the President's been clear.
He thinks Paul Manafort's been treated unfairly."% 867
\footnote{Sarah Sanders, \textit{White House Daily Briefing}, C-SPAN (Aug.~1, 2018).}

On August 16, 2018, the Manafort case was submitted to the jury and deliberations began.
At that time, Giuliani had recently suggested to reporters that the Special Counsel investigation needed to be "done in the next two or three weeks,"% 868
\footnote{Chris Strohm \& Shannon Pettypiece, \textit{Mueller Probe Doesn't Need to Shut Down Before Midterms, Officials Say}, Bloomberg (Aug.~15, 2018).}
and media stories reported that a Manafort acquittal would add to criticism that the Special Counsel investigation was not worth the time and expense, whereas a conviction could show that ending the investigation would be premature.% 869
\footnote{\textit{See, e.g}, Katelyn Polantz et al., \textit{Manafort jury ends first day of deliberations without a verdict}, CNN (Aug.~16, 2018);
David Voreacos, \textit{What Mueller's Manafort Case Means for the Trump Battle to Come}, Bloomberg (Aug.~2, 2018);
Gabby Morrongiello, \textit{What a guilty verdict for Manafort would mean for Trump and Mueller}, Washington Examiner (Aug.~18, 2018).}

On August 17, 2018, as jury deliberations continued, the President commented on the trial from the South Lawn of the White House.
In an impromptu exchange with reporters that lasted approximately five minutes, the President twice called the Special Counsel's investigation a "rigged witch hunt."% 870
\footnote{President Trump Remarks on John Brennan and Mueller Probe, C-SPAN (Aug.~17, 2018).}
When asked whether he would pardon Manafort if he was convicted, the President said, "I don't talk about that now.
I don't talk about that."% 871
\footnote{President Trump Remarks on John Brennan and Mueller Probe, C-SPAN (Aug.~17, 2018).}
The President then added, without being asked a further question,"I think the whole Manafort trial is very sad when you look at what's going on there.
I think it's a very sad day for our country.
He worked for me for a very short period of time.
But you know what, he happens to be a very good person.
And I think it's very sad what they've done to Paul Manafort."% 872
\footnote{President Trump Remarks on John Brennan and Mueller Probe, C-SPAN (Aug.~17, 2018).}
The President did not take further questions.% 873
\footnote{President Trump Remarks on John Brennan and Mueller Probe, C-SPAN (Aug.~17, 2018).}
In response to the President's statements, Manafort's attorney said, "Mr.~Manafort really appreciates the support of President Trump."% 874
\footnote{\textit{Trump calls Manafort "very good person,"} All In with Chris Hayes (Aug.~17, 2018) (transcript);
\textit{Manafort lawyer: We appreciate Trump's support}, CNN (Aug.~17, 2018) (https://www.cnn.com/videos/politics/2018/08/17/paul-manafort-attorney-trump-jury-deliberations-schneider-lead-vpx.cnn).}

On August 21, 2018, the jury found Manafort guilty on eight felony counts.
Also on August 21, Michael Cohen pleaded guilty to eight offenses, including a campaign-finance violation that he said had occurred "in coordination with, and at the direction of, a candidate for federal office."% 875
\footnote{Transcript at 23, \textit{United States v.\ Michael Cohen}, 1:18-cr-602 (S.D.N.Y. Aug.~21, 2018), Doc.~7 (Cohen 8/21/18 Transcript).}
The President reacted to Manafort's convictions that day by telling reporters, "Paul Manafort's a good man" and "it's a very sad thing that happened."% 876
\footnote{\textit{President Trump Remarks on Manafort Trial}, C-SPAN (Aug.~21, 2018).}
The President described the Special Counsel's investigation as "a witch hunt that ends in disgrace."% 877
\footnote{\textit{President Trump Remarks on Manafort Trial}, C-SPAN (Aug.~21, 2018).}
The next day, the President tweeted, "I feel very badly for Paul Manafort and his wonderful family.
'Justice' took a 12 year old tax case, among other things, applied tremendous pressure on him and, unlike Michael Cohen, he refused to 'break' - make up stories in order to get a 'deal.'
Such respect for a brave man!"% 878
\footnote{\@realDonaldTrump 8/22/18 (9:21 a.m.~ET) Tweet.}

In a Fox News interview on August 22, 2018, the President said: "[Cohen] make a better deal when he uses me, like everybody else.
And one of the reasons I respect Paul Manafort so much is he went through that trial - you know they make up stories.
People make up stories.
This whole thing about flipping, they call it, I know all about flipping."% 879
\footnote{\textit{Fox \& Friends Exclusive Interview with President Trump}, Fox News (Aug.~23, 2018) (recorded the previous day).}
The President said that flipping was "not fair" and "almost ought to be outlawed."% 880
\footnote{\textit{Fox \& Friends Exclusive Interview with President Trump}, Fox News (Aug.~23, 2018) (recorded the previous day).}
In response to a question about whether he was considering a pardon for Manafort, the President said, "I have great respect for what he's done, in terms of what he's gone through....
He worked for many, many people many, many years, and I would say what he did, some of the charges they threw against him, every consultant, every lobbyist in Washington probably does."% 881
\footnote{\textit{Fox \& Friends Exclusive Interview with President Trump}, Fox News (Aug.~23, 2018) (recorded the previous day).}
Giuliani told journalists that the President "really thinks Manafort has been horribly treated" and that he and the President had discussed the political fallout if the President pardoned Manafort.% 882
\footnote{Maggie Haberman \& Katie Rogers, \textit{"How Did We End Up Here?" Trump Wonders as the White House Soldiers On}, New York Times (Aug.~22, 2018).}
The next day, Giuliani told the Washington Post that the President had asked his lawyers for advice on the possibility of a pardon for Manafort and other aides, and had been counseled against considering a pardon until the investigation concluded.% 883
\footnote{Carol D. Leonnig \& Josh Dawsey, \textit{Trump recently sought his lawyers' advice on possibility of pardoning Manafort, Giuliani says}, Washington Post (Aug.~23, 2018).}

On September 14, 2018, Manafort pleaded guilty to charges in the District of Columbia and signed a plea agreement that required him to cooperate with investigators.% 884
\footnote{Plea Agreement, \textit{United States v.\ Paul J. Manafort, Jr.}, 1:17-cr-201 (D.D.C. Sept.~14, 2018), Doc.~422.}
Giuliani was reported to have publicly said that Manafort remained in a joint defense agreement with the President following Manafort's guilty plea and agreement to cooperate, and that Manafort's attorneys regularly briefed the President's lawyers on the topics discussed and the information Manafort had provided in interviews with the Special Counsel's Office.% 885
\footnote{Karen Freifeld \& Nathan Layne, \textit{Trump lawyer: Manafort said nothing damaging in Mueller interviews}, Reuters (Oct.~22, 2018);
Michael S. Schmidt et al., \textit{Manafort's Lawyer Said to Brief Trump Attorneys on What He Told Mueller}, New York Times (Nov.~27, 2018);
Dana Bash, \textit{Manafort team briefed Giuliani on Mueller meetings}, CNN, Posted 11/28/18, available at
\url{https://www.cnn.com/videos/politics/2018/11/28/manafort-lawyers-keeping-trump-lawyers-giuliani-updated-mueller-probe-bash-sot-nr-vpx.cnn};
\textit{see} Sean Hannity, \textit{Interview with Rudy Giuliani}, Fox News (Sept.~14, 2018) (Giuliani: "[T]here was a quote put out by a source close to Manafort that the plea agreement has, and cooperation agreement has, nothing to do with the Trump campaign....
Now, I know that because I've been privy to lot of facts I can't repeat.").}
On November 26, 2018, the Special Counsel's Office disclosed in a public court filing that Manafort had breached his plea agreement by lying about multiple subjects.% 886
\footnote{Joint Status Report, \textit{United States v.\ Paul J. Manafort, Jr.}, (D.D.C Nov.~26, 2018), Doc.~455.}
The next day, Giuliani said that the President had been "upset for weeks" about what he considered to be "the un-American, horrible treatment of Manafort."% 887
\footnote{Stephen Collinson, \textit{Trump appears consumed by Mueller investigation as details emerge}, CNN (Nov.~29, 2018).}
In an interview on November 28, 2018, the President suggested that it was "very brave" that Manafort did not "flip":

\begin{quote}
If you told the truth, you go to jail.
You know this flipping stuff is terrible.
You flip and you lie and you get - the prosecutors will tell you 99 percent of the time they can get people to flip.
It's rare that they can't.
But I had three people: Manafort, Corsi - I don't know Corsi, but he refuses to say what they demanded.% 888
\footnote{"Corsi" is a reference to Jerome Corsi, \blackout{Harm to Ongoing Matter} who was involved in efforts to coordinate with WikiLeaks and Assange, and who stated publicly at that time that he had refused a plea offer from the Special Counsel's Office because he was "not going to sign a lie."
Sara Murray \& Eli Watkins, \blackout{Harm to Ongoing Matter} \textit{says he won't agree to plea deal}, CNN (Nov.~26, 2018).}
Manafort, Corsi, \blackout{Harm to Ongoing Investigation}.
It's actually very brave.% 889
\footnote{Marisa Schultz \& Nikki Schwab, \textit{Oval Office Interview with President Trump: Trump says pardon for Paul Manafort still a possibility}, New York Post (Nov.~28, 2018).
That same day, the President tweeted: "While the disgusting Fake News is doing everything within their power not to report it that way, at least 3 major players are intimating that the Angry Mueller Gang of Dems is viciously telling witnesses to lie about facts \& they will get relief.
This is our Joseph McCarthy Era!" \@realDonaldTrump 11/28/18 (8:39 a.m.~ET) Tweet.}
\end{quote}

In response to a question about a potential pardon for Manafort, the President said, "It was never discussed, but I wouldn't take it off the table.
Why would I take it off the table?"% 890
\footnote{Marisa Schultz \& Nikki Schwab, \textit{New York Post Oval Office Interview with President Trump: Trump says pardon for Paul Manafort still a possibility}, New York Post (Nov.~28, 2018).}

\subsubsection{[████████: Harm to Ongoing Matter]}

\blackout{Harm to Ongoing Investigation}% 891
\footnote{\blackout{Harm to Ongoing Matter}}
\blackout{Harm to Ongoing Investigation}% 892
\footnote{\blackout{Harm to Ongoing Matter}}
\blackout{Harm to Ongoing Investigation}% 893
\footnote{\blackout{Harm to Ongoing Matter}}

\blackout{Harm to Ongoing Investigation}% 894
\footnote{\blackout{Harm to Ongoing Matter}}
\blackout{Harm to Ongoing Investigation}% 895
\footnote{\blackout{Harm to Ongoing Matter}}
\blackout{Harm to Ongoing Investigation}% 896
\footnote{\blackout{Harm to Ongoing Matter}}

\blackout{Harm to Ongoing Investigation}

\blackout{Harm to Ongoing Investigation}% 897
\footnote{\blackout{Harm to Ongoing Matter}}
\blackout{Harm to Ongoing Investigation}% 898
\footnote{\blackout{Harm to Ongoing Matter}}
\blackout{Harm to Ongoing Investigation}% 899
\footnote{\blackout{Harm to Ongoing Matter}}
\blackout{Harm to Ongoing Investigation}% 900
\footnote{\blackout{Harm to Ongoing Matter}}

\blackout{Harm to Ongoing Investigation}% 901
\footnote{\blackout{Harm to Ongoing Matter}}
\blackout{Harm to Ongoing Investigation}% 902
\footnote{\blackout{Harm to Ongoing Matter}}
\blackout{Harm to Ongoing Investigation}% 903
\footnote{\blackout{Harm to Ongoing Matter}}

\blackout{Harm to Ongoing Investigation}% 904
\footnote{\blackout{Harm to Ongoing Matter}}

\blackout{Harm to Ongoing Investigation}% 905
\footnote{\blackout{Harm to Ongoing Matter}}
\blackout{Harm to Ongoing Investigation}% 906
\footnote{\blackout{Harm to Ongoing Matter}}
\blackout{Harm to Ongoing Investigation}% 907
\footnote{\blackout{Harm to Ongoing Matter}}
\blackout{Harm to Ongoing Investigation}% 908
\footnote{\blackout{Harm to Ongoing Matter}}

\begin{center}
\textbf{Analysis}
\end{center}

In analyzing the President's conduct towards Flynn, Manafort, \blackout{Harm to Ongoing Investigation}, the following evidence is relevant to the elements of obstruction of justice:

\underline{Obstructive act.}
The President's actions towards witnesses in the Special Counsel's investigation would qualify as obstructive if they had the natural tendency to prevent particular witnesses from testifying truthfully, or otherwise would have the probable effect of influencing, delaying, or preventing their testimony to law enforcement.

With regard to Flynn, the President sent private and public messages to Flynn encouraging him to stay strong and conveying that the President still cared about him before he began to cooperate with the government.
When Flynn's attorneys withdrew him from a joint defense agreement with the President, signaling that Flynn was potentially cooperating with the government, the President's personal counsel initially reminded Flynn's counsel of the President's warm feelings towards Flynn and said "that still remains."
But when Flynn's counsel reiterated that Flynn could no longer share information under a joint defense agreement, the President's personal counsel stated that the decision would be interpreted as reflecting Flynn's hostility towards the President.
That sequence of events could have had the potential to affect Flynn's decision to cooperate, as well as the extent of that cooperation.
Because of privilege issues, however, we could not determine whether the President was personally involved in or knew about the specific message his counsel delivered to Flynn's counsel.

With respect to Manafort, there is evidence that the President's actions had the potential to influence Manafort's decision whether to cooperate with the government.
The President and his personal counsel made repeated statements suggesting that a pardon was a possibility for Manafort, while also making it clear that the President did not want Manafort to "flip" and cooperate with the government.
On June 15, 2018, the day the judge presiding over Manafort's D.C. case was considering whether to revoke his bail, the President said that he "felt badly" for Manafort and stated, "I think a lot of it is very unfair."
And when asked about a pardon for Manafort, the President said, "I do want to see people treated fairly.
That's what it's all about." Later that day, after Manafort's bail was revoked, the President called it a "tough sentence" that was "Very unfair!"
Two days later, the President's personal counsel stated that individuals involved in the Special Counsel's investigation could receive a pardon "if in fact the [P]resident and his advisors ... come to the conclusion that you have been treated unfairly"' - using language that paralleled how the President had already described the treatment of Manafort.
Those statements, combined with the President's commendation of Manafort for being a "brave man" who "refused to" break'," suggested that a pardon was a more likely possibility if Manafort continued not to cooperate with the government.
And while Manafort eventually pleaded guilty pursuant to a cooperation agreement, he was found to have violated the agreement by lying to investigators.

The President's public statements during the Manafort trial, including during jury deliberations, also had the potential to influence the trial jury.
On the second day of trial, for example, the President called the prosecution a "terrible situation" and a "hoax" that "continues to stain our country" and referred to Manafort as a "Reagan/Dole darling" who was "serving solitary confinement" even though he was "convicted of nothing."
Those statements were widely picked up by the press.
While jurors were instructed not to watch or read news stories about the case and are presumed to follow those instructions, the President's statements during the trial generated substantial media coverage that could have reached jurors if they happened to see the statements or learned about them from others.
And the President's statements during jury deliberations that Manafort "happens to be a very good person" and that "it's very sad what they've done to Paul Manafort" had the potential to influence jurors who learned of the statements, which the President made just as jurors were considering whether to convict or acquit Manafort.

\blackout{Harm to Ongoing Investigation}

\underline{Nexus to an official proceeding.}
The President's actions towards Flynn, Manafort, appear to have been connected to pending or anticipated official proceedings involving each individual.
The President's conduct towards Flynn principally occurred when both were under criminal investigation by the Special Counsel's Office and press reports speculated about whether they would cooperate with the Special Counsel's investigation.
And the President's conduct towards Manafort was directly connected to the official proceedings involving him.
The President made statements about Manafort and the charges against him during Manafort's criminal trial.
And the President's comments about the prospect of Manafort "flipping" occurred when it was clear the Special Counsel continued to oversee grand jury proceedings.

\underline{Intent.}
Evidence concerning the President's intent related to Flynn as a potential witness is inconclusive.
As previously noted, because of privilege issues we do not have evidence establishing whether the President knew about or was involved in his counsel's communications with Flynn's counsel stating that Flynn's decision to withdraw from the joint defense agreement and cooperate with the government would be viewed as reflecting "hostility" towards the President.
And regardless of what the President's personal counsel communicated, the President continued to express sympathy for Flynn after he pleaded guilty pursuant to a cooperation agreement, stating that Flynn had "led a very strong life" and the President "fe[lt] very badly" about what had happened to him.

Evidence concerning the President's conduct towards Manafort indicates that the President intended to encourage Manafort to not cooperate with the government.
Before Manafort was convicted, the President repeatedly stated that Manafort had been treated unfairly.
One day after Manafort was convicted on eight felony charges and potentially faced a lengthy prison term, the President said that Manafort was "a brave man" for refusing to "break" and that "flipping" "almost ought to be outlawed."
At the same time, although the President had privately told aides he did not like Manafort, he publicly called Manafort "a good man" and said he had a "wonderful family."
And when the President was asked whether he was considering a pardon for Manafort, the President did not respond directly and instead said he had "great respect for what [Manafort]'s done, in terms of what he's gone through."
The President added that "some of the charges they threw against him, every consultant, every lobbyist in Washington probably does."
In light of the President's counsel's previous statements that the investigations "might get cleaned up with some presidential pardons" and that a pardon would be possible if the President "come[s] to the conclusion that you have been treated unfairly," the evidence supports the inference that the President intended Manafort to believe that he could receive a pardon, which would make cooperation with the government as a means of obtaining a lesser sentence unnecessary.

We also examined the evidence of the President's intent in making public statements about Manafort at the beginning of his trial and when the jury was deliberating.
Some evidence supports a conclusion that the President intended, at least in part, to influence the jury.
The trial generated widespread publicity, and as the jury began to deliberate, commentators suggested that an acquittal would add to pressure to end the Special Counsel's investigation.
By publicly stating on the second day of deliberations that Manafort "happens to be a very good person" and that "it's very sad what they've done to Paul Manafort" right after calling the Special Counsel's investigation a "rigged witch hunt," the President's statements could, if they reached jurors, have the natural tendency to engender sympathy for Manafort among jurors, and a fact finder could infer that the President intended that result.
But there are alternative explanations for the President's comments, including that he genuinely felt sorry for Manafort or that his goal was not to influence the jury but to influence public opinion.
The President's comments also could have been intended to continue sending a message to Manafort that a pardon was possible.
As described above, the President made his comments about Manafort being "a very good person" immediately after declining to answer a question about whether he would pardon Manafort.

\blackout{Harm to Ongoing Investigation}

\subsection{The President's Conduct Involving Michael Cohen}

\begin{center}
\textbf{Overview}
\end{center}

The President's conduct involving Michael Cohen spans the full period of our investigation.
During the campaign, Cohen pursued the Trump Tower Moscow project on behalf of the Trump Organization.
Cohen briefed candidate Trump on the project numerous times, including discussing whether Trump should travel to Russia to advance the deal.
After the media began questioning Trump's connections to Russia, Cohen promoted a "party line" that publicly distanced Trump from Russia and asserted he had no business there.
Cohen continued to adhere to that party line in 2017, when Congress asked him to provide documents and testimony in its Russia investigation.
In an attempt to minimize the President's connections to Russia, Cohen submitted a letter to Congress falsely stating that he only briefed Trump on the Trump Tower Moscow project three times, that he did not consider asking Trump to travel to Russia, that Cohen had not received a response to an outreach he made to the Russian government, and that the project ended in January 2016, before the first Republican caucus or primary.
While working on the congressional statement, Cohen had extensive discussions with the President's personal counsel, who, according to Cohen, said that Cohen should not contradict the President and should keep the statement short and "tight."
After the FBI searched Cohen's home and office in April 2018, the President publicly asserted that Cohen would not "flip" and privately passed messages of support to him.
Cohen also discussed pardons with the President's personal counsel and believed that if he stayed on message, he would get a pardon or the President would do "something else" to make the investigation end.
But after Cohen began cooperating with the government in July 2018, the President publicly criticized him, called him a "rat," and suggested his family members had committed crimes.

\begin{center}
\textbf{Evidence}
\end{center}

\subsubsection{Candidate Trump's Awareness of and Involvement in the Trump Tower Moscow Project}

The President's interactions with Cohen as a witness took place against the background of the President's involvement in the Trump Tower Moscow project.

As described in detail in Volume I, Section IV.A.1, supra, from September 2015 until at least June 2016, the Trump Organization pursued a Trump Tower Moscow project in Russia, with negotiations conducted by Cohen, then-executive vice president of the Trump Organization and special counsel to Donald J. Trump.% 909
\footnote{In August 2018 and November 2018, Cohen pleaded guilty to multiple crimes of deception, including making false statements to Congress about the Trump Tower Moscow project, as described later in this section.
When Cohen first met with investigators from this Office, he repeated the same lies he told Congress about the Trump Tower Moscow project.
Cohen 8/7/18 302, at 12-17.
But after Cohen pleaded guilty to offenses in the Southern District of New York on August 21, 2018, he met with investigators again and corrected the record.
The Office found Cohen's testimony in these subsequent proffer sessions to be consistent with and corroborated by other information obtained in the course of the Office's investigation.
The Office's sentencing submission in Cohen's criminal case stated: "Starting with his second meeting with the [Special Counsel's Office] in September 2018, the defendant has accepted responsibility not only for his false statements concerning the [Trump Tower] Moscow Project, but also his broader efforts through public statements and testimony before Congress to minimize his role in, and what he knew about, contacts between the [Trump Organization] and Russian interests during the course of the campaign....
The information provided by Cohen about the [Trump Tower] Moscow Project in these proffer sessions is consistent with and corroborated by other information obtained in the course of the [Special Counsel's Office's] investigation....
The defendant, without prompting by the [Special Counsel's Office], also corrected other false and misleading statements that he had made concerning his outreach to and contacts with Russian officials during the course of the campaign.
"Gov't Sentencing Submission at 4, \textit{United States v.\ Michael Cohen}, 1:18-cr-850 (S.D.N.Y. Dec.~7, 2018), Doc.~14.
At Cohen's sentencing, our Office further explained that Cohen had "provided valuable information ... while taking care and being careful to note what he knows and what he doesn't know."
Transcript at 19, \textit{United States v.\ Michael Cohen}, 1:18-cr-850 (S.D.N.Y. Dec.~12, 2018), Doc.~17 (Cohen 12/12/18 Transcript).}
The Trump Organization had previously and unsuccessfully pursued a building project in Moscow.% 910
\footnote{\textit{See} Volume I, Section IV.A.1, \textit{supra} (noting that starting in at least 2013, several employees of the Trump Organization, including then-president of the organization Donald J. Trump, pursued a Trump Tower Moscow deal with several Russian counterparties).}
According to Cohen, in approximately September 2015 he obtained internal approval from Trump to negotiate on behalf of the Trump Organization to have a Russian corporation build a tower in Moscow that licensed the Trump name and brand.% 911
\footnote{Cohen 9/12/18 302, at 1-4;
Cohen 8/7/18 302, at 15.}
Cohen thereafter had numerous brief conversations with Trump about the project.% 912
\footnote{Cohen 9/12/18 302, at 2, 4.}
Cohen recalled that Trump wanted to be updated on any developments with Trump Tower Moscow and on several occasions brought the project up with Cohen to ask what was happening on it.% 913
\footnote{Cohen 9/12/18 302, at 4.}
Cohen also discussed the project on multiple occasions with Donald Trump Jr.\ and Ivanka Trump.% 914
\footnote{Cohen 9/12/18 302, at 4, 10.}

In the fall of 2015, Trump signed a Letter of Intent for the project that specified highly lucrative terms for the Trump Organization.% 915
\footnote{MDC-H-000618-25 (10/28/15 Letter of Intent, signed by Donald J. Trump, Trump Acquisition, LLC and Andrey Rozov, I.C. Expert Investment Company);
Cohen 9/12/18 302, at 3;
Written Responses of Donald J. Trump (Nov.~20, 2018), at 15 (Response to Question III, Parts (a) through (g)).}
In December 2015, Felix Sater, who was handling negotiations between Cohen and the Russian corporation, asked Cohen for a copy of his and Trump's passports to facilitate travel to Russia to meet with government officials and possible financing partners.% 916
\footnote{MDC-H-000600 (12/19/15 Email, Sater to Cohen).}
Cohen recalled discussing the trip with Trump and requesting a copy of Trump's passport from Trump's personal secretary, Rhona Graff.% 917
\footnote{Cohen 9/12/18 302, at 5.}

By January 2016, Cohen had become frustrated that Sater had not set up a meeting with Russian government officials, so Cohen reached out directly by email to the office of Dmitry Peskov, who was Putin's deputy chief of staff and press secretary.% 918
\footnote{\textit{See} FS00004 (12/30/15 Text Message, Cohen to Sater), TRUMPORG\_MC\_000233 (1/11/16 Email, Cohen to pr\_peskova\@prpress.gof.ru);
MDC-H-000690 (1/14/16 Email, Cohen to info\@prpress.gov.ru);
TRUMPORG\_MC\_000235 (1/16/16 Email, Cohen to pr\_peskova\@prpress.gov.ru).}
On January 20, 2016, Cohen received an email response from Elena Poliakova, Peskov's personal assistant, and phone records confirm that they then spoke for approximately twenty minutes, during which Cohen described the Trump Tower Moscow project and requested assistance in moving the project forward.% 919
\footnote{1/20/16 Email, Poliakova to Cohen;
Call Records of Michael Cohen.
(Showing a 22-minute call on January 20, 2016, between Cohen and the number Poliakova provided in her email);
Cohen 9/12/18 302, at 2-3.
After the call, Cohen saved Poliakova's contact information in his Trump Organization Outlook contact list.
1/20/16 Cohen Microsoft Outlook Entry (6:22 a.m.).}
Cohen recalled briefing candidate Trump about the call soon afterwards.% 920
\footnote{Cohen 11/20/18 302, at 5.}
Cohen told Trump he spoke with a woman he identified as "someone from the Kremlin," and Cohen reported that she was very professional and asked detailed questions about the project.% 921
\footnote{Cohen 11/20/18 302, at 5-6;
Cohen 11/12/18 302, at 4.}
Cohen recalled telling Trump he wished the Trump Organization had assistants who were as competent as the woman from the Kremlin.% 922
\footnote{Cohen 11/20/18 302, at 5.}

Cohen thought his phone call renewed interest in the project.% 923
\footnote{Cohen 9/12/18 302, at 5.}
The day after Cohen's call with Poliakova, Sater texted Cohen, asking him to "[c]all me when you have a few minutes to chat ...It's about Putin they called today."% 924
\footnote{FS00011 (1/21/16 Text Messages, Sater \& Cohen).}
Sater told Cohen that the Russian government liked the project and on January 25, 2016, sent an invitation for Cohen to visit Moscow "for a working visit."% 925
\footnote{Cohen 9/12/18 302, at 5;
1/25/16 Email, Sater to Cohen (attachment).}
After the outreach from Sater, Cohen recalled telling Trump that he was waiting to hear back on moving the project forward.% 926
\footnote{Cohen 11/20/18 302, at 5.}

After January 2016, Cohen continued to have conversations with Sater about Trump Tower Moscow and continued to keep candidate Trump updated about those discussions and the status of the project.% 927
\footnote{Cohen 9/12/18 302, at 6.
In later congressional testimony, Cohen stated that he briefed Trump on the project approximately six times after January 2016.
\textit{Hearing on Issues Related to Trump Organization Before the House Oversight and Reform Committee}, 116th Cong.\ (Feb.~27, 2019) (CQ Cong.\ Transcripts, at 24) (testimony of Michael Cohen).}
Cohen recalled that he and Trump wanted Trump Tower Moscow to succeed and that Trump never discouraged him from working on the project because of the campaign.% 928
\footnote{Cohen 9/12/18 302, at 6.}
In March or April 2016, Trump asked Cohen if anything was happening in Russia.% 929
\footnote{Cohen 9/12/18 302, at 4.}
Cohen also recalled briefing Donald Trump Jr.\ in the spring - a conversation that Cohen said was not "idle chit chat" because Trump Tower Moscow was potentially a \$1 billion deal.% 930
\footnote{Cohen 9/12/18 302, at 10.}

Cohen recalled that around May 2016, he again raised with candidate Trump the possibility of a trip to Russia to advance the Trump Tower Moscow project.% 931
\footnote{Cohen 9/12/18 302, at 7.}
At that time, Cohen had received several texts from Sater seeking to arrange dates for such a trip.% 932
\footnote{Cohen 9/12/18 302, at 7.}
On May4, 2016, Sater wrote to Cohen, "I had a chat with Moscow.
ASSUMING the trip does happen the question is before or after the convention.....
Obviously the pre meeting trip (you only) can happen any time you want but the 2 big guys[is] the question.
I said I would confirm and revert."% 933
\footnote{FS00015 (5/4/16 Text Message, Sater to Cohen).}
Cohen responded, "My trip before Cleveland.
Trump once he becomes the nominee after the convention."% 934
\footnote{FS00015 (5/4/16 Text Message, Cohen to Sater).}
On May 5, 2016, Sater followed up with a text that Cohen thought he probably read to Trump:

\begin{quote}
Peskov would like to invite you as his guest to the St.~Petersburg Forum which is Russia's Davos it's June 16-19.
He wants to meet there with you and possibly introduce you to either Putin or Medvedev....
This is perfect.
The entire business class of Russia will be there as well.
He said anything you want to discuss including dates and subjects are on the table to discuss.% 935
\footnote{FS00016-17 (5/5/16 Text Messages, Sater \& Cohen).}
\end{quote}

Cohen recalled discussing the invitation to the St.~Petersburg Economic Forum with candidate Trump and saying that Putin or Russian Prime Minister Dmitry Medvedev might be there.% 936
\footnote{Cohen 9/12/18 302, at 7.}
Cohen remembered that Trump said that he would be willing to travel to Russia if Cohen could "lock and load" on the deal.% 937
\footnote{Cohen 9/12/18 302, at 7.}
In June 2016, Cohen decided not to attend the St.~Petersburg Economic Forum because Sater had not obtained a formal invitation for Cohen from Peskov.% 938
\footnote{Cohen 9/12/18 302, at 7-8.}
Cohen said he had a quick conversation with Trump at that time but did not tell him that the project was over because he did not want Trump to complain that the deal was on-again-off-again if it were revived.% 939
\footnote{Cohen 9/12/18 302, at 8.}

During the summer of 2016, Cohen recalled that candidate Trump publicly claimed that he had nothing to do with Russia and then shortly afterwards privately checked with Cohen about the status of the Trump Tower Moscow project, which Cohen found "interesting."% 940
\footnote{Cohen 9/12/18 302, at 2.}
At some point that summer, Cohen recalled having a brief conversation with Trump in which Cohen said the Trump Tower Moscow project was going nowhere because the Russian development company had not secured a piece of property for the project.% 941
\footnote{Cohen 3/19/19 302, at 2.
Cohen could not recall the precise timing of this conversation, but said he thought it occurred in June or July 2016.
Cohen recalled that the conversation happened at some point after candidate Trump was publicly stating that he had nothing to do with Russia.
Cohen 3/19/19 302, at 2.}
Trump said that was "too bad," and Cohen did not recall talking with Trump about the project after that.% 942
\footnote{Cohen 3/19/19 302, at 2.}
Cohen said that at no time during the campaign did Trump tell him not to pursue the project or that the project should be abandoned.% 943
\footnote{Cohen 3/19/19 302, at 2.}

\subsubsection{Cohen Determines to Adhere to a "Party Line" Distancing Candidate Trump From Russia}

As previously discussed, see Volume II, Section II.A, supra, when questions about possible Russian support for candidate Trump emerged during the 2016 presidential campaign, Trump denied having any personal, financial, or business connection to Russia, which Cohen described as the "party line" or "message" to follow for Trump and his senior advisors.% 944
\footnote{Cohen 11/20/18 302, at 1;
Cohen 9/18/18 302, at 3, 5;
Cohen 9/12/18 302, at 9.}

After the election, the Trump Organization sought to formally close out certain deals in advance of the inauguration.% 945
\footnote{Cohen 9/18/18 302, at 1-2;
\textit{see also} Rtskhiladze 4/4/18 302, at 8-9.}
Cohen recalled that Trump Tower Moscow was on the list of deals to be closed out.% 946
\footnote{Cohen 9/18/18 302, at 1-2.}
In approximately January 2017, Cohen began receiving inquiries from the media about Trump Tower Moscow, and he recalled speaking to the President-Elect when those inquiries came in.% 947
\footnote{Cohen 9/18/18 302, at 3.}
Cohen was concerned that truthful answers about the Trump Tower Moscow project might not be consistent with the "message" that the President-Elect had no relationship with Russia.% 948
\footnote{Cohen 9/18/18 302, at 4.}

In an effort to "stay on message," Cohen told a New York Times reporter that the Trump Tower Moscow deal was not feasible and had ended in January 2016.% 949
\footnote{Cohen 9/18/18 302, at 5.
The article was published on February 19, 2017, and reported that Sater and Cohen had been working on plan for a Trump Tower Moscow "as recently as the fall of 2015" but had come to a halt because of the presidential campaign.
Consistent with Cohen's intended party line message, the article stated, "Cohen said the Trump Organization had received a letter of intent for a project in Moscow from a Russian real estate developer at that time but determined that the project was not feasible."
Megan Twohey \& Scott Shane, \textit{A Back-Channel Plan for Ukraine and Russia, Courtesy of Trump Associates}, New York Times (Feb.~19, 2017).}
Cohen recalled that this was part of a "script" or talking point she had developed with President-Elect Trump and others to dismiss the idea of a substantial connection between Trump and Russia.% 950
\footnote{Cohen 9/18/18 302, at 5-6.}
Cohen said that he discussed the talking points with Trump but that he did not explicitly tell Trump he thought they were untrue because Trump already knew they were untrue.% 951
\footnote{Cohen 9/18/18 302, at 6.}
Cohen thought it was important to say the deal was done in January 2016, rather than acknowledge that talks continued in May and June 2016, because it limited the period when candidate Trump could be alleged to have a relationship with Russia to an early point in the campaign, before Trump had become the party's presumptive nominee.% 952
\footnote{Cohen 9/18/18 302, at 10.}

\subsubsection{Cohen Submits False Statements to Congress Minimizing the Trump Tower Moscow Project in Accordance with the Party Line}

In early May 2017, Cohen received requests from Congress to provide testimony and documents in connection with congressional investigations of Russian interference in the 2016 election.% 953
\footnote{P-SCO-000000328 (5/9/17 Letter, HPSCI to Cohen);
P-SCO-000000331 (5/12/17 Letter, SSCI to Cohen).}
At that time, Cohen understood Congress's interest in him to be focused on the allegations in the Steele reporting concerning a meeting Cohen allegedly had with Russian officials in Prague during the campaign.% 954
\footnote{Cohen 11/20/18 302, at 2-3.}
Cohen had never traveled to Prague and was not concerned about those allegations, which he believed were provably false.% 955
\footnote{Cohen 11/20/18 302, at 2-3.}
On May 18, 2017, Cohen met with the President to discuss the request from Congress, and the President instructed Cohen that he should cooperate because there was nothing there.% 956
\footnote{Cohen 11/12/18 302, at 2;
Cohen 11/20/19 302, at 3.}

Cohen eventually entered into a joint defense agreement (JDA) with the President and other individuals who were part of the Russia investigation.% 957
\footnote{Cohen 11/12/18 302, at 2.}
In the months leading up to his congressional testimony, Cohen frequently spoke with the President's personal counsel.% 958
\footnote{Cohen 11/12/18 302, at 2-3;
Cohen 11/20/18, at 2-6.
Cohen told investigators about his conversations with the President's personal counsel after waiving any privilege of his own and after this Office advised his counsel not to provide any communications that would be covered by any other privilege, including communications protected by a joint defense or common interest privilege.
As a result, most of what Cohen told us about his conversations with the President's personal counsel concerned what Cohen had communicated to the President's personal counsel, and not what was said in response.
Cohen described certain statements made by the President's personal counsel, however, that are set forth in this section.
Cohen and his counsel were better positioned than this Office to evaluate whether any privilege protected those statements because they had knowledge of the scope oft heir joint defense agreement and access to privileged communications that may have provided context for evaluating the statements they shared.
After interviewing Cohen about these matters, we asked the President's personal counsel if he wished to provide information to us about his conversations with Cohen related to Cohen's congressional testimony about Trump Tower Moscow.
The President's personal counsel declined and, through his own counsel, indicated that he could not disaggregate information he had obtained from Cohen from information he had obtained from other parties in the JDA.
In view of the admonition this Office gave to Cohen's counsel to withhold communications that could be covered by privilege, the President's personal counsel's uncertainty about the provenance of his own knowledge, the burden on a privilege holder to establish the elements to support a claim of privilege, and the substance of the statements themselves, we have included relevant statements Cohen provided in this report.
If the statements were to be used in a context beyond this report, further analysis could be warranted.}
Cohen said that in those conversations the President's personal counsel would sometimes say that he had just been with the President.% 959
\footnote{Cohen 11/20/18 302, at 6.}
Cohen recalled that the President's personal counsel told him the JDA was working well together and assured him that there was nothing there and if they stayed on message the investigations would come to an end soon.% 960
\footnote{Cohen 11/20/18 302, at 2, 4.}
At that time, Cohen's legal bills were being paid by the Trump Organization,% 961
\footnote{Cohen 11/20/18 302, at 4.}
and Cohen was told not to worry because the investigations would be over by summer or fall of 2017.% 962
\footnote{Cohen 9/18/18 302, at 8;
Cohen 11/20/18 302, at 3-4.}
Cohen said that the President's personal counsel also conveyed that, as part of the JDA, Cohen was protected, which he would not be if he "went rogue."% 963
\footnote{Cohen 11/20/18 302, at 4.}
Cohen recalled that the President's personal counsel reminded him that "the President loves you" and told him that if he stayed on message, the President had his back.% 964
\footnote{Cohen 9/18/18 302, at 11;
Cohen 11/20/18 302, at 2.}

In August 2017, Cohen began drafting a statement about Trump Tower Moscow to submit to Congress along with his document production.% 965
\footnote{P-SCO-000003680 and P-SCO-0000003687 (8/16/17 Email and Attachment, Michael Cohen's Counsel to Cohen).
Cohen said it was not his idea to write a letter to Congress about Trump Tower Moscow.
Cohen 9/18/18 302, at 7.}
The final version of the statement contained several false statements about the project.% 966
\footnote{P-SCO-00009478 (Statement of Michael D. Cohen, Esq.\ (Aug.~28, 2017)).}
First, although the Trump Organization continued to pursue the project until at least June 2016, the statement said, "The proposal was under consideration at the Trump Organization from September 2015 until the end of January 2016.
By the end of January 2016, I determined that the proposal was not feasible for a variety of business reasons and should not be pursued further.
Based on my business determinations, the Trump Organization abandoned the proposal."% 967
\footnote{P-SCO-00009478 (Statement of Michael D. Cohen, Esq.\ (Aug.~28, 2017)).}
Second, although Cohen and candidate Trump had discussed possible travel to Russia by Trump to pursue the venture, the statement said, "Despite overtures by Mr.~Sater, I never considered asking Mr.~Trump to travel to Russia in connection with this proposal.
I told Mr.~Sater that Mr.~Trump would not travel to Russia unless there was a definitive agreement in place."% 968
\footnote{P-SCO-00009478 (Statement of Michael D. Cohen, Esq.\ (Aug.~28, 2017)).}
Third, although Cohen had regularly briefed Trump on the status of the project and had numerous conversations about it, the statement said, "Mr.~Trump was never in contact with anyone about this proposal other than me on three occasions, including signing a non-binding letter of intent in 2015."% 969
\footnote{P-SCO-00009478 (Statement of Michael D. Cohen, Esq.\ (Aug.~28, 2017)).}
Fourth, although Cohen's outreach to Peskov in January 2016 had resulted in a lengthy phone call with a representative from the Kremlin, the statement said that Cohen did "not recall any response to my email [to Peskov], nor any other contacts by me with Mr.~Peskov or other Russian government officials about the proposal."% 970
\footnote{P-SCO-00009478 (Statement of Michael D. Cohen, Esq.\ (Aug.~28, 2017)).}

Cohen's statement was circulated in advance to, and edited by, members of the JDA.% 971
\footnote{Cohen 9/12/18 302, at 8-9.
Cohen also testified in Congress that the President's counsel reviewed and edited the statement.
\textit{Hearing on Issues Related to Trump Organization Before the House Oversight and Reform Committee}, 116th Cong.\ (Feb.~27, 2019) (CQ Cong.\ Transcripts, at 24-25) (testimony by Michael Cohen).
Because of concerns about the common interest privilege, we did not obtain or review all drafts of Cohen's statement.
Based on the drafts that were released through this Office's filter process, it appears that the substance of the four principal false statements described above were contained in an early draft prepared by Cohen and his counsel.
P-SCO-0000003680 and P-SCO-0000003687 (8/16/17 Email and Attachment, Cohen's counsel to Cohen).}
Before the statement was finalized, early drafts contained a sentence stating, "The building project led me to make limited contacts with Russian government officials."% 972
\footnote{P-SCO-0000003687 (8/16/17 Draft Statement of Michael Cohen);
Cohen 11/20/18 302, at 4.}
In the final version of the statement, that line was deleted.% 973
\footnote{Cohen 11/20/18 302, at 4.
A different line stating that Cohen did "not recall any response to my email [to Peskov in January 2016], nor any other contacts by me with Mr.~Peskov or other Russian government officials about the proposal" remained in the draft.
\textit{See} P-SCO-0000009478 (Statement of Michael D. Cohen, Esq.\ (Aug.~28, 2017)).}
Cohen though the was told that it was a decision of the JDA to take out that sentence, and he did not push back on the deletion.% 974
\footnote{Cohen 11/20/18 302, at 4.}
Cohen recalled that he told the President's personal counsel that he would not contest a decision of the JDA.% 975
\footnote{Cohen 11/20/18 302, at 5.}

Cohen also recalled that in drafting his statement for Congress, he spoke with the President's personal counsel about a different issue that connected candidate Trump to Russia: Cohen's efforts to set up a meeting between Trump and Putin in New York during the 2015 United Nations General Assembly.% 976
\footnote{Cohen 9/18/18 302, at 10-11.}
In September 2015, Cohen had suggested the meeting to Trump, who told Cohen to reach out to Putin's office about it.% 977
\footnote{Cohen 9/18/18 302, at 11;
Cohen 11/12/18 302, at 4.}
Cohen spoke and emailed with a Russian official about a possible meeting, and recalled that Trump asked him multiple times for updates on the proposed meeting with Putin.% 978
\footnote{Cohen 9/18/18 302, at 11;
Cohen 11/12/18 302, at 5.}
When Cohen called the Russian official a second time, she told him it would not follow proper protocol for Putin to meet with Trump, and Cohen relayed that message to Trump.% 979
\footnote{Cohen 11/12/18 302, at 5.}
Cohen anticipated he might be asked questions about the proposed Trump- Putin meeting when he testified before Congress because he had talked about the potential meeting on Sean Hannity's radio show.% 980
\footnote{Cohen 9/18/18 302, at 11.}
Cohen recalled explaining to the President's personal counsel the "whole story" of the attempt to set up a meeting between Trump and Putin and Trump's role in it.% 981
\footnote{Cohen 3/19/19 302, at 2.}
Cohen recalled that he and the President's personal counsel talked about keeping Trump out of the narrative, and the President's personal counsel told Cohen the story was not relevant and should not be included in his statement to Congress.% 982
\footnote{Cohen 3/19/19 302, at 2;
\textit{see} Cohen 9/18/18 302, at 11 (recalling that he was told that if he stayed on message and kept the President out of the narrative, the President would have his back).}

Cohen said that his "agenda" in submitting the statement to Congress with false representations about the Trump Tower Moscow project was to minimize links between the project and the President, give the false impression that the project had ended before the first presidential primaries, and shut down further inquiry into Trump Tower Moscow, with the aim of limiting the ongoing Russia investigations.% 983
\footnote{Cohen 9/12/18 302, at 8;
Information at 4-5, \textit{United States v.\ Michael Cohen}, 1:18-cr-850 (S.D.N.Y. Nov.~29, 2018), Doc.~2 (\textit{Cohen} Information).}
Cohen said he wanted to protect the President and be loyal to him by not contradicting anything the President had said.% 984
\footnote{Cohen 11/20/18 302, at 4.}
Cohen recalled he was concerned that if he told the truth about getting a response from the Kremlin or speaking to candidate Trump about travel to Russia to pursue the project, he would contradict the message that no connection existed between Trump and Russia, and he rationalized his decision to provide false testimony because the deal never happened.% 985
\footnote{Cohen 11/20/18 302, at 4;
Cohen 11/12/18 302, at 2-3, 4, 6.}
He was not concerned that the story would be contradicted by individuals who knew it was false because he was sticking to the party line adhered to by the whole group.% 986
\footnote{Cohen 9/12/18 302, at 9.}
Cohen wanted the support of the President and the White House, and he believed that following the party line would help put an end to the Special Counsel and congressional investigations.% 987
\footnote{Cohen 9/12/18 302, at 8-9.}

Between August 18, 2017, when the statement was in an initial draft stage, and August 28, 2017, when the statement was submitted to Congress, phone records reflect that Cohen spoke with the President's personal counsel almost daily.% 988
\footnote{Cohen 11/12/18 302, at 2-3;
Cohen 11/20/18 302, at 5;
Call Records of Michael Cohen (Reflecting three contacts on August 18, 2017 (24 seconds;
5 minutes 25 seconds;
and 10 minutes 58 seconds);
two contacts on August 19 (23 seconds and 24 minutes 26 seconds);
three contacts on August 23 (8 seconds; 20 minutes 33 seconds;
and 5 minutes 8 seconds);
one contact on August 24 (11 minutes 59 seconds);
14 contacts on August 27 (28 seconds;
4 minutes 37 seconds;
1 minute 16 seconds;
1 minutes 35 seconds;
6 minutes 16 seconds;
1 minutes 10 seconds;
3 minutes 5 seconds;
18 minutes 55 seconds;
4 minutes 56 seconds;
11 minutes 6 seconds;
8 seconds;
3 seconds;
2 seconds;
2 seconds).}
On August 27, 2017, the day before Cohen submitted the statement to Congress, Cohen and the President's personal counsel had numerous contacts by phone, including calls lasting three, four, six, eleven, and eighteen minutes.% 989
\footnote{Cohen 11/20/18 302, at 5;
Call Records of Michael Cohen.
(Reflecting 14 contacts on August 27, 2017 (28 seconds;
4 minutes 37 seconds;
1 minute 16 seconds;
1 minutes 35 seconds;
6 minutes 16 seconds;
1 minutes 10 seconds;
3 minutes 5 seconds;
18 minutes 55 seconds;
4 minutes 56 seconds;
11 minutes 6 seconds;
8 seconds;
3 seconds;
2 seconds;
2 seconds)).
}
Cohen recalled telling the President's personal counsel, who did not have first-hand knowledge of the project, that there was more detail on Trump Tower Moscow that was not in the statement, including that there were more communications with Russia and more communications with candidate Trump than the statement reflected.% 990
\footnote{Cohen 11/20/18 302, at 5.}
Cohen stated that the President's personal counsel responded that it was not necessary to elaborate or include those details because the project did not progress and that Cohen should keep his statement short and "tight" and the matter would soon come to an end.% 991
\footnote{Cohen 11/20/18 302, at 5.
Cohen also vaguely recalled telling the President's personal counsel that he spoke with a woman from the Kremlin and that the President's personal counsel responded to the effect of "so what?" because the deal never happened.
Cohen 11/20/18 302, at 5.}
Cohen recalled that the President's personal counsel said "his client" appreciated Cohen, that Cohen should stay on message and not contradict the President, that there was no need to muddy the water, and that it was time to move on.% 992
\footnote{Cohen 11/20/18 302, at 5.}
Cohen said he agreed because it was what he was expected to do.% 993
\footnote{Cohen 11/20/18 302, at 5.}
After Cohen later pleaded guilty to making false statements to Congress about the Trump Tower Moscow project, this Office sought to speak with the President's personal counsel about these conversations with Cohen, but counsel declined, citing potential privilege concerns.% 994
\footnote{9/8/19 email, Counsel for personal counsel to the President to Special Counsel's Office.}

At the same time that Cohen finalized his written submission to Congress, he served as a source for a Washington Post story published on August 27, 2017, that reported in depth for the first time that the Trump Organization was "pursuing a plan to develop a massive Trump Tower in Moscow" at the same time as candidate Trump was "running for president in late 2015 and early 2016."% 995
\footnote{Cohen 9/18/18 302, at 7;
Carol D. Leonnig et al., \textit{Trump's business sought deal on a Trump Tower in Moscow while he ran for president}, Washington Post (Aug.~27, 2017).}
The article reported that "the project was abandoned at the end of January 2016, just before the presidential primaries began, several people familiar with the proposal said."% 996
\footnote{Carol D. Leonnig et al., \textit{Trump's business sought deal on a Trump Tower in Moscow while he ran for president}, Washington Post (Aug.~27, 2017).}
Cohen recalled that in speaking to the Post, he held to the false story that negotiations for the deal ceased in January 2016.% 997
\footnote{Cohen 9/18/18 302, at 7.}

On August 28, 2017, Cohen submitted his statement about the Trump Tower Moscow project to Congress.% 998
\footnote{P-SCO-000009477 - 9478 (8/28/17 Letter and Attachment, Cohen to SSCI).}
Cohen did not recall talking to the President about the specifics of what the statement said or what Cohen would later testify to about Trump Tower Moscow.% 999
\footnote{Cohen 11/12/18 302, at 2;
Cohen 9/12/18 302, at 9.}
He recalled speaking to the President more generally about how he planned to stay on message in his testimony.% 1000
\footnote{Cohen 9/12/18 302, at 9.}
On September 19, 2017, in anticipation of his impending testimony, Cohen orchestrated the public release of his opening remarks to Congress, which criticized the allegations in the Steele material and claimed that the Trump Tower Moscow project "was terminated in January of 2016; which occurred before the Iowa caucus and months before the very first primary."% 1001
\footnote{Cohen 9/18/18 302, at 7;
\textit{see, e.g., READ: Michael Cohen's statement to the Senate intelligence committee}, CNN (Sept.~19, 2017).}
Cohen said the release of his opening remarks was intended to shape the narrative and let other people who might be witnesses know what Cohen was saying so they could follow the same message.% 1002
\footnote{Cohen 9/18/18 302, at 7.}
Cohen said his decision was meant to mirror Jared Kushner's decision to release a statement in advance of Kushner's congressional testimony, which the President's personal counsel had told Cohen the President liked.% 1003
\footnote{Cohen 9/18/18 302, at 7;
Cohen 11/20/18 302, at 6.}
Cohen recalled that on September 20, 2017, after Cohen's opening remarks had been printed by the media, the President's personal counsel told him that the President was pleased with the Trump Tower Moscow statement that had gone out.% 1004
\footnote{Cohen 11/20/18 302, at 6.
Phone records show that the President's personal counsel called Cohen on the morning of September 20, 2017, and they spoke for approximately 11 minutes, and that they had two more contacts that day, one of which lasted approximately 18 minutes.
Call Records of Michael Cohen. (Reflecting three contacts on September 20, 2017, with calls lasting for 11 minutes 3 seconds; 2 seconds; and 18 minutes 38 seconds).}

On October 24 and 25, 2017, Cohen testified before Congress and repeated the false statements he had included in his written statement about Trump Tower Moscow.% 1005
\footnote{\textit{Cohen} Information, at 4;
Executive Session, Permanent Select Committee on Intelligence, U.S. House of Representatives, Interview of Michael Cohen (Oct.~24, 2017), at 10-11, 117-119.}
Phone records show that Cohen spoke with the President's personal counsel immediately after his testimony on both days.% 1006
\footnote{Call Records of Michael Cohen.
(Reflecting two contacts on October 24, 2017 (12 minutes 8 seconds and 8 minutes 27 seconds) and three contacts on October 25, 2017 (1 second; 4 minutes 6 seconds; and 6 minutes 6 seconds)).}

\subsubsection{The President Sends Messages of Support to Cohen}

In January 2018, the media reported that Cohen had arranged a \$130,000 payment during the campaign to prevent a woman from publicly discussing an alleged sexual encounter she had with the President before he ran for office.% 1007
\footnote{\textit{See, e.g.}, Michael Rothfeld \& Joe Palazzolo, \textit{Trump Lawyer Arranged \$130,000 Payment for Adult-Film Star's Silence}, Wall Street Journal (Jan.~12, 2018).}
This Office did not investigate Cohen's campaign- period payments to women.% 1008
\footnote{The Office was authorized to investigate Cohen's establishment and use of Essential Consultants LLC, which Cohen created to facilitate the \$130,000 payment during the campaign, based on evidence that the entity received funds from Russian-backed entities.
Cohen's use of Essential Consultants to facilitate the \$130,000 payment to the woman during the campaign was part of the Office's referral of certain Cohen-related matters to the U.S. Attorney's Office for the Southern District of New York.}
However, those events, as described here, are potentially relevant to the President's and his personal counsel's interactions with Cohen as a witness who later began to cooperate with the government.

On February 13, 2018, Cohen released a statement to news organizations that stated, "In a private transaction in 2016, I used my own personal funds to facilitate a payment of \$130,000 to [the woman].
Neither the Trump Organization nor the Trump campaign was a party to the May-Contain- transaction with [the woman], and neither reimbursed me for the payment, either directly or indirectly."% 1009
\footnote{\textit{See, e.g.}, Mark Berman, \textit{Longtime Trump attorney says he made \$130,000 payment to Stormy Daniels with his money}, Washington Post (Feb.~14, 2018).}
In congressional testimony on February27, 2019, Cohen testified that he had discussed what to say about the payment with the President and that the President had directed Cohen to say that the President "was not knowledgeable ... of [Cohen's] actions" in making the payment.% 1010
\footnote{\textit{Hearing on Issues Related to Trump Organization Before the House Oversight and Reform Committee}, 116th Cong.\ (Feb.~27, 2019) (CQ Cong.\ Transcripts, at 147-148) (testimony of Michael Cohen).
Toll records show that Cohen was connected to a White House phone number for approximately five minutes on January 19, 2018, and for approximately seven minutes on January 30, 2018, and that Cohen called Melania Trump's cell phone several times between January 26, 2018, and January 30, 2018.
Call Records of Michael Cohen.}
On February 19, 2018, the day after the New York Times wrote a detailed story attributing the payment to Cohen and describing Cohen as the President's "fixer," Cohen received a text message from the President's personal counsel that stated, "Client says thanks for what you do.''% 1011
\footnote{9/19/18 Text Message, President's personal counsel to Cohen;
\textit{see} Jim Rutenberg et al., \textit{Tools of Trump's Fixer: Payouts, Intimidation and the Tabloids}, New York Times (Feb.~18, 2018).}

On April 9, 2018, FBI agents working with the U.S. Attorney's Office for the Southern District of New York executed search warrants on Cohen's home, hotel room, and office.% 1012
\footnote{Gov't Opp.\ to Def.\ Mot.\ for Temp.\ Restraining Order, \textit{In the Matter of Search Warrants Executed on April 9, 2018}, 18-mj-3161 (S.D.N.Y. Apr.~13, 2018), Doc.~1
("On April 9, 2018, agents from the New York field office of the Federal Bureau of Investigation... executed search warrants for Michael Cohen's residence, hotel room, office, safety deposit box, and electronic devices.").}
That day, the President spoke to reporters and said that he had "just heard that they broke into the office of one of my personal attorneys - a good man."% 1013
\footnote{Remarks by President Trump Before Meeting with Senior Military Leadership, White House (Apr.~9, 2018).}
The President called the searches "a real disgrace" and said, "It's an attack on our country, in a true sense.
It's an attack on what we all stand for."% 1014
\footnote{Remarks by President Trump Before Meeting with Senior Military Leadership, White House (Apr.~9, 2018).}
Cohen said that after the searches he was concerned that he was "an open book," that he did not want issues arising from the payments to women to "come out," and that his false statements to Congress were ''a big concern."% 1015
\footnote{Cohen, 10/17/18 302, at 11.}

A few days after the searches, the President called Cohen.% 1016
\footnote{Cohen 3/19/19 302, at 4.}
According to Cohen, the President said he wanted to "check in" and asked if Cohen was okay, and the President encouraged Cohen to "hang in there" and "stay strong."% 1017
\footnote{Cohen 3/19/19 302, at 4.}
Cohen also recalled that following the searches he heard from individuals who were in touch with the President and relayed to Cohen the President's support for him.% 1018
\footnote{Cohen 9/12/18 302, at 11.}
Cohen recalled that \blackout{Personal Privacy}, a friend of the President's, reached out to say that he was with "the Boss" in Mar-a-Lago and the President had said "he loves you" and not to worry.% 1019
\footnote{Cohen 9/12/18 302, at 11.}
Cohen recalled that \blackout{Personal Privacy} for the Trump Organization, told him, ``the boss loves you.''% 1020
\footnote{Cohen 9/12/18 302, at 11.}
And Cohen said that \blackout{Personal Privacy}, a friend of the President's, told him, "everyone knows the boss has your back."% 1021
\footnote{Cohen 9/12/18 302, at 11.}

On or about April 17, 2018, Cohen began speaking with an attorney, Robert Costello, who had a close relationship with Rudolph Giuliani, one of the President's personal lawyers.% 1022
\footnote{4/17/18 Email, Citron to Cohen;
4/19/18 Email, Costello to Cohen;
MC-SCO-001 (7/7/18 redacted billing statement from Davidoff, Hutcher \& Citron to Cohen).}
Costello told Cohen that he had a "back channel of communication" to Giuliani, and that Giuliani had said the "channel" was "crucial" and "must be maintained."% 1023
\footnote{4/21/18 Email, Costello to Cohen.}
On April 20, 2018, the New York Times published an article about the President's relationship with and treatment of Cohen.% 1024
\footnote{See Maggie Haberman et al., \textit{Michael Cohen Has Said He Would Take a Bullet for Trump. Maybe Not Anymore.}, New York Times (Apr.~20, 2018).}
The President responded with a series of tweets predicting that Cohen would not "flip":

\begin{quote}
The New York Times and a third rate reporter ... are going out of their way to destroy Michael Cohen and his relationship with me in the hope that he will 'flip.'
They use non- existent 'sources' and a drunk/drugged up loser who hates Michael, a fine person with a wonderful family.
Michael is a businessman for his own account/lawyer who I have always liked \& respected.
Most people will flip if the Government lets them out of trouble, even if it means lying or making up stories.
Sorry, I don't see Michael doing that despite the horrible Witch Hunt and the dishonest media!% 1025
\footnote{\@realDonaldTrump 4/21/18 (9:10 a.m.~ET) Tweets.}
\end{quote}

In an email that day to Cohen, Costello wrote that he had spoken with Giuliani.% 1026
\footnote{4/91/18 Email, Costello to Cohen.}
Costello told Cohen the conversation was "Very Very Positive[.] You are 'loved' ... they are in our corner .... Sleep well tonight[], you have friends in high places."% 1027
\footnote{4/21/18 Email, Costello to Cohen. \blackout{Harm to Ongoing Matter}}

Cohen said that following these messages he believed he had the support of the White House if he continued to toe the party line, and he determined to stay on message and be part of the team.% 1028
\footnote{Cohen 9/12/18 302, at 11.}
At the time, Cohen's understood that his legal fees were still being paid by the Trump Organization, which he said was important to him.% 1029
\footnote{Cohen 9/12/18 302, at 10.}
Cohen believed he needed the power of the President to take care of him, so he needed to defend the President and stay on message.% 1030
\footnote{Cohen 9/12/18 302, at 10.}

Cohen also recalled speaking with the President's personal counsel about pardons after the searches of his home and office had occurred, at a time when the media had reported that pardon discussions were occurring at the White House.% 1031
\footnote{Cohen 11/20/18 302, at 7.
At a White House press briefing on April 23, 2018, in response to a question about whether the White House had "close[d] the door one way or the other on the President pardoning Michael Cohen," Sanders said, "It's hard to close the door on something that hasn't taken place.
I don't like to discuss or comment on hypothetical situations that may or may not ever happen.
I would refer you to personal attorneys to comment on anything specific regarding that case, but we don't have anything at this point."
Sarah Sanders, \textit{White House Daily Briefing}, C-SPAN (Apr.~23, 2018).}
Cohen told the President's personal counsel he had been a loyal lawyer and servant, and he said that after the searches he was in an uncomfortable position and wanted to know what was in it for him.% 1032
\footnote{Cohen 11/20/18 302, at 7;
Cohen 3/19/19 302, at 3.}
According to Cohen, the President's personal counsel responded that Cohen should stay on message, that the investigation was a witch hunt, and that everything would be fine.% 1033
\footnote{Cohen 3/19/19 302, at 3.}
Cohen understood based on this conversation and previous conversations about pardons with the President's personal counsel that as long as he stayed on message, he would be taken care of by the President, either through a pardon or through the investigation being shut down.% 1034
\footnote{Cohen 3/19/19 302, at 3-4.}

On April 24, 2018, the President responded to a reporter's inquiry whether he would consider a pardon for Cohen with, "Stupid question."% 1035
\footnote{Remarks by President Trump and President Macron of France Before Restricted Bilateral Meeting, The White House (Apr.~24, 2018).}
On June 8, 2018, the President said he "hadn't even thought about" pardons for Manafort or Cohen, and continued, "It's far too early to be thinking about that.
They haven't been convicted of anything.
There's nothing to pardon."% 1036
\footnote{\textit{President Donald Trump Holds Media Availability Before Departing for the G-7 Summit}, CQ Newsmaker Transcripts (June 8, 2018).}
And on June 15, 2018, the President expressed sympathy for Cohen, Manafort, and Flynn in a press interview and said, "I feel badly about a lot of them, because I think lot of it is very unfair."% 1037
\footnote{Remarks by President Trump in Press Gaggle, The White House (June 15, 2018).}

\subsubsection{The President's Conduct After Cohen Began Cooperating with the Government}

On July 2, 2018, ABC News reported based on an "exclusive" interview with Cohen that Cohen "strongly signaled his willingness to cooperate with special counsel Robert Mueller and federal prosecutors in the Southern District of New York - even if that puts President Trump in jeopardy."% 1038
\footnote{\textit{EXCLUSIVE: Michael Cohen says family and country, not President Trump, is his 'first loyalty'}, ABC (July 2, 2018).
Cohen said in the interview, "To be crystal clear, my wife, my daughter and my son, and this country have my first loyalty."}
That week, the media reported that Cohen had added an attorney to his legal team who previously had worked as a legal advisor to President Bill Clinton.% 1039
\footnote{\textit{See e.g.}, Darren Samuelsohn, \textit{Michael Cohen hires Clinton scandal veteran Lanny Davis}, Politico (July 5, 2018).}

Beginning on July 20, 2018, the media reported on the existence of a recording Cohen had made of a conversation he had with candidate Trump about a payment made to a second woman who said she had had an affair with Trump.% 1040
\footnote{\textit{See, e.g.}, Matt Apuzzo et al., \textit{Michael Cohen Secretly Taped Trump Discussing Payment to Playboy Model}, New York Times (July 20, 2018).}
On July 21, 2018, the President responded: "Inconceivable that the government would break into a lawyer's office (early in the morning) - almost unheard of.
Even more inconceivable that a lawyer would tape a client - totally unheard of \& perhaps illegal.
The good news is that your favorite President did nothing wrong!"% 1041
\footnote{\@realDonaldTrump 7/21/18 (8:10 a.m.~ET) Tweet.}
On July 27, 2018, after the media reported that Cohen was willing to inform investigators that Donald Trump Jr.\ told his father about the June 9, 2016 meeting to get "dirt" on Hillary Clinton,% 1042
\footnote{\textit{See, e.g.}, Jim Sciutto, \textit{Cuomo Prime Time Transcript}, CNN (July 26, 2018).}
the President tweeted:
"[S]o the Fake News doesn't waste my time with dumb questions, NO, I did NOT know of the meeting with my son, Don jr.
Sounds to me like someone is trying to make up stories in order to get himself out of an unrelated jam (Taxi cabs maybe?).
He even retained Bill and Crooked Hillary's lawyer.
Gee, I wonder if they helped him make the choice!"% 1043
\footnote{\@realDonaldTrump 7/27/18 (7:26 a.m.~ET) Tweet;
\@realDonaldTrump 7/27/18 (7:38 a.m.~ET) Tweet;
\@realDonaldTrump 7/27/18 (7:56 a.m.~ET) Tweet.
At the time of these tweets, the press had reported that Cohen's financial interests in taxi cab medallions were being scrutinized by investigators.
See, e.g., Matt Apuzzo et al., \textit{Michael Cohen Secretly Taped Trump Discussing Payment to Playboy Model}, New York Times (July 20, 2018).}

On August 21, 2018, Cohen pleaded guilty in the Southern District of New York to eight felony charges, including two counts of campaign-finance violations based on the payments he had made during the final weeks of the campaign to women who said they had affairs with the President.% 1044
\footnote{\textit{Cohen} Information.}
During the plea hearing, Cohen stated that he had worked "at the direction of" the candidate in making those payments.% 1045
\footnote{\textit{Cohen} 8/21/18 Transcript, at 23.}
The next day, the President contrasted Cohen's cooperation with Manafort's refusal to cooperate, tweeting, "I feel very badly for Paul Manafort and his wonderful family.
'Justice' took a 12 year old tax case, among other things, applied tremendous pressure on him and, unlike Michael Cohen, he refused to 'break' - make up stories in order to get a 'deal.'
Such respect for a brave man!"% 1046
\footnote{\@realDonaldTrump 8/22/18 (9:21 a.m.~ET) Tweet.}

On September 17, 2018, this Office submitted written questions to the President that included questions about the Trump Tower Moscow project and attached Cohen's written statement to Congress and the Letter of Intent signed by the President.% 1047
\footnote{9/17/18 Letter, Special Counsel's Office to President's Personal Counsel (attaching written questions for the President, with attachments).}
Among other issues, the questions asked the President to describe the timing and substance of discussions he had with Cohen about the project, whether they discussed a potential trip to Russia, and whether the President "at any time direct[ed] or suggest[ed] that discussions about the Trump Moscow project should cease," or whether the President was "informed at any time that the project had been abandoned."% 1048
\footnote{9/17/18 Letter, Special Counsel's Office to President's Personal Counsel (attaching written questions for the President), Question III, Parts (a) through(g).}

On November 20, 2018, the President submitted written responses that did not answer those questions about Trump Tower Moscow directly and did not provide any information about the timing of the candidate's discussions with Cohen about the project or whether he participated in any discussions about the project being abandoned or no longer pursued.% 1049
\footnote{Written Responses of Donald J. Trump (Nov.~20, 2018).}
Instead, the President's answers stated in relevant part:

\begin{quote}
I had few conversations with Mr.~Cohen on this subject.
As I recall, they were brief, and they were not memorable.
I was not enthused about the proposal, and I do not recall any discussion of travel to Russia in connection with it.
I do not remember discussing it with anyone else at the Trump Organization, although it is possible.
I do not recall being aware at the time of any communications between Mr.~Cohen and Felix Sater and any Russian government official regarding the Letter of Intent.% 1050
\footnote{Written Responses of Donald J. Trump (Nov.~20, 2018), at 15 (Response to Question III, Parts (a) through (g)).}
\end{quote}

On November 29, 2018, Cohen pleaded guilty to making false statements to Congress based on his statements about the Trump Tower Moscow project.% 1051
\footnote{\textit{Cohen} Information;
\textit{Cohen} 8/21/18 Transcript.}
In a plea agreement with this Office, Cohen agreed to "provide truthful information regarding any and all matters as to which this Office deems relevant."% 1052
\footnote{Plea Agreement at 4, \textit{United States v.\ Michael Cohen}, 1:18-cr-850 (S.D.N.Y. Nov.~29, 2018).}
Later on November 29, after Cohen's guilty plea had become public, the President spoke to reporters about the Trump Tower Moscow project, saying:

\begin{quote}
I decided not to do the project....
I decided ultimately not to do it.
There would have been nothing wrong if I did do it.
If I did do it, there would have been nothing wrong.
That
May - Certain -
It was an option that I decided not to do....
I decided not to do it.
I was running my
was my business...
The primary reason...
I was focused on running for President...
business while I was campaigning.
There was a good chance that I wouldn't have won, in which case I would've gone back into the business.
And why should I lose lots of opportunities?% 1053
\footnote{\textit{President Trump Departure Remarks}, C-SPAN (Nov.~29, 2018).
In contrast to the President's remarks following Cohen's guilty plea, Cohen's August 28, 2017 statement to Congress stated that Cohen, not the President, "decided to abandon the proposal" in late January 2016;
that Cohen "did not ask or brief Mr.~Trump.... before I made the decision to terminate further work on the proposal";
and that the decision to abandon the proposal was "unrelated" to the Campaign. P-SCO-000009477 (Statement of Michael D. Cohen, Esq.\ (Aug.~28, 2017)).
}
\end{quote}

The President also said that Cohen was "a weak person.
And by being weak, unlike other people that you watch - he is a weak person.
And what he's trying to do is get a reduced sentence.
So he's lying about a project that everybody knew about."% 1054
\footnote{\textit{President Trump Departure Remarks}, C-SPAN (Nov.~29, 2018).}
The President also brought up Cohen's written submission to Congress regarding the Trump Tower Moscow project:
"So here's the story: Go back and look at the paper that Michael Cohen wrote before he testified in the House and/or Senate.
It talked about his position."% 1055
\footnote{\textit{President Trump Departure Remarks}, C-SPAN (Nov.~29, 2018).}
The President added, "Even if [Cohen] was right, it doesn't matter because I was allowed to do whatever I wanted during the campaign."% 1056
\footnote{\textit{President Trump Departure Remarks}, C-SPAN (Nov.~29, 2018).}

In light of the President's public statements following Cohen's guilty plea that he "decided not to do the project," this Office again sought information from the President about whether he participated in any discussions about the project being abandoned or no longer pursued, including when he "decided not to do the project," who he spoke to about that decision, and what motivated the decision.% 1057
\footnote{1/23/19 Letter, Special Counsel's Office to President's Personal Counsel.}
The Office also again asked for the timing of the President's discussions with Cohen about Trump Tower Moscow and asked him to specify "what period of the campaign" he was involved in discussions concerning the project.% 1058
\footnote{1/23/19 Letter, Special Counsel's Office to President's Personal Counsel.}
In response, the President's personal counsel declined to provide additional information from the President and stated that "the President has fully answered the questions at issue."% 1059
\footnote{2/6/19 Letter, President's Personal Counsel to Special Counsel's Office.}

In the weeks following Cohen's plea and agreement to provide assistance to this Office, the President repeatedly implied that Cohen's family members were guilty of crimes.
On December 3, 2018, after Cohen had filed his sentencing memorandum, the President tweeted, "Michael Cohen asks judge for no Prison Time.'
You mean he can do all of the TERRIBLE, unrelated to Trump, things having to do with fraud, big loans, Taxis, etc., and not serve a long prison term?
He makes up stories to get a GREAT \& ALREADY reduced deal for himself, and get his wife and father-in-law (who has the money?) off Scott Free.
He lied for this outcome and should, in my opinion, serve a full and complete sentence."% 1060
\footnote{\@realDonaldTrump 12/3/18 (10:24 a.m.~ET and 10:29 a.m.~ET) Tweets (emphasis added).}
\blackout{Harm to Ongoing Investigation}% 1061
\footnote{\@realDonaldTrump 12/3/18 (10:48 a.m.~ET) Tweet.}

On December 12, 2018, Cohen was sentenced to three years of imprisonment.% 1062
\footnote{\textit{Cohen} 12/12/18 Transcript.}
The next day, the President sent a series of tweets that said:

\begin{quote}
I never directed Michael Cohen to break the law....
Those charges were just agreed to by him in order to embarrass the president and get a much reduced prison sentence, which he did - including the fact that his family was temporarily let off the hook.
As a lawyer, Michael has great liability to me!% 1063
\footnote{\@realDonaldTrump 12/13/18 (8:17 a.m.~ET, 8:25 a.m.~ET, and 8:39 a.m.~ET) Tweets (emphasis added).}
\end{quote}

On December 16, 2018, the President tweeted, "Remember, Michael Cohen only became a 'Rat' after the FBI did something which was absolutely unthinkable \& unheard of until the Witch Hunt was illegally started.
They BROKE INTO AN ATTORNEY'S OFFICE!
Why didn't they break into the DNC to get the Server, or Crooked's office?"% 1064
\footnote{\@realDonaldTrump 12/16/18 (9:39 a.m.~ET) Tweet.}

In January 2019, after the media reported that Cohen would provide public testimony in a congressional hearing, the President made additional public comments suggesting that Cohen's family members had committed crimes.
In an interview on Fox on January 12, 2019, the President was asked whether he was worried about Cohen's testimony and responded:

\begin{quote}
[I]n order to get his sentence reduced, [Cohen] says "I have an idea, I'll ah, tell - I'll give you some information on the president."
Well, there is no information.
But he should give information may be on his father-in-law because that's the one that people want to look at because where does that money - that's the money in the family.
And I guess he didn't want to talk about his father-in-law, he's trying to get his sentence reduced.
So it's ah, pretty sad.
You know, it's weak and it's very sad to watch a thing like that.% 1065
\footnote{\textit{Jeanine Pirro Interview with President Trump}, Fox News (Jan.~12, 2019) (emphasis added).}
\end{quote}

On January 18, 2019, the President tweeted, "Kevin Corke, \@FoxNews 'Don't forget, Michael Cohen has already been convicted of perjury and fraud, and as recently as this week, the Wall Street Journal has suggested that he may have stolen tens of thousands of dollars...."- Lying to reduce his jail time! Watch father-in-law!"% 1066
\footnote{\@realDonaldTrump 1/18/19 (10:02 a.m.~ET) Tweet (emphasis added).}

On January 23, 2019, Cohen postponed his congressional testimony, citing threats against his family.% 1067
\footnote{Statement by Lanny Davis, Cohen's personal counsel (Jan.~23, 2019).}
The next day, the President tweeted, "So interesting that bad lawyer Michael Cohen, who sadly will not be testifying before Congress, is using the lawyer of Crooked Hillary Clinton to represent him - Gee, how did that happen?"% 1068
\footnote{\@realDonaldTrump 1/24/19 (7:48 a.m.~ET) Tweet.}

Also in January 2019, Giuliani gave press interviews that appeared to confirm Cohen's account that the Trump Organization pursued the Trump Tower Moscow project well past January 2016.
Giuliani stated that "it's our understanding that [discussions about the Trump Moscow project] went on throughout 2016.
Weren't a lot of them, but there were conversations.
Can't be sure of the exact date.
But the president can remember having conversations with him about it.
The president also remembers - yeah, probably up - could be up to as far as October, November."% 1069
\footnote{Meet the Press Interview with Rudy Giuliani, NBC (Jan.~20, 2019).}
In an interview with the New York Times, Giuliani quoted the President as saying that the discussions regarding the Trump Moscow project were "going on from the day I announced to the day I won."% 1070
\footnote{Mark Mazzetti et al., \textit{Moscow Skyscraper Talks Continued Through "the Day I Won," Trump Is Said to Acknowledge}, New York Times (Jan.~20, 2019).}
On January 21, 2019, Giuliani issued a statement that said: "My recent statements about discussions during the 2016 campaign between Michael Cohen and candidate Donald Trump about a potential Trump Moscow 'project' were hypothetical and not based on conversations I had with the president."% 1071
\footnote{Maggie Haberman, \textit{Giuliani Says His Moscow Trump Tower Comments Were "Hypothetical"}, New York Times (Jan.~21, 2019).
In a letter to this Office, the President's counsel stated that Giuliani's public comments "were not intended to suggest nor did they reflect knowledge of the existence or timing of conversations beyond that contained in the President's [written responses to the Special Counsel's Office]."
2/6/19 Letter, President's Personal Counsel to Special Counsel's Office.}

\begin{center}
\textbf{Analysis}
\end{center}

In analyzing the President's conduct related to Cohen, the following evidence is relevant to the elements of obstruction of justice.

\underline{Obstructive act.}
We gathered evidence of the President's conduct related to Cohen on two issues:
(i) whether the President or others aided or participated in Cohen's false statements to Congress, and
(ii) whether the President took actions that would have the natural tendency to prevent Cohen from providing truthful information to the government.

First, with regard to Cohen's false statements to Congress, while there is evidence, described below, that the President knew Cohen provided false testimony to Congress about the Trump Tower Moscow project, the evidence available to us does not establish that the President directed or aided Cohen's false testimony.

Cohen said that his statements to Congress followed a "party line" that developed within the campaign to align with the President's public statements distancing the President from Russia.
Cohen also recalled that, in speaking with the President in advance of testifying, he made it clear that he would stay on message - which Cohen believed they both understood would require false testimony.
But Cohen said that he and the President did not explicitly discuss whether Cohen's testimony about the Trump Tower Moscow project would be or was false, and the President did not direct him to provide false testimony.
Cohen also said he did not tell the President about the specifics of his planned testimony.
During the time when his statement to Congress was being drafted and circulated to members of the JDA, Cohen did not speak directly to the President about the statement, but rather communicated with the President's personal counsel - as corroborated by phone records showing extensive communications between Cohen and the President's personal counsel before Cohen submitted his statement and when he testified before Congress.

Cohen recalled that in his discussions with the President's personal counsel on August 27, 2017 - the day before Cohen's statement was submitted to Congress - Cohen said that there were more communications with Russia and more communications with candidate Trump than the statement reflected.
Cohen recalled expressing some concern at that time.
According to Cohen, the President's personal counsel - who did not have first-hand knowledge of the project - responded by saying that there was no need to muddy the water, that it was unnecessary to include those details because the project did not take place, and that Cohen should keep his statement short and tight, not elaborate, stay on message, and not contradict the President.
Cohen's recollection of the content of those conversations is consistent with direction about the substance of Cohen's draft statement that appeared to come from members of the JDA.
For example, Cohen omitted any reference to his outreach to Russian government officials to set up a meeting between Trump and Putin during the United Nations General Assembly, and Cohen believed it was a decision of the JDA to delete the sentence, "The building project led me to make limited contacts with Russian government officials."

The President's personal counsel declined to provide us with his account of his conversations with Cohen, and there is no evidence available to us that indicates that the President was aware of the information Cohen provided to the President's personal counsel.
The President's conversations with his personal counsel were presumptively protected by attorney-client privilege, and we did not seek to obtain the contents of any such communications.
The absence of evidence about the President and his counsel's conversations about the drafting of Cohen's statement precludes us from assessing what, if any, role the President played.

Second, we considered whether the President took actions that would have the natural tendency to prevent Cohen from providing truthful information to criminal investigators or to Congress.

Before Cohen began to cooperate with the government, the President publicly and privately urged Cohen to stay on message and not "flip." Cohen recalled the President's personal counsel telling him that he would be protected so long as he did not go "rogue."
In the days and weeks that followed the April 2018 searches of Cohen's home and office, the President told reporters that Cohen was a "good man" and said he was" a fine person with a wonderful family ... who I have always liked \& respected."
Privately, the President told Cohen to "hang in there" and "stay strong." People who were close to both Cohen and the President passed messages to Cohen that "the President loves you,""the boss loves you," and "everyone knows the boss has your back."
Through the President's personal counsel, the President also had previously told Cohen "thanks for what you do" after Cohen provided information to the media about payments to women that, according to Cohen, both Cohen and the President knew was false.
At that time, the Trump Organization continued to pay Cohen's legal fees, which was important to Cohen.
Cohen also recalled discussing the possibility of a pardon with the President's personal counsel, who told him to stay on message and everything would be fine.
The President indicated in his public statements that a pardon had not been ruled out, and also stated publicly that "[m]ost people will flip if the Government lets them out of trouble" but that he "d[idn't] see Michael doing that."

After it was reported that Cohen intended to cooperate with the government, however, the President accused Cohen of "mak[ing] up stories in order to get himself out of an unrelated jam (Taxi cabs maybe?)," called Cohen a "rat," and on multiple occasions publicly suggested that Cohen's family members had committed crimes.
The evidence concerning this sequence of events could support an inference that the President used inducements in the form of positive messages in an effort to get Cohen not to cooperate, and then turned to attacks and intimidation to deter the provision of information or undermine Cohen's credibility once Cohen began cooperating.

\underline{Nexus to an official proceeding.}
The President's relevant conduct towards Cohen occurred when the President knew the Special Counsel's Office, Congress, and the U.S. Attorney's Office for the Southern District of New York were investigating Cohen's conduct.
The President acknowledged through his public statements and tweets that Cohen potentially could cooperate with the government investigations.

\underline{Intent.}
In analyzing the President's intent in his actions towards Cohen as a potential witness, there is evidence that could support the inference that the President intended to discourage Cohen from cooperating with the government because Cohen's information would shed adverse light on the President's campaign-period conduct and statements.

Cohen's false congressional testimony about the Trump Tower Moscow project was designed to minimize connections between the President and Russia and to help limit the congressional and DOJ Russia investigations - a goal that was in the President's interest, as reflected by the President's own statements.
During and after the campaign, the President made repeated statements that he had "no business" in Russia and said that there were "no deals that could happen in Russia, because we've stayed away."
As Cohen knew, and as he recalled communicating to the President during the campaign, Cohen's pursuit of the Trump Tower Moscow project cast doubt on the accuracy or completeness of these statements.

In connection with his guilty plea, Cohen admitted that he had multiple conversations with candidate Trump to give him status updates about the Trump Tower Moscow project, that the conversations continued through at least June 2016, and that he discussed with Trump possible travel to Russia to pursue the project.
The conversations were not off-hand, according to Cohen, because the project had the potential to be so lucrative.
In addition, text messages to and from Cohen and other records further establish that Cohen's efforts to advance the project did not end in January 2016 and that in May and June 2016, Cohen was considering the timing for possible trips to Russia by him and Trump in connection with the project.

The evidence could support an inference that the President was aware of these facts at the time of Cohen's false statements to Congress.
Cohen discussed the project with the President in early 2017 following media inquiries.
Cohen recalled that on September 20, 2017, the day after he released to the public his opening remarks to Congress - which said the project "was terminated in January of 2016" - the President's personal counsel told him the President was pleased with what Cohen had said about Trump Tower Moscow.
And after Cohen's guilty plea, the President told reporters that he had ultimately decided not to do the project, which supports the inference that he remained aware of his own involvement in the project and the period during the Campaign in which the project was' being pursued.

The President's public remarks following Cohen's guilty plea also suggest that the President may have been concerned about what Cohen told investigators about the Trump Tower Moscow project.
At the time the President submitted written answers to questions from this Office about the project and other subjects, the media had reported that Cohen was cooperating with the government but Cohen had not yet pleaded guilty to making false statements to Congress.
Accordingly, it was not publicly known what information about the project Cohen had provided to the government.
In his written answers, the President did not provide details about the timing and substance of his discussions with Cohen about the project and gave no indication that he had decided to no longer pursue the project.
Yet after Cohen pleaded guilty, the President publicly stated that he had personally made the decision to abandon the project.
The President then declined to clarify the seeming discrepancy to our Office or answer additional questions.
The content and timing of the President's provision of information about his knowledge and actions regarding the Trump Tower Moscow project is evidence that the President may have been concerned about the information that Cohen could provide as a witness.

The President's concern about Cohen cooperating may have been directed at the Southern District of New York investigation into other aspects of the President's dealings with Cohen rather than an investigation of Trump Tower Moscow.
There also is some evidence that the President's concern about Cohen cooperating was based on the President's stated belief that Cohen would provide false testimony against the President in an attempt to obtain a lesser sentence for his unrelated criminal conduct.
The President tweeted that Manafort, unlike Cohen, refused to "break" and "make up stories in order to get a 'deal.'"
And after Cohen pleaded guilty to making false statements to Congress, the President said, "what [Cohen]'s trying to do is get a reduced sentence.
So he's lying about a project that everybody knew about." But the President also appeared to defend the underlying conduct, saying, "Even if [Cohen] was right, it doesn't matter because I was allowed to do whatever I wanted during the campaign."
As described above, there is evidence that the President knew that Cohen had made false statements about the Trump Tower Moscow project and that Cohen did so to protect the President and minimize the President's connections to Russia during the campaign.

Finally, the President's statements insinuating that members of Cohen's family committed crimes after Cohen began cooperating with the government could be viewed as an effort to retaliate against Cohen and chill further testimony adverse to the President by Cohen or others.
It is possible that the President believes, as reflected in his tweets, that Cohen "ma[d]e[] up stories" in order to get a deal for himself and "get his wife and father-in-law ... off Scott Free." It also is possible that the President's mention of Cohen's wife and father-in-law were not intended to affect Cohen as a witness but rather were part of a public-relations strategy aimed at discrediting Cohen and deflecting attention away from the President on Cohen-related matters.
But the President's suggestion that Cohen's family members committed crimes happened more than once, including just before Cohen was sentenced (at the same time as the President stated that Cohen "should, in my opinion, serve a full and complete sentence") and again just before Cohen was scheduled to testify before Congress.
The timing of the statements supports an inference that they were intended at least in part to discourage Cohen from further cooperation.

\subsection{Overarching Factual Issues}

Although this report does not contain a traditional prosecution decision or declination decision, the evidence supports several general conclusions relevant to analysis of the facts concerning the President's course of conduct.

Three features of this case render it atypical compared to the heartland obstruction-of-justice prosecutions brought by the Department of Justice.

First, the conduct involved actions by the President.
Some of the conduct did not implicate the President's constitutional authority and raises garden-variety obstruction-of-justice issues.
Other events we investigated, however, drew upon the President's Article II authority, which raised constitutional issues that we address in Volume II, Section III-B, infra.
A factual analysis of that conduct would have to take into account both that the President's acts were facially lawful and that his position as head of the Executive Branch provides him with unique and powerful means of influencing official proceedings, subordinate officers, and potential witnesses.

Second, many obstruction cases involve the attempted or actual cover-up of an underlying crime.
Personal criminal conduct can furnish strong evidence that the individual had an improper obstructive purpose, see, e.g., United States v.\ Willoughby, 860 F.2d 15, 24 (2d Cir.~1988), or that he contemplated an effect on an official proceeding, see, e.g., United States v.\ Binday, 804 F.3d 558, 591 (2d Cir.~2015).
But proof of such a crime is not an element of an obstruction offense.
See United States v.\ Greer, 872 F.3d 790, 798 (6th Cir.~2017) (stating, in applying the obstruction sentencing guideline, that "obstruction of a criminal investigation is punishable even if the prosecution is ultimately unsuccessful or even if the investigation ultimately reveals no underlying crime").
Obstruction of justice can be motivated by a desire to protect non-criminal personal interests, to protect against investigations where underlying criminal liability falls into a gray area, or to avoid personal embarrassment.
The injury to the integrity of the justice system is the same regardless of whether a person committed an underlying wrong.

In this investigation, the evidence does not establish that the President was involved in an underlying crime related to Russian election interference.
But the evidence does point to a range of other possible personal motives animating the President's conduct.
These include concerns that continued investigation would call into question the legitimacy of his election and potential uncertainty about whether certain events - such as advance notice of WikiLeaks's release of hacked information or the June 9, 2016 meeting between senior campaign officials and Russians - could be seen as criminal activity by the President, his campaign, or his family.

Third, many of the President's acts directed at witnesses, including discouragement of cooperation with the government and suggestions of possible future pardons, occurred in public view.
While it may be more difficult to establish that public-facing acts were motivated by a corrupt intent, the President's power to influence actions, persons, and events is enhanced by his unique ability to attract attention through use of mass communications.
And no principle of law excludes public acts from the scope of obstruction statutes.
If the likely effect of the acts is to intimidate witnesses or alter their testimony, the justice system's integrity is equally threatened.

Although the events we investigated involved discrete acts - e.g., the President's statement to Comey about the Flynn investigation, his termination of Comey, and his efforts to remove the Special Counsel - it is important to view the President's pattern of conduct as a whole.
That pattern sheds light on the nature of the President's acts and the inferences that can be drawn about his intent.

Our investigation found multiple acts by the President that were capable of exerting undue influence over law enforcement investigations, including the Russian-interference and obstruction investigations.
The incidents were often carried out through one-on-one meetings in which the President sought to use his official power outside of usual channels.
These actions ranged from efforts to remove the Special Counsel and to reverse the effect of the Attorney General's recusal; to the attempted use of official power to limit the scope of the investigation; to direct and indirect contacts with witnesses with the potential to influence their testimony.
Viewing the acts collectively can help to illuminate their significance.
For example, the President's direction to McGahn to have the Special Counsel removed was followed almost immediately by his direction to Lewandowski to tell the Attorney General to limit the scope of the Russia investigation to prospective election - interference only - a temporal connection that suggests that both acts were taken with a related purpose with respect to the investigation.

The President's efforts to influence the investigation were mostly unsuccessful, but that is largely because the persons who surrounded the President declined to carry out orders or accede to his requests.
Comey did not end the investigation of Flynn, which ultimately resulted in Flynn's prosecution and conviction for lying to the FBI.
McGahn did not tell the Acting Attorney General that the Special Counsel must be removed, but was instead prepared to resign over the President's order.
Lewandowski and Dearborn did not deliver the President's message to Sessions that he should confine the Russia investigation to future election meddling only.
And McGahn refused to recede from his recollections about events surrounding the President's direction to have the Special Counsel removed, despite the President's multiple demands that he do so.
Consistent with that pattern, the evidence we obtained would not support potential obstruction charges against the President's aides and associates beyond those already filed.

In considering the full scope of the conduct we investigated, the President's actions can be divided into two distinct phases reflecting a possible shift in the President's motives.
In the first phase, before the President fired Comey, the President had been assured that the FBI had not opened an investigation of him personally.
The President deemed it critically important to make public that he was not under investigation, and he included that information in his termination letter to Comey after other efforts to have that information disclosed were unsuccessful.

Soon after he fired Comey, however, the President became aware that investigators were conducting an obstruction-of-justice inquiry into his own conduct.
That awareness marked a significant change in the President's conduct and the start of a second phase of action.
The President launched public attacks on the investigation and individuals involved in it who could possess evidence adverse to the President, while in private, the President engaged in a series of targeted efforts to control the investigation.
For instance, the President attempted to remove the Special Counsel; he sought to have Attorney General Sessions unrecuse himself and limit the investigation; he sought to prevent public disclosure of information about the June 9, 2016 meeting between Russians and campaign officials; and he used public forums to attack potential witnesses who might offer adverse information and to praise witnesses who declined to cooperate with the government.
Judgments about the nature of the President's motives during each phase would be informed by the totality of the evidence.
