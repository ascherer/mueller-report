\section{Legal Defenses to the Application of Obstruction-of-Justice Statutes to the President}

\subsection{Statutory Defenses to the Application of Obstruction-Of-Justice Provisions to the Conduct Under Investigation}

\subsubsection{The Text of Section 1512(c)(2) Prohibits a Broad Range of Obstructive Acts}

\subsubsection{Judicial Decisions Support a Broad Reading of Section 1512(c)(2)}

\subsubsection{The Legislative History of Section 1512(c)(2) Does Not Justify Narrowing Its Texts}

\subsubsection{General Principles of Statutory Construction Do Not Suggest That Section 1512(c)(2) is Inapplicable to the Conduct in this Investigation}

\subsubsection{Other Obstruction Statutes Might Apply to the Conduct in this Investigation}

\subsection{Constitutional Defenses to Applying Obstruction-Of-Justice Statutes to Presidential Conduct}

\subsubsection{The Requirement of a Clear Statement to Apply Statutes to Presidential Conduct Does Not Limit the Obstruction Statutes}

\subsubsection{Separation-of-Powers Principles Support the Conclusion that Congress May Validly Prohibit Corrupt Obstructive Acts Carried Out Through the President’s Official Powers}

\paragraph{The Supreme Court’s Separation-of-Powers Balancing Test Applies In This Contexts}

\paragraph{The Effect of Obstruction-of-Justice Statutes on the President’s Capacity to Perform His Article II Responsibilities is Limited}

\paragraph{Congress Has Power to Protect Congressional, Grand Jury, and Judicial Proceedings Against Corrupt Acts from Any Source}

\subsubsection{Ascertaining Whether the President Violated the Obstruction Statutes Would Not Chill his Performance of his Article II Duties}
