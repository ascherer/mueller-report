\section{Legal Defenses to the Application of Obstruction-of-Justice Statutes to the President}

The President’s personal counsel has written to this Office to advance statutory and constitutional defenses to the potential application of the obstruction-of-justice statutes to the President’s conduct.% 1072
\footnote{6/23/17 Letter, President’s Personal Counsel to Special Counsel’s Office;
see also 1/29/18 Letter, President’s Personal Counsel to Special Counsel’s Office;
2/6/18 Letter, President’s Personal Counsel to Special Counsel’s Office;
8/8/18 Letter, President’s Personal Counsel to Special Counsel’s Office, at 4.}
As a statutory matter, the President’s counsel has argued that a core obstruction-of-justice statute, 18~U.S.C. \S~1512(c)(2), does not cover the President’s actions.% 1073
\footnote{9/6/18 Letter, President’s Personal Counsel to Special Counsel’s Office, at 2--9.
Counsel has also noted that other potentially applicable obstruction statutes, such as 18~U.S.C. \S~1505, protect only pending proceedings.
6/23/17 Letter, President’s Personal Counsel to Special Counsel’s Office, at 7--8.
Section 1512(c)(2) is not limited to pending proceedings, but also applies to future proceedings that the person contemplated.
See Volume~II, Section III.A, \textit{supra}.}
As a constitutional matter, the President’s counsel argued that the President cannot obstruct justice by exercising his constitutional authority to close Department of Justice investigations or terminate the FBI Director.% 1074
\footnote{6/23/17 Letter, President’s Personal Counsel to Special Counsel’s Office, at 1 (“[T]he President cannot obstruct ... by simply exercising these inherent Constitutional powers.”).}
Under that view, any statute that restricts the President’s exercise of those powers would impermissibly intrude on the President’s constitutional role.
The President’s counsel has conceded that the President may be subject to criminal laws that do not directly involve exercises of his Article II authority, such as laws prohibiting bribing witnesses or suborning perjury.% 1075
\footnote{6/23/17 Letter, President’s Personal Counsel to Special Counsel’s Office, at 2 n.1.}
But counsel has made a categorical argument that “the President’s exercise of his constitutional authority here to terminate an FBI Director and to close investigations cannot constitutionally constitute obstruction of justice.% 1076
\footnote{6/23/17 Letter, President’s Personal Counsel to Special Counsel’s Office, at 2 n.1 (dashes omitted);
\textit{see also} 8/8/18 Letter, President’s Personal Counsel to Special Counsel’s Office, at 4 (“[T]he obstruction-of-justice statutes cannot be read so expansively as to create potential liability based on facially lawful acts undertaken by the President in furtherance of his core Article II discretionary authority to remove principal officers or carry out the prosecution function.”).}

In analyzing counsel’s statutory arguments, we concluded that the President’s proposed interpretation of Section 1512(c)(2) is contrary to the litigating position of the Department of Justice and is not supported by principles of statutory construction.

As for the constitutional arguments, we recognized that the Department of Justice and the courts have not definitively resolved these constitutional issues.
We therefore analyzed the President’s position through the framework of Supreme Court precedent addressing the separation of powers.
Under that framework, we concluded, Article II of the Constitution does not categorically and permanently immunize the President from potential liability for the conduct that we investigated.
Rather, our analysis led us to conclude that the obstruction-of-justice statutes can validly prohibit a President’s corrupt efforts to use his official powers to curtail, end, or interfere with an investigation.

\subsection{Statutory Defenses to the Application of Obstruction-Of-Justice Provisions to the Conduct Under Investigation}

The obstruction-of-justice statute most readily applicable to our investigation is 18~U.S.C. \S~1512(c)(2). Section 1512(c) provides:

(c)~Whoever corruptly -

(1)~alters, destroys, mutilates, or conceals a record, document, or other object, or attempts to do so, with the intent to impair the object’s integrity or availability for use in an official proceeding; or

(2)~otherwise obstructs, influences, or impedes any official proceeding, or attempts to do so,

shall be fined under this title or imprisoned not more than 20 years, or both.

The Department of Justice has taken the position that Section 1512(c)(2) states a broad, independent, and unqualified prohibition on obstruction of justice.% 1077
\footnote{See U.S. Br., United States v.\ Kumar, Nos.~06-5482-cr(L), 06-5654—cr(CON) (2d Cir.\ filed Oct.~26, 2007), at pp.~15--28; United States v.\ Singleton, Nos.~H-04-CR-514SS, H-06-cr-80 (S.D. Tex.\ filed June 5, 2006).}
While defendants have argued that subsection (c)(2) should be read to cover only acts that would impair the availability or integrity of evidence because that is subsection (c)(1)’s focus, strong arguments weigh against that proposed limitation.
The text of Section 1512(c)(2) confirms that its sweep is not tethered to Section 1512(c)(1); courts have so interpreted it; its history does not counsel otherwise; and no principle of statutory construction dictates a contrary view.
On its face, therefore, Section 1512(c)(2) applies to all corrupt means of obstructing a proceeding, pending or contemplated---including by improper exercises of official power.
In addition, other statutory provisions that are potentially applicable to certain conduct we investigated broadly prohibit obstruction of proceedings that are pending before courts, grand juries, and Congress.
\textit{See} 18~U.S.C. \S\S~1503, 1505.
Congress has also specifically prohibited witness tampering.
\textit{See} 18~U.S.C. \S~1512(b).

\subsubsection{The Text of Section 1512(c)(2) Prohibits a Broad Range of Obstructive Acts}

Several textual features of Section 1512(c)(2) support the conclusion that the provision broadly prohibits corrupt means of obstructing justice and is not limited by the more specific prohibitions in Section 1512(c)(1), which focus on evidence impairment.

First, the text of Section 1512(c)(2) is unqualified: it reaches acts that “obstruct[], influence[], or impede[] any official proceeding” when committed “corruptly.”
Nothing in Section 1512(c)(2)’s text limits the provision to acts that would impair the integrity or availability of evidence for use in an official proceeding.
In contrast, Section 1512(c)(1) explicitly includes the requirement that the defendant act “with the intent to impair the object’s integrity or availability for use in an official proceeding,” a requirement that Congress also included in two other sections of Section 1512.
\textit{See} 18~U.S.C. \S\S~1512(a)(2)(B)(ii) (use of physical force with intent to cause a person to destroy an object “with intent to impair the integrity or availability of the object for use in an official proceeding”);
1512(b)(2)(B) (use of intimidation, threats, corrupt persuasion, or misleading conduct with intent to cause a person to destroy an object “with intent to impair the integrity or availability of the object for use in an official proceeding”).
But no comparable intent or conduct element focused on evidence impairment appears in Section 1512(c)(2).
The intent element in Section 1512(c)(2) comes from the word “corruptly.”
\textit{See, e.g., United States v.\ McKibbins}, 656 F.3d 707, 711 (7th Cir.~2011) (“The intent element is important because the word ‘corruptly’ is what serves to separate criminal and innocent acts of obstruction.”) (internal quotation marks omitted).
And the conduct element in Section 1512(c)(2) is “obstruct[ing], influenc[ing], or imped[ing]” a proceeding.
Congress is presumed to have acted intentionally in the disparate inclusion and exclusion of evidence-impairment language.
\textit{See Loughrin v.\ United States}, 573 U.S. 351, 358 (2014) (“[W]hen ‘Congress includes particular language in one section of a statute but omits it in another’---let alone in the very next provision---this Court ‘presume[s]’ that Congress intended a difference in meaning”) (quoting \textit{Russello v.\ United States}, 464 U.S. 16, 23 (1983));
\textit{accord Digital Realty Trust, Inc.\ v.\ Somers}, 138 S. Ct.~767, 777 (2018).

Second, the structure of Section 1512 supports the conclusion that Section 1512(c)(2) defines an independent offense.
Section 1512(c)(2) delineates a complete crime with different elements from Section 1512(c)(1) -- and each subsection of Section 1512(c) contains its own “attempt” prohibition, underscoring that they are independent prohibitions.
The two subsections of Section 1512(c) are connected by the conjunction “or,” indicating that each provides an alternative basis for criminal liability.
\textit{See Loughrin}, 573 U.S. at 357 (“ordinary use [of ‘or’] is almost always disjunctive, that is, the words it connects are to be given separate meanings”) (internal quotation marks omitted).
In \textit{Loughrin}, for example, the Supreme Court relied on the use of the word “or” to hold that adjacent and overlapping subsections of the bank fraud statute, 18~U.S.C. \S~1344, state distinct offenses and that subsection 1344(2) therefore should not be interpreted to contain an additional element specified only in subsection 1344(1).
\textit{Id; see also Shaw v.\ United States}, 137 S. Ct.~462, 465--469 (2016) (recognizing that the subsections of the bank fraud statute “overlap substantially” but identifying distinct circumstances covered by each).% 1078
\footnote{The Office of Legal Counsel recently relied on several of the same interpretive principles in concluding that language that appeared in the first clause of the Wire Act, 18~U.S.C. \S~1084, restricting its prohibition against certain betting or wagering activities to “any sporting event or contest,” did not apply to the second clause of the same statute, which reaches other betting or wagering activities.
\textit{See Reconsidering Whether the Wire Act Applies to Non-Sports Gambling} (Nov.~2, 2018), slip op.~7 (relying on plain language);
\textit{id}. at 11 (finding it not “tenable to read into the second clause the qualifier ‘on any sporting event or contest’ that appears in the first clause”);
\textit{id}. at 12 (relying on \textit{Digital Realty}).}
And here, as in \textit{Loughrin}, Section 1512(c)’s “two clauses have separate numbers, line breaks before, between, and after them, and equivalent indentation---thus placing the clauses visually on an equal footing and indicating that they have separate meanings.” 573 U.S. at 359.

Third, the introductory word “otherwise” in Section 1512(c)(2) signals that the provision covers obstructive acts that are different from those listed in Section 1512(c)(1).
\textit{See} Black’s Law Dictionary 1101 (6th ed.~1990) (“otherwise” means“in a different manner; in another way, or in other ways”);
\textit{see also, e.g.}, American Heritage College Dictionary Online (“1. In another way; differently; 2. Under other circumstances”);
\textit{see also Gooch v.\ United States}, 297 U.S. 124, 128 (1936) (characterizing “otherwise” as a “broad term” and holding that a statutory prohibition on kidnapping “for ransom or reward or otherwise” is not limited by the words “ransom” and “reward” to kidnappings for pecuniary benefits);
\textit{Collazos v.\ United States}, 368 F.3d 190, 200 (2d Cir.~2004) (construing “otherwise” in 28~U.S.C. \S~2466(1)(C) to reach beyond the “specific examples” listed in prior subsections, thereby covering the “myriad means that human ingenuity might devise to permit a person to avoid the jurisdiction of a court”);
\textit{cf.~Begay v.\ United States}, 553 U.S. 137, 144 (2006) (recognizing that “otherwise” is defined to mean “in a different way or manner,” and holding that the word “otherwise” introducing the residual clause in the Armed Career Criminal Act, 18~U.S.C. \S~924(e)(2)(B)(ii), can, but need not necessarily, “refer to a crime that is similar to the listed examples in some respects but different in others”).% 1079
\footnote{In \textit{Sykes v.\ United States}, 564 U.S. 1, 15 (2011), the Supreme Court substantially abandoned Begay’s reading of the residual clause, and in \textit{Johnson v.\ United States}, 135 S. Ct.~2551 (2015), the Court invalidated the residual clause as unconstitutionally vague.
\textit{Begay’s} analysis of the word “otherwise” is thus of limited value.}
The purpose of the word “otherwise” in Section 1512(c)(2) is therefore to clarify that the provision covers obstructive acts \textit{other} than the destruction of physical evidence with the intent to impair its integrity or availability, which is the conduct addressed in Section 1512(c)(1).
The word “otherwise” does not signal that Section 1512(c)(2) has less breadth in covering obstructive conduct than the language of the provision implies.

\subsubsection{Judicial Decisions Support a Broad Reading of Section 1512(c)(2)}

Courts have not limited Section 1512(c)(2) to conduct that impairs evidence, but instead have read it to cover obstructive acts in any form.

As one court explained, “[t]his expansive subsection operates as a catch-all to cover ‘otherwise’ obstructive behavior that might not constitute a more specific offense like document destruction, which is listed in (c)(1).”
\textit{United States v.\ Volpendesto}, 746 F.3d 273, 286 (7th Cir.~2014) (some quotation marks omitted).
For example, in \textit{United States v.\ Ring}, 628 F. Supp.~2d 195 (D.D.C. 2009), the court rejected the argument that “\S~1512(c)(2)’s reference to conduct that ‘otherwise obstructs, influences, or impedes any official proceeding’ is limited to conduct that is similar to the type of conduct proscribed by subsection (c)(1)---namely, conduct that impairs the integrity or availability of ‘record[s], documents[s], or other object[s] for use in an official proceeding.”
\textit{Id.} at 224.
The court explained that “the meaning of \S~1512(c)(2) is plain on its face.”
\textit{Id.} (alternations in original), And courts have upheld convictions under Section 1512(c)(2) that did not involve evidence impairment, but instead resulted from conduct that more broadly thwarted arrests or investigations.
\textit{See, e.g., United States v.\ Martinez}, 862 F.3d 223, 238 (2d Cir.~2017) (police officer tipped off suspects about issuance of arrest warrants before “outstanding warrants could be executed, thereby potentially interfering with an ongoing grand jury proceeding”);
\textit{United States v.\ Ahrensfield}, 698 F.3d 1310, 1324--1326 (10th Cir.~2012) (officer disclosed existence of an undercover investigation to its target);
\textit{United States v.\ Phillips}, 583 F.3d 1261, 1265 (10th Cir.~2009) (defendant disclosed identity of an undercover officer thus preventing him from making controlled purchases from methamphetamine dealers).
Those cases illustrate that Section 1512(c)(2) applies to corrupt acts---including by public officials---that frustrate the commence mentor conduct of a proceeding, and not just to acts that make evidence unavailable or impair its integrity.

Section 1512(c)(2)’s breadth is reinforced by the similarity of its language to the omnibus clause of 18~U.S.C. \S~1503, which covers anyone who “corruptly ... obstructs, or impedes, or endeavors to influence, obstruct, or impede, the due administration of justice.” That clause of Section 1503 follows two more specific clauses that protect jurors, judges, and court officers.
The omnibus clause has nevertheless been construed to be “far more general in scope than the earlier clauses of the statute.”
\textit{United States v.\ Aguilar}, 515 U.S. 593, 599 (1995).
“The omnibus clause is essentially a catch-all provision which generally prohibits conduct that interferes with the due administration of justice.”
\textit{United States v.\ Brenson}, 104 F.3d 1267, 1275 (11th Cir.~1997).
Courts have accordingly given it a “non-restrictive reading.”
\textit{United States v.\ Kumar}, 617 F.3d 612, 620 (2d Cir.~2010); id, at 620 n.7 (collecting cases from the Third, Fourth, Sixth, Seventh, and Eleventh Circuits).
As one court has explained, the omnibus clause “prohibits acts that are similar in result, rather than manner, to the conduct described in the first part of the statute.”
\textit{United States v.\ Howard}, 569 F.2d 1331, 1333 (5th Cir.~1978).
While the specific clauses “forbid certain means of obstructing justice ... the omnibus clause aims at obstruction of justice itself, regardless of the means used to reach that result.”
\textit{Id.} (collecting cases).
Given the similarity of Section 1512(c)(2) to Section 1503’s omnibus clause, Congress would have expected Section 1512(c)(2) to cover acts that produced a similar result to the evidence-impairment provisions---i.e., the result of obstructing justice---rather than covering only acts that were similar in manner.
Read this way, Section 1512(c)(2) serves a distinct function in the federal obstruction-of-justice statutes: it captures corrupt conduct, other than document destruction, that has the natural tendency to obstruct contemplated as well as pending proceedings.

Section 1512(c)(2) overlaps with other obstruction statutes, but it does not render them superfluous.
Section 1503, for example, which covers pending grand jury and judicial proceedings, and Section 1505, which covers pending administrative and congressional proceedings, reach “endeavors to influence, obstruct, or impede” the proceedings---a broader test for inchoate violations than Section 1512(c)(2)’s “attempt” standard, which requires a substantial step towards a completed offense.
\textit{See United States v.\ Sampson}, 898 F.3d 287, 302 (2d Cir.~2018) (“[E]fforts to witness tamper that rise to the level of an ‘endeavor’ yet fall short of an ‘attempt’ cannot be prosecuted under \S~1512.”);
\textit{United States v.\ Leisure}, 844 F.2d 1347, 1366--1367 (8th Cir.~1988) (collecting cases recognizing the difference between the “endeavor” and “attempt” standards).
And 18~U.S.C. \S~1519, which prohibits destruction of documents or records in contemplation of an investigation or proceeding, does not require the “nexus” showing under Aguilar, which Section 1512(c)(2) demands.
\textit{See, e.g., United States v.\ Yielding}, 657 F.3d 688, 712 (8th Cir.~2011) (“The requisite knowledge and intent [under Section 1519] can be present even if the accused lacks knowledge that he is likely to succeed in obstructing the matter.”);
\textit{United States v.\ Gray}, 642 F.3d 371, 376--377 (2d Cir.~2011) (“[I]n enacting \S~1519, Congress rejected any requirement that the government prove a link between a defendant’s conduct and an imminent or pending official proceeding.”).
The existence of even “substantial” overlap is not “uncommon” in criminal statutes.
\textit{Loughrin}, 573 U.S. at 359 n.4;
\textit{see Shaw}, 137 S. Ct.~at 458--469;
\textit{Aguilar}, 515 USS. at 616 (Scalia, J., dissenting) (“The fact that there is now some overlap between \S~1503 and \S~1512 is no more intolerable than the fact that there is some overlap between the omnibus clause of \S~1503 and the other provision of \S~1503 itself.”).
But given that Sections 1503, 1505, and 1519 each reach conduct that Section 1512(c)(2) does not, the overlap provides no reason to give Section 1512(c)(2) an artificially limited construction.
\textit{See Shaw}, 137 S. Ct.~at 469.% 1080
\footnote{The Supreme Court’s decision in \textit{Marinello v.\ United States}, 138 S. Ct.~1101 (2018), does not support imposing a non-textual limitation on Section 1512(c)(2).
\textit{Marinello} interpreted the tax obstruction statute, 26~U.S.C. \S~7212(a), to require “a ‘nexus’ between the defendant’s conduct and a particular administrative proceeding.”
\textit{Id}. at 1109.
The Court adopted that construction in light of the similar interpretation given to “other obstruction provisions,” \textit{id}. (citing \textit{Aguilar} and \textit{Arthur Andersen}), as well as considerations of context, legislative history, structure of the criminal tax laws, fair warning, and lenity.
\textit{Id}. at 1106--1108.
The type of “nexus” element the Court adopted in Marinello already applies under Section 1512(c)(2), and the remaining considerations the Court cited do not justify reading into Section 1512(c)(2) language that is not there.
\textit{See Bates v.\ United States}, 522 U.S. 23, 29 (1997) (the Court “ordinarily resist[s] reading words or elements into a statute that do not appear on its face.”).}

\subsubsection{The Legislative History of Section 1512(c)(2) Does Not Justify Narrowing Its Texts}

“Given the straightforward statutory command” in Section 1512(c)(2), “there is no reason to resort to legislative history.”
\textit{United States v.\ Gonzales}, 520 U.S. 1, 6 (1997).
In any event, the legislative history of Section 1512(c)(2) is not a reason to impose extratextual limitations on its reach.

Congress enacted Section 1512(c)(2) as part the Sarbanes-Oxley Act of 2002, Pub.\ L. No.~107-204, Tit.~XI, \S~1102, 116 Stat.~807.
The relevant section of the statute was entitled “Tampering with a Record \textit{or Otherwise Impeding an Official Proceeding}.”
116 Stat.~807 (emphasis added).
That title indicates that Congress intended the two clauses to have independent effect.
Section 1512(c) was added as a floor amendment in the Senate and explained as closing a certain “loophole” with respect to “document shredding.”
\textit{See} 148 Cong.\ Rec.\ S6545 (July 10, 2002) (Sen.~Lott);
\textit{id}. at S6549-S6550 (Sen: Hatch).
But those explanations do not limit the enacted text.
\textit{See Pittston Coal Group v.\ Sebben}, 488 U.S. 105, 115 (1988) (“[I]t is not the law that a statute can have no effects which are not explicitly mentioned in its legislative history.”);
\textit{see also Encino Motorcars, LLC v.\ Navarro}, 138 S. Ct.~1134, 1143 (2018) (“Even if Congress did not foresee all of the applications of the statute, that is no reason not to give the statutory text a fair reading.”).
The floor statements thus cannot detract from the meaning of the enacted text.
\textit{See Barnhart v.\ Sigmon Coal Co.}, 534 U.S. 438, 457 (2002) (“Floor statements from two Senators cannot amend the clear and unambiguous language of a statute.
We see no reason to give greater weight to the views of two Senators than to the collective votes of both Houses, which are memorialized in the unambiguous statutory text.”).
That principle has particular force where one of the proponents of the amendment to Section 1512 introduced his remarks as only “briefly elaborat[ing] on some of the specific provisions contained in this bill.”
148 Cong.\ Rec.\ S6550 (Sen.~Hatch).

Indeed, the language Congress used in Section 1512(c)(2)---prohibiting “corruptly ... obstruct[ing], influenc[ing], or imped[ing] any official proceeding” or attempting to do so--- parallels a provision that Congress considered years earlier in a bill designed to strengthen protections against witness tampering and obstruction of justice.
While the earlier provision is not a direct antecedent of Section 1512(c)(2), Congress’s understanding of the broad scope of the earlier provision is instructive.
Recognizing that “the proper administration of justice may be impeded or thwarted” by a “variety of corrupt methods ... limited only by the imagination of the criminally inclined,” S. Rep.\ No.~532, 97th Cong., 2d Sess.~17--18 (1982), Congress considered a bill that would have amended Section 1512 by making it a crime, \textit{inter alia}, when a person “corruptly ... influences, obstructs, or impedes ... [t]he enforcement and prosecution of federal law,” “administration of a law under which an official proceeding is being or may be conducted,” or the “exercise of a Federal legislative power of inquiry.” \textit{Id}. at 17--19 (quoting S. 2420).

The Senate Committee explained that:

\begin{quote}
[T]he purpose of preventing an obstruction of or miscarriage of justice cannot be fully carried out by a simple enumeration of the commonly prosecuted obstruction offenses.
There must also be protection against the rare type of conduct that is the product of the inventive criminal mind and which also thwarts justice.
\end{quote}

\textit{Id}. at~18.
The report gave examples of conduct “actually prosecuted under the current residual clause [in 18~U.S.C. \S~1503], which would probably not be covered in this series [of provisions] without a residual clause.”
\textit{Id}.
One prominent example was “[a] conspiracy to cover up the Watergate burglary and its aftermath by having the Central Intelligence Agency seek to interfere with an ongoing FBI investigation of the burglary.”
\textit{Id}. (citing \textit{United States v.\ Haldeman}, 559 F.2d 31 (D.C. Cir.~1976)).
The report therefore indicates a congressional awareness not only that residual-clause language resembling Section 1512(c)(2) broadly covers a wide variety of obstructive conduct, but also that such language reaches the improper use of governmental processes to obstruct justice---specifically, the Watergate cover-up orchestrated by White House officials including the President himself.
\textit{See Haldeman}, 559 F.3d at 51, 86--87, 120--129, 162.% 1081
\footnote{The Senate ultimately accepted the House version of the bill, which excluded an omnibus clause.
\textit{See United States v.\ Poindexter}, 951 F.2d 369, 382--383 (D.C. Cir.~1991) (tracing history of the proposed omnibus provision in the witness-protection legislation).
During the floor debate on the bill, Senator Heinz, one of the initiators and primary backers of the legislation, explained that the omnibus clause was beyond the scope of the witness-protection measure at issue and likely “duplicative” of other obstruction laws, 128 Cong.\ Rec.\ 26,810 (1982) (Sen.~Heinz), presumably referring to Sections 1503 and 1505.}

\subsubsection{General Principles of Statutory Construction Do Not Suggest That Section 1512(c)(2) is Inapplicable to the Conduct in this Investigation}

The requirement of fair warning in criminal law, the interest in avoiding due process concerns in potentially vague statutes, and the rule of lenity do not justify narrowing the reach of Section 1512(c)(2)’s text.% 1082
\footnote{In a separate section addressing considerations unique to the presidency, we consider principles of statutory construction relevant in that context.
See Volume~II, Section III.B.1, \textit{infra}.}

a. As with other criminal laws, the Supreme Court has “exercised restraint” in interpreting obstruction-of-justice provisions, both out of respect for Congress’s role in defining crimes and in the interest of providing individuals with “fair warning” of what a criminal statute prohibits.
\textit{Marinello v.\ United States}, 138 S. Ct.~1101, 1106 (2018);
\textit{Arthur Andersen}, 544 U.S. at 703; Aguilar, 515 U.S. at 599--602.
In several obstruction cases, the Court has imposed a nexus test that requires that the wrongful conduct targeted by the provision be sufficiently connected to an official proceeding to ensure the requisite culpability.
\textit{Marinello}, 138 S, Ct.~at 1109;
\textit{Arthur Andersen}, 544 U.S. at 707--708;
\textit{Aguilar}, 515 U.S. at 600--602.
Section 1512(c)(2) has been interpreted to require a similar nexus.
\textit{See, e.g., United States v.\ Young}, 916 F.3d 368, 386 (4th Cir.~2019);
\textit{United States v.\ Petruk}, 781 F.3d 438, 445 (8th Cir.~2015);
\textit{United States v.\ Phillips}, 583 F.3d 1261, 1264 (10th Cir.~2009);
\textit{United States v.\ Reich}, 479 F.3d 179, 186 (2d Cir.~2007).
To satisfy the nexus requirement, the government must show as an objective matter that a defendant acted “in a manner that is likely to obstruct justice,” such that the statute “excludes defendants who have an evil purpose but use means that would only unnaturally and improbably be successful.”
\textit{Aguilar}, 515 U.S. at 601--602 (internal quotation marks omitted);
\textit{see id}. at 599 (“the endeavor must have the natural and probable effect of interfering with the due administration of justice”) (internal quotation marks omitted).
The government must also show as a subjective matter that the actor “contemplated a particular, foreseeable proceeding.”
\textit{Petruk}, 781 F.3d at 445.
Those requirements alleviate fair-warning concerns by ensuring that obstructive conduct has a close enough connection to existing or future proceedings to implicate the dangers targeted by the obstruction laws and that the individual actually has the obstructive result in mind.

b. Courts also seek to construe statutes to avoid due process vagueness concerns.
\textit{See, e.g., McDonnell v.\ United States}, 136 S. Ct.~2355, 2373 (2016);
\textit{Skilling v.\ United States}, 561 U.S. 358, 368, 402--404 (2010).
Vagueness doctrine requires that a statute define a crime “with sufficient definiteness that ordinary people can understand what conduct is prohibited” and “in a manner that does not encourage arbitrary and discriminatory enforcement.”
\textit{Id}. at 402--403 (internal quotation marks omitted).
The obstruction statutes’ requirement of acting “corruptly” satisfies that test.

“Acting ‘corruptly’ within the meaning of \S~1512(c)(2) means acting with an improper purpose and to engage in conduct knowingly and dishonestly with the specific intent to subvert, impede or obstruct” the relevant proceeding.
\textit{United States v.\ Gordon}, 710 F.3d 1124, 1151 (10th Cir.~2013) (some quotation marks omitted).
The majority opinion in \textit{Aguilar} did not address the defendant’s vagueness challenge to the word “corruptly,” 515 U.S. at 600 n.~1, but Justice Scalia’s separate opinion did reach that issue and would have rejected the challenge, \textit{id}. at 616--617 (Scalia, J., joined by Kennedy and Thomas, JJ., concurring in part and dissenting in part).
“Statutory language need not be colloquial,” Justice Scalia explained, and “the term ‘corruptly’ in criminal laws has a longstanding and well-accepted meaning.
It denotes an act done with an intent to give some advantage inconsistent with official duty and the rights of others.”
\textit{Id}. at 616 (internal quotation marks omitted; citing lower court authority and legal dictionaries).
Justice Scalia added that “in the context of obstructing jury proceedings, any claim of ignorance of wrongdoing is incredible.”
\textit{Id}. at 617.
Lower courts have also rejected vagueness challenges to the word “corruptly.”
\textit{See, e.g., United States v.\ Edwards}, 869 F.3d 490, 501--502 (7th Cir.~2017);
\textit{United States v.\ Brenson}, 104 F.3d 1267, 1280--1281 (11th Cir.~1997);
\textit{United States v.\ Howard}, 569 F.2d 1331, 1336 n.9 (Sh Cir.~1978).
This well-established intent standard precludes the need to limit the obstruction statutes to only certain kinds of inherently wrongful conduct.% 1083
\footnote{In United States v.\ Poindexter, 951 F.2d 369 (D.C. Cir.~1991), the court of appeals found the term “corruptly” in 18~U.S.C. \S~1505 vague as applied to a person who provided false information to Congress.
After suggesting that the word “corruptly” was vague on its face, 951 F.2d at 378, the court concluded that the statute did not clearly apply to corrupt conduct by the person himself and the “core” conduct to which Section 1505 could constitutionally be applied was one person influencing another person to violate a legal duty.
\textit{Id}. at 379--386.
Congress later enacted a provision overturning that result by providing that “[a]s used in [S]ection 1505, the term ‘corruptly’ means acting with an improper purpose, personally or by influencing another, including by making false or misleading statement, or withholding, concealing, altering, or destroying a document or other information.”
18~U.S.C. \S~1515(b).
Other courts have declined to follow \textit{Poindexter} either by limiting it to Section 1505 and the specific conduct at issue in that case, \textit{see Brenson}, 104 F.3d at 1280--1281;
reading it as narrowly limited to certain types of conduct, see \textit{United States v.\ Morrison}, 98 F.3d 619, 629--630 (D.C. Cir.~1996);
or by noting that it predated \textit{Arthur Andersen}’s interpretation of the term “corruptly,” \textit{see Edwards}, 869 F.3d at 501--502.
 }

c. Finally, the rule of lenity does not justify treating Section 1512(c)(2) as a prohibition on evidence impairment, as opposed to an omnibus clause.
The rule of lenity is an interpretive principle that resolves ambiguity in criminal laws in favor of the less-severe construction.
\textit{Cleveland v.\ United States}, 531 U.S. 12, 25 (2000).
“[A]s [the Court has] repeatedly emphasized,” however, the rule of lenity applies only if, “after considering text, structure, history and purpose, there remains a grievous ambiguity or uncertainty in the statute such that the Court must simply guess as to what Congress intended.”
\textit{Abramski v.\ United States}, 573 U.S. 169, 188 n.10 (2014) (internal quotation marks omitted).
The rule has been cited, for example, in adopting a narrow meaning of “tangible object” in an obstruction statute when the prohibition’s title, history, and list of prohibited acts indicated a focus on destruction of records.
\textit{See Yates v.\ United States}, 135 S. Ct.~1074, 1088 (2015) (plurality opinion) (interpreting “tangible object” in the phrase “record, document, or tangible object” in 18~U.S.C. \S~1519 to mean an item capable of recording or preserving information).
Here, as discussed above, the text, structure, and history of Section 1512(c)(2) leaves no “grievous ambiguity” about the statute’s meaning.
Section 1512(c)(2) defines a structurally independent general prohibition on obstruction of official proceedings.

\subsubsection{Other Obstruction Statutes Might Apply to the Conduct in this Investigation}

Regardless whether Section 1512(c)(2) covers all corrupt acts that obstruct, influence, or impede pending or contemplated proceedings, other statutes would apply to such conduct in pending proceedings, provided that the remaining statutory elements are satisfied.
As discussed above, the omnibus clause in 18~U.S.C. \S~1503(a) applies generally to obstruction of pending judicial and grand proceedings.% 1084
\footnote{Section 1503(a) provides for criminal punishment of:
\begin{quote}
Whoever ... corruptly or by threats or force, or by any threatening letter or communication, influences, obstructs, or impedes, or endeavors to influence, obstruct, or impede, the due administration of justice.
\end{quote}}
\textit{See Aguilar}, 515 U.S. at 598 (noting that the clause is “far more general in scope” than preceding provisions).
Section 1503(a)’s protections extend to witness tampering and to other obstructive conduct that has a nexus to pending proceedings.
\textit{See Sampson}, 898 F.3d at 298--303 \& n.6 (collecting cases from eight circuits holding that Section 1503 covers witness-related obstructive conduct, and cabining prior circuit authority).
And Section 1505 broadly criminalizes obstructive conduct aimed at pending agency and congressional proceedings.% 1085
\footnote{Section 1505 provides for criminal punishment of:
\begin{quote}
Whoever corruptly ... influences, obstructs, or impedes or endeavors to influence, obstruct, or impede the due and proper administration of the law under which any pending proceeding is being had before any department or agency of the United States, or the due and proper exercise of the power of inquiry under which any inquiry or investigation is being had by either House, or any committee of either House or any joint committee of the Congress.
\end{quote}}
\textit{See, e.g., United States v.\ Rainey}, 757 F.3d 234, 241--247 (St Cir.~2014).

Finally, 18~U.S.C. \S~1512(b)(3) criminalizes tampering with witnesses to prevent the communication of information about a crime to law enforcement.
The nexus inquiry articulated in Aguilar---that an individual has “knowledge that his actions are likely to affect the judicial proceeding,” 515 U.S. at 599---does not apply to Section 1512(b)(3).
\textit{See United States v.\ Byrne}, 435 F.3d 16, 24--25 (1st Cir.~2006).
The nexus inquiry turns instead on the actor’s intent to prevent communications to a federal law enforcement official.
\textit{See Fowler v.\ United States}, 563 U.S. 668, 673--678 (2011).

\hr

In sum, in light of the breadth of Section 1512(c)(2) and the other obstruction statutes, an argument that the conduct at issue in this investigation falls outside the scope of the obstruction---laws lacks merit.

\subsection{Constitutional Defenses to Applying Obstruction-Of-Justice Statutes to Presidential Conduct}

The President has broad discretion to direct criminal investigations.
The Constitution vests the “executive Power” in the President and enjoins him to “take Care that the Laws be faithfully executed.” \textsc{U.S. Const.\ Art.~II}, \S\S~1,~3.
Those powers and duties form the foundation of prosecutorial discretion.
\textit{See United States v.\ Armstrong}, 517 U.S. 456, 464 (1996) (Attorney General and United States Attorneys“ have this latitude because they are designated by statute as the President’s delegates to help him discharge his constitutional responsibility to ‘take Care that the Laws be faithfully executed.’”).
The President also has authority to appoint officers of the United States and to remove those whom he has appointed.
\textsc{U.S. Const.\ Art.~II}, \S~2, cl.~2 (granting authority to the President to appoint all officers with the advice and consent of the Senate, but providing that Congress may vest the appointment of inferior officers in the President alone, the heads of departments, or the courts of law);
\textit{see also Free Enterprise Fund v.\ Public Company Accounting Oversight Board}, 561 U.S. 477, 492--493, 509 (2010) (describing removal authority as flowing from the President’s “responsibility to take care that the laws be faithfully executed”).

Although the President has broad authority under Article II, that authority coexists with Congress’s Article I power to enact laws that protect congressional proceedings, federal investigations, the courts, and grand juries against corrupt efforts to undermine their functions.
Usually, those constitutional powers function in harmony, with the President enforcing the criminal laws under Article II to protect against corrupt obstructive acts.
But when the President’s official actions come into conflict with the prohibitions in the obstruction statutes, any constitutional tension is reconciled through separation-of-powers analysis.

The President’s counsel has argued that “the President’s exercise of his constitutional authority ... to terminate an FBI Director and to close investigations ... cannot constitutionally constitute obstruction of justice.”% 1086
\footnote{6/23/17 Letter, President’s Personal Counsel to Special Counsel’s Office, at 2 n.~1.}
As noted above, no Department of Justice position or Supreme Court precedent directly resolved this issue.
We did not find counsel’s contention, however, to accord with our reading of the Supreme Court authority addressing separation-of-powers issues.
Applying the Court’s framework for analysis, we concluded that Congress can validly regulate the President’s exercise of official duties to prohibit actions motivated by a corrupt intent to obstruct justice.
The limited effect on presidential power that results from that restriction would not impermissibly undermine the President’s ability to perform his Article II functions.

\subsubsection{The Requirement of a Clear Statement to Apply Statutes to Presidential Conduct Does Not Limit the Obstruction Statutes}

Before addressing Article II issues directly, we consider one threshold statutory construction principle that is unique to the presidency: “The principle that general statutes must be read as not applying to the President if they do not expressly apply where application would arguably limit the President’s constitutional role.”
OLC, \textit{Application of 28~U.S.C. \S~458 to Presidential Appointments of Federal Judges}, 19 Op.\ O.L.C. 350, 352 (1995).
This “clear statement rule,” \textit{id}., has its source in two principles: statutes should be construed to avoid serious constitutional questions, and Congress should not be assumed to have altered the constitutional separation of powers without clear assurance that it intended that result.
OLC, \textit{The Constitutional Separation of Powers Between the President and Congress}, 20 Op.\ O.L.C. 124, 178 (1996).

The Supreme Court has applied that clear-statement rule in several cases.
In one leading case, the Court construed the Administrative Procedure Act, 5~U.S.C. \S~701 \textit{et seq}., not to apply to judicial review of presidential action.
\textit{Franklin v.\ Massachusetts}, 505 U.S. 788, 800--801 (1992).
The Court explained that it “would require an express statement by Congress before assuming it intended the President’s performance of his statutory duties to be reviewed for abuse of discretion.”
\textit{Id}. at 801.
In another case, the Court interpreted the word “utilized” in the Federal Advisory Committee Act (FACA), 5 U.S.C. App., to apply only to the use of advisory committees established directly or indirectly by the government, thereby excluding the American Bar Association’s advice to the Department of Justice about federal judicial candidates.
\textit{Public Citizen v.\ United States Department of Justice}, 491 U.S. 440, 455, 462--467 (1989).
The Court explained that a broader interpretation of the term “utilized” in FACA would raise serious questions whether the statute “infringed unduly on the President’s Article II power to nominate federal judges and violated the doctrine of separation of powers.”
\textit{Id}. at 466--467.
Another case found that an established canon of statutory construction applied with “special force” to provisions that would impinge on the President’s foreign-affairs powers if construed broadly.
\textit{Sale v.\ Haitian Centers Council}, 509 U.S. 155, 188 (1993) (applying the presumption against extraterritorial application to construe the Refugee Act of 1980 as not governing in an overseas context where it could affect “foreign and military affairs for which the President has unique responsibility”).
\textit{See Application of 28~U.S.C. \S~458 to Presidential Appointments of Federal Judges}, 19 Op.\ O.L.C. at 353--354 (discussing \textit{Franklin}, \textit{Public Citizen}, and \textit{Sale}).

The Department of Justice has relied on this clear-statement principle to interpret certain statutes as not applying to the President at all, similar to the approach taken in \textit{Franklin}.
\textit{See, e.g.}, Memorandum for Richard T. Burress, Office of the President, from Laurence H. Silberman, Deputy Attorney General, \textit{Re: Conflict of Interest Problems Arising out of the President's Nomination of Nelson A. Rockefeller to be Vice President under the Twenty-Fifth Amendment to the Constitution}, at 2, 5 (Aug.~28, 1974) (criminal conflict-of-interest statute, 18~U.S.C. \S~208, does not apply to the President).
Other OLC opinions interpret statutory text not to apply to certain presidential or executive actions because of constitutional concerns.
See \textit{Application of 28~U.S.C. \S~458 to Presidential Appointments of Federal Judges}, 19 Op.\ O.L.C. at 350--357 (consanguinity limitations on court appointments, 28~U.S.C. \S~458, found inapplicable to “presidential appointments of judges to the federal judiciary”);
\textit{Constraints Imposed by 18~U.S.C. \S~1913 on Lobbying Efforts}, 13 Op.\ O.L.C. 300, 304--306 (1989) (limitation on the use of appropriated funds for certain lobbying programs found in applicable to certain communications by the President and executive officials).

But OLC has also recognized that this clear-statement rule “does not apply with respect to a statute that raises no separation of powers questions were it to be applied to the President,” such as the federal bribery statute, 18~U.S.C. \S~201.
\textit{Application of 28~U.S.C. \S~458 to Presidential Appointments of Federal Judges}, 19 Op.\ O.L.C. at 357 n.11. OLC explained that “[a]pplication of \S~201 raises no separation of powers question, let alone a serious one,” because [t]he Constitution confers no power in the President to receive bribes.” id.
In support of that conclusion, OLC noted constitutional provisions that forbid increases in the President’s compensation while in office, “which is what a bribe would function to do,” \textit{id}. (citing \textsc{U.S. Const.\ Art.~II}, \S~1, cl.~7), and the express constitutional power of “Congress to impeach [and convict] a President for, \textit{inter alia}, bribery,” \textit{id}. (citing \textsc{U.S. Const.\ Art.~II}, \S~4).

Under OLC’s analysis, Congress can permissibly criminalize certain obstructive conduct by the President, such as suborning perjury, intimidating witnesses, or fabricating evidence, because those prohibitions raise no separation-of-powers questions.
\textit{See Application of 28~U.S.C. \S~458 to Presidential Appointments of Federal Judges}, 19 Op.\ O.L.C. at 357 n.~11.
The Constitution does not authorize the President to engage in such conduct, and those actions would transgress the President’s duty to “take Care that the Laws be faithfully executed.”
\textsc{U.S. Const.\ Art.~II}, \S\S~3.
In view of those clearly permissible applications of the obstruction statutes to the President, Franklin’s holding that the President is entirely excluded from statute absent a clear statement would not apply in this context.

A more limited application of a clear-statement rule to exclude from the obstruction statutes only certain acts by the President---for example, removing prosecutors or ending investigations for corrupt reasons---would be difficult to implement as a matter of statutory interpretation.
It is not obvious how a clear-statement rule would apply to an omnibus provision like Section 1512(c)(2) to exclude corruptly motivated obstructive acts only when carried out in the President’s conduct of office.
No statutory term could easily bear that specialized meaning.
For example, the word “corruptly” has a well-established meaning that does not exclude exercises of official power for corrupt ends.
Indeed, an established definition states that “corruptly” means action with an intent to secure an improper advantage “\textit{inconsistent with official duty} and the rights of others.”
BALLENTINE’S LAW DICTIONARY 276 (3d ed.~1969) (emphasis added).
And it would be contrary to ordinary rules of statutory construction to adopt an unconventional meaning of a statutory term only when applied to the President.
\textit{See United States v.\ Santos}, 553 U.S. 507, 522 (2008) (plurality opinion of Scalia, J.) (rejecting proposal to “giv[e] the same word, in the same statutory provision, different meanings in different factual contexts”);
\textit{cf.~Public Citizen}, 491 U.S. at 462- 467 (giving the term “utilized” in the FACA a uniform meaning to avoid constitutional questions).
Nor could such an exclusion draw on a separate and established background interpretive presumption, such as the presumption against extraterritoriality applied in Sale.
The principle that courts will construe a statute to avoid serious constitutional questions “is not a license for the judiciary to rewrite language enacted by the legislature.”
\textit{Salinas v.\ United States}, 522 U.S. 52, 59--60 (1997).
“It is one thing to acknowledge and accept ... well defined (or even newly enunciated), generally applicable, background principles of assumed legislative intent.
It is quite another to espouse the broad proposition that criminal statutes do not have to be read as broadly as they are written, but are subject to case-by-case exceptions.”
\textit{Brogan v.\ United States}, 522 U.S. 398, 406 (1998).

When a proposed construction “would thus function as an extra-textual limit on [a statute’s] compass, “thereby preventing the statute “from applying to a host of cases falling within its clear terms,” \textit{Loughrin}, 573 U.S. at 357, it is doubtful that the construction would reflect Congress’s intent.
That is particularly so with respect to obstruction statutes, which “have been given a broad and all-inclusive meaning.”
\textit{Rainey}, 757 F.3d at 245 (discussing Sections 1503 and 1505) (internal quotation marks omitted).
Accordingly, since no established principle of interpretation would exclude the presidential conduct we have investigated from statutes such as Sections 1503, 1505, 1512(b), and 1512(c)(2), we proceed to examine the separation-of-powers issues that could be raised as an Article II defense to the application of those statutes.

\subsubsection{Separation-of-Powers Principles Support the Conclusion that Congress May Validly Prohibit Corrupt Obstructive Acts Carried Out Through the President’s Official Powers}

When Congress imposes a limitation on the exercise of Article II powers, the limitation’s validity depends on whether the measure “disrupts the balance between the coordinate branches.”
\textit{Nixon v.\ Administrator of General Services}, 433 U.S. 425, 443 (1977).
“Even when branch does not arrogate power to itself, ... the separation-of-powers doctrine requires that a branch not impair another in the performance of its constitutional duties.”
\textit{Loving v.\ United States}, 517 U.S. 748, 757 (1996).
The “separation of powers does not mean,” however, “that the branches ‘ought to have no partial agency in, or no control over the acts of each other.’”
\textit{Clinton v.\ Jones}, 520 U.S. 681, 703 (1997) (quoting James Madison, The Federalist No.~47, pp.~325--326 (J. Cooke ed.~1961) (emphasis omitted)).
In this context, a balancing test applies to assess separation-of-powers issues.
Applying that test here, we concluded that Congress can validly make obstruction-of-justice statutes applicable to corruptly motivated official acts of the President without impermissibly undermining his Article I functions.

\paragraph{The Supreme Court’s Separation-of-Powers Balancing Test Applies In This Contexts}

A congressionally imposed limitation on presidential action is assessed to determine “the extent to which it prevents the Executive Branch from accomplishing its constitutionally assigned functions,” and, if the “potential for disruption is present[,] ... whether that impact is justified by an overriding need to promote objectives within the constitutional authority of Congress.”
\textit{Administrator of General Services}, 433 U.S. at 443;
\textit{see Nixon v.\ Fitzgerald}, 457 U.S. 731, 753--754 (1982);
\textit{United States v.\ Nixon}, 418 U.S. 683, 706--707 (1974).
That balancing test applies to a congressional regulation of presidential power through the obstruction-of-justice laws.% 1087
\footnote{OLC applied such a balancing test in concluding that the President is not subject to criminal prosecution while in office, relying on many of the same precedents discussed in this section.
\textit{See A Sitting President’s Amenability to Indictment and Criminal Prosecution}, 24 Op.\ O.L.C. 222, 237--238, 244--245 (2000) (relying on, \textit{inter alia}, \textit{United States v.\ Nixon}, \textit{Nixon v.\ Fitzgerald}, and \textit{Clinton v.\ Jones}, and quoting the legal standard from \textit{Administrator of General Services v.\ Nixon} that is applied in the text), OLC recognized that “[t]he balancing analysis” it had initially relied on in finding that a sitting President is immune from prosecution had “been adopted as the appropriate mode of analysis by the Court.”
\textit{Id}. at 244.}

When an Article II power has not been “explicitly assigned by the text of the Constitution to be within the sole province of the President, but rather was thought to be encompassed within the general grant to the President of the ‘executive Power,’” the Court has balanced competing constitutional considerations.
\textit{Public Citizen}, 491 U.S. at 484 (Kennedy, J., concurring in the judgment, joined by Rehnquist, C.J., and O’Connor, J.).
As Justice Kennedy noted in Public Citizen, the Court has applied a balancing test to restrictions on “the President’s power to remove Executive officers, a power[that] ... is not conferred by any explicit provision in the text of the Constitution (as is the appointment power), but rather is inferred to be a necessary part of the grant of the ‘executive Power.’”
\textit{Id}. (citing \textit{Morrison v.\ Olson}, 487 U.S. 654, 694 (1988), and \textit{Myers v.\ United States}, 272 U.S. 52, 115--116 (1926)).
Consistent with that statement, \textit{Morrison} sustained a good-cause limitation on the removal of an inferior officer with defined prosecutorial responsibilities after determining that the limitation did not impermissibly undermine the President’s ability to perform his Article II functions.
487 U.S. at 691--693, 695--696.
The Court has also evaluated other general executive-power claims through a balancing test.
For example, the Court evaluated the President’s claim of an absolute privilege for presidential communications about his official acts by balancing that interest against the Judicial Branch’s need for evidence in a criminal case.
\textit{United States v.\ Nixon}, \textit{supra} (recognizing a qualified constitutional privilege for presidential communications on official matters).
The Court has also upheld a law that provided for archival access to presidential records despite a claim of absolute presidential privilege over the records.
\textit{Administrator of General Services}, 433 U.S. at 443--445, 451--455.
The analysis in those cases supports applying a balancing test to assess the constitutionality of applying the obstruction-of-justice statutes to presidential exercises of executive power.

Only in a few instances has the Court applied a different framework.
When the President’s power is “both ‘exclusive’ and ‘conclusive’ on the issue,” Congress is precluded from regulating its exercise.
\textit{Zivotofsky v.\ Kerry}, 135 S. Ct.~2076, 2084 (2015).
In Zivotofsky, for example, the Court followed “Justice Jackson’s familiar tripartite framework” in \textit{Youngstown Sheet \& Tube Co.\ v.\ Sawyer}, 343 U.S. 579, 635--638 (1952) (Jackson, J., concurring), and held that the President’s authority to recognize foreign nations is exclusive.
\textit{Id}. at 2083, 2094.
\textit{See also Public Citizen} 491 U.S. at 485--486 (Kennedy, J., concurring in the judgment) (citing the power to grant pardons under U.S. Const.\ Art.~II, \S~2, cl.~1, and the Presentment Clauses for legislation, \textsc{U.S. Const.\ Art.~I}, \S~7, Cls.~2, 3, as examples of exclusive presidential powers by virtue of constitutional text).

But even when a power is exclusive, “Congress’ powers, and its central role in making laws, give it substantial authority regarding many of the policy determinations that precede and follow” the President’s act.
\textit{Zivotofsky}, 135 S. Ct.~at 2087.
For example, although the President’s power to grant pardons is exclusive and not subject to congressional regulation, \textit{see United States v.\ Klein}, 80 U.S. (13 Wall.) 128, 147--148 (1872), Congress has the authority to prohibit the corrupt use of “anything of value” to influence the testimony of another person in a judicial, congressional, or agency proceeding, 18~U.S.C. \S~201(b)(3)---which would include the offer or promise of a pardon to induce a person to testify falsely or not to testify at all.
The offer of a pardon would precede the act of pardoning and thus be within Congress’s power to regulate even if the pardon itself is not.
Just as the Speech or Debate Clause, \textsc{U.S. Const.\ Art.~I}, \S~6, cl.1, absolutely protects legislative acts, but not a legislator’s “taking or agreeing to take money for a promise to act in a certain way ... for it is taking the bribe, not performance of the illicit compact, that is a criminal act,” \textit{United States v.\ Brewster}, 408 U.S. 501, 526 (1972) (emphasis omitted), the promise of a pardon to corruptly influence testimony would not be a constitutionally immunized act.
The application of obstruction statutes to such promises therefore would raise no serious separation-of-powers issue.

\paragraph{The Effect of Obstruction-of-Justice Statutes on the President’s Capacity to Perform His Article II Responsibilities is Limited}

Under the Supreme Court’s balancing test for analyzing separation-of-powers issues, the first task is to assess the degree to which applying obstruction-of-justice statutes to presidential actions affects the President’s ability to carry out his Article II responsibilities.
\textit{Administrator of General Services}, 433 U.S. at 443.
As discussed above, applying obstruction-of-justice statutes to presidential conduct that does not involve the President’s conduct of office---such as influencing the testimony of witnesses---is constitutionally unproblematic.
The President has no more right than other citizens to impede official proceedings by corruptly influencing witness testimony.
The conduct would be equally improper whether effectuated through direct efforts to produce false testimony or suppress the truth, or through the actual, threatened, or promised use of official powers to achieve the same result.

The President’s action in curtailing criminal investigations or prosecutions, or discharging law enforcement officials, raises different questions.
Each type of action involves the exercise of executive discretion in furtherance of the President’s duty to “take Care that the Laws be faithfully executed.”
\textsc{U.S. Const.\ Art.~II}, \S~3.
Congress may not supplant the President’s exercise of executive power to supervise prosecutions or to remove officers who occupy law enforcement positions.
\textit{See Bowsher v.\ Synar}, 478 U.S. 714, 726--727 (1986) (“Congress cannot reserve for itself the power of removal of an officer charged with the execution of the laws except by impeachment. ... [Because t]he structure of the Constitution does not permit Congress to execute the laws, ... [t]his kind of congressional control over the execution of the laws ... is constitutionally impermissible.”).
Yet the obstruction-of-justice statutes do not aggrandize power in Congress or usurp executive authority.
Instead, they impose a discrete limitation on conduct only when it is taken with the “corrupt” intent to obstruct justice.
The obstruction statutes thus would restrict presidential action only by prohibiting the President from acting to obstruct official proceedings for the improper purpose of protecting his own interests.
See Volume~II, Section III.A.3, \textit{supra}.

The direct effect on the President’s freedom of action would correspondingly be a limited one.
A preclusion of “corrupt” official action is not a major intrusion on Article II powers.
For example, the proper supervision of criminal law does not demand freedom for the President to act with the intention of shielding himself from criminal punishment, avoiding financial liability, or preventing personal embarrassment.
To the contrary, a statute that prohibits official action undertaken for such personal purposes furthers, rather than hinders, the impartial and evenhanded administration of the law.
And the Constitution does not mandate that the President have unfettered authority to direct investigations or prosecutions, with no limits whatsoever, in order to carry out his Article II functions.
\textit{See Heckler v.\ Chaney}, 470 U.S. 821, 833 (1985) (“Congress may limit an agency’s exercise of enforcement power if it wishes, either by setting substantive priorities, or by otherwise circumscribing an agency’s power to discriminate among issues or cases it will pursue.”);
\textit{United States v.\ Nixon}, 418 U.S. at 707 (“[t]o read the Art.~I] powers of the President as providing an absolute privilege [to withhold confidential communications from a criminal trial]... would upset the constitutional balance of ‘a workable government’ and gravely impair the role of the courts under Art.~III”).

Nor must the President have unfettered authority to remove all Executive Branch officials involved in the execution of the laws.
The Constitution establishes that Congress has legislative authority to structure the Executive Branch by authorizing Congress to create executive departments and officer positions and to specify how inferior officers are appointed.
E.g., U.S. Const.\ Art.~I, \S~8, cl.~18 (Necessary and Proper Clause); ART. II, \S~2, cl.~1 (Opinions Clause); Art.~II, \S~2, cl.~2 (Appointments Clause);
\textit{see Free Enterprise Fund}, 561 U.S. at 499. While the President’s removal power is an important means of ensuring that officers faithfully execute the law, Congress has a recognized authority to place certain limits on removal.
\textit{Id}. at 493--495.

The President’s removal powers are at their zenith with respect to principal officers---that is, officers who must be appointed by the President and who report to him directly.
\textit{See Free Enterprise Fund}, 561 U.S. at 493, 500.
The President’s “exclusive and illimitable power of removal” of those principal officers furthers “the President’s ability to ensure that the laws are faithfully executed.”
\textit{Id}. at 493, 498 (internal quotation marks omitted);
\textit{Myers}, 272 U.S. at 627.
Thus, “there are some ‘purely executive’ officials who must be removable by the President at will if he is able to accomplish his constitutional role.”
\textit{Morrison}, 487 U.S. at 690;
\textit{Myers}, 272 U.S. at 134 (the President’s “cabinet officers must do his will,” and “[t]he moment that he loses confidence in the intelligence, ability, judgment, or loyalty of any one of them, he must have the power to remove him without delay”);
\textit{cf.~Humphrey’s Executor v.\ United States}, 295 U.S. 602 (1935) (Congress has the power to create independent agencies headed by principal officers removable only for good cause).
In light of those constitutional precedents, it may be that the obstruction statutes could not be constitutionally applied to limit the removal of a cabinet officer such as the Attorney General.
\textit{See} 5~U.S.C. \S~101; 28~U.S.C. \S~503.
In that context, at least absent circumstances showing that the President was clearly attempting to thwart accountability for personal conduct while evading ordinary political checks and balances, even the highly limited regulation imposed by the obstruction statutes could possibly intrude too deeply on the President’s freedom to select and supervise the members of his cabinet.

The removal of inferior officers, in contrast, need not necessarily be at will for the President to fulfill his constitutionally assigned role in managing the Executive Branch.
“[I]nferior officers are officers whose work is directed and supervised at some level by other officers appointed by the President with the Senate’s consent.”
\textit{Free Enterprise Fund}, 561 U.S. at 510 (quoting Edmond v.\ United States, 520 U.S. 651, 663 (1997)) (internal quotation marks omitted), The Supreme Court has long recognized Congress’s authority to place for-cause limitations on the President’s removal of “inferior Officers” whose appointment maybe vested in the head of a department.
\textsc{U.S. Const.\ Art.~II}, \S~2, cl.~2.
\textit{See United States v.\ Perkins}, 116 U.S. 483, 485 (1886) (“The constitutional authority in Congress to thus vest the appointment[of inferior officers in the heads of departments] implies authority to limit, restrict, and regulate the removal by such laws as Congress may enact in relation to the officers so appointed”) (quoting lower court decision);
\textit{Morrison}, 487 U.S. at 689 n.~27 (citing \textit{Perkins});
\textit{accord id}. at 723--724 \& n.4 (Scalia, J., dissenting) (recognizing that Perkins is “established” law);
\textit{see also Free Enterprise Fund}, 561 U.S. at 493- 495 (citing Perkins and Morrison).
The category of inferior officers includes both the FBI Director and the Special Counsel, each of whom reports to the Attorney General.
\textit{See} 28~U.S.C. \S\S~509, 515(a), 531; 28 C.F.R. Part 600.
Their work is thus “directed and supervised” by a presidentially appointed, Senate-confirmed officer.
\textit{See In re: Grand Jury Investigation}, \textunderscore\textunderscore F.3d \textunderscore\textunderscore, 2019 WL 921692, at *3-*4 (D.C. Cir.\ Feb.~26, 2019) (holding that the Special Counsel is an “inferior officer” for constitutional purposes).

Where the Constitution permits Congress to impose a good-cause limitation on the removal of an Executive Branch officer, the Constitution should equally permit Congress to bar removal for the corrupt purpose of obstructing justice.
Limiting the range of permissible reasons for removal to exclude a “corrupt” purpose imposes a lesser restraint on the President than requiring an affirmative showing of good cause.
It follows that for such inferior officers, Congress may constitutionally restrict the President’s removal authority if that authority was exercised for the corrupt purpose of obstructing justice.
And even if a particular inferior officer’s position might be of such importance to the execution of the laws that the President must have at-will removal authority, the obstruction-of-justice statutes could still be constitutionally applied to forbid removal for a corrupt reason.% 1088
\footnote{Although the FBI director is an inferior officer, he is appointed by the President and removable by him at will, see 28~U.S.C. \S~532 note, and it is not clear that Congress could constitutionally provide the FBI director with good-cause tenure protection.
\textit{See} OLC, \textit{Constitutionality of Legislation Extending the Term of the FBI Director}, 2011 WL2566125, at *3 (O.L.C. June 20, 2011) (“tenure protection for an officer with the FBI Director’s broad investigative, administrative, and policymaking responsibilities would raise a serious constitutional question whether Congress had ‘impede[d] the President’s ability to perform his constitutional duty’ to take care that the laws be faithfully executed”) (quoting \textit{Morrison}, 487 U.S. at 691).}
A narrow and discrete limitation on removal that precluded corrupt action would leave ample room for all other considerations, including disagreement over policy or loss of confidence in the officer’s judgment or commitment.
A corrupt-purpose prohibition therefore would not undermine the President’s ability to perform his Article II functions.
Accordingly, because the separation-of-powers question is “whether the removal restrictions are of such a nature that they impede the President’s ability to perform his constitutional duty,” \textit{Morrison}, 487 U.S. at 691, a restriction on removing an inferior officer for a corrupt reason---a reason grounded in achieving personal rather than official ends---does not seriously hinder the President’s performance of his duties.
The President retains broad latitude to supervise investigations and remove officials, circumscribed in this context only by the requirement that he not act for corrupt personal purposes.% 1089
\footnote{The obstruction statutes do not disqualify the President from acting in a case simply because he has a personal interest in it or because his own conduct may be at issue.
As the Department of Justice has made clear, a claim of a conflict of interest, standing alone, cannot deprive the President of the ability to fulfill his constitutional function.
\textit{See}, e.g, OLC, \textit{Application of 28~U.S.C. \S~458 to Presidential Appointments of Federal Judges}, 19 O.L.C. Op.\ at 356 (citing Memorandum for Richard T. Burress, Office of the President, from Laurence H. Silberman, Deputy Attorney General, \textit{Re: Conflict of Interest Problems Arising out of the President's Nomination of Nelson A. Rockefeller to be Vice President under the Twenty-Fifth Amendment to the Constitution}, at 2, 5 (Aug.~28, 1974)).}

\paragraph{Congress Has Power to Protect Congressional, Grand Jury, and Judicial Proceedings Against Corrupt Acts from Any Source}

Where a law imposes a burden on the President’s performance of Article II functions, separation-of-powers analysis considers whether the statutory measure “is justified by an overriding need to promote objectives within the constitutional authority of Congress.”
\textit{Administrator of General Services}, 433 U.S. at 443.
Here, Congress enacted the obstruction-of-justice statutes to protect, among other things, the integrity of its own proceedings, grand jury investigations, and federal criminal trials.
Those objectives are within Congress’s authority and serve strong governmental interests.

i. Congress has Article I authority to define generally applicable criminal law and apply it to all persons---including the President.
Congress clearly has authority to protect its own legislative functions against corrupt efforts designed to impede legitimate fact-gathering and lawmaking efforts.
\textit{See Watkins v.\ United States}, 354 U.S. 178, 187, 206--207 (1957);
\textit{Chapman v.\ United States}, 5 App.\ D.C. 122, 130 (1895).
Congress also has authority to establish a system of federal courts, which includes the power to protect the judiciary against obstructive acts.
\textit{See} \textsc{U.S. Const.\ Art.~I}, \S~8, cls.~9, 18 (“The Congress shall have Power ... To constitute Tribunals inferior to the supreme Court” and “To make all Laws which shall be necessary and proper for carrying into Execution the foregoing powers”).
The long lineage of the obstruction-of-justice statutes, which can be traced to at least 1831, attests to the necessity for that protection.
\textit{See An Act Declaratory of the Law Concerning Contempts of Court}, 4 Stat.~487--488 \S~2 (1831) (making it a crime if “any person or persons shall corruptly ... endeavor to influence, intimidate, or impede any juror, witness, or officer, in any court of the United States, in the discharge of his duty, or shall, corruptly ... obstruct, or impede, or endeavor to obstruct or impede, the due administration of justice therein”).

ii. The Article III courts have an equally strong interest in being protected against obstructive acts, whatever their source.
As the Supreme Court explained in United States v.\ Nixon, a “primary constitutional duty of the Judicial Branch” is “to do justice in criminal prosecutions.” 418 U.S. at 707;
\textit{accord Cheney v.\ United States District Court for the District of Columbia}, 542 U.S. 367, 384 (2004).
In \textit{Nixon}, the Court rejected the President’s claim of absolute executive privilege because “the allowance of the privilege to withhold evidence that is demonstrably relevant in a criminal trial would cut deeply into the guarantee of due process of law and gravely impair the basic function of the courts.” 407 U.S. at 712.
As \textit{Nixon} illustrates, the need to safeguard judicial integrity is a compelling constitutional interest.
\textit{See id}. at 709 (noting that the denial of full disclosure of the facts surrounding relevant presidential communications threatens “(t]he very integrity of the judicial system and public confidence in the system”).

iii. Finally, the grand jury cannot achieve its constitutional purpose absent protection from corrupt acts.
Serious federal criminal charges generally reach the Article III courts based on an indictment issued by a grand jury.
\textit{Cobbledick v.\ United States}, 309 U.S. 323, 327 (1940) (“The Constitution itself makes the grand jury a part of the judicial process.”).
And the grand jury’s function is enshrined in the Fifth Amendment.
\textsc{U.S. Const.\ Amend.~V}. (“[n]o person shall be held to answer” for a serious crime “unless on a presentment or indictment of a Grand Jury”).
“[T]he whole theory of [the grand jury’s] function is that it belongs to no branch of the institutional government, serving as a kind of buffer or referee between the Government and the people,” \textit{United States v.\ Williams}, 504 U.S. 36, 47 (1992), “pledged to indict no one because of prejudice and to free no one because of special favor.”
\textit{Costello v.\ United States}, 350 U.S. 359, 362 (1956).
If the grand jury were not protected against corrupt interference from all persons, its function as an independent charging body would be thwarted.
And an impartial grand jury investigation to determine whether probable cause exists to indict is vital to the criminal justice process.

\hr

The final step in the constitutional balancing process is to assess whether the separation-of-powers doctrine permits Congress to take action within its constitutional authority notwithstanding the potential impact on Article II functions.
\textit{See Administrator of General Services}, 433 U.S. at 443;
\textit{see also Morrison}, 487 U.S. at 691--693, 695--696;
\textit{United States v.\ Nixon}, 418 U.S. at 711--712.
In the case of the obstruction-of-justice statutes, our assessment of the weighing of interests leads us to conclude that Congress has the authority to impose the limited restrictions contained in those statutes on the President’s official conduct to protect the integrity of important functions of other branches of government.

A general ban on corrupt action does not unduly intrude on the President’s responsibility to “take Care that the Laws be faithfully executed.”
\textsc{U.S. Const.\ Art.~II}, \S\S~3.% 1090
\footnote{As noted above, the President’s selection and removal of principal executive officers may have a unique constitutional status.}
To the contrary, the concept of “faithful execution” connotes the use of power in the interest of the public, not in the office holder’s personal interests.
\textit{See} 1 Samuel Johnson, \textit{A Dictionary of the English Language} 763 (1755) (“faithfully” def.~3: “[w]ith strict adherence to duty and allegiance”).
And immunizing the President from the generally applicable criminal prohibition against corrupt obstruction of official proceedings would seriously impair Congress’s power to enact laws “to promote objectives within [its] constitutional authority,” \textit{Administrator of General Services}, 433 U.S. at 425---i.e., protecting the integrity of its own proceedings and the proceedings of Article III courts and grand juries.

Accordingly, based on the analysis above, we were not persuaded by the argument that the President has blanket constitutional immunity to engage in acts that would corruptly obstruct justice through the exercise of otherwise-valid Article II powers.% 1091
\footnote{A possible remedy through impeachment for abuses of power would not substitute for potential criminal liability after a President leaves office.
Impeachment would remove a President from office, but would not address the underlying culpability of the conduct or serve the usual purposes of the criminal law.
Indeed, the Impeachment Judgment Clause recognizes that criminal law plays an independent role in addressing an official’s conduct, distinct from the political remedy of impeachment.
\textit{See} \textsc{U.S. Const.\ Art}.~I, \S~3, cl.~7.
Impeachment is also a drastic and rarely invoked remedy, and Congress is not restricted to relying only on impeachment, rather than making criminal law applicable to a former President, as OLC has recognized.
\textit{A Sitting President's Amenability to Indictment and Criminal Prosecution}, 24 Op.\ O.L.C. at 255 (“Recognizing an immunity from prosecution for a sitting President would not preclude such prosecution once the President’s term is over or he is otherwise removed from office by resignation or impeachment.”).}

\subsubsection{Ascertaining Whether the President Violated the Obstruction Statutes Would Not Chill his Performance of his Article II Duties}

Applying the obstruction statutes to the President’s official conduct would involve determining as a factual matter whether he engaged in an obstructive act, whether the act had a nexus to official proceedings, and whether he was motivated by corrupt intent.
But applying those standards to the President’s official conduct should not hinder his ability to perform his Article II duties.
\textit{Cf.~Nixon v.\ Fitzgerald}, 457 U.S. at 752--753 \& n.32 (taking into account chilling effect on the President in adopting a constitutional rule of presidential immunity from private civil damages action based on official duties).
Several safeguards would prevent a chilling effect: the existence of settled legal standards, the presumption of regularity in prosecutorial actions, and the existence of evidentiary limitations on probing the President’s motives.
And historical experience confirms that no impermissible chill should exist.

a. As an initial matter, the term “corruptly” sets a demanding standard.
It requires a concrete showing that a person acted with an intent to obtain an “improper advantage for [him]self or someone else, inconsistent with official duty and the rights of others.” BALLENTINE’S LAW DICTIONARY 276 (3d ed.~1969);
\textit{see United States v.\ Pasha}, 797 F.3d 1122, 1132 (D.C. Cir.~2015);
\textit{Aguilar}, 515 U.S. at 616 (Scalia, J., concurring in part and dissenting in part).
That standard parallels the President’s constitutional obligation to ensure the faithful execution of the laws.
And virtually everything that the President does in the routine conduct of office will have a clear governmental purpose and will not be contrary to his official duty.
Accordingly, the President has no reason to be chilled in those actions because, in virtually all instances, there will be no credible basis for suspecting a corrupt personal motive.

That point is illustrated by examples of conduct that would and would not satisfy the stringent corrupt-motive standard.
Direct or indirect action by the President to end a criminal investigation into his own or his family members’ conduct to protect against personal embarrassment or legal liability would constitute a core example of corruptly motivated conduct.
So too would action to halt an enforcement proceeding that directly and adversely affected the President’s financial interests for the purpose of protecting those interests.
In those examples, official power is being used for the purpose of protecting the President’s personal interests.
In contrast, the President’s actions to serve political or policy interests would not qualify as corrupt.
The President’s role as head of the government necessarily requires him to take into account political factors in making policy decisions that affect law-enforcement actions and proceedings.
For instance, the President’s decision to curtail a law-enforcement investigation to avoid international friction would not implicate the obstruction-of-justice statutes.
The criminal law does not seek to regulate the consideration of such political or policy factors in the conduct of government.
And when legitimate interests animate the President’s conduct, those interests will almost invariably be readily identifiable based on objective factors.
Because the President’s conduct in those instances will obviously fall outside the zone of obstruction law, no chilling concern should arise.

b. There is also no reason to believe that investigations, let alone prosecutions, would occur except in highly unusual circumstances when a credible factual basis exists to believe that obstruction occurred.
Prosecutorial action enjoys a presumption of regularity: absent “clear evidence to the contrary, courts presume that [prosecutors] have properly discharged their official duties.”
\textit{Armstrong}, 517 U.S. at 464 (\textit{quoting United States v.\ Chemical Foundation, Inc.}, 272 U.S. 1, 14--15 (1926)).
The presumption of prosecutorial regularity would provide even greater protection to the President than exists in routine cases given the prominence and sensitivity of any matter involving the President and the likelihood that such matters will be subject to thorough and careful review at the most senior levels of the Department of Justice.
Under OLC’s opinion that a sitting President is entitled to immunity from indictment, only a successor Administration would be able to prosecute a former President.
But that consideration does not suggest that a President would have any basis for fearing abusive investigations or prosecutions after leaving office.
There are “obvious political checks” against initiating a baseless investigation or prosecution of a former President.
\textit{See Administrator of General Services}, 433 U.S. at 448 (considering political checks in separation-of-powers analysis).
And the Attorney General holds “the power to conduct the criminal litigation of the United States Government,” \textit{United States v.\ Nixon}, 418 U.S. at 694 (citing 28~U.S.C. \S~516), which provides a strong institutional safeguard against politicized investigations or prosecutions.% 1092
\footnote{Similar institutional safeguards protect Department of Justice officers and line prosecutors against unfounded investigations into prosecutorial acts.
Prosecutors are generally barred from participating in matters implicating their personal interests, \textit{see} 28~C.F.R. \S~45.2, and are instructed not to be influenced by their “own professional or personal circumstances,” Justice Manual \S~9-27.260, so prosecutors would not frequently be in a position to take action that could be perceived as corrupt and personally motivated.
And if such cases arise, criminal investigation would be conducted by responsible officials at the Department of Justice, who can be presumed to refrain from pursuing an investigation absent a credible factual basis.
Those facts distinguish the criminal context from the common-law rule of prosecutorial immunity, which protects against the threat of suit by “a defendant [who] often will transform his resentment at being prosecuted into the ascription of improper and malicious actions.”
\textit{Imbler v.\ Pachtman}, 424 U.S. 409, 425 (1976).
As the Supreme Court has noted, the existence of civil immunity does not justify criminal immunity.
\textit{See O'Sheay v.\ Littleton}, 414 U.S. 488, 503 (1974) (“Whatever may be the case with respect to civil liability generally, ... we have never held that the performance of the duties of judicial, legislative, or executive officers, requires or contemplates the immunization of otherwise criminal deprivation of constitutional rights.”) (citations omitted).}

These considerations distinguish the Supreme Court’s holding in \textit{Nixon v. Fitzgerald} that, in part because inquiries into the President’s motives would be “highly intrusive,” the President is absolutely immune from private civil damages actions based on his official conduct. 457 U.S. at 756--757.
As \textit{Fitzgerald} recognized,“there is a lesser public interest in actions for civil damages than, for example, in criminal prosecutions.”
\textit{Fitzgerald}, 457 U.S. at 754 n.37; see Cheney, 542 U.S. at 384.
And private actions are not subject to the institutional protections of an action under the supervision of the Attorney General and subject to a presumption of regularity.
\textit{Armstrong}, 517 U.S. at 464.

c. In the rare cases in which a substantial and credible basis justifies conducting an investigation of the President, the process of examining his motivations to determine whether he acted for a corrupt purpose need not have a chilling effect.
Ascertaining the President’s motivations would turn on any explanation he provided to justify his actions, the advice he received, the circumstances surrounding the actions, and the regularity or irregularity of the process he employed to make decisions.
But grand juries and courts would not have automatic access to confidential presidential communications on those matters; rather, they could be presented in official proceedings only on a showing of sufficient need.
\textit{Nixon}, 418 U.S. at 712;
\textit{In re Sealed Case}, 121 F.3d 729, 754, 756--757 (D.C. Cir.~1997);
\textit{see also Administrator of General Services}, 433 U.S. at 448--449 (former President can invoke presidential communications privilege, although successor’s failure to support the claim “detracts from [its] weight’).

In any event, probing the President’s intent in a criminal matter is unquestionably constitutional in at least one context: the offense of bribery turns on the corrupt intent to receive a thing of value in return for being influenced in official action.
18~U.S.C. \S~201(b)(2).
There can be no serious argument against the President’s potential criminal liability for bribery offenses, notwithstanding the need to ascertain his purpose and intent.
\textit{See} \textsc{U.S. Const.\ Art.~I}, \S~3; \textsc{Art.~II}, \S~4;
\textit{see also Application of 28~U.S.C. \S~458 to Presidential Appointments of Federal Judges}, 19 Op.\ O.L.C. at 357 n.11 (“Application of \S~201[to the President] raises no separation of powers issue, let alone a serious one.”).

d. Finally, history provides no reason to believe that any asserted chilling effect justifies exempting the President from the obstruction laws.
As a historical matter, Presidents have very seldom been the subjects of grand jury investigations.
And it is rarer still for circumstances to raise even the possibility of a corrupt personal motive for arguably obstructive action through the President’s use of official power.
Accordingly, the President’s conduct of office should not be chilled based on hypothetical concerns about the possible application of a corrupt-motive standard in this context.

\hr

In sum, contrary to the position taken by the President’s counsel, we concluded that, in light of the Supreme Court precedent governing separation-of-powers issues, we had a valid basis for investigating the conduct at issue in this report.
In our view, the application of the obstruction statutes would not impermissibly burden the President’s performance of his Article II function to supervise prosecutorial conduct or to remove inferior law-enforcement officers.
And the protection of the criminal justice system from corrupt acts by any person---including the President---accords with the fundamental principle of our government that “[n]Jo [person] in this country is so high that he is above the law.”
\textit{United States v.\ Lee}, 106 U.S. 196, 220 (1882);
\textit{see also Clinton v.\ Jones}, 520 U.S. at 697;
\textit{United States v.\ Nixon}, \textit{supra}.
