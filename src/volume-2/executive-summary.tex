\section*{Executive Summary to Volume II}
\label{sec:executive-2}
\addcontentsline{toc}{section}{\nameref{sec:executive-2}}
\markboth{Executive Summary to Volume II}{Executive Summary to Volume II}

\subsection*{Factual Results of the Obstruction Investigation}

\subsubsection*{The Campaign’s response to reports about Russian support for Trump.}

During the 2016 presidential campaign, questions arose about the Russian government’s apparent support for candidate Trump.
After WikiLeaks released politically damaging Democratic Party emails that were reported to have been hacked by Russia, Trump publicly expressed skepticism that Russia was responsible for the hacks at the same time that he and other Campaign officials privately sought information about any further planned WikiLeaks releases.
Trump also denied having any business in or connections to Russia, even though as late as June 2016 the Trump Organization had been pursuing a licensing deal for a skyscraper to be built in Russia called Trump Tower Moscow.
After the election, the President expressed concerns to advisors that reports of Russia’s election interference might lead the public to question the legitimacy of his election.

\subsubsection*{Conduct involving FBI Director Comey and Michael Flynn.}

In mid-January 2017, incoming National Security Advisor Michael Flynn falsely denied to the Vice President, other administration officials, and FBI agents that he had talked to Russian Ambassador Sergey Kislyak about Russia’s response to U.S. sanctions on Russia for its election interference.
On January 27, the day after the President was told that Flynn had lied to the Vice President and had made similar statements to the FBI, the President invited FBI Director Comey to a private dinner at the White House and told Comey that he needed loyalty.
On February 14, the day after the President requested Flynn’s resignation, the President told an outside advisor, “Now that we fired Flynn, the Russia thing is over.”
The advisor disagreed and said the investigations would continue.

Later that afternoon, the President cleared the Oval Office to have a one-on-one meeting with Comey.
Referring to the FBI’s investigation of Flynn, the President said, “I hope you can see your way clear to letting this go, to letting Flynn go.
He is a good guy.
I hope you can let this go.”
Shortly after requesting Flynn’s resignation and speaking privately to Comey, the President sought to have Deputy National Security Advisor K.T. McFarland draft an internal letter stating that the President had not directed Flynn to discuss sanctions with Kislyak.
McFarland declined because she did not know whether that was true, and a White House Counsel’s Office attorney thought that the request would look like a quid pro quo for an ambassadorship she had been offered.

\subsubsection*{The President’s reaction to the continuing Russia investigation.}

In February 2017, Attorney General Jeff Sessions began to assess whether he had to recuse himself from campaign related investigations because of his role in the Trump Campaign.
In early March, the President told White House Counsel Donald McGahn to stop Sessions from recusing.
And after Sessions announced his recusal on March~2, the President expressed anger at the decision and told advisors that he should have an Attorney General who would protect him.
That weekend, the President took Sessions aside at an event and urged him to “unrecuse.”
Later in March, Comey publicly disclosed at a congressional hearing that the FBI was investigating “the Russian government’s efforts to interfere in the 2016 presidential election,” including any links or coordination between the Russian government and the Trump Campaign.
In the following days, the President reached out to the Director of National Intelligence and the leaders of the Central Intelligence Agency (CIA) and the National Security Agency (NSA) to ask them what they could do to publicly dispel the suggestion that the President had any connection to the Russian election-interference effort.
The President also twice called Comey directly, notwithstanding guidance from McGahn to avoid direct contacts with the Department of Justice.
Comey had previously assured the President that the FBI was not investigating him personally, and the President asked Comey to “lift the cloud” of the Russia investigation by saying that publicly.

\subsubsection*{The President’s termination of Comey.}

On May 3, 2017, Comey testified in a congressional hearing, but declined to answer questions about whether the President was personally under investigation.
Within days, the President decided to terminate Comey.
The President insisted that the termination letter, which was written for public release, state that Comey had informed the President that he was not under investigation.
The day of the firing, the White House maintained that Comey’s termination resulted from independent recommendations from the Attorney General and Deputy Attorney General that Comey should be discharged for mishandling the Hillary Clinton email investigation.
But the President had decided to fire Comey before hearing from the Department of Justice.
The day after firing Comey, the President told Russian officials that he had “faced great pressure because of Russia,” which had been “taken off by Comey’s firing.
The next day, the President acknowledged in a television interview that he was going to fire Comey regardless of the Department of Justice’s recommendation and that when he “decided to just do it,” he was thinking that “this thing with Trump and Russia is a made-up story.”
In response to a question about whether he was angry with Comey about the Russia investigation, the President said, “As far as I’m concerned, I want that thing to be absolutely done properly,” adding that firing Comey “might even lengthen out the investigation.”

\subsubsection*{The appointment of a Special Counsel and efforts to remove him.}

On May 17, 2017, the Acting Attorney General for the Russia investigation appointed a Special Counsel to conduct the investigation and related matters.
The President reacted to news that a Special Counsel had been appointed by telling advisors that it was “the end of his presidency” and demanding that Sessions resign.
Sessions submitted his resignation, but the President ultimately did not accept it.
The President told aides that the Special Counsel had conflicts of interest and suggested that the Special Counsel therefore could not serve.
The President’s advisors told him the asserted conflicts were meritless and had already been considered by the Department of Justice.

On June 14, 2017, the media reported that the Special Counsel’s Office was investigating whether the President had obstructed justice.
Press reports called this “a major turning point” in the investigation: while Comey had told the President he was not under investigation, following Comey’s firing, the President now was under investigation.
The President reacted to this news with a series of tweets criticizing the Department of Justice and the Special Counsel’s investigation.
On June 17, 2017, the President called McGahn at home and directed him to call the Acting Attorney General and say that the Special Counsel had conflicts of interest and must be removed.
McGahn did not carry out the direction, however, deciding that he would resign rather than trigger what he regarded as a potential Saturday Night Massacre.

\subsubsection*{Efforts to curtail the Special Counsel’s investigation.}

Two days after directing McGahn to have the Special Counsel removed, the President made another attempt to affect the course of the Russia investigation.
On June 19, 2017, the President met one-on-one in the Oval Office with his former campaign manager Corey Lewandowski, a trusted advisor outside the government, and dictated a message for Lewandowski to deliver to Sessions.
The message said that Sessions should publicly announce that, notwithstanding his recusal from the Russia investigation, the investigation was “very unfair” to the President, the President had done nothing wrong, and Sessions planned to meet with the Special Counsel and “let [him] move forward with investigating election meddling for future elections.”
Lewandowski said he understood what the President wanted Sessions to do.

One month later, in another private meeting with Lewandowski on July 19, 2017, the President asked about the status of his message for Sessions to limit the Special Counsel investigation to future election interference.
Lewandowski told the President that the message would be delivered soon.
Hours after that meeting, the President publicly criticized Sessions in an interview with the New York Times, and then issued a series of tweets making it clear that Sessions’s job was in jeopardy.
Lewandowski did not want to deliver the President’s message personally, so he asked senior White House official Rick Dearborn to deliver it to Sessions.
Dearborn was uncomfortable with the task and did not follow through.

\subsubsection*{Efforts to prevent public disclosure of evidence.}

In the summer of 2017, the President learned that media outlets were asking questions about the June 9, 2016 meeting at Trump Tower between senior campaign officials, including Donald Trump~Jr., and a Russian lawyer who was said to be offering damaging information about Hillary Clinton as “part of Russia and its government’s support for Mr.~Trump.”
On several occasions, the President directed aides not to publicly disclose the emails setting up the June 9 meeting, suggesting that the emails would not leak and that the number of lawyers with access to them should be limited.
Before the emails became public, the President edited a press statement for Trump~Jr. by deleting a line that acknowledged that the meeting was with “an individual who [Trump~Jr.] was told might have information helpful to the campaign” and instead said only that the meeting was about adoptions of Russian children.
When the press asked questions about the President’s involvement in Trump~Jr.’s statement, the President’s personal lawyer repeatedly denied the President had played any role.

\subsubsection*{Further efforts to have the Attorney General take control of the investigation.}

In early summer 2017, the President called Sessions at home and again asked him to reverse his recusal from the Russia investigation. Sessions did not reverse his recusal.
In October 2017, the President met privately with Sessions in the Oval Office and asked him to “take [a] look” at investigating Clinton.
In December 2017, shortly after Flynn pleaded guilty pursuant to a cooperation agreement, the President met with Sessions in the Oval Office and suggested, according to notes taken by a senior advisor, that if Sessions unrecused and took back supervision of the Russia investigation, he would be a “hero.”
The President told Sessions, “I’m not going to do anything or direct you to do anything.
I just want to be treated fairly.”
In response, Sessions volunteered that he had never seen anything “improper” on the campaign and told the President there was a “whole new leadership team” in place.
He did not unrecuse.

\subsubsection*{Efforts to have McGahn deny that the President had ordered him to have the Special Counsel removed.}

In early 2018, the press reported that the President had directed McGahn to have the Special Counsel removed in June 2017 and that McGahn had threatened to resign rather than carry out the order.
The President reacted to the news stories by directing White House officials to tell McGahn to dispute the story and create a record stating he had not been ordered to have the Special Counsel removed.
McGahn told those officials that the media reports were accurate in stating that the President had directed McGahn to have the Special Counsel removed.
The President then met with McGahn in the Oval Office and again pressured him to deny the reports.
In the same meeting, the President also asked McGahn why he had told the Special Counsel about the President’s effort to remove the Special Counsel and why McGahn took notes of his conversations with the President.
McGahn refused to back away from what he remembered happening and perceived the President to be testing his mettle.

\subsubsection*{Conduct towards Flynn, Manafort, [$\blacksquare\blacksquare\blacksquare\blacksquare\blacksquare\blacksquare\blacksquare\blacksquare$: HOM]}

After Flynn withdrew from a joint defense agreement with the President and began cooperating with the government, the President’s personal counsel left a message for Flynn’s attorneys reminding them of the President’s warm feelings towards Flynn, which he said “still remains,” and asking for a “heads up” if Flynn knew “information that implicates the President.”
When Flynn’s counsel reiterated that Flynn could no longer share information pursuant to a joint defense agreement, the President’s personal counsel said he would make sure that the President knew that Flynn’s actions reflected “hostility” towards the President.
During Manafort’s prosecution and when the jury in his criminal, trial was deliberating, the President praised Manafort in public, said that Manafort was being treated unfairly, and declined to rule out a pardon.
After Manafort was convicted, the President called Manafort “a brave man” for refusing to “break” and said that “flipping” “almost ought to be outlawed.”
\blackout{Harm to Ongoing Matter}

\subsubsection*{Conduct involving Michael Cohen.}

The President’s conduct towards Michael Cohen, a former Trump Organization executive, changed from praise for Cohen when he falsely minimized the President’s involvement in the Trump Tower Moscow project, to castigation of Cohen when he became a cooperating witness.
From September 2015 to June 2016, Cohen had pursued the Trump Tower Moscow project on behalf of the Trump Organization and had briefed candidate Trump on the project numerous times, including discussing whether Trump should travel to Russia to advance the deal.
In 2017, Cohen provided false testimony to Congress about the project, including stating that he had only briefed Trump on the project three times and never discussed travel to Russia with him, in an effort to adhere to a “party line” that Cohen said was developed to minimize the President’s connections to Russia.
While preparing for his congressional testimony, Cohen had extensive discussions with the President’s personal counsel, who, according to Cohen, said that Cohen should “stay on message” and not contradict the President.
After the FBI searched Cohen’s home and office in April 2018, the President publicly asserted that Cohen would not “flip,” contacted him directly to tell him to “stay strong,” and privately passed messages of support to him.
Cohen also discussed pardons with the President’s personal counsel and believed that if he stayed on message he would be taken care of.
But after Cohen began cooperating with the government in the summer of 2018, the President publicly criticized him, called him a “rat,” and suggested that his family members had committed crimes.

\subsubsection*{Overarching factual issues.}

We did not make a traditional prosecution decision about these facts, but the evidence we obtained supports several general statements about the President’s conduct.

Several features of the conduct we investigated distinguish it from typical obstruction-of-justice cases.
First, the investigation concerned the President, and some of his actions, such as firing the FBI director, involved facially lawful acts within his Article II authority, which raises constitutional issues discussed below.
At the same time, the President’s position as the head of the Executive Branch provided him with unique and powerful means of influencing official proceedings, subordinate officers, and potential witnesses---all of which is relevant to a potential obstruction-of-justice analysis.
Second, unlike cases in which a subject engages in obstruction of justice to cover up a crime, the evidence we obtained did not establish that the President was involved in an underlying crime related to Russian election interference.
Although the obstruction statutes do not require proof of such a crime, the absence of that evidence affects the analysis of the President’s intent and requires consideration of other possible motives for his conduct.
Third, many of the President’s acts directed at witnesses, including discouragement of cooperation with the government and suggestions of possible future pardons, took place in public view.
That circumstance is unusual, but no principle of law excludes public acts from the reach of the obstruction laws.
If the likely effect of public acts is to influence witnesses or alter their testimony, the harm to the justice system’s integrity is the same.

Although the series of events we investigated involved discrete acts, the overall pattern of the President’s conduct towards the investigations can shed light on the nature of the President’s acts and the inferences that can be drawn about his intent.
In particular, the actions we investigated can be divided into two phases, reflecting a possible shift in the President’s motives.
The first phase covered the period from the President’s first interactions with Comey through the President’s firing of Comey.
During that time, the President had been repeatedly told he was not personally under investigation.
Soon after the firing of Comey and the appointment of the Special Counsel, however, the President became aware that his own conduct was being investigated in an obstruction-of-justice inquiry.
At that point, the President engaged in a second phase of conduct, involving public attacks on the investigation, non-public efforts to control it, and efforts in both public and private to encourage witnesses not to cooperate with the investigation.
Judgments about the nature of the President’s motives during each phase would be informed by the totality of the evidence.

\subsection{Statutory and Constitutional Defenses}

The President’s counsel raised statutory and constitutional defenses to a possible obstruction-of-justice analysis of the conduct we investigated.
We concluded that none of those legal defenses provided a basis for declining to investigate the facts.

\subsubsection*{Statutory defenses.}

Consistent with precedent and the Department of Justice’s general approach to interpreting obstruction statutes, we concluded that several statutes could apply here.
See 18 U.S.C. \S\S 1503, 1505, 1512(b)(3), 1512(c)(2).
Section 1512(c)(2) is an omnibus obstruction-of-justice provision that covers a range of obstructive acts directed at pending or contemplated official proceedings.
No principle of statutory construction justifies narrowing the provision to cover only conduct that impairs the integrity or availability of evidence.
Sections 1503 and 1505 also offer broad protection against obstructive acts directed at pending grand jury, judicial, administrative, and congressional proceedings, and they are supplemented by a provision in Section 1512(b) aimed specifically at conduct intended to prevent or hinder the communication to law enforcement of information related to a federal crime.

\subsubsection*{Constitutional defenses.}

As for constitutional defenses arising from the President’s status as the head of the Executive Branch, we recognized that the Department of Justice and the courts have not definitively resolved these issues.
We therefore examined those issues through the framework established by Supreme Court precedent governing separation-of-powers issues.
The Department of Justice and the President’s personal counsel have recognized that the President is subject to statutes that prohibit obstruction of justice by bribing a witness or suborning perjury because that conduct does not implicate his constitutional authority.
With respect to whether the President can be found to have obstructed justice by exercising his powers under Article II of the Constitution, we concluded that Congress has authority to prohibit a President’s corrupt use of his authority in order to protect the integrity of the administration of justice.

Under applicable Supreme Court precedent, the Constitution does not categorically and permanently immunize a President for obstructing justice through the use of his Article II powers.
The separation-of-powers doctrine authorizes Congress to protect official proceedings, including those of courts and grand juries, from corrupt, obstructive acts regardless of their source.
We also concluded that any inroad on presidential authority that would occur from prohibiting corrupt acts does not undermine the President’s ability to fulfill his constitutional mission.
The term “corruptly” sets a demanding standard.
It requires a concrete showing that a person acted with an intent to obtain an improper advantage for himself or someone else, inconsistent with official duty and the rights of others.
A preclusion of “corrupt” official action does not diminish the President’s ability to exercise Article II powers.
For example, the proper supervision of criminal law does not demand freedom for the President to act with a corrupt intention of shielding himself from criminal punishment, avoiding financial liability, or preventing personal embarrassment.
To the contrary, a statute that prohibits official action undertaken for such corrupt purposes furthers, rather than hinders, the impartial and evenhanded administration of the law.
It also aligns with the President’s constitutional duty to faithfully execute the laws.
Finally, we concluded that in the rare case in which a criminal investigation of the President’s conduct is justified, inquiries to determine whether the President acted for a corrupt motive should not impermissibly chill his performance of his constitutionally assigned duties.
The conclusion that Congress may apply the obstruction laws to the President’s corrupt exercise of the powers of office accords with our constitutional system of checks and balances and the principle that no person is above the law.

\subsection*{Conclusion}

Because we determined not to make a traditional prosecutorial judgment, we did not draw ultimate conclusions about the President’s conduct.
The evidence we obtained about the President’s actions and intent presents difficult issues that would need to be resolved if we were making traditional prosecutorial judgment.
At the same time, if we had confidence after a thorough investigation of the facts that the President clearly did not commit obstruction of justice, we would so state.
Based on the facts and the applicable legal standards, we are unable to reach that judgment.
Accordingly, while this report does not conclude that the President committed a crime, it also does not exonerate him.
