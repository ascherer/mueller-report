\documentclass[12pt]{article}

% support Cyrillic
\usepackage{fontspec} % loaded by polyglossia, but included here for transparency
\usepackage{polyglossia}
\setmainlanguage{english}
\setotherlanguage{russian}
\setmainfont[Ligatures=TeX]{Times New Roman}
\newfontfamily\cyrillicfont{Times New Roman}[Script=Cyrillic]

% no hyphenation
\usepackage[none]{hyphenat}

% \tolerance=1
% \emergencystretch=\maxdimen
% \hyphenpenalty=10000
% \hbadness=10000

% censorship tools
\usepackage{censor}

% enumerate with Roman numerals
\usepackage{enumerate}

% horizontal rule
\newcommand{\hr}{\begin{center} \line(1,0){50} \end{center}}

% do not number the sections
\setcounter{secnumdepth}{0}

% table of contents is clickable
\usepackage{hyperref}
\hypersetup{
  colorlinks,
  citecolor=black,
  filecolor=black,
  linkcolor=black,
  urlcolor=black
}

\begin{document}

\pagenumbering{roman}
\tableofcontents
\newpage
\pagenumbering{arabic}

\section{Introduction to Volume I}

This report is submitted to the Attorney General pursuant to 28 C.F.R. \S 600.8(c), which states that, "[a]t the conclusion of the Special Counsel's work, he ... shall provide the Attorney General a confidential report explaining the prosecution or declination decisions [the Special Counsel] reached."

The Russian government interfered in the 2016 presidential election in sweeping and systematic fashion.
Evidence of Russian government operations began to surface in mid-2016.
In June, the Democratic National Committee and its cyber response team publicly announced that Russian hackers had compromised its computer network. Releases of hacked materials-hacks that public reporting soon attributed to the Russian government-began that same month.
Additional releases followed in July through the organization WikiLeaks, with further releases in October and November.

In late July 2016, soon after WikiLeaks's first release of stolen documents, a foreign government contacted the FBI about a May 2016 encounter with Trump Campaign foreign policy advisor George Papadopoulos.
Papadopoulos had suggested to a representative of that foreign government that the Trump Campaign had received indications from the Russian government that it could assist the Campaign through the anonymous release of information damaging to Democratic presidential candidate Hillary Clinton.
That information prompted the FBI on July 31, 2016, to open an investigation into whether individuals associated with the Trump Campaign were coordinating with the Russian government in its interference activities.

That fall, two federal agencies jointly announced that the Russian government "directed recent compromises of e-mails from US persons and institutions, including US political organizations," and, "[t]hese thefts and disclosures are intended to interfere with the US election process."
After the election, in late December 2016, the United States imposed sanctions on Russia for having interfered in the election. By early 2017, several congressional committees were examining Russia's interference in the election.

Within the Executive Branch, these investigatory efforts ultimately led to the May 2017 appointment of Special Counsel Robert S. Mueller, III.
The order appointing the Special Counsel authorized him to investigate "the Russian government's efforts to interfere in the 2016 presidential election," including any links or coordination between the Russian government and individuals associated with the Trump Campaign.

As set forth in detail in this report, the Special Counsel's investigation established that Russia interfere in the 2016 presidential election principally through two operations.
First, a Russian entity carried out a social media campaign that favored presidential candidate Donald J. Trump and disparaged presidential candidate Hillary Clinton.
Second, a Russian intelligence service conducted computer-intrusion operations against entities, employees, and volunteers working on the Clinton Campaign and then released stolen documents.
The investigation also identified numerous links between the Russian government and the Trump Campaign.
Although the investigation established that the Russian government perceived it would benefit from a Trump presidency and worked to secure that outcome, and that the Campaign expected it would benefit electorally from information stolen and released through Russian efforts, the investigation did not establish that members of the Trump Campaign conspired or coordinated with the Russian government in its election interference activities.

\hr

Below we describe the evidentiary considerations underpinning statements about the results of our investigation and the Special Counsel's charging decisions, and we then provide an overview of the two volumes of our report.

The report describes actions and events that the Special Counsel's Office found to be supported by the evidence collected in our investigation.
In some instances, the report points out the absence of evidence or conflicts in the evidence about a particular fact or event.
In other instances, when substantial, credible evidence enabled the Office to reach a conclusion with confidence, the report states that the investigation established that certain actions or events occurred.
A statement that the investigation did not establish particular facts does not mean there was no evidence of those facts.

In evaluating whether evidence about collective action of multiple individuals constituted a crime, we applied the framework of conspiracy law, not the concept of "collusion."
In so doing, the Office recognized that the word "collud[e]" was used in communications with the Acting Attorney General confirming certain aspects of the investigation's scope and that the term has frequently been invoked in public reporting about the investigation.
But collusion is not a specific offense or theory of liability found in the United States Code, nor is it a term of art in federal criminal law.
For those reasons, the Office's focus in analyzing questions of joint criminal liability was on conspiracy as defined in federal law.
In connection with that analysis, we addressed the factual question whether members of the Trump Campaign" coordinat[ed]" - a term that appears in the appointment order - with Russian election interference activities.
Like collusion, "coordination" does not have a settled definition in federal criminal law.
We understood coordination to require an agreement - tacit or express - between the Trump Campaign and the Russian government on election interference.
That requires more than the two parties taking actions that were informed by or responsive to the other's actions or interests.
We applied the term coordination in that sense when stating in the report that the investigation did not establish that the Trump Campaign coordinated with the Russian government in its election interference activities.

\hr

The report on our investigation consists of two volumes:

\textit{Volume I} describes the factual results of the Special Counsel's investigation of Russia's interference in the 2016 presidential election and its interactions with the Trump Campaign.
Section I describes the scope of the investigation.
Sections II and III describe the principal ways Russia interfered in the 2016 presidential election.
Section IV describes links between the Russian government and individuals associated with the Trump Campaign.
Section V sets forth the Special Counsel's charging decisions.

\textit{Volume II} addresses the President's actions towards the FBI's investigation into Russia's interference in the 2016 presidential election and related matters, and his actions towards the Special Counsel's investigation.
Volume II separately states its framework and the considerations that guided that investigation.

\section{Executive Summary to Volume I}

\subsection{Russian Social Media Campaign}

The Internet Research Agency (IRA) carried out the earliest Russian interference operations identified by the investigation - a social media campaign designed to provoke and amplify political and social discord in the United States.
The IRA was based in St. Petersburg, Russia, and received funding from Russian oligarch Yevgeniy Prigozhin and companies he controlled.
Prigozhin is widely reported to have ties to Russian President Vladimir Putin, \xblackout{Harm to Ongoing Matter: Lorem ipsum dolor sit amet, consectetur adipiscing elit, sed do eiusmod tempor}

In mid-2014, the IRA sent employees to the United States on an intelligence-gathering mission with instructions \xblackout{Harm to Ongoing Matter: Lorem ipsum dolor sit amet, consectetur adipiscing elit, sed do eiusmod tempor incididunt ut labore et dolore magna aliqua.}

The IRA later used social media accounts and interest groups to sow discord in the U.S. political system through what it termed "information warfare."
The campaign evolved from a generalized program designed in 2014 and 2015 to undermine the U.S. electoral system, to a targeted operation that by early 2016 favored candidate Trump and disparaged candidate Clinton.
The IRA's operation also included the purchase of political advertisements on social media in the names of U.S. persons and entities, as well as the staging of political rallies inside the United States.
To organize those rallies, IRA employees posed as U.S. grassroots entities and persons and made contact with Trump supporters and Trump Campaign officials in the United States.
The investigation did not identify evidence that any U.S. persons conspired or coordinated with the IRA.
Section II of this report details the Office's investigation of the Russian social media campaign.

\subsection{Russian Hacking Operations}

At the same time that the IRA operation began to focus on supporting candidate Trump in early 2016, the Russian government employed a second form of interference: cyber intrusions (hacking) and releases of hacked materials damaging to the Clinton Campaign.
The Russian intelligence service known as the Main Intelligence Directorate of the General Staff of the Russian Army (GRU) carried out these operations.

In March 2016, the GRU began hacking the email accounts of Clinton Campaign volunteers and employees, including campaign chairman John Podesta.
In April 2016, the GRU hacked into the computer networks of the Democratic Congressional Campaign Committee (DCCC) and the Democratic National Committee (DNC).
The GRU stole hundreds of thousands of documents from the compromised email accounts and networks.
Around the time that the DNC announced in mid-June 2016 the Russian government's role in hacking its network, the GRU began disseminating stolen materials through the fictitious online personas "DCLeaks" and "Guccifer 2.0."
The GRU later released additional materials through the organization WikiLeaks.

The presidential campaign of Donald J. Trump ("Trump Campaign" or "Campaign") showed interest in WikiLeaks's releases of documents and welcomed their potential to damage candidate Clinton. Beginning in June 2016, \xblackout{Harm to Ongoing Matter: Lorem ipsum dolor sit amet} forecast to senior Campaign officials that WikiLeaks would release information damaging to candidate Clinton.
WikiLeaks's first release came in July 2016.
Around the same time, candidate Trump announced that he hoped Russia would recover emails described as missing from a private server used by Clinton when she was Secretary of State (he later said that he was speaking sarcastically).
\xblackout{Harm to Ongoing Matter: Lorem ipsum dolor sit amet, consectetur adipiscing elit, sed do eiusmod tempor} WikiLeaks began releasing Podesta's stolen emails on October 7, 2016, less than one hour after a U.S. media outlet released video considered damaging to candidate Trump.
Section III of this Report details the Office's investigation into the Russian hacking operations, as well as other efforts by Trump Campaign supporters to obtain Clinton-related emails.

\subsection{Russian Contacts with the Campaign}

The social media campaign and the GRU hacking operations coincided with a series of contacts between Trump Campaign officials and individuals with ties to the Russian government.
The Office investigated whether those contacts reflected or resulted in the Campaign conspiring or coordinating with Russia in its election-interference activities.
Although the investigation established that the Russian government perceived it would benefit from a Trump presidency and worked to secure that outcome, and that the Campaign expected it would benefit electorally from information stolen and released through Russian efforts, the investigation did not establish that members of the Trump Campaign conspired or coordinated with the Russian government in its election interference activities.

The Russian contacts consisted of business connections, offers of assistance to the Campaign, invitations for candidate Trump and Putin to meet in person, invitations for Campaign officials and representatives of the Russian government to meet, and policy positions seeking improved U.S.-Russian relations.
Section IV of this Report details the contacts between Russia and the Trump Campaign during the campaign and transition periods, the most salient of which are summarized below in chronological order.

\textbf{2015}.
Some of the earliest contacts were made in connection with a Trump Organization real-estate project in Russia known as Trump Tower Moscow.
Candidate Trump signed a Letter of Intent for Trump Tower Moscow by November 2015, and in January 2016 Trump Organization executive Michael Cohen emailed and spoke about the project with the office of Russian government press secretary Dmitry Peskov.
The Trump Organization pursued the project through at least June 2016, including by considering travel to Russia by Cohen and candidate Trump.

\textbf{Spring 2016}.
Campaign foreign policy advisor George Papadopoulos made early contact with Joseph Mifsud, a London-based professor who had connections to Russia and traveled to Moscow in April 2016.
Immediately upon his return to London from that trip, Mifsud told Papadopoulos that the Russian government had "dirt" on Hillary Clinton in the form of thousands of emails.
One week later, in the first week of May 2016, Papadopoulos suggested to a representative of a foreign government that the Trump Campaign had received indications from the Russian government that it could assist the Campaign through the anonymous release of information damaging to candidate Clinton.
Throughout that period of time and for several months thereafter, Papadopoulos worked with Mifsud and two Russian nationals to arrange a meeting between the Campaign and the Russian government.
No meeting took place.

\textbf{Summer 2016}.
Russian outreach to the Trump Campaign continued into the summer of 2016, as candidate Trump was becoming the presumptive Republican nominee for President.
On June 9, 2016, for example, a Russian lawyer met with senior Trump Campaign officials Donald Trump Jr., Jared Kushner, and campaign chairman Paul Manafort to deliver what the email proposing the meeting had described as "official documents and information that would incriminate Hillary."
The materials were offered to Trump Jr. as "part of Russia and its government's support for Mr. Trump."
The written communications setting up the meeting showed that the Campaign anticipated receiving information from Russia that could assist candidate Trump's electoral prospects, but the Russian lawyer's presentation did not provide such information.

Days after the June 9 meeting, on June 14, 2016, a cybersecurity firm and the DNC announced that Russian government hackers had infiltrated the DNC and obtained access to opposition research on candidate Trump, among other documents.

In July 2016, Campaign foreign policy advisor Carter Page traveled in his personal capacity to Moscow and gave the keynote address at the New Economic School.
Page had lived and worked in Russia between 2003 and 2007.
After returning to the United States, Page became acquainted with at least two Russian intelligence officers, one of whom was later charged in 2015 with conspiracy to act as an unregistered agent of Russia.
Page's July 2016 trip to Moscow and his advocacy for pro-Russian foreign policy drew media attention.
The Campaign then distanced itself from Page and, by late September 2016, removed him from the Campaign.

July 2016 was also the month WikiLeaks first released emails stolen by the GRU from the DNC.
On July 22, 2016, WikiLeaks posted thousands of internal DNC documents revealing information about the Clinton Campaign.
Within days, there was public reporting that U.S. intelligence agencies had "high confidence" that the Russian government was.behind the theft of emails and documents from the DNC.
And within a week of the release, a foreign government informed the FBI about its May 2016 interaction with Papadopoulos and his statement that the Russian government could assist the Trump Campaign.
On July 31, 2016, based on the foreign government reporting, the FBI opened an investigation into potential coordination between the Russian government and individuals associated with the Trump Campaign.

Separately, on August 2, 2016, Trump campaign chairman Paul Manafort met in New York City with his long-time business associate Konstantin Kilimnik, who the FBI assesses to have ties to Russian intelligence.
Kilimnik requested the meeting to deliver in person a peace plan for Ukraine that Manafort acknowledged to the Special Counsel's Office was a "backdoor" way for Russia to control part of eastern Ukraine; both men believed the plan would require candidate Trump's assent to succeed (were he to be elected President).
They also discussed the status of the Trump Campaign and Manafort's strategy for winning Democratic votes in Midwestern states.
Months before that meeting, Manafort had caused internal polling data to be shared with Kilimnik, and the sharing continued for some period of time after their August meeting.

\textbf{Fall 2016}.
On October 7, 2016, the media released video of candidate Trump speaking in graphic terms about women years earlier, which was considered damaging to his candidacy.
Less than an hour later, WikiLeaks made its second release: thousands of John Podesta' s emails that had been stolen by the GRU in late March 2016.
The FBI and other U.S. government institutions were at the time continuing their investigation of suspected Russian government efforts to interfere in the presidential election.
That same day, October 7, the Department of Homeland Security and the Office of the Director of National Intelligence issued a joint public statement "that the Russian Government directed the recent compromises of e-mails from US persons and institutions, including from US political organizations."
Those "thefts" and the "disclosures" of the hacked materials through online platforms such as WikiLeaks, the statement continued, "are intended to interfere with the US election process."

\textbf{Post-2016 Election}.
Immediately after the November 8 election, Russian government officials and prominent Russian businessmen began trying to make inroads into the new administration.
The most senior levels of the Russian government encouraged these efforts.
The Russian Embassy made contact hours after the election to congratulate the President-Elect and to arrange a call with President Putin.
Several Russian businessmen picked up the effort from there.

Kirill Dmitriev, the chief executive officer of Russia's sovereign wealth fund, was among the Russians who tried to make contact with the incoming administration.
In early December, a business associate steered Dmitriev to Erik Prince, a supporter of the Trump Campaign and an associate of senior Trump advisor Steve Bannon.
Dmitriev and Prince later met face-to-face in January 2017 in the Seychelles and discussed U.S.-Russia relations.
During the same period, another business associate introduced Dmitriev to a friend of Jared Kushner who had not served on the Campaign or the Transition Team.
Dmitriev and Kushner's friend collaborated on a short written reconciliation plan for the United States and Russia, which Dmitriev implied had been cleared through Putin.
The friend gave that proposal to Kushner before the inauguration, and Kushner later gave copies to Bannon and incoming Secretary of State Rex Tillerson.

On December 29, 2016, then-President Obama imposed sanctions on Russia for having interfered in the election.
Incoming National Security Advisor Michael Flynn called Russian Ambassador Sergey Kislyak and asked Russia not to escalate the situation in response to the sanctions.
The following day, Putin announced that Russia would not take retaliatory measures in response to the sanctions at that time.
Hours later, President-Elect Trump tweeted, "Great move on delay (by V. Putin)."
The next day, on December 31, 2016, Kislyak called Flynn and told him the request had been received at the highest levels and Russia had chosen not to retaliate as a result of Flynn's request.

\hr

On January 6, 2017, members of the intelligence community briefed President-Elect Trump on a joint assessment-drafted and coordinated among the Central Intelligence Agency, FBI, and National Security Agency-that concluded with high confidence that Russia had intervened in the election through a variety of means to assist Trump's candidacy and harm Clinton's.
A declassified version of the assessment was publicly released that same day.

Between mid-January 2017 and early February 2017, three congressional committees-the House Permanent Select Committee on Intelligence (HPSCI), the Senate Select Committee on Intelligence (SSCI), and the Senate Judiciary Committee (SJC)-announced that they would conduct inquiries, or had already been conducting inquiries, into Russian interference in the election.
Then-FBI Director James Corney later confirmed to Congress the existence of the FBI's investigation into Russian interference that had begun before the election.
On March 20, 2017, in open-session testimony before HPSCI, Corney stated:

\begin{quote}
I have been authorized by the Department of Justice to confirm that the FBI, as part of our counterintelligence mission, is investigating the Russian government's efforts to interfere in the 2016 presidential election, and that includes investigating the nature of any links between individuals associated with the Trump campaign and the Russian government and whether there was any coordination between the campaign and Russia's efforts.
...
As with any counterintelligence investigation, this will also include an assessment of whether any crimes were committed.
\end{quote}

The investigation continued under then-Director Corney for the next seven weeks until May 9, 2017, when President Trump fired Corney as FBI Director-an action which is analyzed in Volume II of the report.

On May 17, 2017, Acting Attorney General Rod Rosenstein appointed the Special Counsel and authorized him to conduct the investigation that Corney had confirmed in his congressional testimony, as well as matters arising directly from the investigation, and any other matters within the scope of 28 C.F.R. \S 600.4(a), which generally covers efforts to interfere with or obstruct the investigation.

President Trump reacted negatively to the Special Counsel's appointment.
He told advisors that it was the end of his presidency, sought to have Attorney General Jefferson (Jeff) Sessions unrecuse from the Russia investigation and to have the Special Counsel removed, and engaged in efforts to curtail the Special Counsel's investigation and prevent the disclosure of evidence to it, including through public and private contacts with potential witnesses.
Those and related actions are described and analyzed in Volume II of the report.

\hr

\subsection{The Special Counsel's Charging Decisions}

In reaching the charging decisions described in Volume 1 of the report, the Office determined whether the conduct it found amounted to a violation of federal criminal law chargeable under the Principles of Federal Prosecution.
See Justice Manual \S 9-27.000 et seq. (2018).
The standard set forth in the Justice Manual is whether the conduct constitutes a crime; if so, whether admissible evidence would probably be sufficient to obtain and sustain a conviction; and whether prosecution would serve a substantial federal interest that could not be adequately served by prosecution elsewhere or through non-criminal alternatives.
See Justice Manual \S 9-27 .220.

Section V of the report provides detailed explanations of the Office's charging decisions, which contain three main components.

First, the Office determined that Russia's two principal interference operations in the 2016 U.S. presidential election-the social media campaign and the hacking-and-dumping operations-violated U.S. criminal law.
Many of the individuals and entities involved in the social media campaign have been charged with participating in a conspiracy to defraud the United States by undermining through deceptive acts the work of federal agencies charged with regulating foreign influence in U.S. elections, as well as related counts of identity theft. See \textit{United States v. Internet Research Agency, et al.}, No. 18-cr-32 (D.D.C.).
Separately, Russian intelligence officers who carried out the hacking into Democratic Party computers and the personal email accounts of individuals affiliated with the Clinton Campaign conspired to violate, among other federal laws, the federal computer-intrusion statute, and they have been so charged.
\textit{See United States v. Netyksho, et al.}, No. 18-cr-215 (D.D.C.).

\xblackout{Harm to Ongoing Matter: Lorem ipsum dolor sit amet, consectetur adipiscing elit, sed do eiusmod tempor incididunt ut labore et dolore magna aliqua.}
\xblackout{Personal Privacy: Lorem ipsum dolor sit amet, consectetur adipiscing elit, sed do eiusmod tempor incididunt ut labore et dolore magna aliqua.}

Second, while the investigation identified numerous links between individuals with ties to the Russian government and individuals associated with the Trump Campaign, the evidence was not sufficient to support criminal charges.
Among other things, the evidence was not sufficient to charge any Campaign official as an unregistered agent of the Russian government or other Russian principal.
And our evidence about the June 9, 2016 meeting and WikiLeaks' s releases of hacked materials was not sufficient to charge a criminal campaign-finance violation.
Further, the evidence was not sufficient to charge that any member of the Trump Campaign conspired with representatives of the Russian government to interfere in the 2016 election.

Third, the investigation established that several individuals affiliated with the Trump Campaign lied to the Office, and to Congress, about their interactions with Russian-affiliated individuals and related matters.
Those lies materially impaired the investigation of Russian election interference.
The Office charged some of those lies as violations of the federal false-statements statute.
Former National Security Advisor Michael Flynn pleaded guilty to lying about his interactions with Russian Ambassador Kislyak during the transition period.
George Papadopoulos, a foreign policy advisor during the campaign period, pleaded guilty to lying to investigators about, inter alia, the nature and timing of his interactions with Joseph Mifsud, the professor who told Papadopoulos that the Russians had dirt on candidate Clinton .in the form of thousands of emails.
Former Trump Organization attorney Michael Cohen pleaded guilt to making false statements to Congress about the Trump Moscow project.
\xblackout{Harm to Ongoing Matter: Lorem ipsum dolor sit amet, consectetur adipiscing elit, sed do eiusmod tempor incididunt ut labore et dolore magna aliqua.}
And in February 2019, the U.S. District Court for the District of Columbia found that Manafort lied to the Office and the grand jury concerning his interactions and communications with Konstantin Kilimnik about Trump Campaign polling data and a peace plan for Ukraine.

\hr

The Office investigated several other events that have been publicly repot1ed to involve potential Russia-related contacts.
For example, the investigation established that interactions between Russian Ambassador Kislyak and Trump Campaign officials both at the candidate's April 2016 foreign policy speech in Washington, D.C., and during the week of the Republican National Convention were brief, public, and non-substantive.
And the investigation did not establish that one Campaign official's efforts to dilute a portion of the Republican Party platform on providing assistance to Ukraine were undertaken at the behest of candidate Trump or Russia.
The investigation also did not establish that a meeting between Kislyak and Sessions in September 2016 at Sessions's Senate office included any more than a passing mention of the presidential campaign.

The investigation did not always yield admissible information or testimony, or a complete picture of the activities undertaken by subjects of the investigation.
Some individuals invoked their Fifth Amendment right against compelled self-incrimination and were not, in the Office' s judgment, appropriate candidates for grants of immunity.
The Office limited its pursuit of other witnesses and information-such as information known to attorneys or individuals claiming to be members of the media-in light of internal Depa11ment of Justice policies.
See, e.g., Justice Manual\S\S 9-13.400, 13.410. Some of the information obtained via court process, moreover, was presumptively covered by legal privilege and was screened from investigators by a filter ( or "taint") team.
Even when individuals testified or agreed to be interviewed, they sometimes provided information that was false or incomplete, leading to some of the false-statements charges described above.
And the Office faced practical limits on its ability to access relevant evidence as well-numerous witnesses and subjects lived abroad, and documents were held outside the United States.

Further, the Office learned that some of the individuals we interviewed or whose conduct we investigated - including some associated with the Trump Campaign - deleted relevant communications or communicated during the relevant period using applications that feature encryption or that do not provide for long-term retention of data or communications records.
In such cases, the Office was not able to corroborate witness statements through comparison to contemporaneous communications or fully question witnesses about statements that appeared inconsistent with other known facts.

Accordingly, while this report embodies factual and legal determinations that the Office believes to be accurate and complete to the greatest extent possible, given these identified gaps, the Office cannot rule out the possibility that the unavailable information would shed additional light on (or cast in a new light) the events described in the report.

\section{The Special Counsel's Investigation}

On May 17, 2017, Deputy Attorney General Rod J. Rosenstein-then serving as Acting Attorney General for the Russia investigation following the recusal of former Attorney General Jeff Sessions on March 2, 2016-appointed the Special Counsel "to investigate Russian interference with the 2016 presidential election and related matters."
Office of the Deputy Att'y Gen., Order No. 3915-2017, Appointment of Special Counsel to Investigate Russian Interference with the 2016 Presidential Election and Related Matters, May 17, 2017) ("Appointment Order").
Relying on "the authority vested" in the Acting Attorney General," including 28 U.S.C. \S\S 509, 510, and 515," the Acting Attorney General ordered the appointment of a Special Counsel " in order to discharge [the Acting Attorney General' s] responsibility to provide supervision and management of the Department of Justice, and to ensure a full and thorough investigation of the Russian government's efforts to interfere in the 2016 presidential election." Appointment Order (introduction).
"The Special Counsel," the Order stated, "is authorized to conduct the investigation confirmed by then-FBI Director James B. Corney in testimony before the House Permanent Select Committee on Intelligence on March 20, 2017," including:

\begin{enumerate}[i]
  \item any links and/or coordination between the Russian government and individuals associated with the campaign of President Donald Trump; and
  \item any matters that arose or may arise directly from the investigation; and
  \item any other matters within the scope of 28 C.F.R. \S 600.4(a).
\end{enumerate}

Appointment Order \P (b). Section 600.4 affords the Special Counsel "the authority to investigate and prosecute federal crimes committed in the course of, and with intent to interfere with, the Special Counsel's investigation, such as perjury, obstruction of justice, destruction of evidence, and intimidation of witnesses." 28 C.F.R. \S 600.4(a).
The authority to investigate "any matters that arose ... directly from the investigation," Appointment Order \P (b)(ii), covers similar crimes that may have occurred during the course of the FBI's confirmed investigation before the Special Counsel's appointment.
"If the Special Counsel believes it is necessary and appropriate," the Order further provided, "the Special Counsel is authorized to prosecute federal crimes arising from the investigation of these matters." Id. \P ( c ).
Finally, the Acting Attorney General made applicable " Sections 600.4 through 600.10 of Title 28 of the Code of Federal Regulations." Id. \P (d).

The Acting Attorney General further clarified the scope of the Special Counsel's investigatory authority in two subsequent memoranda.
A memorandum dated August 2, 2017, explained that the Appointment Order had been "worded categorically in order to permit its public release without confirming specific investigations involving specific individuals."
It then confirmed that the Special Counsel had been authorized since his appointment to investigate allegations that three Trump campaign officials-Carter Page, Paul Manafort, and George Papadopoulos-"committed a crime or crimes by colluding with Russian government officials with respect to the Russian government's efforts to interfere with the 2016 presidential election."
The memorandum also confirmed the Special Counsel's authority to investigate certain other matters, including two additional sets of allegations involving Manafort (crimes arising from payments he received from the Ukrainian government and crimes arising from his receipt of loans from a bank whose CEO was then seeking a position in the Trump Administration); allegations that Papadopoulos committed a crime or crimes by acting as an unregistered agent of the Israeli government; and four sets of allegations involving Michael Flynn, the former National Security Advisor to President Trump.

On October 20, 2017, the Acting Attorney General confirmed in a memorandum the Special Counsel's investigative authority as to several individuals and entities.
First," as part of a full and thorough investigation of the Russian government's efforts to interfere in the 2016 presidential election," the Special Counsel was authorized to investigate "the pertinent activities of Michael Cohen, Richard Gates, \xblackout{Personal Privacy}, Roger Stone, and \xblackout{Personal Privacy}" "Confirmation of the authorization to investigate such individuals," the memorandum stressed, "does not suggest that the Special Counsel has made a determination that any of them has committed a crime."
Second, with respect to Michael Cohen, the memorandum recognized the Special Counsel's authority to investigate "leads relate[d] to Cohen's establishment and use of Essential Consultants LLC to, inter alia, receive funds from Russian-backed entities."
Third, the memorandum memorialized the Special Counsel's authority to investigate individuals and entities who were possibly engaged in "jointly undertaken activity" with existing subjects of the investigation, including Paul Manafort.
Finally, the memorandum described an FBI investigation opened before the Special Counsel's appointment into "allegations that [then-Attorney General Jeff Sessions] made false statements to the United States Senate[,]" and confirmed the Special Counsel's authority to investigate that matter.

The Special Counsel structured the investigation in view of his power and authority "to exercise all investigative and prosecutorial functions of any United States Attorney." 28 C.F.R: § 600.6. Like a U.S. Attorney's Office, the Special Counsel's Office considered a range of classified and unclassified information available to the FBI in the course of the Office's Russia investigation, and the Office structured that work around evidence for possible use in prosecutions of federal crimes (assuming that one or more crimes were identified that warranted prosecution).
There was substantial evidence immediately available to the Special Counsel at the inception of the investigation in May 2017 because the FBI had, by that time, already investigated Russian election interference for nearly 10 months.
The Special Counsel's Office exercised its judgment regarding what to investigate and did not, for instance, investigate every public report of a contact between the Trump Campaign and Russian-affiliated individuals and entities.

The Office has concluded its investigation into links and coordination between the Russian government and individuals associated with the Trump Campaign.
Certain proceedings associated with the Office's work remain ongoing.
After consultation with the Office of the Deputy Attorney General, the Office has transferred responsibility for those remaining issues to other components of the Department of Justice and FBI.
Appendix D lists those transfers.

Two district courts confirmed the breadth of the Special Counsel's authority to investigate Russia election interference and links and/or coordination with the Trump Campaign.
See United States v. Manafort, 312 F. Supp. 3d 60, 79-83 (D.D.C. 2018); United States v. Manafort, 321 F. Supp. 3d 640, 650-655 (E.D. Va. 2018).
In the course of conducting that investigation, the Office periodically identified evidence of potential criminal activity that was outside the scope of the Special Counsel's authority established by the Acting Attorney General.
After consultation with the Office of the Deputy Attorney General, the Office referred that evidence to appropriate law enforcement authorities, principally other components of the Department of Justice and to the FBI.
Appendix D summarizes those referrals.

\hr

To carry out the investigation and prosecution of the matters assigned to him, the Special Counsel assembled a team that at its high point included 19 attorneys-five of whom joined the Office from private practice and 14 on detail or assigned from other Department of Justice components.
These attorneys were assisted by a filter team of Department lawyers and FBI personnel who screened materials obtained via court process for privileged information before turning those materials over to investigators; a support staff of three paralegals on detail from the Department's Antitrust Division; and an administrative staff of nine responsible for budget, finance, purchasing, human resources, records, facilities, security, information technology, and administrative support.
The Special Counsel attorneys and support staff were co-located with and worked alongside approximately 40 FBI agents, intelligence analysts, forensic accountants, a paralegal, and professional staff assigned by the FBI to assist the Special Counsel's investigation.
Those "assigned" FBI employees remained under FBI supervision at all times; the matters on which they assisted were supervised by the Special Counsel.%
\footnote{FBI personnel assigned to the Special Counsel's Office were required to adhere to all applicable federal law and all Department and FBI regulations, guidelines, and policies.
An FBI attorney worked on FBI-related matters for the Office, such as FBI compliance with all FBI policies and procedures, including the FBI's Domestic Investigations and Operations Guide (DIOG).
That FBI attorney worked under FBI legal supervision, not the Special Counsel's supervision.}

During its investigation the Office issued more than 2,800 subpoenas under the auspices of a grand jury sitting in the District of Columbia; executed nearly 500 search-and-seizure warrants; obtained more than 230 orders for communications records under 18 U.S.C. § 2703(d); obtained almost 50 orders authorizing use of pen registers; made 13 requests to foreign governments pursuant to Mutual Legal Assistance Treaties; and interviewed approximately 500 witnesses, including almost 80 before a grand jury.

\hr

From its inception, the Office recognized that its investigation could identify foreign intelligence and counterintelligence information relevant to the FBI's broader national security mission.
FBI personnel who assisted the Office established procedures to identify and convey such information to the FBI.
The FBI's Counterintelligence Division met with the Office regularly for that purpose for most of the Office's tenure.
For more than the past year, the FBI also embedded personnel at the Office who did not work on the Special Counsel's investigation, but whose purpose was to review the results of the investigation and to send-in writing-summaries of foreign intelligence and counterintelligence information to FBIHQ and FBI Field Offices.
Those communications and other correspondence between the Office and the FBI contain information derived from the investigation, not all of which is contained in this Volume.
This Volume is a summary.
It contains, in the Office's judgment, that information necessary to account for the Special Counsel's prosecution and declination decisions and to describe the investigation's main factual results.

\section{Russian ``Active Measures'' Social Media Campaign}

The first form of Russian election influence came principally from the Internet Research Agency, LLC (IRA), a Russian organization funded by Yevgeniy Viktorovich Prigozhin and companies he controlled, including Concord Management and Consulting LLC and Concord Catering (collectively "Concord").%
\footnote{The Office is aware of reports that other Russian entities engaged in similar active measures operations targeting the United States.
Some evidence collected by the Office corroborates those reports, and the Office has shared that evidence with other offices in the Department of Justice and FBI.}
The IRA conducted social media operations targeted at large U.S. audiences with the goal of sowing discord in the U.S. political system.%
\footnote{\xblackout{Harm to Ongoing Matter: Lorem ipsum dolor sit amet, consectetur adipiscing elit}
see also SM-2230634, serial 44 (analysis).
The FBI case number cited here, and other FBI case numbers identified in the report, should be treated as law enforcement sensitive given the context.
The report contains additional law enforcement sensitive information.}
These operations constituted "active measures" (активные мероприятия), a term that typically refers to operations conducted by Russian security services aimed at influencing the course of international affairs.%
\footnote{As discussed in Part V below, the active measures investigation has resulted in criminal charges against 13 individual Russian nationals and three Russian entities, principally for conspiracy to defraud the United States, in violation of 18 U.S.C. § 371.
See Volume I, Section V.A, infra; Indictment, United States v. Internet Research Agency, et al., 1 :18-cr-32 (D.D.C. Feb. 16, 2018), Doc. 1 ("Internet Research Agency Indictment").}

The IRA and its employees began operations targeting the United States as early as 2014.
Using fictitious U.S. personas, IRA employees operated social media accounts and group pages designed to attract U.S. audiences.
These groups and accounts, which addressed divisive U.S. political and social issues, falsely claimed to be controlled by U.S. activists.
Over time, these social media accounts became a means to reach large U.S. audiences.
IRA employees travelled to the United States in mid-2014 on an intelligence-gathering mission to obtain information and photographs for use in their social media posts.

IRA employees posted derogatory information about a  number of candidates in the 2016 U.S. presidential election.
By early to mid-2016, IRA operations included supporting the Trump Campaign and disparaging candidate Hillary Clinton.
The IRA made various expenditures to carry out those activities, including buying political advertisements on social media in the names of U.S. persons and entities.
Some IRA employees, posing as U.S. persons and without revealing their Russian association, communicated electronically with individuals associated with the Trump Campaign and with other political activists to seek to coordinate political activities, including the staging of political rallies.%
\footnote{Internet Research Agency Indictment \S\S 52, 54, 55(a), 56, 74; \xblackout{Harm to Ongoing Matter: Lorem ipsum dolor sit amet, consectetur adipiscing elit, sed do eiusmod tempor}}
The investigation did not identify evidence that any U.S. persons knowingly or intentionally coordinated with the IRA's interference operation.

By the end of the 2016 U.S. election, the IRA had the ability to reach millions of U.S. persons through their social media accounts.
Multiple IRA-controlled Facebook groups and Instagram accounts had hundreds of thousands of U.S. participants.
IRA-controlled Twitter accounts separately had tens of thousands of followers, including multiple U.S. political figures who retweeted IRA-created content.
In November 2017, a Facebook representative testified that Facebook had identified 470 IRA-controlled Facebook accounts that collectively made 80,000 posts between January 2015 and August 2017.
Facebook estimated the IRA reached as many as 126 million persons through its Face book accounts.%
\footnote{Social Media Influence in the 2016 US. Election, Hearing Before the Senate Select Committee on Intelligence, 115th Cong. 13 (11/1/17) (testimony of Colin Stretch, General Counsel of Facebook)
("We estimate that roughly 29 million people were served content in their News Feeds directly from the IRA's 80,000 posts over the two years.
Posts from these Pages were also shared, liked, and followed by people on Facebook, and, as a result, three times more people may have been exposed to a story that originated from the Russian operation.
Our best estimate is that approximately 126 million people may have been served content from a Page associated with the IRA at some point during the two-year period.").
The Facebook representative also testified that Facebook had identified 170 Instagram accounts that posted approximately 120,000 pieces of content during that time.
Facebook did not offer an estimate of the audience reached via Instagram.}
In January 2018, Twitter announced that it had identified 3,814 IRA-controlled Twitter accounts and notified approximately 1 .4 million people Twitter believed may have been in contact with an iRA-controlled account.%
\footnote{Twitter, Update on Twitter's Review of the 2016 US Election (Jan. 31, 2018).}

\subsection{Structure of the Internet Research Agency}

\xblackout{Harm to Ongoing Matter: Lorem ipsum dolor sit amet, consectetur adipiscing elit, sed do eiusmod tempor}%
\footnote{See SM-2230634, serial 92.}
\xblackout{Harm to Ongoing Matter: Lorem ipsum dolor sit amet, consectetur adipiscing elit, sed do eiusmod tempor}%
\footnote{\xblackout{Harm to Ongoing Matter: Lorem ipsum dolor sit amet, consectetur adipiscing elit, sed do eiusmod tempor}}
\xblackout{Harm to Ongoing Matter: Lorem ipsum dolor sit amet, consectetur adipiscing elit, sed do eiusmod tempor}%
\footnote{\xblackout{Harm to Ongoing Matter: Lorem ipsum dolor sit amet, consectetur adipiscing elit, sed do eiusmod tempor}}

The organization quickly grew.
\xblackout{Harm to Ongoing Matter: Lorem ipsum dolor sit amet, consectetur adipiscing elit, sed do eiusmod tempor}%
\footnote{See SM-2230634, serial 86 \xblackout{Harm to Ongoing Matter: Lorem ipsum dolor sit amet, consectetur adipiscing elit, sed do eiusmod tempor}}
\xblackout{Harm to Ongoing Matter: Lorem ipsum dolor sit amet, consectetur adipiscing elit, sed do eiusmod tempor}%
\footnote{\xblackout{Harm to Ongoing Matter: Lorem ipsum dolor sit amet, consectetur adipiscing elit, sed do eiusmod tempor}}

The growth of the organization also led to a more detailed organizational structure.
\xblackout{Harm to Ongoing Matter: Lorem ipsum dolor sit amet, consectetur adipiscing elit, sed do eiusmod tempor}

\end{document}
